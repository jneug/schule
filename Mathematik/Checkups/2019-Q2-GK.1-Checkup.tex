\documentclass[11pt, a4paper]{scrartcl}

\usepackage{vorschule}
\usepackage[
    typ=ab,
    fach=Mathematik,
    lerngruppe={Q2 GK},
    nummer={1},
    module={Symbole},
]{schule}

\usepackage[
	typ=checkup,
	kuerzel=Ngb,
	reihe={Analytische Geometrie},
	version={2019-10-20}
]{ngbschule}


\author{J. Neugebauer}
\title{\Nummer. Klausur}
\date{\Heute}

\begin{document}
\CheckupBild\CheckupTitel

Kreuzen sie jeweils an, wie sicher sie sich bei den einzelnen \textbf{Themenschwerpunkten} fühlen (von \enquote{sehr sicher} \usym{1F604} bis \enquote{sehr unsicher} \usym{1F641}). Nutzen sie die \textbf{Aufgaben und Informationen} zum Wiederholen und Lernen von Themen, bei denen sie noch unsicher sind.

Die \textbf{Lösungen} zu den Aufgaben finden sie hinten im Buch. Weitere Aufgaben (ohne Lösungen) finden sie im Buch in den Kapiteln zum Thema.

\begin{checkup}
	\ichkann{die Begriffe \enquote{Vektor}, \enquote{Ortsvektor}, \enquote{Richtungsvektor} und \enquote{Verschiebungsvektor} erklären.}{
		\bu{158-159}{} \\
		\bu{218}{}
	}
	\ichkann{Vektoren darstellen und aus Darstellungen ablesen.}{
		\bu{154-155}{} \\
		\bu{221}{1}
	}
	\ichkann{die Formel zur Längenberechnung eines Vektors erklären und anwenden.}{
		\bu{159-160,164}{} \\
		\bu{218}{}
	}
	\ichkann{Vektoren addieren, subtrahieren und skalieren.}{
		\bu{162-163,169}{} \\
		\bu{218}{}
	}
	\ichkann{Verschiebungsvektoren aus zwei Punkten berechnen.}{
		\bu{157}{} \\
		\bu{218}{}
	}
	\ichkann{die Parameterdarstellung einer Geraden aufstellen.}{
		\bu{174-175}{} \\
		\bu{219}{} \\
		\bu{221}{2}
	}
	\ichkann{die Punktprobe durchführen (Liegt ein Punkt auf einer Geraden).}{
		\bu{174-175}{}\\
		\bu{221}{3,4}
	}
	\ichkann{zwei Geraden auf Schnittpunkte prüfen.}{
		\bu{181, 2. Schritt}{}
	}
\end{checkup}

\end{document}
