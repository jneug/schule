\documentclass[a4paper,ngerman,fontsize=10pt]{scrartcl}

\usepackage[checkup,typo,theme,qrcodes]{arbeitsblatt}

\aboptionen{
	fach=Mathematik,
	name=J. Neugebauer,
	kuerzel=Ngb,
	kurs=7c,
	nummer=6,
	reihe={Zinserechnung},
	titel={zur \Nummer. Arbeit},
	version={2021-04-13},
	lizenz=cc-by-nc-sa-eu-4,
}

\newcommand{\af}[1]{Aufg.#1}
\newcommand{\bu}[2]{Buch S.#1\ifthenelse{\equal{#2}{}}{}{, \af{#2}}}
\newcommand{\ah}[2]{AH S.#1\ifthenelse{\equal{#2}{}}{}{, \af{#2}}}
\newcommand{\ab}[2][AB]{#1 \enquote{#2}}

\begin{document}
\CheckupBild\CheckupTitel

\vspace{5cm}

\begin{checkup}
	\ichkann{Die Grundaufgaben der Prozentrechnung erkennen und lösen (prozentsatz, Grundwert, Prozentwert).}{
		Buch Kapitel III.1 - III.4 \\
		Material in Teams
	}
	\ichkann{Die Begriffe der Prozentrechnung auf die Zinsrechnung anwenden (Zinsen, Zinssatz, Guthaben, ...)}{
		Kapitel III.5 im Buch \\
		Material in Teams
	}
	\ichkann{Unterjährige Zinsen berechnen (z.B. für einen Monat)}{
		Kapitel III.5 im Buch \\
		Material in Teams
	}
	\ichkann{Zinseszinsen berechnen}{
		Kapitel III.6 im Buch  \\
		Material in Teams
	}
\end{checkup}

\end{document}
