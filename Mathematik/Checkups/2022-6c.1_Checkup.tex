\documentclass[checkup, 11pt, a4paper]{arbeitsblatt}

\ladeModule{typo,theme,qrcodes}
\aboptionen{
	fach		= Mathematik,
	name		= {J. Neugebauer},
	kuerzel		= Ngb,
	kurs		= {6c},
	nummer		= 1,
	reihe		= {Körper und Brüche},
	titel		= {zur 1. Arbeit},
	version		= {2022-09-04},
	lizenz		= {cc-by-nc-sa-eu-4},
}

\newcommand{\af}[1]{Aufg.\,#1}
\newcommand{\bu}[1]{Buch S.\,#1}
\newcommand{\buaf}[2]{Buch S.\,#1, \af{#2}}
\newcommand{\ah}[1]{AH S.\,#1}
\newcommand{\ahaf}[2]{AH S.\,#1, \af{#2}}
\newcommand{\ab}[2][AB]{#1 \enquote{#2}}

\begin{document}
\CheckupBild\CheckupTitel

Gehe noch einmal die Aufgaben der letzten Wochen durch. Kreuze jeweils an, wie
sicher du dich bei den einzelnen \textbf{Themenschwerpunkten} fühlst (von
\enquote{sehr sicher} \iconLachen\ bis \enquote{sehr unsicher} \iconTraurig).
Nutze die \textbf{Aufgaben und Informationen} zur Wiederholung der Themen,
\emph{bei denen du noch unsicher bist}.

Vor allem bieten sich dazu die \enquote{Teste dich!} Aufgaben im Buch, die
\emph{Tests} am Ende eines Kapitels und das \emph{Arbeitsheft}
an. Die \textbf{Lösungen} findest du hinten im Buch ab Seite 244, bzw. im
Arbeitsheft im dazugehörigen Lösungsheft.

\begin{checkup}
	\teiler{Körper}
	\ichkann{mit Volumeneinheiten umgehen.}{
		\ahaf{7}{5}
	}
	\ichkann{das Volumen eines Quaders berechnen.}{
		\ahaf{7}{4/6}
	}
	\ichkann{den Oberflächeninhalt eines Quaders bestimmen.}{
		\ahaf{7}{4} \\
		Arbeitsblätter
	}
	\teiler{Brüche}
	\ichkann{die Grundbegiffe der Bruchrechnung verwenden.}{
		Bruch, Zähler, Nenner
	}
	\ichkann{Anteile auf verschiedene Weise darstellen.}{
		\bu{8/9} \\
		\ah{9}
	}
	\ichkann{Anteile ablesen.}{
		\bu{8/9} \\
		\ah{9}
	}
	\ichkann{mit gemischten Brüchen umgehen.}{
		Arbeitsblätter
	}
\end{checkup}

\end{document}
