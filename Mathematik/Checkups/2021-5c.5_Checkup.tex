\documentclass[checkup, 11pt, a4paper]{arbeitsblatt}

\ladeModule{typo,theme,qrcodes}
\aboptionen{
	fach		= Mathematik,
	name		= {J. Neugebauer},
	kuerzel		= Ngb,
	kurs		= {5c},
	nummer		= 5,
	reihe		= {Rechnen},
	titel		= {zur 5. Arbeit},
	version		= {2022-03-28},
	lizenz		= {cc-by-nc-sa-eu-4},
}

\newcommand{\af}[1]{Aufg.\,#1}
\newcommand{\bu}[1]{Buch S.\,#1}
\newcommand{\buaf}[2]{Buch S.\,#1, \af{#2}}
\newcommand{\ah}[1]{AH S.\,#1}
\newcommand{\ahaf}[2]{AH S.\,#1, \af{#2}}
\newcommand{\ab}[2][AB]{#1 \enquote{#2}}

\begin{document}
\CheckupBild\CheckupTitel

Gehe noch einmal die Aufgaben der letzten Wochen durch. Kreuze jeweils an, wie
sicher du dich bei den einzelnen \textbf{Themenschwerpunkten} fühlst (von
\enquote{sehr sicher} \iconLachen\ bis \enquote{sehr unsicher} \iconTraurig).
Nutze die \textbf{Aufgaben und Informationen} zur Wiederholung der Themen,
\emph{bei denen du noch unsicher bist}.

Vor allem bieten sich dazu die \enquote{Teste dich!} Aufgaben im Buch, die
\emph{Tests} am Ende eines Kapitels und das \emph{Arbeitsheft}
an. Die \textbf{Lösungen} findest du hinten im Buch ab Seite 245, bzw. im
Arbeitsheft im dazugehörigen Lösungsheft.

\begin{checkup}
	\teiler{Schriftliches Rechnen}
	\ichkann{schriftlich dividieren.}{
		\bu{115/116} \\
		\buaf{117}{7} \\
		\buaf{118}{18} \\
		\ah{31}
	}
	\teiler{Rechenvorteile nutzen}
	\ichkann{das Kommutativ- und Assoziativgesetz anwenden.}{
		\bu{91/92} \\
		\buaf{92}{4} \\
		\buaf{93}{10/11} \\
		\ah{22}
	}
	\ichkann{das Distributivgesetz anwenden (Ausklammern und Ausmultiplizieren).}{
		\bu{94/95} \\
		\buaf{95}{7-9} \\
		\buaf{96}{16} \\
		\ah{23/24}
	}
	\ichkann{mit Potenzen rechnen.}{
		\bu{97/98} \\
		\buaf{98}{8/9} \\
		\buaf{99}{16} \\
		\ah{25}
	}
	\teiler{Sachaufgaben}
	\ichkann{Sachaufgaben systematisch lösen.}{
		\bu{119/120} \\
		\buaf{117}{8} \\
		\buaf{118}{19} \\
		\buaf{121}{6} \\
		\buaf{122}{12} \\
		\buaf{123}{8} \\
		\ah{32}
	}
\end{checkup}

\end{document}
