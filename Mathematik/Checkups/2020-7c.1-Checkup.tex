\documentclass[11pt, a4paper]{scrartcl}

\usepackage{vorschule}
\usepackage[
    typ=ab,
    fach=Mathematik,
    lerngruppe={Jg.7},
    nummer={1},
    module={Symbol},
]{schule}

\usepackage[
	typ=checkup,
	kuerzel=Ngb,
	reihe={Rationale Zahlen},
]{ngbschule}


\author{J. Neugebauer}
\title{Rationale Zahlen}
\date{\Heute}

\begin{document}
\CheckupBild\CheckupTitel

Gehe noch einmal die Aufgaben der letzten Wochen durch. Kreuze jeweils an, wie sicher du dich bei den einzelnen \textbf{Themenschwerpunkten} fühlst (von \enquote{sehr sicher} \usym{1F604} bis \enquote{sehr unsicher} \usym{1F641}).

Notiere dir unter \textbf{Aufgaben und Informationen} Hilfen zur Wiederholung und Lernen der Themen, \emph{bei denen du noch unsicher bist}. Vor allem bieten sich dazu die \enquote{Teste dich!} Aufgaben im Buch, die \emph{Tests} am Ende eines Kapitels und das \emph{Arbeitsheft} an. Die \textbf{Lösungen} findest du hinten im Buch ab Seite 238, bzw. im Arbeitsheft im dazugehörigen Lösungsheft.

\begin{checkup}
	\teiler{Rationale Zahlen}
	\ichkann{mit rationalen Zahlen rechnen (positive und negative Zahlen, Brüche, Dezimalzahlen; Addition, Subtraktion, Multiplikation, Division).}{\Zeilenabstand[1.5cm]}
	\ichkann{Rechenvorteile auch bei rationalen Zahlen anwenden, um geschickt zu rechnen.}{\Zeilenabstand[1.5cm]}
	
	\teiler{Zuordnungen}
	\ichkann{beschreiben was ein Zuordnung ist.}{\Zeilenabstand[1.5cm]}
	\ichkann{beschreiben was die Eindeutigkeit einer Zuordnung bedeutet.}{\Zeilenabstand[1.5cm]}
	\ichkann{Zuordnungen als Pfeildiagramm, Wertetabelle und Graph darstellen.}{\Zeilenabstand[1.5cm]}
	\ichkann{Pfeildiagramme, Wertetabellen und Graphen von Zuordnungen lesen und interpretieren.}{\Zeilenabstand[1.5cm]}
	\ichkann{die Formel einer Zuordnung benutzen, um Wertepaare zu berechnen.}{\Zeilenabstand[1.5cm]}
\end{checkup}

\end{document}
