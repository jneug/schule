\documentclass[checkup, 11pt, a4paper]{arbeitsblatt}

\ladeModule{typo,theme,qrcodes}
\aboptionen{
	fach		= Mathematik,
	name		= {J. Neugebauer},
	kuerzel		= Ngb,
	kurs		= {06c},
	nummer		= 2,
	reihe		= {Bruchrechnung},
	titel		= {zur 2. Arbeit},
	version		= {2022-10-25},
	lizenz		= {cc-by-nc-sa-eu-4},
}

\newcommand{\af}[1]{Aufg.\,#1}
\newcommand{\bu}[1]{Buch S.\,#1}
\newcommand{\buaf}[2]{Buch S.\,#1, \af{#2}}
\newcommand{\ah}[1]{AH S.\,#1}
\newcommand{\ahaf}[2]{AH S.\,#1, \af{#2}}
\newcommand{\ab}[2][AB]{#1 \enquote{#2}}

\begin{document}
\CheckupBild\CheckupTitel

Gehe noch einmal die Aufgaben der letzten Wochen durch. Kreuze jeweils an, wie
sicher du dich bei den einzelnen \textbf{Themenschwerpunkten} fühlst (von
\enquote{sehr sicher} \iconLachen\ bis \enquote{sehr unsicher} \iconTraurig).
Nutze die \textbf{Aufgaben und Informationen} zur Wiederholung der Themen,
\emph{bei denen du noch unsicher bist}.

Vor allem bieten sich dazu die \enquote{Teste dich!} Aufgaben im Buch, die
\emph{Tests} am Ende eines Kapitels und das \emph{Arbeitsheft}
an. Die \textbf{Lösungen} findest du hinten im Buch ab Seite 244, bzw. im
Arbeitsheft im dazugehörigen Lösungsheft. Eine Überschrift des Kapitels findest Du auf Seite 36 im Buch.

\begin{checkup}
	\teiler{Anteile auf verschiedene Arten darstellen:}
	\ichkann{grafisch und als Bruchzahl}{
		\buaf{15}{8} \\
		\buaf{32}{1} \\
		\buaf{32}{1}
	}
	\ichkann{als Prozentzahl}{
		\buaf{32}{7} \\
		\buaf{32}{10} \\
		\buaf{25}{14}
	}
	\ichkann{als Quotient}{
		\buaf{27}{5}
	}
	\ichkann{als Verhältnis}{
		\buaf{33}{9} \\
		\buaf{32}{10}
	}
	\teiler{\ }
	\ichkann{Brüche erweitern und (vollständig) kürzen}{
		\buaf{15}{8/9} \\
		\buaf{17}{20/21} \\
		\buaf{32}{5} \\
		\buaf{32}{7} \\
		\buaf{32}{4}
	}
	\ichkann{Brüche vergleichen und ordnen}{
		\buaf{20}{7/8} \\
		\buaf{21}{16} \\
		\buaf{32}{11}
	}
	\ichkann{Brüche auf dem Zahlenstrahl anordnen}{
		\buaf{30}{5} \\
		\buaf{31}{12} \\
		\buaf{32}{8}
	}
\end{checkup}

\end{document}
