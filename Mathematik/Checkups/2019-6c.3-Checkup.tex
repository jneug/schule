\documentclass[11pt, a4paper]{scrartcl}

\usepackage{vorschule}
\usepackage[
    typ=ab,
    fach=Mathematik,
    lerngruppe={Jg.6},
    nummer={3},
    module={Symbol},
]{schule}

\usepackage[
	typ=checkup,
	kuerzel=Ngb,
	reihe={Kreise, Winkel und Dezimalzahlen},
]{ngbschule}


\author{J. Neugebauer}
\title{3. Mathearbeit}
\date{\Heute}

\begin{document}
\CheckupTitel

\CheckupBild Kreuze jeweils an, wie sicher du dich bei den einzelnen \textbf{Themenschwerpunkten} fühlst (von \enquote{sehr sicher} \usym{1F604} bis \enquote{sehr unsicher} \usym{1F641}). Nutze die \textbf{Aufgaben und Informationen} zum Wiederholen und Lernen von Themen, \emph{bei denen du noch unsicher bist}. Die \textbf{Lösungen} für die \enquote{Teste dich!} Aufgaben findest du hinten im Buch ab Seite 206. Die Lösungen für die Aufgaben im Arbeitsheft findest du im dazugehörigen Lösungsheft.

\begin{checkup}
	\ichkann{mit Brüchen rechnen, Brüche kürzen und mit Prozentzahlen umgehen.}{
		Checkup zur zweiten Arbeit.
	}
	
	\teiler{Winkel und Kreise}
	\ichkann{Arten von Winkeln erkennen und zeichnen.}{
		\bu{77/78}{} \\
		\bu{79}{6c)} \\
		\bu{84}{8a)}
	}
	\ichkann{Winkel mit dem Geodreieck zeichnen.}{
		\bu{82/83}{} \\
		\bu{84}{9} \\
		\bu{84}{8a)}
	}
	\ichkann{Winkel mit dem Geodreieck messen.}{
		\bu{82/83}{} \\
		\bu{79}{6a+b)} \\
		\bu{84}{8} \\
		\bu{86}{14a+b)}
	}
	\ichkann{Winkel berechnen.}{
		\bu{81}{11} \\
		\bu{86}{14c)}
	}
	\ichkann{Kreise und Kreismuster mit dem Zirkel zeichnen}{
		\bu{72/73}{} \\
		\bu{74}{8/9}
	}
	\ichkann{Kreise mit dem Geodreieck ausmessen (Radius, Durchmesser).}{s.o.}
	
	\teiler{Dezimalzahlen}
	\ichkann{Dezimalzahlen im Stellenwertsystem eintragen.}{
		\bu{14/15}{} \\
		\ah{11}{1}
	}
	\ichkann{Dezimalzahlen als Bruch mit Zehnerpotenz schrieben.}{
		\bu{14/15}{} \\
		\bu{16}{8}
	}
	\ichkann{Brüche mit Zehnerpotenz im Nenner als Dezimalzahl schreiben}{
		\bu{16}{8} \\
		\bu{17}{18a)} \\
		\ah{11}{2+5}
	}
	\ichkann{Brüche durch schriftliche Division als Dezimalzahl schreiben.}{
		\bu{14/15}{} \\
		\ah{11}{1}
	}
\end{checkup}

\end{document}
