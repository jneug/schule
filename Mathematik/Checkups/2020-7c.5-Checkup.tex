\documentclass[a4paper,ngerman,fontsize=10pt]{scrartcl}

\usepackage[checkup,typo,theme,qrcodes]{arbeitsblatt}

\aboptionen{
	fach=Mathematik,
	name=J. Neugebauer,
	kuerzel=Ngb,
	kurs=7c,
	nummer=5,
	reihe={Terme und Gleichungen},
	titel={zur \Nummer. Arbeit},
	version={2021-04-13},
	lizenz=cc-by-nc-sa-eu-4,
}

\newcommand{\af}[1]{Aufg.#1}
\newcommand{\bu}[2]{Buch S.#1\ifthenelse{\equal{#2}{}}{}{, \af{#2}}}
\newcommand{\ah}[2]{AH S.#1\ifthenelse{\equal{#2}{}}{}{, \af{#2}}}
\newcommand{\ab}[2][AB]{#1 \enquote{#2}}

\begin{document}
\CheckupBild\CheckupTitel

Gehe noch einmal die Aufgaben der letzten Wochen durch. Kreuze jeweils an, wie
sicher du dich bei den einzelnen \textbf{Themenschwerpunkten} fühlst (von
\enquote{sehr sicher} \iconLachen\ bis \enquote{sehr unsicher} \iconTraurig).

Notiere dir unter \textbf{Aufgaben und Informationen} Hilfen zur Wiederholung
und Lernen der Themen, \emph{bei denen du noch unsicher bist}. Vor allem bieten
sich dazu die \enquote{Teste dich!} Aufgaben im Buch, die \emph{Tests} am Ende
eines Kapitels und das \emph{Arbeitsheft} an. Die \textbf{Lösungen} findest du
hinten im Buch ab Seite 238, bzw. im Arbeitsheft im dazugehörigen Lösungsheft.

Wir haben in der Schule auch einige Apps auf dem iPad eingesetzt, die für das
Üben hilfreich sein können:

\qrlink{https://apps.apple.com/de/app/maphi-die-mathe-app/id1276181580}[\enquote{Maphi} im AppStore]{
	Gleichungen durch Wischgesten lösen.
}
\qrlink{https://apps.apple.com/de/app/algebra-touch-all-ages/id1512834904}[\enquote{Algebra Touch} im AppStore]{
	Gleichungen durch Wischgesten lösen.
}
\qrlink{https://apps.apple.com/de/app/lineare-gleichungen/id1199448971}[\enquote{Lineare Gleichungen} im AppStore]{
	Rechenaufgaben zu Gleichungen
}

\begin{checkup}
	\ichkann{Terme mit Rechenregeln umformen und vereinfachen}{
		Kapitel IV.2 und IV.3 im Buch
	}
	\ichkann{Gleichungen mit Äquivalenzumformungen lösen}{
		Kapitel IV.5 im Buch \\
		\qrcode[nolink]{https://www.youtube.com/watch?v=Lws4kWu2Z9g} \\[2.5em]
	}
	\ichkann{Gleichungen zu Sachaufgaben aufstellen und lösen}{
		Kapitel IV.4 und IV.7 im Buch
	}
\end{checkup}

\end{document}
