\documentclass[11pt, a4paper]{scrartcl}

\usepackage{vorschule}
\usepackage[
    typ=ab,
    fach=Mathematik,
    lerngruppe={Jg.6},
    nummer={5},
    module={Symbole},
]{schule}

\usepackage[
	typ=checkup,
	kuerzel=Ngb,
	reihe={Rechnen},
]{ngbschule}


\author{J. Neugebauer}
\title{5. Mathearbeit}
\date{\Heute}

\begin{document}
\CheckupBild\CheckupTitel

Kreuze jeweils an, wie sicher du dich bei den einzelnen \textbf{Themenschwerpunkten} fühlst (von \enquote{sehr sicher} \usym{1F604} bis \enquote{sehr unsicher} \usym{1F641}). Nutze die \textbf{Aufgaben und Informationen} zum Wiederholen und Lernen von Themen, \emph{bei denen du noch unsicher bist}.

Die \textbf{Lösungen} für die \enquote{Teste dich!} Aufgaben findest du hinten im Buch ab Seite 206. Die Lösungen für die Aufgaben im Arbeitsheft findest du im dazugehörigen Lösungsheft.

\begin{checkup}
	\ichkann{\dots mit Größen umgehen.}{
		\bu{26/27}{} \\
		\bu{201/204/205}{}
	}
	\ichkann{\dots mit Brüchen rechnen.}{
		\bu{42/43}{} \\
		\bu{98/99}{} \\
		\bu{102/103}{} \\
		\bu{106/107}{}
	}
	\ichkann{\dots Dezimalzahlen multiplizieren.}{
		\bu{114/115}{} \\
		\bu{116}{6/7} \\
		\bu{117}{15} \\
		\bu{126}{3/7}
	}
	\ichkann{\dots Dezimalzahlen dividieren.}{
		\bu{118/119}{} \\
		\bu{120}{6-8} \\
		\bu{121}{16}  \\
		\bu{128}{23} \\
		\bu{133}{2}
	}
	\ichkann{\dots Sachaufgaben strukturiert bearbeiten.}{
		\bu{117}{15} \\
		\bu{121}{16} \\
		\bu{126}{7/8}\\
		\bu{128}{23}
	}
	\ichkann{\dots mit Rechengesetzen Vorteile beim Rechnen ausnutzen.}{
		\bu{122/123}{}\\
		\bu{124}{5}\\
		\bu{125}{12/13}
	}
\end{checkup}

\end{document}
