\documentclass[11pt, a4paper]{scrartcl}

\usepackage{vorschule}
\usepackage[
    typ=ab,
    fach=Mathematik,
    lerngruppe={Jg.6},
    nummer={5},
    module={Symbol},
]{schule}

\usepackage[
	typ=checkup,
	kuerzel=Ngb,
	reihe={Beurteilende Statistik},
]{ngbschule}


\author{J. Neugebauer}
\title{Checkup zum Distanzlernen}
\date{\Heute}

\begin{document}
\CheckupTitel

Gehe noch einmal die Aufgaben der letzten Wochen durch. Kreuze jeweils an, wie sicher du dich bei den einzelnen \textbf{Themenschwerpunkten} fühlst (von \enquote{sehr sicher} \usym{1F604} bis \enquote{sehr unsicher} \usym{1F641}).

\hinweis{Notiere dir zu jedem Thema Fragen, die du im Präsenzunterricht besprechen möchtest.}

\begin{longtable}{|p{6cm}|c|p{7cm}|} \hline
		\rowcolor{ngb.tabelle.kopf.hg}
		\rmfamily\textbf{Ich kann ...}
		&
		& \rmfamily\textbf{Fragen} \\ \hline\hline\endhead
	\ichkann{Daten als Tabelle darstellen.}{\Zeilenabstand[1.5cm]}
	\ichkann{Daten als Säulendiagramm darstellen.}{\Zeilenabstand[1.5cm]}
	\ichkann{absolute und relative Häufigkeiten erklären und berechnen.}{\Zeilenabstand[1.5cm]}
	\ichkann{den Winkel von Kreisabschnitten eines Kreisdiagramms berechnen.}{\Zeilenabstand[1.5cm]}
	\ichkann{ein Kreisdiagramm zeichnen.}{\Zeilenabstand[1.5cm]}
	\ichkann{Säulen- und Kreisdiagramme lesen und interpretieren.}{\Zeilenabstand[1.5cm]}
	\ichkann{das arithmetische Mittel berechnen.}{\Zeilenabstand[1.5cm]}
	\ichkann{den Median bestimmen.}{\Zeilenabstand[1.5cm]}
	\ichkann{das obere und untere Quartil bestimmen.}{\Zeilenabstand[1.5cm]}
	\ichkann{einen Boxplot zeichnen.}{\Zeilenabstand[1.5cm]}
	\ichkann{Boxplots lesen und interpretieren.}{\Zeilenabstand[1.5cm]}
\end{longtable}

\end{document}
