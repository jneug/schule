\documentclass[a4paper,ngerman,10pt]{scrartcl}

\usepackage[checkup,typo,theme,qrcodes]{arbeitsblatt}

\aboptionen{
	fach=Mathematik,
	name=J. Neugebauer,
	kuerzel=Ngb,
	kurs=5c,
	nummer=1,
	reihe={Grundwissen},
	titel={zur \Nummer. Arbeit},
	version={2021-09-27},
	lizenz=cc-by-nc-sa-eu-4,
}

\newcommand{\af}[1]{Aufg.#1}
\newcommand{\bu}[1]{Buch S.#1}
\newcommand{\buaf}[2]{Buch S.#1, \af{#2}}
\newcommand{\ah}[2]{AH S.#1, \af{#2}}
\newcommand{\ab}[2][AB]{#1 \enquote{#2}}

\begin{document}
\CheckupBild\CheckupTitel

Gehe noch einmal die Aufgaben der letzten Wochen durch. Kreuze jeweils an, wie
sicher du dich bei den einzelnen \textbf{Themenschwerpunkten} fühlst (von
\enquote{sehr sicher} \iconLachen\ bis \enquote{sehr unsicher} \iconTraurig).
Nutze die \textbf{Aufgaben und Informationen} zur Wiederholung der Themen, \emph{bei
denen du noch unsicher bist}.

Vor allem bieten sich dazu die \enquote{Teste dich!} Aufgaben im Buch, die \emph{Tests}
am Ende eines Kapitels und das \emph{Arbeitsheft} an. Die \textbf{Lösungen} findest du
hinten im Buch ab Seite 238, bzw. im Arbeitsheft im dazugehörigen Lösungsheft.

\begin{checkup}
	\teiler{Diagramme}
	\ichkann{Säulendiagramme lesen und beschreiben.}{
		\bu{8/9}
		\ah{3}{2}
	}

	\teiler{Zahlen}
	\ichkann{Zahlen vom Zahlenstrahl ablesen und dort eintragen.}{
		\bu{12/13} \\
		\buaf{14}{5, 6, 11}
	}
	\ichkann{auch große Zahlen lesen und vergleichen.}{
		\ah{11}{4} \\
		\ah{11}{6}
	}
	\ichkann{Zahlen richtig runden.}{
		\bu{15/16} \\
		\buaf{17}{6, 7} \\
		\buaf{18}{14} \\
		\ah{11}{5}
	}

	\teiler{Grundrechenarten}
	\ichkann{die Fachbegriffe zu den Grundrechenarten benutzen.}{
		\bu{19/20} \\
		\ah{11}{8}
	}
	\ichkann{mit den Grundrechenarten rechnen (Addition, Subtraktion, Multiplikation, Division).}{
		\bu{19/20} \\
		\buaf{21}{7} \\
		\buaf{18}{14} \\
		\ah{12}{9}
	}
	\ichkann{schriftlich addieren und subtrahieren.}{
		\bu{107/108} \\
		\ah{29}{1, 2}
	}
	\ichkann{Umkehraufgaben finden und berechnen.}{
		\buaf{21}{8}
	}

	\teiler{Größen}
	\ichkann{Einheiten ineinander umfromen.}{
		AH S.7 bis 10
	}
	\ichkann{mit Größen rechnen (Längen, Gewichte, Zeit, Geld).}{
		\bu{40} \\
		\ah{12}{12} \\
		AH S.7 bis 10
	}
\end{checkup}

\end{document}
