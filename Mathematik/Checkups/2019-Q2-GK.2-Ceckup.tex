\documentclass[11pt, a4paper]{scrartcl}

\usepackage{vorschule}
\usepackage[
    typ=ab,
    fach=Mathematik,
    lerngruppe={Q2 GK},
    nummer={2},
    module={Symbole},
]{schule}

\usepackage[
	typ=checkup,
	kuerzel=Ngb,
	reihe={Analytische Geometrie},
	version={2019-11-29}
]{ngbschule}


\author{J. Neugebauer}
\title{\Nummer. Klausur}
\date{\Heute}

\begin{document}
\CheckupBild\CheckupTitel

Kreuzen sie jeweils an, wie sicher sie sich bei den einzelnen \textbf{Themenschwerpunkten} fühlen (von \enquote{sehr sicher} \usym{1F604} bis \enquote{sehr unsicher} \usym{1F641}). Nutzen sie die \textbf{Aufgaben und Informationen} zum Wiederholen und Lernen von Themen, bei denen sie noch unsicher sind.

Die \textbf{Lösungen} zu den Aufgaben finden sie hinten im Buch. Weitere Aufgaben (ohne Lösungen) finden sie im Buch in den Kapiteln zum Thema.

\begin{checkup}
    \ichkann{mit Vektoren umgehen.}{
    	Siehe Checkup zur ersten Klausur.\\
    	\bu{218}{}
    }
    \ichkann{die Spurpunkte von Geraden bestimmen.}{
    	\bu{176}{} \\
    	\bu{221}{4} \\
    	Übungen zur Klausur
    }
    \ichkann{Vektoren auf kollinearität prüfen.}{
    	\bu{221}{3}
    }
    \ichkann{Geradengleichungen aufstellen.}{
	    \bu{219 oben}{} \\
    	\bu{221}{2} \\
		Übungen zur Klausur
    }
    \ichkann{die Lagebeziehung von Geraden Prüfen. (Inklusive Bestimmung eines Schnittpunktes.}{
    	\bu{182}{} \\
    	\bu{219}{} \\
    	\bu{221}{} \\
    	Übungen zur Klausur
    }
    \ichkann{Zwei Vektoren (bzw. Geraden) auf orthogonalität prüfen.}{
    	\bu{220}{} \\
    	\bu{221}{5+6} \\
		Übungen zur Klausur
    }
    \ichkann{das Skalarprodukt von Vektoren berechnen.}{
    	\bu{190/191}{} \\
    	\bu{220}{} \\
    	\bu{221}{5+6}
    }
    \ichkann{die Gleichungen von Geraden aufstellen, die parallel oder orthogonal zu einer gegebenen Geraden sind.}{
    	\bu{221}{5+6} \\
		Übungen zur Klausur
    }
\end{checkup}

\end{document}
