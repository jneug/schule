\documentclass[11pt, a4paper]{scrartcl}

\usepackage{vorschule}
\usepackage[
    typ=ab,
    fach=Mathematik,
    lerngruppe={Jg.6},
    nummer={1},
    module={Symbol},
]{schule}

\usepackage[
	typ=checkup,
	kuerzel=Ngb,
	reihe={Anteile und Brüche},
]{ngbschule}


\author{J. Neugebauer}
\title{1. Mathearbeit}
\date{\Heute}

\begin{document}
\CheckupBild\CheckupTitel

Kreuze jeweils an, wie sicher du dich bei den einzelnen \textbf{Themenschwerpunkten} fühlst (von \enquote{sehr sicher} \usym{1F604} bis \enquote{sehr unsicher} \usym{1F641}). Nutze die \textbf{Aufgaben und Informationen} zum Wiederholen und Lernen von Themen, \emph{bei denen du noch unsicher bist}.

Aufgaben zum Üben findest Du dieses Mal auf den Arbeitsblättern aus dem Unterricht und im Arbeitsheft aus dem letzten Schuljahr.

\begin{checkup}
	\ichkann{\dots Flächen von Dreiecken, Parallelogrammen und Rechtecken berechnen (auch zusammengesetzte).}{
	}\hline
	\ichkann{\dots Anteile aus Figuren ablesen und sie in Figuren einzeichnen.}{
	}
	\ichkann{\dots Anteile von einem Ganzen berechnen.}{
	}
	\ichkann{\dots aus einem Anteil das Ganze berechnen.}{
	}
	\ichkann{\dots Brüche kürzen und erweitern.}{
	}
	\ichkann{\dots Brüche vergleichen und nach der Größe ordnen.}{
	}
	\ichkann{\dots Brüche als Prozentzahlen schreiben und Prozentzahlen als Brüche.}{
	}
	\ichkann{\dots Brüche auf dem Zahlenstrahl anordnen.}{
	}
\end{checkup}

\end{document}
