\documentclass[checkup, 10pt, a4paper, ngerman]{arbeitsblatt}

\ladeModule{typo,theme,qrcodes}
\aboptionen{
	fach=Mathematik,
	name=J. Neugebauer,
	kuerzel=Ngb,
	kurs=5c,
	nummer=2,
	reihe={Grundwissen},
	titel={zur \Nummer. Arbeit},
	version={2021-11-15},
	lizenz=cc-by-nc-sa-eu-4,
}

\newcommand{\af}[1]{Aufg.#1}
\newcommand{\bu}[1]{Buch S.#1}
\newcommand{\buaf}[2]{Buch S.#1, \af{#2}}
\newcommand{\ah}[1]{AH S.#1}
\newcommand{\ahaf}[2]{AH S.#1, \af{#2}}
\newcommand{\ab}[2][AB]{#1 \enquote{#2}}

\begin{document}
\CheckupBild\CheckupTitel

Gehe noch einmal die Aufgaben der letzten Wochen durch. Kreuze jeweils an, wie
sicher du dich bei den einzelnen \textbf{Themenschwerpunkten} fühlst (von
\enquote{sehr sicher} \iconLachen\ bis \enquote{sehr unsicher} \iconTraurig).
Nutze die \textbf{Aufgaben und Informationen} zur Wiederholung der Themen,
\emph{bei denen du noch unsicher bist}.

Vor allem bieten sich dazu die \enquote{Teste dich!} Aufgaben im Buch, die
\emph{Tests} am Ende eines Kapitels und das \emph{Arbeitsheft}
an. Die \textbf{Lösungen} findest du hinten im Buch ab Seite 238, bzw. im
Arbeitsheft im dazugehörigen Lösungsheft.

\begin{checkup}
	\teiler{Größen}
	\ichkann{mit Größen rechnen (Längen, Gewichte, Zeit, Geld).}{
		\textbf{Geld}: \bu{23} \\ \buaf{24}{5/6} \\
		\textbf{Längen}: \bu{26} \\ \buaf{28}{7/8} \\ \buaf{29}{16} \\
		\textbf{Gewicht}: \bu{30}  \\ \buaf{32}{7/8} \\ \buaf{33}{17} \\
		\textbf{Zeitangaben}: \bu{34} \\ \buaf{35}{5/6} \\ \buaf{36}{11} \\
	}
	\ichkann{Einheiten ineinander umfromen und dazu auch Zehnerpotenzen einsetzen.}{
		\emph{Zehnerpotenzen im Heft} \\
		\bu{40} \\
		\buaf{41}{3 (Runde 1)} \\
		\buaf{41}{1 (Runde 2)} \\
		\buaf{37}{4}
	}

	\teiler{Senkrechte und Parallele}
	\ichkann{Senkrechte und parallele Geraden erkennen und messen.}{
		\bu{50/51} \\
		\buaf{53}{7} \\
		\buaf{77}{2a) (Runde 2)} \\
		\ahaf{13}{1/2}
	}
	\ichkann{Senkrechte und parallele Geraden zeichnen.}{
		\emph{Miniheft} \\
		\bu{52 oben} \\
		\buaf{53}{8} \\
		\ahaf{13}{2/4} \\
		\ahaf{14}{1a)}
	}
\end{checkup}

\end{document}
