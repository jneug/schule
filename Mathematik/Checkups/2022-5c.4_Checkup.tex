\documentclass[checkup, 10pt, a4paper, ngerman]{arbeitsblatt}

\ladeModule{typo,theme,qrcodes}
\aboptionen{
	fach=Mathematik,
	name=J. Neugebauer,
	kuerzel=Ngb,
	kurs=5c,
	nummer=4,
	reihe={Rechnen},
	titel={zur \Nummer. Arbeit},
	version={2022-02-21},
	lizenz=cc-by-nc-sa-eu-4,
}

\newcommand{\af}[1]{Aufg.#1}
\newcommand{\bu}[1]{Buch S.#1}
\newcommand{\buaf}[2]{Buch S.#1, \af{#2}}
\newcommand{\ah}[1]{AH S.#1}
\newcommand{\ahaf}[2]{AH S.#1, \af{#2}}
\newcommand{\ab}[2][AB]{#1 \enquote{#2}}

\begin{document}
\CheckupBild\CheckupTitel

Gehe noch einmal die Aufgaben der letzten Wochen durch. Kreuze jeweils an, wie
sicher du dich bei den einzelnen \textbf{Themenschwerpunkten} fühlst (von
\enquote{sehr sicher} \iconLachen\ bis \enquote{sehr unsicher} \iconTraurig).
Nutze die \textbf{Aufgaben und Informationen} zur Wiederholung der Themen,
\emph{bei denen du noch unsicher bist}.

Vor allem bieten sich dazu die \enquote{Teste dich!} Aufgaben im Buch, die
\emph{Tests} am Ende eines Kapitels und das \emph{Arbeitsheft}
an. Die \textbf{Lösungen} findest du hinten im Buch ab Seite 245, bzw. im
Arbeitsheft im dazugehörigen Lösungsheft.

\begin{checkup}
	\teiler{Schätzen}
	\ichkann{sinnvolle Schätzungen zu einer Aufgabe machen.}{
		Schätzaufgaben \\
		aus dem Unterricht
	}
	\ichkann{eine Überschlagsrechnung machen.}{
		\buaf{109}{8} \\
		\buaf{123}{4}
	}
	\teiler{Rechnen}
	\ichkann{schriftlich addieren und subtrahieren.}{
		\bu{107/108} \\
		\buaf{109}{8/9} \\
		\ah{28/29}
	}
	\ichkann{schriftlich multiplizieren.}{
		\bu{111/112} \\
		\buaf{113}{8/9} \\
		\ah{30}
	}
	\teiler{Teilbarkeit und Primzahlen}
	\ichkann{die Teilbarkeitsregeln nutzen.}{
		\bu{100/101} \\
		\buaf{102}{8/9} \\
		\ah{26}
	}
	\ichkann{Primzahlen erkennen.}{
		\bu{104/105} \\
		\buaf{105}{8} \\
		\buaf{123}{3} \\
		\ahaf{27}{1}
	}
	\ichkann{Zahlen in Primfaktoren zerlegen.}{
		\bu{104/105} \\
		\buaf{105}{9} \\
		\buaf{106}{14} \\
		\buaf{123}{3} \\
		\ahaf{27}{2-5}
	}
	\ichkann{die Teiler einer Zahl bestimmen.}{
		\bu{104/105} \\
		\buaf{106}{14} \\
		\buaf{123}{3} \\
		\ahaf{27}{5}
	}
\end{checkup}

\end{document}
