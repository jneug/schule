\documentclass[checkup, 11pt, a4paper]{arbeitsblatt}

\ladeModule{typo,theme,qrcodes}
\aboptionen{
	fach		= Mathematik,
	name		= {J. Neugebauer},
	kuerzel		= Ngb,
	kurs		= {06c},
	nummer		= 3,
	reihe		= {Bruchrechnung},
	titel		= {zur 3. Arbeit},
	version		= {2022-12-13},
	lizenz		= {cc-by-nc-sa-eu-4},
}

\newcommand{\af}[1]{Aufg.\,#1}
\newcommand{\bu}[1]{Buch S.\,#1}
\newcommand{\buaf}[2]{Buch S.\,#1, \af{#2}}
\newcommand{\ah}[1]{AH S.\,#1}
\newcommand{\ahaf}[2]{AH S.\,#1, \af{#2}}
\newcommand{\ab}[2][AB]{#1 \enquote{#2}}

\begin{document}
\CheckupBild\CheckupTitel

Gehe noch einmal die Aufgaben der letzten Wochen durch. Kreuze jeweils an, wie
sicher du dich bei den einzelnen \textbf{Themenschwerpunkten} fühlst (von
\enquote{sehr sicher} \iconLachen\ bis \enquote{sehr unsicher} \iconTraurig).
Nutze die \textbf{Aufgaben und Informationen} zur Wiederholung der Themen,
\emph{bei denen du noch unsicher bist}.

Vor allem bieten sich dazu die \enquote{Teste dich!} Aufgaben im Buch, die
\emph{Tests} am Ende eines Kapitels und das \emph{Arbeitsheft}
an. Die \textbf{Lösungen} findest du hinten im Buch ab Seite 244, bzw. im
Arbeitsheft im dazugehörigen Lösungsheft.

\begin{checkup}
	\teiler{Brüche}
	\ichkann{mit Brüchen umgehen (Anteile, Prozentzahlen, Verhältnisse, Kürzen, Erweitern, ...)}{
		\bu{8-38}
	}
	\ichkann{Brüche addieren und subtrahieren.}{
		\bu{72/73} \\
		\buaf{74}{7/8} \\
		\buaf{76}{17/20} \\
		\buaf{90}{1/2} \\
		\ah{27}
	}
	\teiler{Dezimalzahlen}
	\ichkann{Dezimalzahlen in das Stellenwertsystem eintragen.}{
		\bu{44/45} \\
		\ahaf{19}{1}
	}
	\ichkann{Zwischen Dezimalzahlen und Brüchen umwandeln (mit Hilfe des Stellenwertsystems).}{
		\bu{44/45} \\
		\buaf{46}{8-10} \\
		\buaf{47}{18} \\
		\buaf{61}{1/2}
	}
	\ichkann{Brüche durch Division in abbrechende und periodische Dezimalzahlen umwandeln.}{
		\bu{52/53} \\
		\buaf{54}{6} \\
		\ah{22}
	}
	\ichkann{Dezimalzahlen vergleichen und runden.}{
		\bu{48/49} \\
		\buaf{50}{8-10} \\
		\buaf{54}{7} \\
		\buaf{61}{4/6}
	}
\end{checkup}

\end{document}
