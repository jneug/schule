\documentclass[11pt, a4paper]{scrartcl}

\usepackage{vorschule}
\usepackage[
    typ=ab,
    fach=Mathematik,
    lerngruppe={Jg.5},
    nummer={6},
    module={Symbol},
]{schule}

\usepackage[
	typ=checkup,
	kuerzel=Ngb,
	reihe={Flächen und Volumen},
]{ngbschule}


\author{J. Neugebauer}
\title{2. Mathearbeit}
\date{\Heute}

\begin{document}
\CheckupBild\CheckupTitel

Kreuze jeweils an, wie sicher du dich bei den einzelnen \textbf{Themenschwerpunkten} fühlst (von \enquote{sehr sicher} \usym{1F604} bis \enquote{sehr unsicher} \usym{1F641}). Nutze die \textbf{Aufgaben und Informationen} zum Wiederholen und Lernen von Themen, \emph{bei denen du noch unsicher bist}.

Die \textbf{Lösungen} für die \enquote{Teste dich!} Aufgaben findest du hinten im Buch ab Seite 206. Die Lösungen für die Aufgaben im Arbeitsheft findest du im dazugehörigen Lösungsheft.

\begin{checkup}
	\ichkann{mit Brüchen und Prozentzahlen umgehen.}{
		Checkup zur ersten Arbeit.
	}
	\ichkann{Brüche vefielfachen und teilen.}{
		\bu{98/99}{} \\
		\bu{100}{8/9} \\
		\bu{101}{16/17}
	}
	\ichkann{Brüche mit Brüchen multiplizieren.}{
		\bu{102/103}{} \\
		\bu{104}{9/10} \\
		\bu{127}{10/11} \\
		\bu{133}{1/4}
	}
	\ichkann{Brüche durch Brüche teilen.}{
		\bu{106/107}{} \\
		\bu{126}{4} \\
		\bu{133}{1/4}
	}
	\ichkann{Brüche mit Brüchen addieren und subtrahieren.}{
		\bu{42/43}{} \\
		\bu{44}{7} \\
		\bu{46}{19/20}
	}
	\ichkann{Sachaufgaben strukturiert bearbeiten.}{
		\bu{56/57}{} \\
		\bu{58}{6} \\
		\bu{59}{12} \\
		\bu{104}{10} \\
		\bu{105}{18} \\
		\bu{108}{10}
	}
\end{checkup}

\end{document}
