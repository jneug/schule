\documentclass[11pt, a4paper]{scrartcl}

\usepackage{vorschule}
\usepackage[
    typ=ab,
    fach=Mathematik,
    lerngruppe={Jg.5},
    nummer={5},
    module={Symbole},
]{schule}

\usepackage[
	typ=checkup,
	kuerzel=Ngb,
	reihe={Rechnen und Flächen},
]{ngbschule}


\author{J. Neugebauer}
\title{5. Mathearbeit}
\date{\Heute}

\begin{document}
\CheckupBild\CheckupTitel

Kreuze jeweils an, wie sicher du dich bei den einzelnen \textbf{Themenschwerpunkten} fühlst (von \enquote{sehr sicher} \usym{1F604} bis \enquote{sehr unsicher} \usym{1F641}). Nutze die \textbf{Aufgaben und Informationen} zum Wiederholen und Lernen von Themen, \emph{bei denen du noch unsicher bist}.

Die \textbf{Lösungen} für die \enquote{Teste dich!} Aufgaben findest du hinten im Buch ab Seite 240. Die Lösungen für die Aufgaben im Arbeitsheft findest du im dazugehörigen Lösungsheft.

\begin{checkup}
	\ichkann{\dots schriftlich addieren, subtrahieren, multiplizieren und dividieren.}{
	}
	\ichkann{\dots Rechengesetze nutzen, um geschickt zu rechnen.}{
	}
	\ichkann{\dots mit Potenzen rechnen.}{
	}
	\ichkann{\dots Flächeninhalte vergleichen (Kästchen zählen).}{
		\bu{130/131}{} \\
		\bu{132}{6/10} \\
		\ah{34}{}
	}
	\ichkann{\dots mit Flächeneinheiten rechnen.}{
		\bu{133/134}{} \\
		\bu{135}{10/11} \\
		\bu{137}{23/24} \\
		\ah{35/36}{}
	}
	\ichkann{\dots Flächen von Rechtecken berechnen.}{
		\bu{138/139}{} \\
		\bu{140}{9/10} \\
		\bu{141}{20/21} \\
		\ah{37}{}
	}
\end{checkup}

\end{document}
