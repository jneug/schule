\documentclass[11pt, a4paper]{scrartcl}

\usepackage{vorschule}
\usepackage[
    typ=ab,
    fach=Mathematik,
    lerngruppe={Jg.5},
    nummer={6},
    module={Symbol},
]{schule}

\usepackage[
	typ=checkup,
	kuerzel=Ngb,
	reihe={Flächen und Volumen},
]{ngbschule}


\author{J. Neugebauer}
\title{6. Mathearbeit}
\date{\Heute}

\begin{document}
\CheckupBild\CheckupTitel

Kreuze jeweils an, wie sicher du dich bei den einzelnen \textbf{Themenschwerpunkten} fühlst (von \enquote{sehr sicher} \usym{1F604} bis \enquote{sehr unsicher} \usym{1F641}). Nutze die \textbf{Aufgaben und Informationen} zum Wiederholen und Lernen von Themen, \emph{bei denen du noch unsicher bist}.

Die \textbf{Lösungen} für die \enquote{Teste dich!} Aufgaben findest du hinten im Buch ab Seite 240. Die Lösungen für die Aufgaben im Arbeitsheft findest du im dazugehörigen Lösungsheft.

\begin{checkup}
	\ichkann{\dots schriftlich dividieren mit Rest.}{
	}
	\ichkann{\dots mit Flächeneinheiten rechnen.}{
		\bu{133/134}{} \\
		\bu{155}{2/3}
	}
	\ichkann{\dots Flächen von Rechtecken berechnen.}{
		\bu{138/139}{} \\
		\bu{155}{6/7} \\
		\bu{156}{12/13}
	}
	\ichkann{\dots Flächen von Dreiecken und Parallelogrammen berechnen.}{
		\bu{142/143}{} \\
		\bu{145}{8/9} \\
		\bu{146}{19} \\
		\bu{155}{4/5} \\
		\bu{156}{10}
	}
	\ichkann{\dots Umfang von Figuren bestimmen.}{
		\bu{147/148}{} \\
		\bu{149}{9/10} \\
		\bu{150}{20/21} \\
		\bu{155}{6}\\
		\bu{156}{12a}
	}
	\ichkann{\dots mit Maßstäben umgehen.}{
		\bu{151/152}{} \\
		\bu{153}{5/6} \\
		\bu{154}{12} \\
		\bu{155}{8} \\
		\bu{157}{22}
	}
	\ichkann{\dots Würfel- und Quadernetze zeichnen und prüfen.}{
		\bu{166/167}{} \\
		\bu{168}{5/6} \\
		\bu{169}{13}
	}
	\ichkann{\dots Schrägbilder von Würfeln und Quadern zeichnen.}{
		\bu{170/171}{} \\
		\bu{171}{3} \\
		\bu{172}{8} \\
		\bu{187}{4}
	}
	\ichkann{\dots Oberflächeninhalt von Würfeln und Quadern berechnen.}{
		\bu{184}{} \\
		\bu{188}{10} \\
		\bu{187}{9}
	}
\end{checkup}

\end{document}
