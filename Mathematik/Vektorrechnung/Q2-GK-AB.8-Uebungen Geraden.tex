\documentclass[11pt, a5paper, landscape]{scrartcl}

% put to a5 pages on one a4 sheet
%\usepackage{pgfpages}
%\pgfpagesuselayout{2 on 1}[a4paper]

\usepackage{vorschule}
\usepackage[
    typ=ab,
    fach=Mathematik,
    lerngruppe={Q1 GK},
    nummer=8,
    seitenzahlen=keine,
	loesungen=folgend,
    module={Symbole}
]{schule}

\usepackage[
	kuerzel={Ngb},
	reihe={Analytische Geometrie},
	version={2019-12-01},
	hinweise=keine
]{ngbschule}

\author{J. Neugebauer}
\title{Vermischte Übungen zur Klausur}
\date{\Heute}

\setzeAufgabentemplate{ngbohne}

\chead{\Titel}

\begin{document}

%% Aufgabe 1
\begin{aufgabe}
    \begin{teilaufgaben}
    \teilaufgabe \operator{Bestimmen} sie die Gleichung einer Geraden $g$ durch die Punkte \pkt[A](7|-3|2) und \pkt[B](-4|5|-2).
    
    \teilaufgabe \operator{Bestimmen} sie die Gleichung einer Geraden $h$, die aus $g$ hervorgeht, wenn $g$ um den Vektor $\vec{v} = \vector(-1|-4|8)$ verschoben wird.
    
    \teilaufgabe \operator{Bestimmen} sie die Gleichung einer Geraden $i$, die zu $g$ orthogonal ist und durch \pkt[C](1|2|3) verläuft.
    
    \teilaufgabe \operator{Bestimmen} sie die Gleichung einer Geraden $j$, die zu $i$ parallel ist und durch den Punkt A von oben verläuft.
    
    \teilaufgabe \operator{Berechnen} sie alle Spurpunkte der Geraden $g$.
    \end{teilaufgaben}
\end{aufgabe}
\clearpage

%% Hinweise Aufgabe 1
\begin{loesung}
	\begin{multicols}{2}
	\begin{teilaufgaben}
		\teilaufgabe \[ g:\vec{x} = \vector(7|-3|2) + s\cdot \vector(-11|8|-4) \]
		
		\teilaufgabe Berechne $\vec{OA} + \vec{v}$. Der Richtungsvektor bleibt gleich.
		\[ h:\vec{x} = \vector(6|1|10) + s\cdot \vector(-11|8|-4) \]
		
		\teilaufgabe z.B.
		\[ i:\vec{x} = \vector(1|2|3) + s\cdot \vector(0|1|2) \]
		
		\teilaufgabe z.B.
		\[ j:\vec{x} = \vector(7|-3|2) + s\cdot \vector(0|1|2) \]
		
		\teilaufgabe \begin{gather*}
		\pkt[S_{x_1x_2}](1,5|1|0)\\ \pkt[S_{x_2x_3}](0|2,09|-0,55)\\ \pkt[S_{x_1x_3}](2,88|0|0,5)
		\end{gather*}
	\end{teilaufgaben}
	\end{multicols}
\end{loesung}
\clearpage

\begin{aufgabe}
    Prüfen sie die Lage der Geraden $g$ und $h$ zueinander. Prüfen sie auch auf orthogonalität, falls die Geraden windschief sind und geben sie den Schnittpunkt an, falls sich die Geraden schneiden.
    
    \begin{teilaufgaben}
    \teilaufgabe \[ g: \vec{x} = \vector(0|8|-7) + s\cdot \vector(1|2|-2)\quad h:\vec{x} = \vector(-9|0|7) + t\cdot \vector(3|1|-4) \]
    
    \teilaufgabe \[ g: \vec{x} = \vector(3|7|3) + s\cdot \vector(6|9|-12)\quad h:\vec{x} = \vector(9|14|4) + t\cdot \vector(-8|-12|16) \]
    
    \teilaufgabe \[ g: \vec{x} = \vector(1|2|-2) + s\cdot \vector(1|0|-1)\quad h:\vec{x} = \vector(10|-1|4) + t\cdot \vector(3|4|3) \]
    \end{teilaufgaben}
\end{aufgabe}
\clearpage

\begin{loesung}
	\begin{teilaufgaben}
		\teilaufgabe Die Geraden schneiden sich in \pkt(-3|2|-1).
		
		Sie sind nicht orthogonal. Das Skalarprodukt der Richtungsvektoren ist 13.
		
		\teilaufgabe Die Geraden sind echt parallel. Die Richtungsvektor sind mit dem Faktor $\dfrac{4}{3}$ skaliert.
		
		\teilaufgabe Die Geraden sind windschief zueinander.
		
		Sie sind orthogonal, das das Skalarprodukt der Richtungsvektoren gleich null ist: $1\cdot 3 + 0\cdot 4 + (-1)\cdot 3 = 3 - 3 = 0$.
	\end{teilaufgaben}
\end{loesung}
\clearpage

% Aufgabe 3
\begin{aufgabe}
	Gegeben ist ein Dreieck $ABD$ mit \pkt[A](-1|-1|1), \pkt[B](2|-2|1) und \pkt[D](2,5|-0,5|1).
	
	\begin{teilaufgaben}
		\teilaufgabe \operator{Berechnen} sie die Länge der drei Seiten des Dreiecks.
		\teilaufgabe \operator{Bestimmen} sie die Koordinaten eines Punktes $C$, so dass $ABCD$ ein \emph{Parallelogramm} ist.
		\teilaufgabe \operator{Prüfen} sie, ob das Parallelogramm $ABCD$ ein Rechteck ist.
		\teilaufgabe \operator{Bestimmen} sie Gleichungen für die Diagonalengeraden von $ABCD$ und ermitteln sie deren Schnittpunkt.
		\teilaufgabe Mit einem Punkt $E$, der senkrecht über dem Schnittpunkt der Diagonalen liegt, bildet $ABCD$ eine Pyramide. \operator{Bestimmen} sie den Punkt $E$ so, dass das Volumen der Pyramide $10$ ist.
	\end{teilaufgaben}
\end{aufgabe}
\clearpage

\begin{loesung}
	\begin{multicols}{2}
	\begin{teilaufgaben}
		\teilaufgabe $\abs{\vec{AB}} \approx 3,16$, $\abs{\vec{BD}} \approx 1,58$, $\abs{\vec{AD}} \approx 3,54$
		\teilaufgabe \[ \vec{OA} + \vec{BD} = \vector(-0,5|0,5|1) \]
		\teilaufgabe Ja, denn $\vec{AB}\ast \vec{BD} = 0$
		\teilaufgabe \begin{gather*}
		 g:\vec{x} = \vec{OA} + s\cdot \vec{AD}\\
		 h:\vec{x} = \vec{OB} + s\cdot \vec{BC}\\
		 \pkt[S](0,75|-0,75|1)
		\end{gather*}
		\teilaufgabe $E$ liegt über $S$, also $\pkt[E](0,75|-0,75|1+h)$. Das Volumen der Pyramide berechnet sich durch 
		\[ V_P = 10 = \dfrac{1}{3}\cdot (\abs{\vec{AB}}\cdot \abs{\vec{BD}})\cdot h \]
		Nach h umgestellt ergibt sich $h \approx 6$, also $\pkt[E](0,75|-0,75|7)$.
	\end{teilaufgaben}
	\end{multicols}
\end{loesung}
\clearpage


% Aufgabe 4
%\begin{aufgabe}
%\end{aufgabe}
\clearpage

%\begin{loesung}
%\end{loesung}
\clearpage

\end{document}