\documentclass[11pt, a5paper, landscape]{scrartcl}

% put to a5 pages on one a4 sheet
%\usepackage{pgfpages}
%\pgfpagesuselayout{2 on 1}[a4paper]

\usepackage{vorschule}
\usepackage[
    typ=ab,
    fach=Mathematik,
    lerngruppe={Q1 GK},
    nummer=8,
    seitenzahlen=keine,
%    loesungen=folgend,
    module={Symbole}
]{schule}

\usepackage[
	kuerzel={Ngb},
	reihe={Analytische Geometrie},
	version={2019-12-01},
	hinweise=keine
]{ngbschule}

\author{J. Neugebauer}
\title{Vermischte Übungen zur Klausur}
\date{\Heute}

\setzeAufgabentemplate{ngbohne}

\chead{\Titel}

\begin{document}

%% Aufgabe 1
\begin{aufgabe}
    \begin{teilaufgaben}
    \teilaufgabe \operator{Bestimmen} sie die Gleichung einer Geraden $g$ durch die Punkte \pkt[A](7|-3|2) und \pkt[B](-4|5|-2).
    
    \teilaufgabe \operator{Bestimmen} sie die Gleichung einer Geraden $h$, die aus $g$ hervorgeht, wenn $g$ um den Vektor $\vec{v} = \vector(-1|-4|8)$ verschoben wird.
    
    \teilaufgabe \operator{Bestimmen} sie die Gleichung einer Geraden $i$, die zu $g$ orthogonal ist und durch \pkt[C](1|2|3) verläuft.
    
    \teilaufgabe \operator{Bestimmen} sie die Gleichung einer Geraden $j$, die zu $i$ parallel ist und durch den Punkt A von oben verläuft.
    
    \teilaufgabe \operator{Berechnen} sie alle Spurpunkte der Geraden $g$.
    \end{teilaufgaben}
\end{aufgabe}
\clearpage

%% Hinweise Aufgabe 1
\begin{loesung}
\end{loesung}
\clearpage

\begin{aufgabe}
    Prüfen sie die Lage der Geraden $g$ und $h$ zueinander. Prüfen sie auch auf orthogonalität, falls die Geraden windschief sind und geben sie den Schnittpunkt an, falls sich die Geraden schneiden.
    
    \begin{teilaufgaben}
    \teilaufgabe \[ g: \vec{x} = \vector(0|8|-7) + s\cdot \vector(1|2|-2)\quad h:\vec{x} = \vector(-9|0|7) + t\cdot \vector(3|1|-4) \]
    
    \teilaufgabe \[ g: \vec{x} = \vector(3|7|3) + s\cdot \vector(6|9|-12)\quad h:\vec{x} = \vector(9|14|4) + t\cdot \vector(-8|-12|16) \]
    
    \teilaufgabe \[ g: \vec{x} = \vector(1|2|-2) + s\cdot \vector(1|0|-1)\quad h:\vec{x} = \vector(10|-1|4) + t\cdot \vector(3|4|3) \]
    \end{teilaufgaben}
\end{aufgabe}

% Aufgabe 3
\begin{aufgabe}

\end{aufgabe}

\begin{loesung}
\end{loesung}


% Aufgabe 4
\begin{aufgabe}

\end{aufgabe}

\begin{loesung}
\end{loesung}

\end{document}