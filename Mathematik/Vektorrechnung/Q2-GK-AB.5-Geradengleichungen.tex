\documentclass[10pt, a4paper]{scrartcl}

\usepackage{vorschule}
\usepackage[
    typ=ab,
    fach=Mathematik,
    lerngruppe={Q2-GK},
    nummer=5,
    module={Symbole,Lizenzen},
    seitenzahlen=keine,
    farbig,
    lizenz=cc-by-nc-sa-4,
]{schule}

\usepackage[
	kuerzel=Ngb,
	reihe={Analytische Geometrie},
	version={2019-11-17},
]{ngbschule}

\author{J. Neugebauer}
\title{Geradengleichungen}
\date{\Heute}

\setzeAufgabentemplate{ngbnormal}

%\usepackage{qrcode}
%\usepackage{tinspire}


\begin{document}

\ReiheTitel

\begin{aufgabe}\label{afg:geraden_bestimmen}
	\operator{Bestimmen} sie eine Gleichung in Parameterdarstellung für eine Gerade ...
	\begin{enumerate}
		\item ... durch die Punkte \pkt[A](5,8|7|-8) und \pkt[B](19|3|-8): $\quad g:$\linie[6cm]
		\item ... durch den Punkt \pkt[C](1|-6,5|32) mit der Richtung $\vec{v} = \vector(13,2|-4|0)$:  $\quad h:$\linie[6cm]
		\item ... entlang des Vektors $\vec{w} = \vector(-6,38|-7,7|8,8)$: $\quad i:$\linie[6cm]
	\end{enumerate} 
\end{aufgabe}

\begin{aufgabe}
	\begin{infobox}\small
		Die Spurpunkte einer Geraden sind die \emph{Schnittpunkte der Geraden mit den Koordinatenebenen}. \emph{Mindestens eine} Koordinate jedes Spurpunktes ist also null. Eine Gerade kann einen, zwei oder drei Spurpunkte haben. Die Spurpunkte helfen dabei, Geraden zu zeichnen.
	\end{infobox}
	
	\begin{enumerate}
		\item Überlegen sie, welche Möglichkeiten und Sonderfälle es für die Spurpunkte einer Geraden geben kann.
		\item \operator{Berechnen} sie die \emph{Spurpunkte} der Geraden $g$ aus \prettyref{afg:geraden_bestimmen}.
	\end{enumerate}
	
	\vspace{4cm}
\end{aufgabe}

\begin{aufgabe}
	\operator{Bestimmen} sie die paarweise Lage aller Geraden aus \prettyref{afg:geraden_bestimmen} zueinander. 
	
	\begin{itemize}
		\item $g$ und $h$ sind \linie zueinander.
		\item $h$ und $i$ sind \linie zueinander.
		\item $g$ und $i$ sind \linie zueinander.
	\end{itemize}
\end{aufgabe}

\end{document}
