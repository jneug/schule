\documentclass[11pt, a5paper, landscape]{scrartcl}

\usepackage{qrcode}
\qrset{height=4.5cm, nolink, padding}

% put to a5 pages on one a4 sheet
%\usepackage{pgfpages}
%\pgfpagesuselayout{2 on 1}[a4paper]

\usepackage{vorschule}
\usepackage[
    typ=ab,
    fach=Mathematik,
    lerngruppe={Q1 GK},
    nummer=6,
    seitenzahlen=keine,
%    loesungen=folgend,
    module={Symbole}
]{schule}

\usepackage[
	kuerzel={Ngb},
	reihe={Analytische Geometrie},
	version={2019-10-20},
	hinweise=keine
]{ngbschule}

\author{J. Neugebauer}
\title{Vermischte Übungen zur Klausur}
\date{\Heute}

\setzeAufgabentemplate{ngbohne}

\chead{\Titel}

\begin{document}
	%% Aufgabe 1
	\begin{aufgabe}
		\begin{enumeratea}
			\item \operator{Zeichnen} sie die ein räumliches Koordinatensystem von \num{-8} bis \num{8} auf jeder Achse ($\SI{1}{LE} = \SI{1}{\centi\meter}$).
			
			\item \operator{Zeichnen} sie ein Rechteck mit der Breite \SI{6}{LE} und Höhe \SI{4}{LE} in die $x_2x_3$-Ebene. Die untere linke Ecke soll im Koordinatenursprung anliegen. \operator{Bestimmen} sie die \emph{Ortsvektoren} der vier Eckpunkte $A, B, C$ und $D$.
			
			\item Erweitern sie das Rechteck in Richtung der $x_1$-Achse zu einem Quader mit der Tiefe \SI{5}{LE}. \operator{Zeichnen} sie den Quader und bestimmen sie die Koordinaten der neuen Eckpunkte.
			
			\item \operator{Berechnen} sie die Länge der Diagonalen der Seiten des Quaders, die in den Koordinatenebenen liegen.
			
			\item \operator{Beschreiben} sie den vom Ursprung am weitesten entfernten Eckpunkt auf drei unterschiedliche Weisen (z.B. durch Addition von anderen Vektoren).
			
			\item \operator{Berechnen} sie die neuen Koordinaten des Quaders, nachdem er um den \emph{Vektor} $\vec{v} = \begin{pmatrix} 6,3 \\ -4,2 \\ -5,6 \end{pmatrix}$ verschoben wurde. 
		\end{enumeratea}
	\end{aufgabe}
	\clearpage
		
	%% Hinweise Aufgabe 1
	\hspace{1em}
	\clearpage
	
	%% Aufgabe 2
	\begin{aufgabe}
		Rechnen sie möglichst viele der Aufgaben ohne Einsatz des GTR.
		\begin{multicols}{2}
%			\operator{Berechnen} sie den Ergebnisvektor.
%			\begin{enumeratea}
%				\item $\begin{pmatrix} 5 \\ -17 \\ 31 \end{pmatrix} + 3\cdot \begin{pmatrix} 1 \\ 0,5 \\ -2 \end{pmatrix}$
%
%				\item $4,75\cdot \begin{pmatrix} -6,8 \\ 15,2 \\ 2 \end{pmatrix} - 3,25\cdot \begin{pmatrix} 5 \\ 7 \\ -1 \end{pmatrix}$
%				
%				\item $\begin{pmatrix} 1,5 \\ 0 \\ -0,5 \end{pmatrix} + k\cdot \begin{pmatrix} 0,5 \\ 0 \\ 2 \end{pmatrix}$\\ für $k = 4$ und $k = 12$
%			\end{enumeratea}
			
			\operator{Bestimmen} sie, ob der Punkt $P$ auf der Geraden $g$ liegt.
			\begin{enumeratea}
				\item $g:\vec{x} = \begin{pmatrix} 3 \\ 5 \\ 2 \end{pmatrix} + r\cdot \begin{pmatrix} 2 \\ 2 \\ 3 \end{pmatrix}\qquad P\pkt(1|3|-1)$
				
				\item $g:\vec{x} = \begin{pmatrix} 2 \\ 3 \\ 1 \end{pmatrix} + q\cdot \begin{pmatrix} 2 \\ 1 \\ 2 \end{pmatrix}\qquad P\pkt(1|2,5|3)$
				
				\item $g:\vec{x} = \begin{pmatrix} -2 \\ 4 \\ -3 \end{pmatrix} + s\cdot \begin{pmatrix} -2 \\ 3 \\ 1 \end{pmatrix}\quad P\pkt(4|-5|-6)$
			\end{enumeratea}
			
			\vspace{2cm}
			\operator{Bestimmen} sie, ob die Geraden $g$ und $h$ einen Schnittpunkt besitzen.
			\begin{enumeratea}
				\item $g:\vec{x} = \begin{pmatrix} -2 \\ 2 \\ -5 \end{pmatrix} + r\cdot \begin{pmatrix} 2 \\ -3 \\ 1 \end{pmatrix}$\\ $h:\vec{x} = \begin{pmatrix} -3 \\ 2 \\ -16 \end{pmatrix} + s\cdot \begin{pmatrix} -1 \\ 2 \\ 3 \end{pmatrix}$
				
				\item $g:\vec{x} = \begin{pmatrix} 3 \\ -4 \\ 1 \end{pmatrix} + r\cdot \begin{pmatrix} 5 \\ 2 \\ -4 \end{pmatrix}$\\ $h:\vec{x} = \begin{pmatrix} 3 \\ -4 \\ 1 \end{pmatrix} + s\cdot \begin{pmatrix} -2 \\ 6 \\ -1 \end{pmatrix}$
			\end{enumeratea}
		\end{multicols}
	\end{aufgabe}
	\clearpage
	
	%% Hinweise Aufgabe 2
	\begin{center}
		\begin{multicols}{2}
			Punktprobe \\
			\qrcode{Ortsvektor mit Gerade gleichsetzen:
			
			P = g
			
			Gleichungssystem aufstellen und nach der Unbekannten aufloesen.
			Gibt es eine Loesung, dann liegt der Punkt auf der Geraden.}
			\columnbreak
			
			Schnittpunkt \\
			\qrcode{Die Geraden gleichsetzen:
			
			g = h
			
			Gleichungssystem aufstellen und nach den beiden Unbekannten aufloesen.
			Gibt es genau eine Loesung, dann liegt der Punkt auf der Geraden.}
		\end{multicols}
	\end{center}
	\clearpage
	
	%% Aufgabe 3
	\begin{aufgabe}
		Bezogen auf ein lokales Koordinatensystem mit der Einheit \si{\meter} kann die Flugroute eines Sportflugzeugs nach dem Start näherungsweise durch die Gerade
		\[ g: \vec{x} = \begin{pmatrix} 420 \\ -630 \\ 120 \end{pmatrix} + r\cdot \begin{pmatrix} 40 \\ 50 \\ 11 \end{pmatrix}\]
		angegeben werden.
		
		In der Nähe des Flugplatzes steht ein Windrad. Der Fußpunkt des Windrads befindet sich im Punkt P\pkt(1380|570|0), der höchste Punkt der Umlaufbahn der Rotorblätter liegt \SI{170}{\meter} über dem Boden.
		
		\begin{teilaufgaben}
			\teilaufgabe \operator{Prüfen} sie, ob die Spitze des Windrads auf der Flugbahn des Sportflugzeugs liegt.
		 
			\teilaufgabe \operator{Überprüfen} sie, ob das Flugzeug bei gleichbleibendem Kurs genau über das Windrad hinweg fliegt. Wenn ja, \operator{bestimmen} sie den Abstand, in der es das Windrad überfliegt.
		\end{teilaufgaben}
	\end{aufgabe}
	\clearpage
	
	%% Hinweise Aufgabe 3
	\begin{center}
		\begin{multicols}{3}
			Teilaufgabe a)\\
			\qrcode{Spitze des Windrads: S(1380|570|170)
			
			Punktprobe durchfuehren: OS mit g gleichsetzen.}
			\columnbreak
			
			Teilaufgabe b) Ansatz\\
			\qrcode{Geradengleichung fuer das Windrad aufstellen.
			
			Flugbahn und Gerade auf Schnittpunkte pruefen (gleich setzen).
			
			Gibt es einen Schnittpunkt, fliegt das Flugzeug ueber das Rad.
			
			Abstand: Laenge des Differenzvektors zwischen Schnittpunkt
			und Spitze des Windrads.}
			\columnbreak
			
			Teilaufgabe b) Lösungen\\
			\qrcode{Das Flugzeug ueberfliegt das windrad in einer Hoehe von 384m.
			Der Abstand zur Spitze betraegt dann 214m.}
		\end{multicols}
	\end{center}
	\clearpage
	
	%% Aufgabe 4
	\begin{aufgabe}\small
		Von einem Flugplatz, der in der $x_1x_2$-Ebene liegt, hebt ein Sportflugzeug im Punkt A\pkt(4|1|0) von der Startbahn ab. Es fliegt in den ersten drei Minuten auf einem Kurs, der annähernd durch die Gerade $g: \vec{x} = \begin{pmatrix} 4 \\ 1 \\ 0 \end{pmatrix} + r\cdot \begin{pmatrix} 18 \\ 14 \\ 3 \end{pmatrix}$ ($r$ in Minuten ab dem Abheben), beschrieben werden kann. Die Längeneinheit beträgt \SI{100}{\meter}. Nach drei Minuten ändert der Pilot seinen Kurs und fliegt in den nächsten 20 Minuten ohne weitere Kursänderung pro Minute um den Vektor $\vec{u} = \begin{pmatrix} 22 \\ 19 \\ 1,2 \end{pmatrix}$ weiter.
		
		\begin{teilaufgaben}
			\teilaufgabe \operator{Berechnen} sie: Mit welcher Geschwindigkeit hebt die Maschine vom Boden ab? In welchem Punkt befindet sich das Flugzeug 10 Minuten nach dem Abheben?
			
			\teilaufgabe Ein zweites Flugzeug befindet sich in dem Moment, in dem as Sportflugzeug in $A$ abhebt, im Punkt B\pkt(220|-180|32). Es bewegt sich über längere Zeit pro Minute um den Vektor $\vec{v} = \begin{pmatrix} 14 \\ 25 \\ 0 \end{pmatrix}$ weiter. \operator{Bestimmen} sie, wie weit die beiden Flugzeuge 10 Minuten nach dem Abheben des Sportflugzeuges voneinander entfernt sind.
			
			\teilaufgabe \operator{Untersuchen} sie, ob es zu einer Kollision kommen könnte, wenn die beiden Flugzeuge ihren Kurs beibehalten.
		\end{teilaufgaben}
	\end{aufgabe}
	\clearpage
	
	%% Hinweise Aufgabe 4
	\begin{center}
		\begin{multicols}{4}
			Teilaufgabe a)\\
			\qrcode[height=3cm]{Die Geschwindigkeit ist gleich
			der Laenge des Richtungsvektors der Flugbahn.
			
			Achten sie bei der Bestimmung auf die korrekten Einheiten.
			
			Fuer die Position nach 10 Minuten bestimmen sie eine neue
			Gerade mit der Position nach 3 Minuten als Stuetzvektor
			und dem neuen Richtungsvektor.}
			\columnbreak
			
			Teilaufgabe b)\\
			\qrcode[height=3cm]{Flugbahn des zweiten Flugzeugs:
			Position + r * Richtung
			
			Abstand:
			Position nach 10 Minuten berechnen, Differenzvektor zur
			Position des anderen Flugzeuges bilden, Laenge berechnen}
			\columnbreak
			
			Teilaufgabe c)\\
			\qrcode[height=3cm]{Flugbahnen (Geraden) gleichsetzen und pruefen,
			ob es eine Loesung (einen Schnittpunkt) gibt.}
			\columnbreak
			
			Lösungen\\
			\qrcode[height=3cm]{a) 2300 m/min = 138 km/h
			(212|176|17,4)
			2909 m/min = 174,5 km/h
			
			b) Pos zweites Flugzeug: (360|70|32)
			Abstand: ca. 18260 m
			
			c) Es gibt keinen Schnittpunkt, die Flugbahnen kreuzen sich nicht.}
		\end{multicols}
	\end{center}
\end{document}