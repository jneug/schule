\documentclass[10pt, a4paper, ngerman]{arbeitsblatt}

\ladeModule{theme,typo,icons,boxen,aufgaben}
\ladeFach[]{mathematik}

\aboptionen{
	name 		= {J. Neugebauer},
	kuerzel 	= {Ngb},
	titel 		= {Geradengleichungen},
	reihe 		= {Analytische Geometrie},
	fach 		= {Mathematik},
	kurs 		= {Q2},
	nummer 		= {III.5},
	lizenz 		= {cc-by-nc-sa-4},
	version 	= {2021-05-14},
}


\begin{document}
\ReiheTitel

\begin{aufgabe}
	\label{aufg:geraden-bestimmen}
	Bestimme eine Gleichung in Parameterdarstellung für eine Gerade \dots
	\begin{enumn}
		\item \dots durch die Punkte \punkt[A](5,8|7|-8) und \punkt[B](19|3|-8):
		      $\quad g:$\linie[6cm]
		\item \dots durch den Punkt \punkt[C](1|-6,5|32) mit der Richtung
		      $\vec{v} = \vector(13,2|-4|0)$:	$\quad h:$\linie[6cm]
		\item \dots entlang des Vektors $\vec{w} = \vector(-6,38|-7,7|8,8)$:
		      $\quad i:$\linie[6cm]
	\end{enumn}
\end{aufgabe}

\begin{aufgabe}
	\begin{infobox}\small
		Die Spurpunkte einer Geraden sind die \emph{Schnittpunkte der Geraden mit den Koordinatenebenen}.
		\emph{Mindestens eine} Koordinate jedes Spurpunktes ist also null. Eine Gerade
		kann einen, zwei oder drei Spurpunkte haben. Die Spurpunkte helfen dabei,
		Geraden zu zeichnen.
	\end{infobox}

	\begin{enumerate}
		\item Überlege, welche Möglichkeiten und Sonderfälle es für die Spurpunkte einer
		      Geraden geben kann.
		\item Berechne die \emph{Spurpunkte} der Geraden
		      $g$ aus \prettyref{aufg:geraden-bestimmen}.
	\end{enumerate}

	\vspace{4cm}
\end{aufgabe}

\begin{aufgabe}
	Bestimme die paarweise Lage aller Geraden aus
	\prettyref{aufg:geraden-bestimmen} zueinander.

	\begin{itemize}
		\item $g$ und $h$ sind \linie zueinander.
		\item $h$ und $i$ sind \linie zueinander.
		\item $g$ und $i$ sind \linie zueinander.
	\end{itemize}
\end{aufgabe}

\end{document}
