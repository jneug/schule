\documentclass[12pt,a5paper,landscape]{scrartcl}

\usepackage{vorschule}
\usepackage[
    typ=ohne,
    fach=Mathematik,
    lerngruppe={7},
	nummer=4,
    module={Lizenzen,Symbole},
	seitenzahlen=keine,
	farbig,
    lizenz=cc-by-nc-sa-4,
]{schule}

\usepackage[
	typ=lerntheke,
	kuerzel=Ngb,
	reihe={Terme und Gleichungen},
	version={2020-12-03},
]{ngbschule}

\author{J. Neugebauer}
\title{0.1 (Terme und Gleichungen)}
\date{\Heute}

\begin{document}

\begin{hilfekarte}{Terme, Gleichungen und Variablen}{begriffe}
	Ein \textbf{Term} ist ein \emph{sinnvoller} Rechenausdruck aus Zahlen, Rechenzeichen, Vorzeichen, Klammern und Variablen.
	
	Gültig: $5+(3,4 - (-4)) + x$\\
	Ungültig: $5+\cdot5)-4$
	
	\vspace{1cm}
	Eine \textbf{Gleichung} besteht aus zwei Termen, die durch ein Gleichzeichen ($=$) verbunden sind. Eine Gleichung kann \emph{wahr} sein (z.B. $5+3 = 9-1$) oder \emph{falsch} sein (z.B. $4 = 1+1$).
	
	\vspace{1cm}
	Eine \textbf{Variable} ist ein Platzhalter für einen Wert, der erstmal unbekannt oder nicht festgelegt ist.
\end{hilfekarte}

\begin{loesungskarte}[Terme, Gleichungen und Variablen]
	Beispiele für Terme:
	\begin{multicols}{2}
		\textbf{sinnvolle Ausdrücke (Terme)}
		
		\textbf{unsinnige Ausdrücke (keine Terme)}
	\end{multicols}
\end{loesungskarte}


\begin{hilfekarte}{Variablen}{variablen}
	\textbf{Variablen} sind Platzhalter für Werte, die noch unbekannt sind. Wir benutzen kleine Buchstaben für Variablen (z.B: $x,y,a,b,c,\dots$). Kommt ein Buchstabe mehrfach vor, dann steht er immer für denselben Wert.
	
	Es gibt zwei Arten von Variablen:
	
	In einem \emph{Term} ist eine Variable ein Platzhalter für beliebige Werte, die du dir selber aussuchen kannst. Du kannst für die Variablen Zahlen einsetzen und den Wert des Terms berechnen. z.B.
	
	\[ 5\cdot a + 4 \overset{a = 3}{\longrightarrow} 5\cdot 3 + 4 = 19 \]
	
	In einer \emph{Gleichung} steht eine Variable für einen bestimmten Wert, der gefunden werden muss. So, dass die Gleichung wahr ist. z.B.
	
	\[ 5\cdot x + 4 = 34 \Rightarrow x = 4 \text{, da } 5\cdot 4 + 4 = 34\text{ wahr ist} \]
\end{hilfekarte}

\leereKarte

\begin{hilfekarte}{Termumformungen}{termumf}
\end{hilfekarte}

\leereKarte

\begin{hilfekarte}{Äquivalenzumformungen}{aequivumf}
\end{hilfekarte}

\leereKarte

\begin{karte1}{Terme }
	
\end{karte1}

\begin{loesungskarte}{Erste Karte}

\end{loesungskarte}


\end{document}
