\documentclass[11pt, a4paper]{scrartcl}

\usepackage{vorschule}
\usepackage[
    typ=ab,
    fach=Mathematik,
    lerngruppe={Q1 GK},
    nummer={4},
    module={Symbole},
]{schule}

\usepackage[
	typ=checkup,
	kuerzel=Ngb,
	reihe={Analysis III},
]{ngbschule}


\author{J. Neugebauer}
\title{\Nummer. Klausur}
\date{\Heute}

\begin{document}
\CheckupBild\CheckupTitel

Kreuzen sie jeweils an, wie sicher sie sich bei den einzelnen \textbf{Themenschwerpunkten} fühlen (von \enquote{sehr sicher} \usym{1F604} bis \enquote{sehr unsicher} \usym{1F641}). Nutzen sie die \textbf{Aufgaben und Informationen} zum Wiederholen und Lernen von Themen, bei denen sie noch unsicher sind.

Die \textbf{Lösungen} zu den Aufgaben finden sie hinten im Buch. Weitere Aufgaben (ohne Lösungen) finden sie im Buch in den Kapiteln zum Thema.

\begin{checkup}
	\ichkann{\dots Ableitungsregeln anwenden (insbesondere von Exponentialfunktionen und inklusive Produkt- und Kettenregel).}{
		\bu{117-119}{} \\
		\bu{123}{} \\
		\bu{133}{} \\
		AB \enquote{Exp.fkt. abl. I + II} \\
		AB \enquote{Ableitungsregeln} \\
		\bu{149}{1}
	}
	\ichkann{\dots Stammfunktionen von Exponentialfunktionen (insbesondere der e-Funktion) ermitteln.}{
		\bu{117-119}{} \\
		\bu{123}{} \\
		\bu{133}{} \\
		AB \enquote{Ableitungsregeln} \\
		\bu{149}{2} \\
		\bu{150}{6b}
	}
	\ichkann{\dots den Hauptsatz der Differential- und Integralrechnung anwenden.}{
		$\int_a^b\!f(x)\d{x} = F(b) - F(a)$
	}
	\ichkann{\dots Exponentielles Wachstum und Abnahme mit e-Funktionen modellieren.}{
		\bu{123/123}{} \\
		\bu{126/127}{} \\
		\bu{150}{7, 8}\\
		AB \enquote{Exp.fkt. abl. I + II}
	}
	\ichkann{\dots exponentielle Wachstumsprozesse untersuchen (Nullstellen, Extrempunkte, \dots) und Ergebnisse im Sachzusammenhang interpretieren.}{
		\bu{138/139}{} \\
		\bu{142-144}{} \\
		\bu{150}{7, 8}
	}
\end{checkup}

\end{document}
