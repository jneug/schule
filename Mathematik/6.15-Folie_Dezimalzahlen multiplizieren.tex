\documentclass[11pt, a4paper]{scrartcl}

\usepackage{vorschule}
\usepackage[
    typ=folie,
    fach=Mathematik,
    lerngruppe={6d},
    nummer=15,
    module={Symbole,Papiertypen},
]{schule}

\usepackage[
	kuerzel=Ngb,
	reihe={Mit Brüchen und Dezimalzahlen rechnen},
	version={2019-03-14},
]{ngbschule}

\author{J. Neugebauer}
\title{Dezimalzahlen multiplizieren}
\date{\Heute}

\begin{document}
	\LARGE

	Hannes soll in seinem Kinderzimmer einen neuen Teppich bekommen. Er hat ausgemessen, dass sein Zimmer \unit[$4,5$]{m} breit und \unit[$3,6$]{m} lang ist. Ein Quadratmeter des Teppichs, den er sich ausgesucht hat, kostet \unit[$4,99$]{\euro}.
	
	\begin{center}
	\includegraphics{6_15-Abb_Renovieren.jpg}
	
	\textbf{Wie viel kostet der neue Teppich insgesamt?}
	\end{center}
	
\end{document}