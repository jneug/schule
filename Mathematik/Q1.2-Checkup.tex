\documentclass[11pt, a4paper]{scrartcl}

\usepackage{vorschule}
\usepackage[
    typ=ab,
    fach=Mathematik,
    lerngruppe={Q1 GK},
    nummer={2},
    module={Symbole},
]{schule}

\usepackage[
	typ=checkup,
	kuerzel=Ngb,
	reihe={Analysis},
]{ngbschule}


\author{J. Neugebauer}
\title{2. Klausur}
\date{\Heute}

\begin{document}
\CheckupBild\CheckupTitel

Kreuzen sie jeweils an, wie sicher sie sich bei den einzelnen \textbf{Themenschwerpunkten} fühlen (von \enquote{sehr sicher} \usym{1F604} bis \enquote{sehr unsicher} \usym{1F641}). Nutzen sie die \textbf{Aufgaben und Informationen} zum Wiederholen und Lernen von Themen, bei denen sie noch unsicher sind.

Die \textbf{Lösungen} zu den Aufgaben finden sie hinten im Buch. Weitere Aufgaben (ohne Lösungen) finden sie im Buch im Kapitel 1.1.3, 1.3 und 1.4.

\begin{checkup}
	\ichkann{... Potenzfunktionen mit negativen Exponenten ableiten.}{
		\bu{27/28}{} \\
		\bu{64}{7}
	}
	\ichkann{... Aspekte der Funktionsuntersuchung anwenden (vor allem hinreichende und notwendige Bedingungen für Extrem- und Wendestellen).}{
		\bu{23-25}{} \\
		\bu{13-15}{} \\
		\bu{18-21}{}
	}
	\ichkann{... Lösungsverfahren für lineare Gleichungssysteme mit zwei Unbekannten anwenden.}{
		\bu{42/43}{}
	}
	\ichkann{... erkennen, ob ein Gleichungssystem eine, keine oder unendlich viele Lösungen hat.}{
		\bu{42/43}{}\\
		\bu{47/48}{}
	}
	\ichkann{... den Gauß-Algorithmus zur Lösung von Gleichungssystemen mit mehr als zwei Unbekannten anwenden (auch auf einer Koeffizientenmatrix).}{
		\bu{44/45}{}\\
		\bu{65}{9}
	}
	\ichkann{... Polynomfunktionen anhand eines \enquote{Steckbriefs} ermitteln.}{
		\bu{51-53}{}\\
		\bu{65}{10}\\
		\bu{66}{16}
	}
	\ichkann{... den GTR zur Lösung von Gleichungssystemen nutzen.}{
		\bu{65}{13}
	}
\end{checkup}

\end{document}
