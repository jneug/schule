\documentclass[11pt, a5paper, landscape, final]{scrartcl}

% put to a5 pages on one a4 sheet
\usepackage{pgfpages}
\pgfpagesuselayout{2 on 1}[a4paper]

\usepackage{vorschule}
\usepackage[
    typ=ab,
    fach=Mathematik,
    lerngruppe={Q1 GK},
%    nummer=14,
    seitenzahlen=keine,
%    loesungen=folgend,
    module={Symbole},
]{schule}

\usepackage[
	kuerzel={Ngb},
	reihe={Analysis},
	version={2019-03-31},
]{ngbschule}

\author{J. Neugebauer}
\title{Vermischte Übungen zur Klausur}
\date{10.01.2018}

\setzeAufgabentemplate{ngbohne}

\chead{\Titel}

\usepackage{qrcode}
\qrset{height=4.5cm, nolink, padding}

\begin{document}
	%% Aufgabe 1
	\begin{aufgabe}
		\operator{Bestimmen} sie die ersten und zweiten Ableitungen der Funktionen.
		\begin{multicols}{2}
			\begin{enumerate}[label=\alph*)]
				\item $f(x) = 2x^{-2} + 5x^{-3}$
				\item $f(x) = \frac{3}{4x^2}$
				\item $f(x) = 4x^{-5} + 2x^3 - \frac{1}{x^3}$
				\item $f(x) = \frac{1}{3x^8} - 25x^4 - \frac{20}{x^5}$
			\end{enumerate}
		\end{multicols}
	\end{aufgabe}
	\clearpage
	
	%% Aufgabe 2
	\begin{aufgabe}
		\operator{Berechnen} sie die Lösungen der Gleichungssysteme \emph{ohne Einsatz des GTR}. \operator{Begründen} sie gegebenenfalls, warum es keine oder unendlich viele Lösungen gibt.
		\begin{multicols}{2}
			\begin{enumerate}[label=\alph*)]
				\item 
				\[ \left| \begin{alignedat}{4}
				-x + && y + && z = & 0 \\
				x - && 3y - && 2z = & 5 \\
				5x + && y + && 4z = & 3
				\end{alignedat} \right| \]
				
				\item 
				\[ \left| \begin{alignedat}{4}
				-2x - && 2y + &&  z = & 10 \\
				3x + && 12y - &&  7z = & 42 \\
				-16x - && 16y + && 8z = & 56
				\end{alignedat} \right| \]
			\end{enumerate}
		\end{multicols}
	\end{aufgabe}
	\clearpage
	
	%% Hinweise Aufgabe 1
	\begin{center}
		Beachten sie, dass x\textasciicircum-3 gleichbedeutend ist mit $x^{-3}$.
		\begin{multicols}{2}
			1. Ableitung \\
			\qrhinweis[a) f'(x) = -4 x\^-3 - 15 x\^-4
				b) f'(x) = -3/2 x\^-3
				c) f'(x) = -20 x\^-6 + 6 x\^2 + 3 x\^-4
				d) f'(x) = -8/3 x\^-9 - 100 x\^3 + 100 x\^-6 ]{
				\begin{enumerate}[label=\alph*)]
					\item $f'(x) = -4 x^-3 - 15x^-4$
					\item $f'(x) = -\frac{3}{2} x^-3$
					\item $f'(x) = -20 x^-6 + 6 x^2 + 3 x^-4$
					\item $f'(x) = -\frac{8}{3} x^-9 - 100 x^3 + 100 x^-6$
				\end{enumerate}			
			}
			\columnbreak
			
			2. Ableitung\\
			\qrhinweis[a) f''(x) = 12 x\^-4 + 60 x\^-5
				b) f''(x) = 9/2 x\^-4
				c) f''(x) = 120 x\^-7 + 12 x - 12 x\^-5
				d) f''(x) = 24 x\^-10 - 300x\^2 - 600 x\^-7 ]{
				\begin{enumerate}[label=\alph*)]
					\item $f''(x) = 12 x^-4 + 60 x^-5$
					\item $f''(x) = \frac{9}{2} x^-4$
					\item $f''(x) = 120 x^-7 + 12 x - 12 x^-5$
					\item $f''(x) = 24 x^-10 - 300x^2 - 600 x^-7$
				\end{enumerate}
			}
		\end{multicols}
	\end{center}
	\clearpage
	
	%% Hinweise Aufgabe 2
	\begin{center}
		\begin{multicols}{2}
			Lösung zu a) \\
			\qrcode{x = -1
				y = -4
				z = 3
			
				L=\{(-1|-4|3)\} }
			\columnbreak
			
			Lösung zu b) \\
			\qrcode{Keine Loesung
				
				L = \{\}
				
				Teilt man die letzte Gleichung durch 8 entsteht ein Widerspruch zur Gleichung 1}
		\end{multicols}
	\end{center}
	\clearpage
	
	%% Aufgabe 3
	\begin{aufgabe}
		\operator{Bestimmen} sie die Gleichung einer ganzrationalen Funktion 3. Grades. Der Graph der Funktion geht durch den Punkt $(3|-28)$, sie hat eine Nullstelle bei $x = 1$ und an der Stelle $x = 0,5$ eine Wendestelle mit der Steigung $1,5$.
	\end{aufgabe}
	\clearpage

	%% Aufgabe 4
	\begin{aufgabe}
		\operator{Bestimmen} sie die Gleichung der unten abgebildeten ganzrationalen Funktion.
		
		\begin{center}
		\includegraphics{Q1-AB-Uebungen_Graph_Afg3.pdf}
		\end{center}
	\end{aufgabe}
	\clearpage
	
	%% Hinweise Aufgabe 3
	\begin{center}
		\begin{multicols}{3}
			Gleichungen\\
			\qrcode{f(3) = -28
				f(1) = 0
				f''(0,5)=0
				f'(0,5)=1,5 }
			\columnbreak
			
			Koeffizientenmatrix\\
			\qrcode{1     1   1   1    0
				27     9   3   1  -28
				0,75  1   1   0    1,5
				3     2   0   0    0 }
			\columnbreak
			
			Lösung\\
			\qrcode{a = -2
				b = 3
				c = 0
				d = -1 }
		\end{multicols}
	\end{center}
	\clearpage
	
	%% Hinweise Aufgabe 4
	\begin{center}
		Beachten sie, dass x\textasciicircum-4 gleichbedeutend ist mit $x^{-4}$.
		\begin{multicols}{3}
			Gleichungen\\
			\qrcode{Das Bild deutet auf Grad 4 hin. Wegen Symmetrie dann:
			f(x) = ax^4 + bx^2 + c
			Es lassen sich z.B. diese Punkte ablesen:
			f(0) = -2
			f(2) = 5
			f'(2) = 0}
			\columnbreak
			
			Koeffizientenmatrix\\
			\qrcode{ 0    0    1    -2
				16    4    1     5
				32    4    0     0 }
			\columnbreak
			
			Lösung\\
			\qrcode{a = -7/16
				b = 7/2
				c = -2  }
		\end{multicols}
	\end{center}
\end{document}