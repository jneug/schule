\documentclass[10pt, a4paper, landscape, twocolumn]{scrartcl}

\usepackage{vorschule}
\usepackage[
    typ=ab,
    fach=Mathematik,
    lerngruppe={Q1 GK},
    nummer=4,
    module={Symbole,Lizenzen},
    seitenzahlen=keine,
    farbig,
    lizenz=cc-by-nc-sa-4,
]{schule}

\usepackage[
	kuerzel={Ngb},
	reihe={Analysis},
	version={2019-06-17},
]{ngbschule}

\author{J. Neugebauer}
\title{Die $e$-Funktion}
\date{\Heute}

\renewcommand{\arraystretch}{1.6}

\begin{document}
    \section*{Exponentialfunktionen}
    \[ f(x) = a\cdot b^x = a\cdot e^{\ln{b}x} \]
    \begin{multicols}{2}
    \begin{description}
        \item[$a$] Startwert
        \item[$e$] $\approx 2,7183$
        \item[$b>1$] Wachstumsfaktor ($\ln{b}>0$)
        \item[$0<b<1$] Abnahmefaktor ($\ln{b}<0$)
    \end{description}
    \end{multicols}
    
    \section*{Eigenschaften der $e$-Funktion}
    Für $f(x)=e^{k\cdot x},\ k \geq 1$ gilt:
    \begin{multicols}{2}
        \begin{tikzpicture}[scale=0.4]
            \geoInit[xmin=-5,xmax=5,ymin=-5,ymax=5]
            \tkzDrawXY[label={},]
            %\geoGitter
            \tkzFct[]{2.71828 ** \x}
        \end{tikzpicture}
        
        \columnbreak
            
        \begin{itemize}
            \item $f(0)=1$
            \item $f(x)>0$ für alle $x\in \R$
            \item $f$ wächst steng monoton
            \item $f(x) \longrightarrow \infty$ für $x \longrightarrow \infty$
            \item $f(x) \longrightarrow 0$ für $x \longrightarrow -\infty$
        \end{itemize}
    \end{multicols}
    
     Für $f(x)=e^{k\cdot x},\ k \leq -1$ gilt:
    \begin{multicols}{2}
        \begin{tikzpicture}[scale=0.4]
            \geoInit[xmin=-5,xmax=5,ymin=-5,ymax=5]
            \tkzDrawXY[label={},]
            %\geoGitter
            \tkzFct[]{2.71828 ** (-1 * \x)}
        \end{tikzpicture}
        
        \columnbreak
        
        \begin{itemize}
            \item $f(0)=1$
            \item $f(x)>0$ für alle $x\in \R$
            \item $f$ fällt steng monoton
            \item $f(x) \longrightarrow 0$ für $x \longrightarrow \infty$
            \item $f(x) \longrightarrow -\infty$ für $ex \longrightarrow -\infty$
        \end{itemize}
    \end{multicols}
    
    \section*{Kombinationen der $e$-Funktion}
    \begin{tabular}{lc}
        $f(x) = x^k\cdot e^x$ & \\
        $f(x) = k\cdot e^x$ & \\
        $f(x) = k\cdot e^x + k\cdot e^{-x}$ & \\
        $f(x) = e^{k\cdot x} - e^{k\cdot -x}$ & \\
    \end{tabular}
\end{document}