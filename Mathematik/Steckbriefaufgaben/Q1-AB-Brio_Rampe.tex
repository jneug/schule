\documentclass[11pt, a4paper]{scrartcl}

\usepackage{vorschule}
\usepackage[
    typ=ab,
    fach=Mathematik,
    lerngruppe={Q1 GK},
    nummer=11,
    loesungen=folgend,
    seitenzahlen=keine,
    zitate=quotes,
    module={Symbole},
]{schule}

\usepackage[
	kuerzel=Ngb,
	reihe={Analysis},
	version=1,
]{ngbschule}

\author{J. Neugebauer}
\title{BRIO Eisenbahnrampe}
\date{\Heute}

\setzeAufgabentemplate{ngbohne}

\begin{document}
	\TITEL

	\begin{aufgabe}
		\operator{Bestimmen} sie eine Modellfunktion zur Beschreibung einer Eisenbahnrampe für eine Kinder-Holzeisenbahn.

		\includegraphics[width=\textwidth]{Q1-Abb-Brio_Rampe.pdf}

		\begin{loesung}
			Die gesuchte Funktion hat mindestens Grad 3, da sie zwei Stellen hat mit $f'(x) = 0$. Also:

			\[ f(x) = ax^3 + bx^2 + cx + d \]

			Es werden also vier Gleichungen benötigt, da in der Funktion vier Unbekannte zu bestimmen sind. Die beiden Stellen mit Steigung $0$ liefern jeweils zwei Gleichungen. Die Punkte können durch Messen der Schiene bestimmt werden und die Ableitung kann bestimmt und gleich $0$ gesetzt werden:

			\[ f'(x) = 3ax^2 + 2bx + c \]

			Angenommen die gemessenen Werte sind $h = 7$ und $l = 20$, dann ergeben sich die Gleichungen:

			\begin{align*}
				f(0) = 7 &\Leftrightarrow a\cdot 0^3 + b\cdot 0^2 + c\cdot 0 + d = 7 \\
				f(20) = 0 &\Leftrightarrow a\cdot 20^3 + b\cdot 20^2 + c\cdot 20 + d = 0 \\
				f'(0) = 0 &\Leftrightarrow 3a\cdot 0^2 + 2b\cdot 0 + c = 0 \\
				f'(20) = 0 &\Leftrightarrow 3a\cdot 20^2 + 2b\cdot 20 + c = 0
			\end{align*}

			Nun lässt sich die erweiterte Koeffizientenmatrix aufstellen und mit dem Gauß-Algorithmus lösen:
			\[ \begin{gmatrix}[p]
			0 & 0 & 0 & 1 & 7 \\
			8000 & 400 & 20 & 1 & 0 \\
			0 & 0 & 1 & 0 & 0 \\
			1200 & 40 & 1 & 0 & 0
			\rowops
			\swap{0}{3}
			\end{gmatrix} \]
			\[ \begin{gmatrix}[p]
			1200 & 40 & 1 & 0 & 0 \\
			8000 & 400 & 20 & 1 & 0 \\
			0 & 0 & 1 & 0 & 0 \\
			0 & 0 & 0 & 1 & 7
			\end{gmatrix} \]

			Durch Verwendung des GTR-Befehls \texttt{rref} kann das Gleichungssystem gelöst werden. Wir erhalten:
			\[ \begin{gmatrix}[p]
			1 & 0 & 0 & 0 & \frac{7}{4000} \\
			0 & 1 & 0 & 0 & \frac{-21}{400} \\
			0 & 0 & 1 & 0 & 0 \\
			0 & 0 & 0 & 1 & 7
			\end{gmatrix} \]

			Die gesuchte Funktion ist also:
			\begin{align*}
				f(x) &= \frac{7}{4000}x^3 - \frac{21}{400}x^2 + 7 \\
				f'(x) &= \frac{21}{4000}x^2 - \frac{42}{400}x
			\end{align*}

			\begin{center}
				\begin{tikzpicture}[scale=.6]
					\tkzInit[xmin=-1,xmax=25,ymin=-1,ymax=9]
					\tkzGrid
					\tkzAxeXY[font=\sffamily\small]
					\tkzFct[color=NavyBlue,line width=1.2pt,domain=0:20]{(7 * x ** 3)/4000 - (21 * x ** 2)/400 + 7}
					\tkzFct[color=WildStrawberry,line width=.9pt,style=dashed]{(21 * x ** 2)/4000 - (42 * x)/400}
					\geoKoordinate(3,7){A}\geoPunktBeschriften[above](A){$f(x)$}
					\geoKoordinate(1,0){B}\geoPunktBeschriften[above](B){$f'(x)$}
				\end{tikzpicture}
			\end{center}

			Probe:
			\begin{align*}
			f(0) = \frac{7}{4000}\cdot 0^3 - \frac{21}{400}\cdot 0^2 + 7 = 7 \ \checkmark\\
			f(20) = \frac{7}{4000}\cdot 20^3 - \frac{21}{400}\cdot 20^2 + 7 = 0 \ \checkmark\\
			f'(0) = \frac{21}{4000}\cdot 0^2 - \frac{42}{400}\cdot 0 = 0 \ \checkmark \\
			f'(20) = \frac{21}{4000}\cdot 20^2 - \frac{42}{400}\cdot 20 = 0 \ \checkmark
			\end{align*}
		\end{loesung}
	\end{aufgabe}
\end{document}
