\documentclass[12pt, a4paper]{scrartcl}

\usepackage{vorschule}
\usepackage[
    typ=ab,
    fach=Mathematik,
    lerngruppe={Q1 GK},
    nummer=10,
    loesungen=seite,
    module={Lizenzen},
    lizenz={cc-by-nc-sa-4}
]{schule}

\usepackage[
	kuerzel=Ngb,
	reihe={Analysis},
	version={2019-03-31},
]{ngbschule}

\author{J. Neugebauer}
\title{Vermischte Übungen}
\date{\Heute}

\setzeAufgabentemplate{ngbnormal}

\begin{document}
	\ReiheTitel
	
	\begin{aufgabe}[subtitle=Bilden sie die Ableitung der Funktion]
		\begin{tasks}(2)
			\task $f(x) = \frac{1}{x}$
			\task $f(x) = \frac{4}{x}$
			\task $f(x) = 3x^{-4}$
			\task $f(x) = 7x^5 + 8x^{-2}$
			\task $f(x) = \frac{-3}{x^5} - 4x^3 + 3$
			\task $f(x) = 9x^6 - 7x^{-3} + \frac{2}{x^2} - 18$
		\end{tasks}
	
		\begin{loesung}
			\begin{tasks}(2)
				\task $f'(x) = -\frac{1}{x^2} = -1x^{-2}$
				\task $f'(x) = -\frac{4}{x^2} = -4x^{-2}$
				\task $f'(x) = -12x^{-5} = -\frac{12}{x^5}$
				\task $f'(x) = 35x^4 - 16x^{-3}$
				\task $f'(x) = 15x^{-6} - 12x^2$
				\task $f'(x) = 54x^5 + 21x^{-4} - 4x^{-3}$
			\end{tasks}
		\end{loesung}
	\end{aufgabe}

	\begin{aufgabe}[subtitle=,print=false]
	\end{aufgabe}

	\begin{aufgabe}[subtitle=,print=false]
	\end{aufgabe}

	\begin{aufgabe}[subtitle=,print=false]
	\end{aufgabe}

	\begin{aufgabe}[subtitle=,print=false]
	\end{aufgabe}

	\begin{aufgabe}[subtitle=Bringen sie das Gleichungssystem in Dreiecksgestalt]
		\begin{tasks}(2)
			\task
			\[ \left|\begin{alignat}{4} 
			x + && y - && z && = 3 \\
			x + && 2y - && 2z && = 2 \\
			2x - && y + && 2z && = 15
			\end{alignat}\right| \]
			
			\task
			\[ \left|\begin{alignat}{4} 
			-2x + && y + && z = && -1 \\ 
			6x - && 2y - && 2z = && 6 \\
			3x - && 2y + && 2z = && 4
			\end{alignat}\right| \]
		\end{tasks}
	
		\begin{loesung}
			\begin{enumeratea}
				\item Mit der erweiterten Koeffizientenmatrix:
				\[ \begin{gmatrix}[p]
				1 & 1 & -1 & 3 \\ 
				1 & 2 & -2 & 2 \\
				2 & -1 & 2 & 15
				\rowops
					\mult{0}{\cdot -1}
					\add{0}{1}
					\mult{0}{\cdot -2}
					\add{0}{2}
				\end{gmatrix} \]
				\[ \begin{gmatrix}[p]
				1 & 1 & -1 & 3 \\ 
				0 & 1 & -1 & -1 \\
				0 & -3 & 4 & 9
				\rowops
					\mult{1}{\cdot 3}
					\add{1}{2}
				\end{gmatrix} \]
				\[ \begin{gmatrix}[p]
				1 & 1 & -1 & 3 \\ 
				0 & 1 & -1 & -1 \\
				0 & 0 & 1 & 6
				\end{gmatrix} \]
				
				Einsetzen von unten nach oben ergibt:
				\begin{align*}
				z &= 6 \\
				y - z = y - 6 &= -1 \\
					&\Leftrightarrow y = 5 \\
				x + y - z = x + 5 - 6 &= 3 \\
					&\Leftrightarrow x = 4
				\end{align*}
				
				\item Mit der erweiterten Koeffizientenmatrix:
				\[ \begin{gmatrix}[p]
				-2 & 1 & 1 & -1 \\ 
				6 & -2 & -2 & 6 \\
				3 & -2 & 2 & 4
				\rowops
				\mult{0}{\cdot 3}
				\add{0}{1}
				\mult{0}{\cdot \nicefrac{3}{2}}
				\add{0}{2}
				\end{gmatrix} \]
				\[ \begin{gmatrix}[p]
				-2 & 1 & 1 & -1 \\ 
				0 & 1 & 1 & 3 \\
				0 & -0,5 & 3,5 & 2,5
				\rowops
				\mult{1}{\cdot 0,5}
				\add{1}{2}
				\end{gmatrix} \]
				\[ \begin{gmatrix}[p]
				-2 & 1 & 1 & -1 \\ 
				0 & 1 & 1 & 3 \\
				0 & 0 & 4 & 4
				\end{gmatrix} \]
				
				Einsetzen von unten nach oben ergibt:
				\begin{align*}
				4z &= 4 \\
				&\Leftrightarrow z = 1 \\
				y + z = y + 1 &= 3 \\
				&\Leftrightarrow y = 2 \\
				-2x + y + z = -2x + 2 + 1 &= -1 \\
				&\Leftrightarrow -2x = -4 \\
				&\Leftrightarrow x = 2
				\end{align*}
			\end{enumeratea}
		\end{loesung}
	\end{aufgabe}
\end{document}