\documentclass[11pt, a4paper, ngerman]{arbeitsblatt}

\ladeModule{theme,muster}

\aboptionen{
	name 		= {J. Neugebauer},
	kuerzel 	= {Ngb},
	titel 		= {Mit Brüchen rechnen},
	reihe 		= {Bruchrechnung},
    fach 		= {Mathematik},
    kurs 		= {Jg.6},
    nummer 		= {6},
    lizenz 		= {cc-by-nc-sa-4},
	version 	= {2021-04-30},
}


\xsimsetup{
	aufgabe/template = abkompakt
}


\begin{document}

\ReiheTitel

\medskip
\begin{minipage}{0.7\textwidth}
Anne hat drei Geschwister, die fünf, sechs und acht Jahre alt sind. Als sie
alleine ihre Oma besucht steckt die ihr einen Geldschein zu und sagt: \medskip

\enquote{\textit{Schön das du heute da warst, Anne. Das Geld ist für dich und deine
Geschwister. Ich möchte das du jedem den Anteil abgibst, der seinem Alter entspricht.}} \medskip

\enquote{\textit{Och Oma}}, ruft Anne entsetzt, \enquote{\textit{dann bekomme ich ja am wenigsten von allen!}}
\end{minipage}\hfill
\begin{minipage}{4cm}
	\includegraphics[width=4cm]{6-AB.6-Abb_1}
\end{minipage}

\begin{center}\footnotesize
Falls ihr nicht weiter kommt:\\
Zu jeder Aufgabe gibt es vorne einen Tipp mit der entsprechenden Aufgabennummer.
\end{center}

\begin{aufgabe}
	Prüft mit den \textbf{Bruchsteinen} Annes Befürchtung. Hat sie Recht?
\end{aufgabe}
\medskip

\begin{aufgabe}
	Welchen Anteil muss Anne insgesamt abgeben? \textbf{Notiert eine Rechnung:}

	\liniert{1}
\end{aufgabe}
\medskip

\begin{aufgabe}
	\enquote{Haben Brüche denselben Nenner, dann kann man sie leicht addieren und subtrahieren.}

	Prüft diese Aussage mit verschiedenen \textbf{Kombinationen von Bruchsteinen}. Notiert jeweils eine Rechnung und das Ergebnis als Bruch (z.B. $\frac{1}{4} + \frac{1}{4} = $...).

	\liniert{4}
\end{aufgabe}
\medskip

\begin{aufgabe}
	Formuliert eine Regel, wie man Brüche mit gleichem Nenner (gleichnamige Brüche) addieren und subtrahieren kann.

	Man kann gleichnamige Brüche addieren und subtrahieren, indem man ...

	\liniert{6}
\end{aufgabe}

\end{document}
