\documentclass[10pt, a4paper]{scrartcl}

\usepackage{vorschule}
\usepackage[
    typ=ab,
    fach=Mathematik,
    lerngruppe={Q1 GK},
    nummer=18,
    module={Symbole,Lizenzen},
    seitenzahlen=keine,
    farbig,
    lizenz=cc-by-nc-sa-4,
]{schule}

\usepackage[
	kuerzel={Ngb},
	reihe={Analysis III},
	version={2019-05-05},
]{ngbschule}

\author{J. Neugebauer}
\title{Exponentialfunktionen ableiten II}
\date{\Heute}

\setzeAufgabentemplate{ngbohne}
%\setzeAufgabentemplate{schule-binnen}

\usepackage{qrcode}

\renewcommand{\qrhinweis}[1]{%
	\begin{wrapfigure}{r}{0pt}
		\qrcode[height=1cm]{#1}
	\end{wrapfigure}%
}

\begin{document}
	\ReiheTitel
	
	Auf einem Konto werden \EUR{10000} zu \SI{1,5}{\percent\pa} angelegt. Das Wachstum des Guthabens kann durch eine \emph{Exponentialfunktion} abhängig von der Zeit $t$ modelliert werden:
	\[ g(t) = \luecke{8cm} \]
	
	Mit welcher Rate wächst das Guthaben nach \SI{2}{Jahren}?
	\begin{rahmen}
	\paragraph{Lösungsansatz}
	\begin{center}
		\begin{smallenumerate}[label=\Alph*)]\itemsep -1ex
			\item Exponentialfunktion als $e$-Funktion darstellen.
			\item Ableitung bilden.
			\item $t=2$ in die Ableitung einsetzen.
		\end{smallenumerate}
	\end{center}
	\end{rahmen}

	\qrhinweis{g(t) = 10000 * 1,015^t = 10000 * e^ln(1,015)^t = 10000 * e^(ln(1,015) * t)}
	\paragraph{A)} Nutzen sie den \emph{natürlichen Logarithmus} und die \emph{Potenzgesetze}, um die Funktion $g(t)$ als $e$-Funktion darzustellen. (Stellen sie die Basis $1,015$ als Potenz von $e$ dar.)
	\[ g(t) = \luecke{6cm} = \luecke{6cm} \]
	
	\qrhinweis{f(x) = a * b^x = a * e^ln(b)^x = a * e^(ln(b) * x)}
	\symStern\ Leiten sie aus dem Beispiel eine \emph{allgemeine Regel} zur Umformung von Exponentialfunktionen in eine $e$-Funktion ab.
	\[ f(x) = a\cdot b^x = \luecke{6cm} \]
	
	\qrhinweis{f1'(x) = k * e^(kx)}
	\paragraph{B)} Ermitteln sie mit \emph{GeoGebra}\footnote{Nutzen sie die Konstruktionsanleitung auf der Rückseite.} die Ableitung(en) von \[ f_1(x) = e^{k\cdot x} \qquad f_1'(x) = \luecke{6cm} \]
	
	\qrhinweis{f2'(x) = k * e^(kx+n)}
	\[ \symStern\ f_2(x) = e^{k\cdot x + n} \qquad f_2'(x) = \luecke{6cm} \]
	
	\qrhinweis{g'(x) = 10000 * ln(1,015) * e^(ln(1,015) * t)}
	Wenden sie die Regel\footnote{Denken sie auch an die anderen Ableitungsregeln.} auf $g(t)$ (in der Darstellung als $e$-Funktion) an:
	\[ g'(t) = \luecke{6cm} \]
	
	\qrhinweis{g'(2) = 153,39}
	\paragraph{C)} Berechnen sie $g'(2)$ und interpretieren sie das Ergebnis im Sachzusammenhang.
	\[ g'(2) = \luecke{8cm} \]
	
	\clearpage
	
	\section*{Untersuchung von $e^{kx}$ in GeoGebra}
	\begin{enumeraten}
		\item Starten sie GeoGebra in der \appfunktion{Grafikrechner}-Ansicht.
		\item In der Seitenleiste links können sie Eingaben machen (neben dem \appfunktion{+}-Symbol).
		\item Klicken sie in das Eingabefeld und geben sie ein:
		\begin{center}\large
			\verb!f(x)=e^kx!
		\end{center}
		\hinweis{Geben sie die Symbole genauso ein. GeoGebra macht den Rest.}
		\item Bestätigen sie mit \taste{ENTER}. GeoGebra zeigt die Funktion an und erstellt einen \enquote{Schieberegler}, mit dem $k$ verändert werden kann.
		\item Das nächste Eingabefeld sollte automatisch ausgewählt worden sein. Geben sie hier ein
		\begin{center}\large
			\verb!P=(0,f(0))!
		\end{center}
		und bestätigen sie wieder mit \taste{Enter}.
		\item Geben sie im nächsten Feld ein
		\begin{center}\large
			\verb!g=Tangente(P,f)!
		\end{center}
		\item Und zuletzt
		\begin{center}\large
			\verb!Steigung(g)!
		\end{center}
		\item Untersuchen sie nun die Steigung der Tangente an der Stelle \num{0} für verschiedene Werte von $k$, indem sie den Schieberegler manipulieren.
	\end{enumeraten}

	\section*{Untersuchung von $e^{kx+n}$ in GeoGebra}
	Gehen sie vor wie oben beschrieben, aber geben sie in Schritt 3) die neue Funktion ein:
	\begin{center}\large
		\verb!f(x)=e^kx+n!
	\end{center}


	\section*{Vermutungen überprüfen}
	\textbf{\llap{\Large\symInfo}\space}Um ihre Vermutungen zu überprüfen kann GeoGebra ihnen die Ableitung der Funktion direkt anzeigen mit der Eingabe
	\begin{center}\large
			\verb!Ableitung(f)!
	\end{center}
\end{document}