\documentclass[11pt, a5paper, landscape, final]{scrartcl}

% put to a5 pages on one a4 sheet
\usepackage{pgfpages}
\pgfpagesuselayout{2 on 1}[a4paper]

\usepackage{vorschule}
\usepackage[
    typ=ab,
    fach=Mathematik,
    lerngruppe={Q1 GK},
    nummer=20,
    seitenzahlen=keine,
%    loesungen=folgend,
    module={Symbole},
]{schule}

\usepackage[
	kuerzel={Ngb},
	reihe={Analysis III},
	version={2019-05-19},
	hinweise=keine
]{ngbschule}

\author{J. Neugebauer}
\title{Vermischte Übungen zur Klausur}
\date{\Heute}

\setzeAufgabentemplate{ngbohne}

\chead{\Titel}

\usepackage{qrcode}
\qrset{height=4.5cm, nolink, padding}

\begin{document}
	%% Aufgabe 1
	\begin{aufgabe}
		\begin{multicols}{2}
			\operator{Bestimmen} sie die ersten Ableitungen der Funktionen (ohne Verwendung des GTR).
			\begin{enumeratea}
				\item $f_1(x) = 5x^2 + 2e^{4x}$
				\item $f_2(x) = 8x^4 + 6x^3 - e^{0,2x+3,4}$
				\item $f_3(x) = 2,5x^4\cdot e^{3x}$
				\item $f_4(x) = 4x\cdot e^x - cos(2x)$
			\end{enumeratea}
		
			\operator{Bestimmen} sie die eine Stammfunktion der Funktionen (ohne Verwendung des GTR).
			\begin{enumeratea}
				\item $f_5(x) = 2e^{2x}$
				\item $f_6(x) = e^{4x} + 2x^2$
				\item $f_7(x) = 3e^{5x+9,3}$
				\item $f_8(x) = cos(x) - e^{4x} + x^6$
			\end{enumeratea}
		\end{multicols}
	\end{aufgabe}
	\clearpage
	
	%% Aufgabe 2
	\begin{aufgabe}
		\operator{Bestimmen} sie $k$ (ohne Verwendung des GTR).
		\begin{multicols}{2}
			\begin{enumeratea}
				\item \[ \int_0^1\! ke^x - e\d{x} = 0 \]
				\item \[ \int_0^1\! e^x - k\d{x} = 1 \]
				\item \[ \int_{-1}^1\! e^x + kx\d{x} = e \]
				\item \[ \int_0^k\! e^x \d{x} = e + 1 \]
			\end{enumeratea}
		\end{multicols}
	\end{aufgabe}
	\clearpage
	
	%% Hinweise Aufgabe 1
	\begin{center}
		Beachten sie, dass e\textasciicircum 4x gleichbedeutend ist mit $e^{4x}$.
		\begin{multicols}{2}
			Ableitungen \\
			\qrcode{a) f1'(x) = 10x + 8e^4x
			b) f2'(x) = 32x^3 + 18x^2 - 0,2e^(0,2x+3,4)
			c) f3'(x) = 10x^3 * e^(3x) + 2,5x^4 * 3e^(3x)
			d) f4'(x) = x * e^x + 4x * e^x + 2sin(x)
				Erfordert die Anwendung der Produkt-
				und Kettenregel! }
			\columnbreak
			
			Stammfunktionen \\
			\qrcode{a) F5(x) = e^(2x)
			b) F6(x) = 1/4 e^(4x) + 2/3 x^3
			c) F7(x) = 3/5 e^(5x+9,3)
			d) F8(x) = sin(x) - 1/4 e^(4x) + 1/7 x^7}
		\end{multicols}
	\end{center}
	\clearpage
	
	%% Hinweise Aufgabe 2
	\begin{center}
		Beachten sie, dass x\textasciicircum-3 gleichbedeutend ist mit $x^{-3}$.
		\begin{multicols}{3}
			Lösungsansatz \\
			\qrcode{Wenden sie den Hauptsatz an,
			um das Integral umzuformen. Bilden sie 
			dazu die Stammfunktion.}
			\columnbreak
			
			Lösungshinweis \\
			\qrcode{Die Stammfunktion von e^x ist e^x
			Potenzen: e^0 = 1, e^1 = e
			Logarithmus: ln(e^b) = e^ln(b) = b
			Kehrwert: e^-1 = 1/e}
			\columnbreak
		
			Lösungen \\
			\qrcode{a) k = e/(e-1)
			b) k = e-2
			c) k = 1/e
			d) k = ln(e+2)}
		\end{multicols}
	\end{center}
	\clearpage
	
	%% Aufgabe 3
	\begin{aufgabe}
		Das Wachstum einer Wasseralgenart wurde auf einem \SI{20}{\square\meter} großen See gemessen. Es kann durch die Funktion
		\[ f(t) = 4te^{0,02t} \]
		beschrieben werden. Wobei $t$ die Zeit in \si{Tagen} darstellt, und $f(t)$ die belegte Wasserfläche in \si{\square\meter}.
		
		\begin{teilaufgaben}
			\teilaufgabe \operator{Berechnen} sie den Funktionswert nach \SI{1,5}{Tagen} und interpretieren wie das Ergebnis im Sachzusammenhang. 
			\teilaufgabe \operator{Bestimmen} sie die Wachstumsrate der Algen zum Zeitpunkt $t=0$.
			\teilaufgabe Ungefähr wann haben die Algen die Oberfläche des Teichs vollständig belegt?\\
			\hinweis{Nutzen sie den GTR!}
			\teilaufgabe Haben die Algen den Teich vollständig belegt fangen sie mangels Ausbreitungsfläche an, abzusterben. Pro Tag sterben \SI{3}{\%} der Algen ab. \operator{Modellieren} sie die Abnahme mit einer e-Funktion.
		\end{teilaufgaben}
	\end{aufgabe}
	\clearpage

	%% Aufgabe 4
	\begin{aufgabe}
		Ein Fußballderby wird an einem Freitagabend vor 80 645 Zuschauern ausgetragen. Das Spiel beginnt um 20:30 Uhr, die EIngägne werden schon um 18:00 Uhr geöffnet.
		
		Der Andrang der Fans an den Eingängen des Fußballstadions kann näherungsweise durch die Funktion $f$ mit $f(x) = 40x\cdot e^{-0,02x}$ beschrieben werden. Dabei wird $x$ in Minuten seit der Öffnung der Eingänge um 18:00 Uhr und $f(x)$ in Zuschauer pro Minute gemessen.
		
		\begin{teilaufgaben}
			\teilaufgabe Zeichnen sie den Graphen der Funktion im GTR und \operator{beschreiben} sie den Verlauf mit eigenen Worten im Sachzusammenhang.
			\teilaufgabe \operator{Bestimmen} sie den Zeitpunkt, an dem der Zuschauerandrang am größten war. Wie viele Zuschauerinnen und Zuschauer kamen zu diesem Zeitpunkt an den Eingängen an?
			\teilaufgabe Wie viele Personen waren nach diesem Modell um 20:30 Uhr im Stadion?\\
			\hinweis{Nutzen sie den GTR!}
		\end{teilaufgaben}
	\end{aufgabe}
	\clearpage
	
	%% Hinweise Aufgabe 3
	\begin{center}
		Beachten sie, dass x\textasciicircum ln(0,96)t gleichbedeutend ist mit $x^{\ln{0,96}t}$.
		\begin{multicols}{4}
			Teilaufgabe a)\\
			\qrcode[height=3cm]{f(1,5) = 6,1827}
			\columnbreak
			
			Teilaufgabe b)\\
			\qrcode[height=3cm]{f'(t) mit der Produktregel bilden:
			f'(t) = (4 + 0,08t) * e^0,02x
		
			t = 0 einsetzen:
			f'(0) = 4}
			\columnbreak
			
			Teilaufgabe c)\\
			\qrcode[height=3cm]{f(t) = 20
			
			Probieren im GTR ergibt:
			t = 4,56}
			\columnbreak
			
			Teilaufgabe d)\\
			\qrcode[height=3cm]{g(t)
			= 20 * (1-0,03)^t
			= 20 * e^ln(0,97)t
			= 20 * e^-0,0305t}
			\columnbreak
		\end{multicols}
	\end{center}
	\clearpage
	
	%% Hinweise Aufgabe 4
	\begin{center}
		Beachten sie, dass x\textasciicircum ln(0,96)t gleichbedeutend ist mit $x^{\ln{0,96}t}$.
		\begin{multicols}{4}
			Teilaufgabe a)\\
			\qrcode[height=3cm]{Am Graphen der Funktion erkennt man:
			In der ersten halben Stunde nach Oeffnung steigt der Andrang stark an.
			Der Hoechststand ist etwas nach 50 min.
			Danach nimmt der Andrang ab.
			Bei Spielbeginn (x=150) kommen immer noch
			ca. 300 Zuschauer pro Minute an.}
			\columnbreak
			
			Teilaufgabe b) Ableitung bilden\\
			\qrcode[height=3cm]{f'(t) mit der Produktregel bilden:
			f'(t) = 40 * (1 - 0,02x) * e^-0,02x}
			\columnbreak
			
			Teilaufgabe b) Nullstellen\\
			\qrcode[height=3cm]{f'(t) = 0 kann nur erfuellt 
			sein, wenn:
			(1 - 0,02x) = 0
		
			Also x = 50
			
			Hinreichende Bedingung bestaetigt HP}
			\columnbreak
			
			Teilaufgabe c)\\
			\qrcode[height=3cm]{Gesucht ist das
			Integral von 0 bis 150 von f(x) dx
		
			Da wir keine Regel kennen, um die Stamm-
			funktion zu bilden, nutzen wir den GTR.
			
			Loesung: ca. 80085 Zuschauer}
			\columnbreak
		\end{multicols}
	\end{center}
\end{document}