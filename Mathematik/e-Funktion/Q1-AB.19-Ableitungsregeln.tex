\documentclass[10pt, a5paper, landscape]{scrartcl}

\usepackage{vorschule}
\usepackage[
    typ=ab,
    fach=Mathematik,
    lerngruppe={Q1 GK},
    nummer=19,
    module={Symbole,Lizenzen},
    seitenzahlen=keine,
    farbig,
    lizenz=cc-by-nc-sa-4,
]{schule}

\usepackage[
	kuerzel={Ngb},
	reihe={Analysis III},
	version={2019-05-11},
]{ngbschule}

\author{J. Neugebauer}
\title{Ableitungsregeln}
\date{\Heute}

\setzeAufgabentemplate{ngbohne}

\begin{document}
	\begin{multicols}{2}
		\paragraph{Die Produktregel:}
		\begin{align*}
			f(x) &= u(x)\cdot v(x) \\
			f'(x) &= u'(x)\cdot v(x) + u(x)\cdot v'(x)
		\end{align*}
		
		\paragraph{Beispiele:}
		\begin{align*}
		f_1(x)  &= 2x^2\cdot e^x \\
		f_1'(x) &= 4x\cdot e^x + 2x^2\cdot e^x\\[1em]
		f_2(x)  &= \sin{x}\cdot 4x^3 \\
		f_2'(x) &= \cos{x}\cdot 4x^3 + \sin{x}\cdot 12x^2\\[1em]
		f_3(x)  &= 3e^x\cdot 7x^4 \\
		f_3'(x) &= 
		\end{align*}
	\end{multicols}

	Aufgaben:
	\begin{enumeraten}
		\item[\Large\symEinzel] Studieren sie zunächst alleine die Ableitungsregel und die Beispiele.
		\item[\Large\symPartner] Tauschen sie sich mit ihrer Sitznachbarin über die Regel aus und klären sie Fragen. Ermitteln sie gemeinsam die Ableitung $f_3'(x)$. Finden sie gemeinsam ein weiteres Beispiel für die \emph{Produktregel}.
		\item[\Large\symGruppe] Setzen sie sich mit einer Gruppe zusammen, die die \emph{Kettenregel} behandelt hat und erklären sie sich gegenseitig ihre Ableitungsregeln.
	\end{enumeraten}
	
	\clearpage
	\begin{multicols}{2}
		\paragraph{Die Kettenregel:}
		\begin{align*}
		f(x) &= u(v(x)) \\
		f'(x) &= v'(x)\cdot u'(v(x))
		\end{align*}
		
		\paragraph{Beispiele:}
		\begin{align*}
		f_1(x)  &= \sin{2x^3} \\
		f_1'(x) &= 6x^2\cdot \cos{2x^3}\\[1em]
		f_2(x)  &= e^{2x+2} \\
		f_2'(x) &= 2\cdot e^{2x+2}\\[1em]
		f_3(x)  &= e^{2x^2} \\
		f_3'(x) &= 
		\end{align*}
	\end{multicols}
	
	Aufgaben:
	\begin{enumeraten}
		\item[\Large\symEinzel] Studieren sie zunächst alleine die Ableitungsregel und die Beispiele.
		\item[\Large\symPartner] Tauschen sie sich mit ihrer Sitznachbarin über die Regel aus und klären sie Fragen. Ermitteln sie gemeinsam die Ableitung $f_3'(x)$. Finden sie gemeinsam ein weiteres Beispiel für die \emph{Kettenregel}.
		\item[\Large\symGruppe] Setzen sie sich mit einer Gruppe zusammen, die die \emph{Produktregel} behandelt hat und erklären sie sich gegenseitig ihre Ableitungsregeln.
	\end{enumeraten}
\end{document}