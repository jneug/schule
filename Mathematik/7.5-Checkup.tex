\documentclass[11pt, a4paper]{scrartcl}

\usepackage{vorschule}
\usepackage[
    typ=ab,
    fach=Mathematik,
    lerngruppe={Jg.7},
    nummer={5},
    module={Symbole,Lizenzen},
    lizenz=cc-by-nc-sa-4,
]{schule}

\usepackage[
	typ=checkup,
	kuerzel=Ngb,
	reihe={Terme und Gleichungen},
	version={2019-04-09}
]{ngbschule}

\chead{\Titel am 03.05.2019}

\author{J. Neugebauer}
\title{5. Mathearbeit}
\date{\Heute}

\begin{document}
\CheckupBild\CheckupTitel

Kreuze jeweils an, wie sicher du dich bei den einzelnen \textbf{Themenschwerpunkten} fühlst (von \enquote{sehr sicher} \usym{1F604} bis \enquote{sehr unsicher} \usym{1F641}). Nutze die \textbf{Aufgaben und Informationen} zum Wiederholen und Lernen von Themen, \emph{bei denen du noch unsicher bist}.

Die \textbf{Lösungen} für die \enquote{Bist du sicher?} Aufgaben und \enquote{Selbsttrainings} findest du hinten im Buch ab Seite 235. Auf Seite 142 sind alle Informationen im \enquote{Rückblick} zusammengefasst.

Das \textbf{Arbeitsheft} bietet sich an, um selbstständig weitere Aufgaben zum Thema zu üben.

\begin{checkup}
	\ichkann{\dots mit Brüchen rechnen (addieren, subtrahieren, multiplizieren und dividieren).}{
		Zum Teil \bu{225}{}\\
		\bu{233}{16}
	}
	\ichkann{\dots Terme mit maximal einer Variablen aufstellen.}{
		\bu{112/113}{} \\
		\bu{143}{2/3} \\
		\bu{233}{11} \\
		AB \enquote{Die 7b bastelt}
	}
	\ichkann{\dots den Wert eines Terms berechnen.}{
		\bu{118}{2} \\
		\bu{233}{13} \\
		Siehe auch Kapitel III
	}
	\ichkann{\dots Terme umformen, um sie zu vereinfachen.}{
		\bu{116/117}{} \\
		\bu{120/121}{} \\
		\bu{118}{1/2} \\
		\bu{122}{2}\\
		\bu{143}{6}
	}
	\ichkann{\dots Gleichungen auf Äquivalenz prüfen.}{
		\bu{125/126}{} \\
		\bu{118}{1} \\
		\bu{122}{1} \\
		\bu{128}{2}\\
		\bu{143}{4}
	}
	\ichkann{\dots die Lösung für Gleichungen mit einer Variablen bestimmen.}{
		\bu{125/126}{} \\
		\bu{128}{1}\\
		\bu{143}{5}\\
		\bu{143}{5}\\
		\bu{233}{17}
	}
\end{checkup}

\end{document}
