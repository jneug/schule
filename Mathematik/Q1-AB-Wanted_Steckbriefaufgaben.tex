\documentclass[11pt, a4paper, landscape, twocolumn]{scrartcl}

\usepackage{vorschule}
\usepackage[
    typ=ab,
    fach=Mathematik,
    lerngruppe={Q1 GK},
    nummer=11,
    loesungen=folgend,
    seitenzahlen=keine,
    zitate=quotes,
    module={Symbole},
]{schule}

\usepackage[
	kuerzel=Ngb,
	reihe={Analysis},
	version=1,
]{ngbschule}

\author{J. Neugebauer}
\title{WANTED Steckbriefaufgaben}
\date{\Heute}

\setzeAufgabentemplate{ngbohne}

\begin{document}
	\TITEL
	
	\begin{aufgabe}
		\begin{description}\itemsep 2em
			\item[Ganove A]
			Die Zeugin Ana L. hat folgendes Phantombild gezeichnet:
			\begin{center}
				\includegraphics{Q1-Abb_Ganove_A.jpg}
			\end{center}
		
			\item[Ganove B]
			Er hat in den Punkten $A(0|0)$, $B(1|1)$ und $C(2|3)$ Verbrechen begangen!
			
			\item[Ganove C]
			Ganove C scheint 2. Grades zu sein. Nach einer Nullrunde bei $x=4$ wurde das
			maximale Verbrechen bei $x=2$ begangen. Zudem wurde der Ganove im Punkt
			$P(3|1,5)$ beobachtet.
		\end{description}
	
		\begin{rahmen}
			Finden sie jeweils die ganzrationale Funktion, die zu den Steckbriefen passt.
			Die gesuchten Funktionen haben jeweils den geringsmöglichen Grad.
		\end{rahmen}
		\bigskip

		\begin{loesung}
			\begin{description}
				\item[Ganove A]
				Laut Abbildung besitzt die Funktion (mindestens) vier Nullstellen. Außerdem (mindestens) drei Extrempunkte. Die Funktion hat also mindestens Grad vier:
				
				\[ f_A(x) = ax^4 + bx^3 + cx^2 + dx + e \]
				
				Wir können aber weitere Beobachtungen der Funktionsuntersuchung nutzen, um den 
				Funktionsterm zu vereinfachen:\\
				Da die Funktion sehr wahrscheinlich an der y-Achse gespiegelt ist (nur gerade Exponenten), 
				hat die Funktion vermutlich die Form:

				\[ f_A(x) = ax^4 + cx^2 + e \]
				
				Es werden also drei Gleichungen benötigt, da in der Funktion drei Unbekannte zu bestimmen sind. Wir können aus der Abbildung mehrere Punkte mit großer Sicherheit ablesen. zum Beispiel an den Stellen $x = 0$ und $x = 2$. An $x = 2$ befindet sich auch ein Extremum, das bedeutet $f'_A(2) = 0$. Diese drei Informationen ergeben (zusammen mit $f'_A(x) = 4ax^3 + 2cx$):
				
				\begin{align*}
				f_A(2) = -3  &\Leftrightarrow a\cdot 2^4 + c\cdot 2^2 + e = -3 \\
				f'_A(2) = 0 &\Leftrightarrow 4a\cdot 2^3 + 2c\cdot 2 = 0 \\
				f_A(0) = 1   &\Leftrightarrow a\cdot 0^4 + c\cdot 0^2 + e = 1
				\end{align*}
				
				\textit{Es lassen sich noch andere Punkte aus der Abbildung entnehmen und für das 
				Aufstellen eines Gleichungssystems nutzen. Die Lösung wird beispielhaft an den oben 
				gewählten Gleichungen gezeigt.}
				
				Daraus lässt sich die erweiterte Koeffizientenmatrix aufstellen und mit dem Gauß-Algorithmus lösen:
				\[ \begin{gmatrix}[p]
				16 & 4 & 1 & -3 \\
				32 & 4 & 0 & 0 \\
				0 & 0 & 1 & 1
				\rowops
				\mult{0}{-1}
				\add{0}{1}
				\end{gmatrix} \]
				\[ \begin{gmatrix}[p]
				16 & 4 & 1 & -3 \\
				0 & -4 & -2 & 6 \\
				0 & 0 & 1 & 1
				\end{gmatrix} \]
				
				Einsetzen von unten nach oben ergibt:
				\begin{align*}
				e &= 1 \\
				-4c - 2e = -4c - 2 &= 6 \\
				&\Leftrightarrow c = -2 \\
				16a + 4c + e = 16a - 8 + 1 &= -3 \\
				&\Leftrightarrow a = 0,25
				\end{align*}
				
				Die gesuchte Funktion ist also:
				\begin{align*}
				f_A(x) &= 0,25x^4 - 2x^2 + 1 \\
				f'_A(x) &= x^3 - 4x
				\end{align*}
				
				Probe:
				\begin{align*}
				f_A(2) = 0,25\cdot 2^4 - 2\cdot 2^2 + 1 = -3\ \checkmark \\
				f'_A(2) = 2^3 - 4\cdot 2 = 0\ \checkmark \\
				f_A(0) = 0,25\cdot 0^4 - 2\cdot 0^2 + 1 = 1\ \checkmark
				\end{align*}
				
				\item[Ganove B]
				Da die drei Punkte nicht auf einer Geraden liegen können (ggf. im GTR betrachten!), muss die gesuchte Funktion mindestens Grad 2 haben. Also:
				
				\[ f_B(x) = ax^2 + bx + c \]
				
				Es werden also drei Gleichungen benötigt, da in der Funktion drei Unbekannte zu bestimmen sind. Zum Glück sind drei Punkte gegeben, die wir nutzen können:
				
				\begin{align*}
					f_B(0) = 0 &\Leftrightarrow a\cdot 0^2 + b\cdot 0 + c = 0 \\
					f_B(1) = 1 &\Leftrightarrow a\cdot 1^2 + b\cdot 1 + c = 1 \\
					f_B(2) = 3 &\Leftrightarrow a\cdot 2^2 + b\cdot 2 + c = 3
				\end{align*}
				
				Daraus lässt sich die erweiterte Koeffizientenmatrix aufstellen und mit dem Gauß-Algorithmus lösen:
				\[ \begin{gmatrix}[p]
				0 & 0 & 1 & 0 \\ 
				1 & 1 & 1 & 1 \\
				4 & 2 & 1 & 3
				\rowops
				\swap{0}{2}
				\end{gmatrix} \]
				\[ \begin{gmatrix}[p]
				1 & 1 & 1 & 1 \\
				4 & 2 & 1 & 3 \\
				0 & 0 & 1 & 0 
				\rowops
				\mult{0}{\cdot -4}
				\add{0}{1}
				\end{gmatrix} \]
				\[ \begin{gmatrix}[p]
				1 & 1 & 1 & 1 \\
				0 & -2 & -3 & -1 \\
				0 & 0 & 1 & 0 
				\end{gmatrix} \]
				
				Einsetzen von unten nach oben ergibt:
				\begin{align*}
				c &= 0 \\
				-2b - 3c = -2b &= -1 \\
				&\Leftrightarrow b = 0,5 \\
				a + b + c = a + 0,5 &= 1 \\
				&\Leftrightarrow a = 0,5
				\end{align*}
				
				Die gesuchte Funktion ist also:
				\[ f_B(x) = 0,5x^2 + 0,5x \]
				
				Probe:
				\begin{align*}
				f_B(0) = 0,5\cdot 0 + 0,5\cdot 0 &= 0 \ \checkmark \\
				f_B(1) = 0,5\cdot 1 + 0,5\cdot 1 &= 1 \ \checkmark \\
				f_B(2) = 0,5\cdot 2^2 + 0,5\cdot 2 = 2 + 1 &= 3 \ \checkmark
				\end{align*}
				
				\item[Ganove C]
				Die gesuchte Funktion hat laut Aufgabenstellung den Grad 2. Also:
				
				\[ f_C(x) = ax^2 + bx + c \]
				
				Es werden also drei Gleichungen benötigt, da in der Funktion drei Unbekannte zu bestimmen sind. Ein Punkt und eine Nullstelle sind direkt gegeben:
				\begin{align*}
					f_C(3) = 1,5 &\Leftrightarrow a\cdot 3^2 + b\cdot 3 + c = 1,5 \\
					f_C(4) = 0 &\Leftrightarrow a\cdot 4^2 + b\cdot 4 + c = 0
				\end{align*}
				
				Es ist eine weitere Gleichung zu finden. Dabei hilft die Information, dass das \enquote{maximale Verbrechen} bei $x=2$ begangen wurde. Dies deutet auf einen Hochpunkt hin. Wir wissen (laut der hinreichenden Bedingung für Extrempunkte), dass die erste Ableitung an dieser Stelle null sein muss. Also:
				\begin{align*}
					f'_C(x) &= 2ax + b \\
					f'_C(2) &= 0 \Leftrightarrow 2a\cdot 2^2 + b = 0
				\end{align*}
				
				Nun lässt sich die erweiterte Koeffizientenmatrix aufstellen und mit dem Gauß-Algorithmus lösen:
				\[ \begin{gmatrix}[p]
				9 & 3 & 1 & 1,5 \\ 
				16 & 4 & 1 & 0 \\
				8 & 1 & 0 & 0
				\rowops
				\mult{0}{\cdot -1}
				\add{0}{1}
				\end{gmatrix} \]
				\[ \begin{gmatrix}[p]
				9 & 3 & 1 & 1,5 \\ 
				7 & 1 & 0 & -1,5 \\
				8 & 1 & 0 & 0
				\rowops
				\mult{1}{\cdot -1}
				\add{1}{2}
				\end{gmatrix} \]
				\[ \begin{gmatrix}[p]
				9 & 3 & 1 & 1,5 \\ 
				7 & 1 & 0 & -1,5 \\
				1 & 0 & 0 & 1,5
				\end{gmatrix} \]
				
				\textit{Beachten sie, dass hier die Koeffizientenmatrix auch in Dreiecksgestalt ist, auch wenn das Dreieck aus Nullen diesmal nicht unten links, sondern unten rechts zu finden ist!
				Wichtig ist, dass in der untersten Zeile genau ein Eintrag der ersten drei ungleich null ist,
				in der vorletzten Zeile maximal zwei ungleich null und in der ersten maximal drei.}
			
			
				Einsetzen von unten nach oben ergibt:
				\begin{align*}
				a &= 1,5 \\
				7a + 1b = 10,5 + b &= -1,5 \\
				&\Leftrightarrow b = -12 \\
				9a + 3b + 1c = 13,5 - 36 + c &= 1,5 \\
				&\Leftrightarrow c = 24
				\end{align*}
				
				Die gesuchte Funktion ist also:
				\begin{align*}
					f_C(x) &= 1,5x^2 - 12x + 24 \\
					f'_C(x) &= 3x - 12
				\end{align*}
				
				Probe:
				\begin{align*}
				f_C(3) = 1,5\cdot 3^2 - 12\cdot 3 + 24 &= 1,5 \ \checkmark \\
				f_C(4) = 1,5\cdot 4^2 - 12\cdot 4 + 24 &= 0 \ \checkmark \\
				f'_C(2) = 3\cdot 2^2 - 12 &= 0 \ \checkmark
				\end{align*}
			\end{description}
		\end{loesung}
	\end{aufgabe}
\end{document}