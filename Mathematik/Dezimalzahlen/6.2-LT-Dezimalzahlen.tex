\documentclass[12pt,a5paper,landscape]{scrartcl}

\usepackage{vorschule}
\usepackage[
    typ=ohne,
    fach=Mathematik,
    lerngruppe={6},
	nummer=2,
    module={Lizenzen,Symbole},
	seitenzahlen=keine,
	farbig,
    lizenz=cc-by-nc-sa-4,
]{schule}

\usepackage[
	typ=lerntheke,
	kuerzel=Ngb,
	reihe={Dezimalzahlen},
	version={2020-02-12},
]{ngbschule}

\author{J. Neugebauer}
\title{Dezimalzahlen}
\date{\Heute}

\begin{document}

	\begin{hilfekarte}{Dezimalzahlen}{dezimalzahlen}

	\end{hilfekarte}
	
	\begin{hilfekarte}{Brüche umwandeln}{umwandeln}
		Manchmal ist es wichtig zu wissen, wie ein Bruch in einen Dezimalbruch umgewandelt werden kann. Maik bestellt bei einem Metzger $\tSI{1}{4}{\kilo\gram}$ Hackfleisch, wie viel Kilogramm sind das? 
		
		Um dies zu lösen kannst du zwei Methoden verwenden.
		\begin{enumerate}
			\item Erweitere oder kürze den Nenner auf 10, 100, 1000 usw. Beispiel:
			
			\[ \dfrac{1}{4} = \dfrac{25	}{100} = 0,25 \]
			\item Wandle einen Bruch durch schriftliche Division in einen Dezimalbruch um. Beispiel:
			
			\[ \dfrac{1}{4} = \longdivision{1}{4} \]
		\end{enumerate}
	\end{hilfekarte}
			
	\begin{hilfekarte}{Abbrechende und periodische Dezimalzahlen}{abbrechend-und-periodisch}

	\end{hilfekarte}
	
	\begin{hilfekarte}{Runden und überschlagen}{runden}
		Gerade beim Einkaufen ist es hilfreich, dass du mit gerundeten Zahlen schnell die ungefähre Summe \textbf{überschlagen} kann. Rundet dabei immer so, dass du leicht im Kopf rechnen kannst.
		
		Runden von Dezimalbrüchen:
		\begin{smallenumerate}
			\item Lege die Stelle fest, auf die gerundet werden soll (Zehntel, Hundertstel, Tausendstel).
			\item Betrachte die nächstfolgende Ziffer:
			\begin{smallitemize}
				\item Ist die Ziffer 0, 1, 2, 3 oder 4, so \textbf{rundest} du \textbf{ab}.
				\item Ist die Ziffer 5, 6, 7, 8 oder 9, so \textbf{rundest} du \textbf{auf}.
			\end{smallitemize}
		\end{smallenumerate}
		
		\begin{wrapfig}
			\begin{wrapfigure}[12]{r}{0pt}		
				\includegraphics[width=3cm]{6.2-LT-Abb_Kassenzettel_0}
			\end{wrapfigure}
			\textbf{Beispiel:} Sonja kauft für ihr Meerschweinchen ein. An der Kasse stellt sie fest, dass sie nur \SI{11}{ Euro} bei sich hat. Um zu überschlagen, ob das Geld reicht, rundet sie die Beträge auf und rechnet $2+3+6=11$. Sie weiß nun, dass sie etwas weniger als \EUR{11} zahlen muss.
		\end{wrapfig}
	\end{hilfekarte}
	
	\begin{hilfekarte}{Dezimalzahlen ordnen}{ordnen}

	\end{hilfekarte}
	
	\begin{hilfekarte}{Addition und Subtraktion}{addition-und-subtraktion}
		\begin{wrapfigure}{r}{0pt}		
			\includegraphics[height=.45\textheight]{6.2-LT-Abb_Addition}
		\end{wrapfigure}
		
		
		\includegraphics[height=.45\textheight]{6.2-LT-Abb_Subtraktion}
	\end{hilfekarte}
	
	\begin{hilfekarte}{Multiplikation und Division}{multiplikation-und-division}

	\end{hilfekarte}
	
	
	\begin{karte1}{Überschlagrechnung}\hilfeMarke{runden}
	Im Alltag muss man oft Kosten abschätzen. Wenn es schnell gehen soll, kann man eine \emph{Überschlagsrechnung} durchführen.

	\smallskip
	\textbf{Beispiel:} statt \EUR{5,10} + \EUR{12,98} + \EUR{7,90} rechne \EUR{5} + \EUR{13} + \EUR{8} = \EUR{26} also \EUR{5,10} + \EUR{12,98} + \EUR{7,90} $\approx$ \EUR{26}

	\smallskip
	Überschlage und runde auf ganze Euro.
	\begin{enumeratea}
		\item \EUR{2,98} + \EUR{17,05} + \EUR{4,95} $\approx$ \linie[1cm]
		\item \EUR{14,90} - \EUR{6,10} - \EUR{2,95} $\approx$ \linie[1cm]
		\item \EUR{23,89} - \EUR{8,90} + \EUR{3,99} $\approx$ \linie[1cm]
		\item 3$\cdot$ \EUR{4,99} + 7$\cdot$ \EUR{3,19} $\approx$ \linie[1cm]
	\end{enumeratea}
	\end{karte1}
	
	\begin{loesungskarte}
	\begin{enumeratea}
		\item \EUR{2,98} + \EUR{17,05} + \EUR{4,95} $\approx$ \EUR{3} + \EUR{17} + \EUR{5} = \EUR{25}
		\item \EUR{14,90} - \EUR{6,10} - \EUR{2,95} $\approx$ \EUR{15} - \EUR{6} - \EUR{3} = \EUR{6}
		\item \EUR{23,89} - \EUR{8,90} + \EUR{3,99} $\approx$ \EUR{24} - \EUR{9} + \EUR{4} = \EUR{19}
	\end{enumeratea}
	\end{loesungskarte}
	
	\begin{karte1}{Addieren und Subtrahieren}
	\end{karte1}
	
	\begin{loesungskarte}
	\end{loesungskarte}
	
	\begin{karte1}{Multiplizieren und Dividieren}
	\end{karte1}
	
	\begin{loesungskarte}
	\end{loesungskarte}
	
	\begin{karte2}[\symUhr]{Rechnen auf Zeit}
		\begin{wrapfig}
			\begin{wrapfigure}[4]{r}{0pt}
				\includegraphics[width=2cm]{6.2-LT-Abb_Stoppuhr}
			\end{wrapfigure}
			Berechne die Lösung der Aufgaben \emph{möglichst schnell}. Stopp deine Zeit mit einer Uhr oder deinem Handy. Schaffst du all Aufgaben in 10 Minuten?
		\end{wrapfig}
		
		\begin{tasks}(2)
			\task $5,62\cdot  3,42$
			\task $(12,23 - 10,22)\cdot 4,1$
			\task $17,433 + 5,2\cdot 3,41$
			\task $2,35 : 0,5 + 1,3$
			\task $\dfrac{3}{5} + 3,46 - \dfrac{1}{10}$
			\task $4,55 + \dfrac{1}{20}- 0,4$
		\end{tasks}
		
		\infotext{\enquote{Chronometer} von \url{http://schulbilder.org}}
	\end{karte2}
	
	\begin{loesungskarte}
		\begin{tasks}(2)
			\task $19,2204$
			\task $8,241$
			\task $35,165$
			\task $6$
			\task $3,96$
			\task $4,2$
		\end{tasks}
	\end{loesungskarte}
	
	\begin{karte1}{Kontoauszüge}
		Berechne das grau markierte Feld im Kontoauszug.
		
		\begin{multicols}{2}\footnotesize\centering
		
		\begin{tabular}{|l|l|r|} \hline
		\multicolumn{2}{|l}{\small\textbf{Sparkasse Ranstadt}} & Kontonr.: 1487523 \\ \hline
		\multicolumn{2}{|r|}{\texttt{Alter Kontostand:}} & +\color{green!50!black}\EUR{52,71} \\ \hline
		\textbf{Datum} & \textbf{Art} & \textbf{Betrag} \\ \hline
		2.2.19 & Gehalt & +\color{green!50!black}\EUR{2012,54} \\ \hline
		3.2.19 & Einzahlung & +\color{green!50!black}\EUR{124,95} \\ \hline
		5.2.19 & Zinsgutschrift & +\color{green!50!black}\EUR{741,69} \\ \hline
		\multicolumn{2}{|r|}{\texttt{Neuer Kontostand:}} & \cellcolor{lightgray} \\ \hline
		\end{tabular}
		
		\medskip
		\begin{tabular}{|l|l|r|} \hline
		\multicolumn{2}{|l}{\small\textbf{Sparkasse Hungen}} & Kontonr.: 657485420 \\ \hline
		\multicolumn{2}{|r|}{\texttt{Alter Kontostand:}} & +\color{green!50!black}\EUR{2874,11} \\ \hline
		\textbf{Datum} & \textbf{Art} & \textbf{Betrag} \\ \hline
		12.3.19 & Miete & -\color{red}\EUR{1255,68} \\ \hline
		14.3.19 & Bausparen & -\color{red}\EUR{457,21} \\ \hline
		16.3.19 & Krankenversicherung & -\color{red}\EUR{298,61} \\ \hline
		20.3.19 & Überweisung & -\color{red}\EUR{317,04} \\ \hline
		\multicolumn{2}{|r|}{\texttt{Neuer Kontostand:}} & \cellcolor{lightgray} \\ \hline
		\end{tabular}
		
		\columnbreak
		
		\begin{tabular}{|l|l|r|} \hline
		\multicolumn{2}{|l}{\small\textbf{Sparkasse Friedberg}} & Kontonr.: 25479632 \\ \hline
		\multicolumn{2}{|r|}{\texttt{Alter Kontostand:}} & +\color{green!50!black}\EUR{3457,54} \\ \hline
		\textbf{Datum} & \textbf{Art} & \textbf{Betrag} \\ \hline
		2.2.19 & Miete & -\color{red}\EUR{874,53} \\ \hline
		4.2.19 & Bausparen & -\color{red}\EUR{165,23} \\ \hline
		4.2.19 & Krankenversicherung & -\color{red}\EUR{456,87} \\ \hline
		\multicolumn{2}{|r|}{\texttt{Neuer Kontostand:}} & \cellcolor{lightgray} \\ \hline
		\end{tabular}
		
		\medskip
		\begin{tabular}{|l|l|r|} \hline
		\multicolumn{2}{|l}{\small\textbf{Sparkasse Borgstedt}} & Kontonr.: 7734185 \\ \hline
		\multicolumn{2}{|r|}{\texttt{Alter Kontostand:}} & +\color{green!50!black}\EUR{1,25} \\ \hline
		\textbf{Datum} & \textbf{Art} & \textbf{Betrag} \\ \hline
		1.5.19 & Gehalt & +\color{green!50!black}\EUR{3125,47} \\ \hline
		1.5.19 & Miete & -\color{red}\EUR{891,00} \\ \hline
		2.5.19 & Strom und Wasser & -\color{red}\EUR{259,72} \\ \hline
		5.5.19 & Verkauf Fahrrad & +\color{green!50!black}\EUR{477,50} \\ \hline
		\multicolumn{2}{|r|}{\texttt{Neuer Kontostand:}} & \cellcolor{lightgray} \\ \hline
		\end{tabular}
		
		\end{multicols}
		
		\loesung{R: \EUR{2931,89}, F: \EUR{1960,91}, H: \EUR{545,57}, B: \EUR{2453,50}}
	\end{karte1}

	\leereKarte
%	\begin{loesungskarte}
%		\begin{itemize}
%			\item \textbf{Sparkasse Ranstadt}: \lsgText{\EUR{2931,89}}
%			\item \textbf{Sparkasse Friedberg}: \lsgText{\EUR{1960,91}}
%			\item \textbf{Sparkasse Hungen}: \lsgText{\EUR{545,57}}
%			\item \textbf{Sparkasse Borgstedt}: \lsgText{\EUR{2094,34}}
%		\end{itemize}
%	\end{loesungskarte}

	\begin{karte1}{Kassenzettel}
		\rotatebox{90}{\includegraphics[height=\textwidth]{6.2-LT-Abb_Kassenzettel_1}}
		
		Auf dem Kassenzettel oben wurde der zu zahlende Betrag verwischt. Berechne die Summe des Einkaufs.
		
		\loesung{\EUR{19,94}}
	\end{karte1}

	\leereKarte

	\begin{karte2}{Kassenzettel 2}
		\begin{center}
			\rotatebox{90}{\includegraphics[height=\textheight]{6.2-LT-Abb_Kassenzettel_2}}
		\end{center}
		
		Auf dem Kassenzettel oben sind zwei Werte verwischt worden. Kannst du die fehlenden Zahlen herausfinden?
		
		\loesung{\EUR{2,79} und \EUR{2,19}}
	\end{karte2}
	
	\leereKarte

	\begin{karte3}[\symPartner]{Zahlen raten}
		Such dir für diese Karte einen Partner.
		
		\begin{multicols}{2}
			\begin{enumerate}
				\item Überlegt euch jeder \textbf{sechs} unterschiedliche Dezimalzahlen und schreibt sie auf einen Zettel.	
				\item Addiert jeder \textbf{drei} der Dezimalzahlen zusammen und notiert die Summe auf einem anderen Zettel.
				\item Tauscht beide Zettel mit eurem Partner.
				\item Versucht die drei Zahlen zu ermitteln, die zusammen die Summe ergeben.
			\end{enumerate}
			
			\columnbreak
			
			\begin{center}
				\includegraphics[width=6cm]{6.2-LT-Abb_Rechnen}
			\end{center}
		\end{multicols}
		
		\infotext{\enquote{Rechnen} von \url{http://schulbilder.org}}
	\end{karte3}
	
	\leereKarte
	
	\begin{karte2}{Netzwerkkabel}
	content
	\end{karte2}

\end{document}
