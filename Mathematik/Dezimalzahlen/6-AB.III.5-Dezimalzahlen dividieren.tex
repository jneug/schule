\documentclass[10pt, a4paper]{scrartcl}

\usepackage{vorschule}
\usepackage[
    typ=ab,
    fach=Mathematik,
    lerngruppe={Jg.6},
    nummer=III.5,
    module={Lizenzen,Symbole,Papiertypen},
   	seitenzahlen=keine,
   	farbig,
   	lizenz=cc-by-nc-sa-4,
]{schule}

\usepackage[
	kuerzel=Ngb,
	reihe={Mit Brüchen und Dezimalzahlen rechnen},
	version={2020-01-29},
]{ngbschule}

\author{J. Neugebauer}
\title{Dezimalzahlen dividieren}
\date{\Heute}

\setzeAufgabentemplate{ngbnormal}


\begin{document}
	\ReiheTitel
	
	\begin{aufgabe}[symbol=\symPartner] 
		\label{aufg:1}Findet die Rechnungen mit demselben Ergebnis und markiert sie.
		
		\begin{center}
		\begin{tikzpicture}[xscale=2]
			\node at (0,0.2){$50:20$};
			\node at (2,0){$5000:2000$};
			\node at (5,0.8){$2:5$};
			\node at (1,1.5){$20000:5000$};
			\node at (3.2,1.6){$200:50$};
			\node at (5.5,1.8){$0,5:0,2$};
			\node at (0.5,2.4){$2000:500$};
			\node at (1.9,2.5){$2000:5000$};
			\node at (2.8,3){$5:2$};
			\node at (4,3.2){$0,2:0,05$};
			\node at (-.5,4){$20:50$};
			\node at (1.3,3.9){$500:200$};
			\node at (1.4,4.6){$2:0,5$};
			\node at (2.5,4.2){$20:5$};
			\node at (4.2,4.7){$200:500$};
			\node at (5.1,3.9){$0,2:0,5$};
		\end{tikzpicture}
		\end{center}
		
		\tipp{Formt die Rechnungen in Brüche um.}
	\end{aufgabe}

	\begin{aufgabe}[symbol=\symPartner]
		Betrachtet die Rechnungen und Ergebnisse aus \prettyref{aufg:1}. Könnt ihr Regelmäßigkeiten erkennen?\smallskip
		
		Formuliert eine Regel, wann das Ergebnis verschiedener \emph{Divisionen von Dezimalzahlen}, gleich ist.\smallskip
		
		\feldLin[1cm]{4}
		
		\hinweis{Achtet auf die Position des Kommas.}
	\end{aufgabe}

	\begin{aufgabe}[symbol=\symPartner]
		Notiert vier Divisionen, die dasselbe Ergebnis haben.
		
		\feldLin[1cm]{3}
	\end{aufgabe}
	
	\begin{aufgabe}[symbol=\symGruppe]
		Diskutiert gemeinsam: Wie könnt ihr die Regel oben ausnutzen, um Dezimalzahlen zu dividieren? Formuliert eine Regel.
				
		\feldLin[1cm]{3}
	\end{aufgabe}
\end{document}