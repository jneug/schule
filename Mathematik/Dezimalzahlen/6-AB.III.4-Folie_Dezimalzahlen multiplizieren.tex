\documentclass[11pt, a4paper]{scrartcl}

\usepackage{vorschule}
\usepackage[
    typ=folie,
    fach=Mathematik,
    lerngruppe={Jg.6},
    nummer=III.4,
    module={Lizenzen,Symbole,Papiertypen},
   	seitenzahlen=keine,
   	farbig,
   	lizenz=cc-by-nc-sa-4,
]{schule}

\usepackage[
	kuerzel=Ngb,
	reihe={Mit Brüchen und Dezimalzahlen rechnen},
	version={2020-01-29},
]{ngbschule}

\author{J. Neugebauer}
\title{Dezimalzahlen multiplizieren}
\date{\Heute}

\begin{document}
	\LARGE

	Hannes soll in seinem Kinderzimmer einen neuen Teppich bekommen. Er hat ausgemessen, dass sein Zimmer \SI{4,5}{\meter} breit und \SI{3,6}{\meter} lang ist. Ein Quadratmeter des Teppichs, den er sich ausgesucht hat, kostet \EUR{4,99}.
	
	\begin{center}
	\includegraphics{6-AB.III.4-Abb_Renovieren}
	
	\textbf{Wie viel kostet der neue Teppich insgesamt?}
	\end{center}
	
\end{document}