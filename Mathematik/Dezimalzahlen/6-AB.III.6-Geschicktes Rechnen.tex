\documentclass[10pt, a5paper, landscape]{scrartcl}

\usepackage{vorschule}
\usepackage[
	typ=ab,
	fach=Mathematik,
	lerngruppe={6},
	nummer=III.6,
	module={Symbole,Lizenzen,Papiertypen},
	seitenzahlen=keine,
	farbig,
	lizenz=cc-by-nc-sa-4,
]{schule}

\usepackage[
	kuerzel=Ngb,
	reihe={Dezimalzahlen},
	version={2020-02-26},
]{ngbschule}

\author{J. Neugebauer}
\title{Geschicktes Rechnen}
\date{\Heute}

\setzeAufgabentemplate{ngbnormal}


\begin{document}
\ReiheTitel

Sina macht Himbeerbowle. Nach Rezept braucht sie $\SI{0,25}{\liter}$ Himbeerpüree, $\SI{0,4}{\liter}$ Cranberry-Nektar, $\SI{0,33}{\liter}$ Apfelsaft, $\SI{0,67}{\liter}$ Himbeersirup und $\SI{0,6}{\liter}$ Holunder-Bionade. Sie überlegt, wie viel Himbeer-Bowle sie nach diesem Rezept erhält und rechnet 
\[ 0,25 + 0,4 + 0,33 + 0,67 + 0,6 = 0,65 + 0,33 + 0,67 + 0,6 = 0,98 + 0,67 + 0,6 = 1,65 + 0,6 = 2,25\ (\si{\liter}) \]

Ihr Freund Leo fragt: \enquote{Warum hast du nicht zuerst die Zahlen vertauscht und Klammern gesetzt?}

\subsubsection*{Aufgabe 1}
Warum ist Leo Vorschlag geschickter als Sina Rechnung? Erinnert euch gemeinsam an das \emph{Kommutativ-} und das \emph{Assoziativgesetz} und erklärt, wie es hier angewandt werden kann.

\subsubsection*{Aufgabe 2}
Formuliert eine Regel, wann es geschickt ist, zu vertauschen (Kommutativgesetz) oder Klammern zu setzen (Assoziativgesetz) wenn man mit Dezimalzahlen und Brüchen rechnet.

\clearpage
\ReiheTitel

Auf Fridas Party wurden bisher \SI{0,15}{\litre}, \SI{0,4}{\litre} und \SI{0,2}{\litre} getrunken. Sie möchte wissen, wie viel von den ursprünglich \SI{1,25}{\litre} noch übrig sind. Sie rechnet

\[ 1,75 - 0,15 - 0,4 - 0,2 = 1,6 - 0,4 - 0,2 = 1,2 - 0,2 = 1,0\ (\si{\litre}) \]

Ihre Freundin Paula fragt: \enquote{Warum hast du nicht zuerst die getrunkenen Mengen addiert?}

\subsubsection*{Aufgabe 1}
Warum ist Mahins Vorschlag geschickter als Fridas Rechnung? Erinnert euch gemeinsam an die  \emph{Minusklammerregel} und erklärt, wie sie hier angewandt werden kann.

\subsubsection*{Aufgabe 2}
Formuliert eine Regel, wann es geschickt ist, die Minusklammerregel anzuwenden, wenn man mit Dezimalzahlen und Brüchen rechnet.

\clearpage
\ReiheTitel

Harun möchte wissen, wie viel Quadratmeter seine Zweizimmerwohnung hat. Das eine Zimmer hat die Abmessungen $\SI{2,4}{\meter} \times  \SI{3,6}{\meter}$ und das andere Zimmer die Maße $\SI{3,2}{\meter}\times \SI{3,6}{\meter}$. Er rechnet 

\[ 2,8\cdot 3,6 + 3,2\cdot 3,6 = 10,08 + 11,52 = 21,6\ (\si{\square\meter}) \]

Sein Freund Helge fragt: \enquote{Warum hast du nicht zuerst ausgeklammert?}

\subsubsection*{Aufgabe 1}
Warum ist Helges Vorschlag geschickter als Haruns Rechnung? Erinnert euch gemeinsam an das \emph{Distributivgesetz} und erklärt, wie es hier angewandt werden kann.

\subsubsection*{Aufgabe 2}
Formuliert eine Regel, wann es geschickt ist, das Ausklammern (Distributivgesetz) anzuwenden, wenn man mit Dezimalzahlen und Brüchen rechnet.

\clearpage
\ReiheTitel

Boris möchte den Hasenkäfig vergrößern. Der aktuelle Käfig ist $\tSI{3}{5}{\meter}$ breit und $\tSI{5}{7}{\meter}$ lang. Laut Webseite ist der Anbau genauso breit und $\tSI{5}{3}{\meter}$ lang. Für die Größe des neuen Käfigs rechnet er daher

\[ \dfrac{3}{5}\cdot (\dfrac{5}{7} + \dfrac{5}{3}) = \dfrac{3}{5}\cdot \dfrac{50}{21} = \dfrac{10}{7}\ (\si{\square\meter}) \]

Seine Freundin Mahin fragt: \enquote{Warum hast du nicht zuerst ausmultipliziert?}

\subsubsection*{Aufgabe 1}
Warum ist Mahins Vorschlag geschickter als Boris' Rechnung? Erinnert euch gemeinsam an das \emph{Distributivgesetz} und erklärt, wie es hier angewandt werden kann.

\subsubsection*{Aufgabe 2}
Formuliert eine Regel, wann es geschickt ist, das Ausmultiplizieren (Distributivgesetz) anzuwenden, wenn man mit Dezimalzahlen und Brüchen rechnet.

\end{document}
