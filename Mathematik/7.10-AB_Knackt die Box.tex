\documentclass[11pt, a4paper]{scrartcl}

\usepackage{vorschule}
\usepackage[
    typ=ab,
    fach=Mathematik,
    lerngruppe={Jg.7},
    nummer=10,
    module={Symbole,Papiertypen,Lizenzen},
    lizenz=cc-by-nc-sa-4,
]{schule}

\usepackage[
	kuerzel=Ngb,
	reihe={Terme und Gleichungen},
	version={2019-04-01},
]{ngbschule}

\author{J. Neugebauer}
\title{Knackt die Box}
\date{\Heute}

\setzeAufgabentemplate{ngbnormal}


\begin{document}
	\ReiheTitel
	
	\begin{aufgabe}
		Lest gemeinsam im Buch auf Seite 107 den \enquote{Forschungsauftrag 1: Boxen füllen}. Geht dann wie folgt vor:
		\begin{smallitemize}
			\item Einer überlegt sich eine Boxengleichung.
			\item Einer löst die Boxengleichung.
			\item Der Rest beobachtet und notiert die Boxengleichung und die Lösungsschritte als Bilder im Heft (wie in \emph{Fig. 1}).
			\item Wechselt die Rollen, so dass jeder einmal eine Boxengleichung legen durfte.
		\end{smallitemize}
	\end{aufgabe}

	\begin{aufgabe}
		Lest gemeinsam im Buch auf Seite 107 den \enquote{Forschungsauftrag 2: Boxen und Gleichungen}. Geht dabei wie folgt vor:
		\begin{smallitemize}
			\item Ordnet gemeinsam die Gleichungen in \emph{Fig. 3} den Boxengleichungen zu.
			\item Überlegt euch \emph{gemeinsam} die Schritte zur Lösung und notiert sie im Heft.
			
			\hinweis{Als Hilfe könnt ihr die Gleichungen mit echten Schachteln und Hölzern nachlegen.}
		\end{smallitemize}
	\end{aufgabe}
	
	\begin{aufgabe}
		Lest gemeinsam im Buch auf Seite 107 den \enquote{Forschungsauftrag 1: Boxen füllen}. Geht dann wie folgt vor:
		\begin{smallitemize}
			\item Notiert die Boxengleichungen aus Aufgabe 1 als Gleichungen mit Zahlen und Variablen im Heft.
			\item Notiert auch die Lösungsschritte aus Aufgabe 1 und 2 als Terme und Gleichungen.
		\end{smallitemize}
	\end{aufgabe}

	\begin{aufgabe}[symbol=\symStern]
		Bisher bestehen unsere Boxengleichungen aus Summen von positiven Zahlen und Variablen. Überlegt gemeinsam, wie man auch negative Zahlen mit Streichhölzern darstellen könnte.
	\end{aufgabe}
\end{document}