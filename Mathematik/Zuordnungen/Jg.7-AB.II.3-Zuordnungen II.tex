\documentclass[10pt, a4paper]{scrartcl}

\usepackage{vorschule}
\usepackage[
	typ=ab,
	fach=Mathematik,
	lerngruppe={Jg.7},
	nummer={II.3},
	module={Symbole,Lizenzen,Papiertypen},
	seitenzahlen=keine,
	%farbig,
	lizenz=cc-by-nc-sa-4,
]{schule}

\usepackage[
	kuerzel=Ngb,
	reihe={Zuordnungen},
	version={2020-09-27},
]{ngbschule}

\author{J. Neugebauer}
\title{Zuordnungen}
\date{\Heute}

\setzeAufgabentemplate{ngbkompakt}


\begin{document}


% Seite 1
\begin{rahmen}
	Herr Müller verdient als Anlagemechaniker täglich bei achtstündiger Arbeitszeit \EUR{104,80}. Wie viel verdient er im Monat, wenn er insgesamt \num{176} Stunden arbeitet.
\end{rahmen}

\begin{aufgabe}[symbol=\symPartner]
	Löst die Aufgabe oben. Erstellt dann für die Zuordnung \emph{Arbeitszeit in \si{\hour}} $\rightarrow$ \emph{Lohn in Euro} eine \emph{Formel}, füllt die \emph{Wertetabelle} unten aus und zeichnet den \emph{Graphen} der Zuordnung auf das Arbeitsblatt.
	
	Achte dabei auf sinnvolle Einheiten für die $x$- und $y$-Achsen. (Ihr müsst nicht alle Werte der Tabelle einzeichnen, wenn der Platz nicht reicht.)
\end{aufgabe}
\begin{center}
\begin{tabularx}{.9\textwidth}{c|*{11}{|>{\centering\bfseries\arraybackslash}X}}
	$x$ & 0 & 1 & 2 & 3 & 4 & 5 & 6 & 7 & 8 & 9 & 10 \\\hline
	$y$\Zeilenabstand & & & & & & & & & & &
\end{tabularx}
\end{center}

\feldMil{10}

\begin{aufgabe}[symbol=\symGruppe]
	Zeichnet die Graphen zu den vier Aufgaben an die Tafel und vergleicht sie miteinander. Welche Gemeinsamkeiten gibt es? Wie lassen sich die Unterschiede erklären? Diskutiert gemeinsam.
\end{aufgabe}

\begin{aufgabe}[symbol=\symEinzel]
	Lies im Buch Seite 58 und 59. Fülle dann den Lückentext aus.
	
	\begin{rahmen}
	Eine Zuordnung, deren Graph die Gestalt einer \luecke{2cm} hat, die durch den Punkt (\luecke{.5cm}|\luecke{.5cm}) verläuft, nennt man eine \luecke{2cm} Zuordnung.
	
	Eine \luecke{2cm} Zuordnung ordnet dem $n$-fachen des \luecke{.5cm}-Wertes das $n$-fache des \luecke{.5cm}-Wertes zu.
	
	Die Formel einer solchen Zuordnung ist $y=$\luecke{2cm}. Der Wert \luecke{.5cm} heißt \luecke{2cm}faktor der Zuordnung. Er kann aus der \emph{Wertetabelle} berechnet werden durch $q=$\luecke{2cm}.
	\end{rahmen}
\end{aufgabe}

\begin{aufgabe}[symbol=\symPartner]
	Vergleicht eure Lückentexte und korrigert gegebenfalls.
	
	Bearbeitet dann im Buch: \bfseries S.60, Aufg.1 bis 4 und S.61, Aufg.5.
\end{aufgabe}

% Seite 2
\clearpage\setcounter{aufgabe}{0}
\begin{rahmen}
	\SI{1}{\square\meter} Teppichboden kosten \EUR{24}. Wie viel muss Familie Meier bezahlen, um das Kinderzimmer mit \SI{18}{\square\meter} komplett mit Teppichboden auszulegen?
\end{rahmen}

\begin{aufgabe}[symbol=\symPartner]
	Löst die Aufgabe oben. Erstellt dann für die Zuordnung \emph{Fläche in \si{\square\meter}} $\rightarrow$ \emph{Kosten in Euro} eine \emph{Formel}, füllt die \emph{Wertetabelle} unten aus und zeichnet den \emph{Graphen} der Zuordnung auf das Arbeitsblatt.
	
	Achte dabei auf sinnvolle Einheiten für die $x$- und $y$-Achsen. (Ihr müsst nicht alle Werte der Tabelle einzeichnen, wenn der Platz nicht reicht.)
\end{aufgabe}
\begin{center}
\begin{tabularx}{.9\textwidth}{c|*{11}{|>{\centering\bfseries\arraybackslash}X}}
	$x$ & 0 & 1 & 2 & 3 & 4 & 5 & 6 & 7 & 8 & 9 & 10 \\\hline
	$y$\Zeilenabstand & & & & & & & & & & &
\end{tabularx}
\end{center}

\feldMil{10}

\begin{aufgabe}[symbol=\symGruppe]
	Zeichnet die Graphen zu den vier Aufgaben an die Tafel und vergleicht sie miteinander. Welche Gemeinsamkeiten gibt es? Wie lassen sich die Unterschiede erklären? Diskutiert gemeinsam.
\end{aufgabe}

\begin{aufgabe}[symbol=\symEinzel]
	Lies im Buch Seite 58 und 59. Fülle dann den Lückentext aus.
	
	\begin{rahmen}
	Eine Zuordnung, deren Graph die Gestalt einer \luecke{2cm} hat, die durch den Punkt (\luecke{.5cm}|\luecke{.5cm}) verläuft, nennt man eine \luecke{2cm} Zuordnung.
	
	Eine \luecke{2cm} Zuordnung ordnet dem $n$-fachen des \luecke{.5cm}-Wertes das $n$-fache des \luecke{.5cm}-Wertes zu.
	
	Die Formel einer solchen Zuordnung ist $y=$\luecke{2cm}. Der Wert \luecke{.5cm} heißt \luecke{2cm}faktor der Zuordnung. Er kann aus der \emph{Wertetabelle} berechnet werden durch $q=$\luecke{2cm}.
	\end{rahmen}
\end{aufgabe}

\begin{aufgabe}[symbol=\symPartner]
	Vergleicht eure Lückentexte und korrigert gegebenfalls.
	
	Bearbeitet dann im Buch: \bfseries S.60, Aufg.1 bis 4 und S.61, Aufg.5.
\end{aufgabe}

% Seite 3
\clearpage\setcounter{aufgabe}{0}
\begin{rahmen}
	Eine Taxifahrt kostet pro angefangenem Kilometer \EUR{1,13}. Wie teuer ist die Fahrt vom Bielefelder Obersee zum Tierpark Olderdissen (ca. \SI{9,6}{\kilo\meter})?
\end{rahmen}

\begin{aufgabe}[symbol=\symPartner]
	Löst die Aufgabe oben. Erstellt dann für die Zuordnung \emph{Strecke in \si{\kilo\meter}} $\rightarrow$ \emph{Lohn in Euro} eine \emph{Formel}, füllt die \emph{Wertetabelle} unten aus und zeichnet den \emph{Graphen} der Zuordnung auf das Arbeitsblatt.
	
	Achte dabei auf sinnvolle Einheiten für die $x$- und $y$-Achsen. (Ihr müsst nicht alle Werte der Tabelle einzeichnen, wenn der Platz nicht reicht.)
\end{aufgabe}
\begin{center}
%\begin{tabular}{c|*{11}{|>{\centering\bfseries\arraybackslash}p{1cm}}}
\begin{tabularx}{.9\textwidth}{c|*{11}{|>{\centering\bfseries\arraybackslash}X}}
	$x$ & 0 & 1 & 2 & 3 & 4 & 5 & 6 & 7 & 8 & 9 & 10 \\\hline
	$y$\Zeilenabstand & & & & & & & & & & &
\end{tabularx}
\end{center}

\feldMil{10}

\begin{aufgabe}[symbol=\symGruppe]
	Zeichnet die Graphen zu den vier Aufgaben an die Tafel und vergleicht sie miteinander. Welche Gemeinsamkeiten gibt es? Wie lassen sich die Unterschiede erklären? Diskutiert gemeinsam.
\end{aufgabe}

\begin{aufgabe}[symbol=\symEinzel]
	Lies im Buch Seite 58 und 59. Fülle dann den Lückentext aus.
	
	\begin{rahmen}
	Eine Zuordnung, deren Graph die Gestalt einer \luecke{2cm} hat, die durch den Punkt (\luecke{.5cm}|\luecke{.5cm}) verläuft, nennt man eine \luecke{2cm} Zuordnung.
	
	Eine \luecke{2cm} Zuordnung ordnet dem $n$-fachen des \luecke{.5cm}-Wertes das $n$-fache des \luecke{.5cm}-Wertes zu.
	
	Die Formel einer solchen Zuordnung ist $y=$\luecke{2cm}. Der Wert \luecke{.5cm} heißt \luecke{2cm}faktor der Zuordnung. Er kann aus der \emph{Wertetabelle} berechnet werden durch $q=$\luecke{2cm}.
	\end{rahmen}
\end{aufgabe}

\begin{aufgabe}[symbol=\symPartner]
	Vergleicht eure Lückentexte und korrigert gegebenfalls.
	
	Bearbeitet dann im Buch: \bfseries S.60, Aufg.1 bis 4 und S.61, Aufg.5.
\end{aufgabe}

% Seite 4
\clearpage\setcounter{aufgabe}{0}
\begin{rahmen}
	Herbert möchte Fotos von seiner Digitalkamera ausdrucken. Er zahlt pro Foto \SI{6}{Cent}. Wie viel muss er für seine Bestllung von 83 Fotos insgesamt bezahlen?
\end{rahmen}

\begin{aufgabe}[symbol=\symPartner]
	Löst die Aufgabe oben. Erstellt dann für die Zuordnung \emph{Anzahl Fotos} $\rightarrow$ \emph{Kosten in Euro} eine \emph{Formel}, füllt die \emph{Wertetabelle} unten aus und zeichnet den \emph{Graphen} der Zuordnung auf das Arbeitsblatt.
	
	Achte dabei auf sinnvolle Einheiten für die $x$- und $y$-Achsen. (Ihr müsst nicht alle Werte der Tabelle einzeichnen, wenn der Platz nicht reicht.)
\end{aufgabe}
\begin{center}
%\begin{tabular}{c|*{11}{|>{\centering\bfseries\arraybackslash}p{1cm}}}
\begin{tabularx}{.9\textwidth}{c|*{11}{|>{\centering\bfseries\arraybackslash}X}}
	$x$ & 0 & 1 & 2 & 3 & 4 & 5 & 6 & 7 & 8 & 9 & 10 \\\hline
	$y$\Zeilenabstand & & & & & & & & & & &
\end{tabularx}
\end{center}

\feldMil{10}

\begin{aufgabe}[symbol=\symGruppe]
	Zeichnet die Graphen zu den vier Aufgaben an die Tafel und vergleicht sie miteinander. Welche Gemeinsamkeiten gibt es? Wie lassen sich die Unterschiede erklären? Diskutiert gemeinsam.
\end{aufgabe}

\begin{aufgabe}[symbol=\symEinzel]
	Lies im Buch Seite 58 und 59. Fülle dann den Lückentext aus.
	
	\begin{rahmen}
	Eine Zuordnung, deren Graph die Gestalt einer \luecke{2cm} hat, die durch den Punkt (\luecke{.5cm}|\luecke{.5cm}) verläuft, nennt man eine \luecke{2cm} Zuordnung.
	
	Eine \luecke{2cm} Zuordnung ordnet dem $n$-fachen des \luecke{.5cm}-Wertes das $n$-fache des \luecke{.5cm}-Wertes zu.
	
	Die Formel einer solchen Zuordnung ist $y=$\luecke{2cm}. Der Wert \luecke{.5cm} heißt \luecke{2cm}faktor der Zuordnung. Er kann aus der \emph{Wertetabelle} berechnet werden durch $q=$\luecke{2cm}.
	\end{rahmen}
\end{aufgabe}

\begin{aufgabe}[symbol=\symPartner]
	Vergleicht eure Lückentexte und korrigert gegebenfalls.
	
	Bearbeitet dann im Buch: \bfseries S.60, Aufg.1 bis 4 und S.61, Aufg.5.
\end{aufgabe}

\end{document}
