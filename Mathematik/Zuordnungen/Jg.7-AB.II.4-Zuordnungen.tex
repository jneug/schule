\documentclass[10pt, a4paper]{scrartcl}

\usepackage{vorschule}
\usepackage[
	typ=ab,
	fach=Mathematik,
	lerngruppe={Jg.7},
	nummer={II.4},
	module={Symbole,Lizenzen,Papiertypen},
	seitenzahlen=keine,
	%farbig,
	lizenz=cc-by-nc-sa-4,
]{schule}

\usepackage[
	kuerzel=Ngb,
	reihe={Zuordnungen},
	version={2020-10-06},
]{ngbschule}

\author{J. Neugebauer}
\title{Zuordnungen II}
\date{\Heute}

\setzeAufgabentemplate{ngbkompakt}


\begin{document}

\begin{rahmen}
	Auf einer Baustelle arbeiten drei Schaufelbagger, um eine Grube auszuheben. Gemeinsam brauchen sie acht Stunden für die Arbeit. Pro Bagger kostet die Firma eine Arbeitsstunde \EUR{128}. Um die Arbeit zu beschleunigen überlegt die Firma nun, zwei weitere Bagger zu beauftragen. 

	Wie lange brauchen fünf Bagger für die Grube? Wie viel muss die Firma für fünf Bagger pro Stunde bezahlen?
\end{rahmen}

\begin{aufgabe}[symbol=\symPartner]
	In der Aufgabe lassen sich zwei Zuordnungen finden: \emph{Anzahl Bagger $\rightarrow$ Zeit in \si{\hour}} und \emph{Anzahl Bagger $\rightarrow$ Kosten in \si{Euro\per\hour}}.
	
	Findet jeweils eine \emph{Formel}, füllt die \emph{Wertetabelle} aus und zeichnet den \emph{Graphen}. (Wenn der Platz nicht reicht, dann zeichnet im Heft.)

	Unten findet ihr einen Hinweis zur zweiten Zuordnung, falls ihr nicht weiter kommt.
\end{aufgabe}

\begin{multicols}{2}
\begin{center}
\emph{Anzahl Bagger $\rightarrow$ Kosten in \si{Euro\per\hour}}
\begin{tabularx}{.9\columnwidth}{c|*{5}{|>{\centering\bfseries\arraybackslash}X}}
	$x$ & 1 & 2 & 3 & 4 & 5 \\\hline
	$y$\Zeilenabstand & & & & &
\end{tabularx}
\end{center}

\feldMil{5}

\begin{center}
\emph{Anzahl Bagger $\rightarrow$ Zeit in \si{\hour}}
\begin{tabularx}{.9\columnwidth}{c|*{5}{|>{\centering\bfseries\arraybackslash}X}}
	$x$ & 1 & 2 & 3 & 4 & 5 \\\hline
	$y$\Zeilenabstand & & & & &
\end{tabularx}
\end{center}

\feldMil{5}
\end{multicols}
\vspace*{-2em}
\begin{aufgabe}[symbol=\symGruppe]
	Beantwortet anhand der Wertetabellen die Fragen der Firma.
	
	Wieviel müsste die Firma für \num{100}, \num{1000} oder \num{10000} Bagger pro Stunde zahlen? Wie viel Zeit brauchen so viele Bagger für die Grube? Machen diese Zahlen im \emph{Sachzusammenhang} Sinn?
	
	Überlegt zuerst zu zweit und diskutiert dann gemeinsam.
\end{aufgabe}

\begin{aufgabe}[symbol=\symGruppe]
	Eine der Zuordnungen ist \emph{proportional}, die andere nicht. Überlegt, wie man am Graphen erkennen kann, welche Zuordnung proportional ist.
\end{aufgabe}

\begin{aufgabe}[symbol=\symEinzel]
	Die andere Art von Zuordnung nennt man \textbf{antiproportional}. Vielleicht ahnst du schon, wieso.
	
	Lies im Buch Seite 63 und 64. Fülle dann den Lückentext aus. 
	
	\begin{rahmen}
	Eine Zuordnung heißt \textbf{antiproportionale}, wenn dem $n$-fachen des \luecke{.5cm}-Wertes das $\dfrac{1}{n}$-fache des \luecke{.5cm}-Wertes zugeordnet wird. (Zum Beispiel dem \enquote{Doppelten} die \enquote{Hälfte} oder dem \enquote{Vierfachen} ein \enquote{\luecke{2cm}}).
	
	Die Formel einer solchen Zuordnung ist $y=$\luecke{2cm}. Der Wert $p$ heißt \luecke{3cm}konstante der Zuordnung. Sie kann aus der \emph{Wertetabelle} berechnet werden durch $p=$\luecke{2cm}.
	\end{rahmen}
\end{aufgabe}

\begin{aufgabe}[symbol=\symPartner]
	Vergleicht eure Lückentexte und korrigert gegebenfalls.
	
	Bearbeitet dann im Buch: \bfseries S.65, Aufg.1 bis 4 und S.66, Aufg.5.
\end{aufgabe}

\vfill

\begin{center}
\raisebox{\depth}{\rotatebox[origin=c]{180}{\footnotesize\color{black!50}Hinweis zu Aufgabe 1: 1 Bagger braucht dreimal so lange, wie drei Bagger. Fünf Bagger brauchen ein Fünftel so lange, wie ein Bagger.}}
\end{center}

\end{document}
