\documentclass[10pt, a5paper, landscape]{arbeitsblatt}

\ladeModule{theme}
\ladeFach{mathematik}

\aboptionen{
	kuerzel=Ngb,
	name={J. Neugebauer},
	title={Prozentrechnung}
	reihe={Prozentrechnung} ,
	version={2021-04-11} ,
	fach=Mathematik,
	lerngruppe={Jg.7} ,
	nummer={IV.1},
	lizenz=cc-by-nc-sa-4,
}

\begin{document}
\TITEL

Berechnet jeweils mit dem Dreisatz. Findet ihr auch eine Rechnung \emph{ohne Dreisatz}?

\begin{aufgabe}[subtitle=Räume: 1 | 4 | 7]
	\label{aufg:prozentsatz}
	Der Preis für ein Paar Sneakers wird von \EUR{50} auf \EUR{30} reduziert. Um wie viel Prozent ist der Preis günstiger geworden?
\end{aufgabe}

\begin{aufgabe}[subtitle=Räume: 2 | 5 | 8]
	\label{aufg:prozentwert}
	Ein Legoset des imperialen Shuttles von Star Wars kostet normalerweise \EUR{60}, ist aber gerade um \prozent{24} reduziert. Wie teuer ist es dann noch?
\end{aufgabe}

\begin{aufgabe}[subtitle=Räume: 3 | 6 | 9]
	\label{aufg:grundwert}
	Am letzten Tag des Jahrmarktes wird der Eintritt um \prozent{15} auf \EUR{11,05} reduziert. Wie teuer war der Eintritt vorher?
\end{aufgabe}

\end{document}
