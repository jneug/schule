\documentclass[10pt, a5paper, landscape]{scrartcl}

\usepackage{vorschule}
\usepackage[
	typ=ab,
	fach=Mathematik,
	lerngruppe={Jg.7} ,
	nummer={IV.1},
	module={Symbole,Lizenzen},
	seitenzahlen=keine,
	farbig,
	lizenz=cc-by-nc-sa-4,
]{schule}

\usepackage[
	kuerzel=Ngb,
	reihe={Prozentrechnung} ,
	version={2021-04-11} ,
]{ngbschule}

\author{J. Neugebauer}
\title{Prozentrechnung}
\date{\Heute}

\setzeAufgabentemplate{ngbnormal}


\begin{document}
\TITEL

Berechnet jeweils mi dem Dreisatz. Findet ihr auch eine Rechnung \emph{ohne Dreisatz}?

\begin{aufgabe}[subtitle=Räume: 1 | 4 | 7]
	\label{aufg:prozentsatz}
	Der Preis für ein Paar Sneakers wird von \EUR{50} auf \EUR{30} reduziert. Um wie viel Prozent ist der Preis günstiger geworden?
\end{aufgabe}

\begin{aufgabe}[subtitle=Räume: 2 | 5 | 8]
	\label{aufg:prozentwert}
	Ein Legoset des imperialen Shuttles von Star Wars kostet normalerweise \EUR{60}, ist aber gerade um \prozent{24} reduziert. Wie teuer ist es dann noch?
\end{aufgabe}

\begin{aufgabe}[subtitle=Räume: 3 | 6 | 9]
	\label{aufg:grundwert}
	Am letzten Tag des Jahrmarktes wird der Eintritt um \prozent{15} auf \EUR{11,05} reduziert. Wie teuer war der Eintritt vorher?
\end{aufgabe}

\end{document}
