\begin{aufgabe}[subtitle=Kuchenverkauf]
	Jeder Kuchen in einer Konditorei wird in 16 Stücke geschnitten.
	
	Vom Obstkuchen wird $\frac{3}{4}$ verkauft, vom Butterkuchen $\frac{1}{2}$, 
	von der Torte $\frac{3}{8}$, von dem Käsekuchen nur $2$ Stück, von der Cremetorte
	$\frac{1}{2}$, $0$ Stück Erdbeer-Sahne Torte.
	\begin{teilaufgaben}
		\teilaufgabe
		Wie viele Stücke wurden verkauft? Wie viele sind noch im Laden?
		
		\teilaufgabe
		Welcher Bestandteil von allen 6 Kuchen wurde verkauft?
		
		\teilaufgabe
		Es gilt folgende Preisliste:
		
		\begin{center}
		\begin{tabular}{lc}
			& je Stück \\
			Obst- und Käsekuchen & \EUR{2,30} \\
			Butterkuchen & \EUR{1,90} \\
			Torten & \EUR{2,90} \\
		\end{tabular}
		\end{center}
	
		Für wie viel Euro wurde Kuchen verkauft?
	\end{teilaufgaben}
	
	\begin{loesung}
		\begin{teilaufgaben}
		\end{teilaufgaben}
	\end{loesung}
	
	\begin{erwartungen}
	\end{erwartungen}
\end{aufgabe}