\documentclass[11pt, a4paper]{scrartcl}

\usepackage{vorschule}
\usepackage[
    typ=ab,
    fach=Informatik,
    lerngruppe={9Diff},
		nummer=2,
    module={Symbole},
		seitenzahlen=keine,
		farbig,
]{schule}

\usepackage[
	typ=checkup,
	kuerzel=Ngb,
	reihe={Imperative Programmierung},
	version={2019-11-26},
]{ngbschule}


\author{J. Neugebauer}
\title{2. Informatikarbeit}
\date{\Heute}

\begin{document}
\CheckupTitel

\begin{wrapfig}
\CheckupBild Kreuze jeweils an, wie sicher du dich bei den einzelnen \textbf{Themenschwerpunkten} fühlst (von \enquote{sehr sicher} \usym{1F604} bis \enquote{sehr unsicher} \usym{1F641}). Nutze die \textbf{Aufgaben und Informationen} zum Wiederholen und Lernen von Themen, \emph{bei denen du noch unsicher bist}.
\end{wrapfig}

\bigskip
\begin{checkup}
	\ichkann{erklären, was ein Algorithmus ist.}{
		Eigene Unterlagen \\
		\href{http://link.ngb.schule/algorithmus}{http://link.ngb.schule/algorithmus}
	}
	\ichkann{Merkmale von Algorithmen beschreiben.}{
		Eigene Unterlagen \\
		\href{http://link.ngb.schule/algorithmus}{http://link.ngb.schule/algorithmus}
	}
	\ichkann{Probleme strukturiert (in Teilprobleme) zerlegen.}{
		Eigene Unterlagen \\
		\ab[PDF]{Strukturierte Zerlegung}
	}
	\ichkann{Algorithmen als Programmablaufplan (PAP) darstellen.}{
		\ab{Ablaufdiagramme}
	}
	\ichkann{mit dem Python-Turtle Probleme lösen.}{
		Buch Kapitel 1-4 \\
		Übungen aus dem Unterricht
	}
	\ichkann{Zählschleifen benutzen.}{
		Buch Kapitel 3 \\
		Übungen aus dem Unterricht
	}
	\ichkann{Animationen mit \code{delay} und \code{clear} implementieren.}{
		\ab{Animationen} \\
		\bu{25ff und 35ff}{}
	}
	\ichkann{eigene Methoden (auch mit Parametern) in Python implementieren.}{
		Buch Kapitel 4 \\
		Übungen aus dem Unterricht
	}
\end{checkup}

\end{document}
