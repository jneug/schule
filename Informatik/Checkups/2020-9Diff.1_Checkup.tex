\documentclass[11pt, a4paper]{scrartcl}

\usepackage{vorschule}
\usepackage[
    typ=ab,
    fach=Informatik,
    lerngruppe={9Diff},
	nummer=1,
    module={Symbole},
	seitenzahlen=keine,
	farbig,
]{schule}

\usepackage[
	typ=checkup,
	kuerzel=Ngb,
	reihe={Binärzahlen und Maschinencode},
	version={2020-09-15},
]{ngbschule}


\author{J. Neugebauer}
\title{1. Informatikarbeit}
\date{\Heute}

\begin{document}
\CheckupTitel

\begin{wrapfig}
\CheckupBild Kreuze jeweils an, wie sicher du dich bei den einzelnen \textbf{Themenschwerpunkten} fühlst (von \enquote{sehr sicher} \usym{1F604} bis \enquote{sehr unsicher} \usym{1F641}). Nutze die \textbf{Aufgaben und Informationen} zum Wiederholen und Lernen von Themen, \emph{bei denen du noch unsicher bist}.
\end{wrapfig}

\bigskip
\begin{checkup}
	\teiler{Binärzahlen}
	\ichkann{Binärzahlen in Dezimalzahlen umrechnen.}{}
	\ichkann{Dezimalzahlen in Binärzahlen umrechnen.}{}
	\ichkann{Binärzahlen im Binärsystem addieren.}{}
	\ichkann{mit dem Überlauf der Binäraddition umgehen.}{}
	\ichkann{mit negative Binärzahlen in Vorzeichndarstellung umgehen.}{}
	\ichkann{das Zweierkomplement von Binärzahlen bilden.}{}
	\ichkann{mit negative Binärzahlen in Zweierkomplementdarstellung umgehen.}{}

	\teiler{Maschinencode}
	\ichkann{die Befehle des \programm{Know How Computers} erklären.}{}
	\ichkann{Ablaufdiagramme lesen.}{}
	\ichkann{Ablaufdiagramme zeichnen.}{}
	\ichkann{kurze Programme (Addition, Subtraktion, ...) für den \programm{Know How Computers} implementieren.}{}
\end{checkup}

\end{document}
