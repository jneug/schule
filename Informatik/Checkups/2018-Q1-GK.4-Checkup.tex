\documentclass[11pt, a4paper]{scrartcl}

\usepackage{vorschule}
\usepackage[
    typ=ab,
    fach=Informatik,
    lerngruppe={Q1 GK},
    nummer={4},
    module={Symbole},
]{schule}

\usepackage[
	typ=checkup,
	kuerzel=Ngb,
	reihe={Informatik und Gesellschaft},
]{ngbschule}


\author{J. Neugebauer}
\title{\Nummer. Klausur}
\date{\Heute}

\begin{document}
\CheckupBild\CheckupTitel

Kreuzen sie jeweils an, wie sicher sie sich bei den einzelnen \textbf{Themenschwerpunkten} fühlen (von \enquote{sehr sicher} \usym{1F604} bis \enquote{sehr unsicher} \usym{1F641}). Nutzen sie die \textbf{Aufgaben und Informationen} zum Wiederholen und Lernen von Themen, bei denen sie noch unsicher sind.

\begin{checkup}
	\ichkann{\dots Pro- und Contraargumente für Artikel 17 (vormals 13) der EU Urheberrechtsreform nennen und erklären.}{
		Texte und Videos zur Reform
	}
	\ichkann{\dots erklären, was ein \enquote{Uploadfilter} ist und welche technischen Probleme mit deren Umsetzung verbunden sind.}{
		AB \enquote{Uploadfilter}
	}
	\ichkann{\dots die Datenschutzprinzipien (Transparenz, Datenminimierung, \dots) nennen, beschreiben und an Beispielen verdeutlichen.}{
		Mind-Map und Texte \\ zur DSGVO
	}
	\ichkann{\dots die Konzepte der \enquote{Erforderlichkeit} und \enquote{Verbot mit Erlaubnisvorbehalt} erklären.}{
		Mind-Map und Texte \\ zur DSGVO
	}
	\ichkann{\dots die Ethischen Leitlinien der GI beschreiben.}{
		Webseite der GI \\
		AB aus dem Unterricht
	}
	\ichkann{\dots begründet und strukturiert zu Fallbeispielen Stellung nehmen.}{
		Vorgehensmodelle für \\ Beurteilungen
	}
\end{checkup}

\end{document}
