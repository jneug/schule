\documentclass[11pt, a4paper]{scrartcl}

\usepackage{vorschule}
\usepackage[
    typ=ab,
    fach=Informatik,
    lerngruppe={EF},
		nummer=1,
    module={Symbole},
		seitenzahlen=keine,
		farbig,
]{schule}

\usepackage[
	typ=checkup,
	kuerzel=Ngb,
	reihe={Objektorientierte Programmierung},
	version={2019-11-26},
]{ngbschule}


\author{J. Neugebauer}
\title{1. Informatikarbeit}
\date{\Heute}

\begin{document}
\CheckupTitel

\begin{wrapfig}
\CheckupBild Kreuze jeweils an, wie sicher du dich bei den einzelnen \textbf{Themenschwerpunkten} fühlst (von \enquote{sehr sicher} \usym{1F604} bis \enquote{sehr unsicher} \usym{1F641}). Nutze die \textbf{Aufgaben und Informationen} zum Wiederholen und Lernen von Themen, \emph{bei denen du noch unsicher bist}.
\end{wrapfig}

\bigskip
\begin{checkup}
	\ichkann{eine Situationsbeschreibung auf Objekte, Klassen, Eigenschaften und Fähigkeiten analysieren.}{
		\bu{20ff}{} \\
		\ab{Mars-Rover}
	}	
	\ichkann{den Zustand von Objekten als Objektdiagtramm darstellen.}{
		\bu{20ff}{} \\
		\ab{Mars-Rover}
	}
	\ichkann{Klassen mit ihren Eigenschaften und Fähigkeiten als Klassendiagramm darstellen.)}{
		\bu{20ff}{} \\
		\ab{Mars-Rover}
	}
	\ichkann{Probleme strukturiert (in Teilprobleme) zerlegen.}{
		\bu{59ff}{}
	}
	\ichkann{Anfragen und Aufträge in Java nutzen, um mit dem Mars-Rover Probleme zu lösen.}{
		\bu{23ff}{}
	}
	\ichkann{logische Aussagen und Verknüpfungen interpretieren und aufstellen.}{
		\bu{53ff}{}\\
		\ab{Wahrheitswerte}
	}
	\ichkann{\dots Bedingte-Anweisungen und -Schleifen in Java nutzen.}{
		\bu{38ff}{} \\
		(ohne Abschnitt 3.3!)
	}
\end{checkup}

\end{document}
