\documentclass[10pt, a4paper, ngerman]{arbeitsblatt}

\ladeModule{theme}
\aboptionen{
	name		= {J. Neugebauer},
	kuerzel		= {Ngb},
	titel		= {SQL-Injections},
	reihe		= {Die Methoden der Hacker},
	fach		= {Informatik},
	lerngruppe	= {Q2},
	nummer		= {6},
	lizenz		= {cc-by-nc-sa-4},
	version		= {2021-10-24},
}

\begin{document}
\ReiheTitel

Kopiere zur Vorbereitung die Datei \datei{fussballem.db} und das Projekt \ordner{SuperSecureServer} aus dem Tauschordner. Du kannst eine der Datenbankdateien durch Doppelklick im Programm \programm{DB Browser for SQLite} öffnen. Hier kannst du dir den Inhalt der Datenbank anzeigen lassen und SQL-Anfragen ausführen.

\begin{aufgabe}[icon=\iconPartner\,\iconLaptop]
	\begin{teilaufgaben}
		\teilaufgabe
		Öffne die Datenbank \datei{fussballem.db} und erkunde das Programm. Sende dann einige \code{SELECT} Anfragen an die Datenbank.

		\teilaufgabe
		Nutze den \code{INSERT} Befehl, um neue Daten in die Datenbank einzufügen. Suche dazu im Internet nach Ergebnissen anderer EM-Jahrgänge.

		\teilaufgabe
		Nutze den \code{UPDATE} Befehl, um einige Datensätze zu verändern.

		\teilaufgabe
		Nutze den \code{DELETE} Befehl, um einige (oder alle) Datensätze zu löschen.
	\end{teilaufgaben}

	\hinweis{Unter \url{https://link.ngb.schule/sqlbefehle} findest du eine Übersicht der SQL-Syntax und Befehle.}

	\hinweis{Falls du die Datenbank beim Arbeiten löscht oder \enquote{kaputt} machst, kannst du dir die Originalversion erneut aus dem Tauschordner kopieren.}
\end{aufgabe}

\begin{aufgabe}[icon=\iconPartner\,\iconLaptop]
	Öffnen die Seite \url{https://link.ngb.schule/sqlinjection} und lies den Text bis zur Überschrift \enquote{Wie können Sie eine SQL-Injection verhindern?}. Erkläre dir selber, was eine \enquote{SQL-Injection} ist und welche Schwachstelle sie ausnutzt.

	\hinweis{Eine alternative Erklärung mit weiteren Beispielen findest du unter \url{https://link.ngb.schule/sqlinjection2}.}
\end{aufgabe}

\begin{aufgabe}[icon=\iconPartner\,\iconLaptop]
	\begin{wrapfigure}[4]{r}{0pt}
	\includegraphics[width=3cm]{Q2-AB.06-ABB_S3-Logo.png}
	\end{wrapfigure}
	Öffne das Projekt \ordner{SuperSecureServer} in \programm{BlueJ}.


	\begin{teilaufgaben}
		\teilaufgabe
		Erstelle einen neuen \code{SuperSecureServer} und probiere die Anmeldung aus. Die Datenbank mit Nutzerkonten ist im Projektordner unter dem Namen \datei{auth.db} gespeichert.

		\teilaufgabe
		Studiere die Klasse \code{SuperSecureServer} und analysiere sie auf mögliche SQL-Injection Schwachstellen.

		\teilaufgabe
		Versuch einen Weg zu finden, dich erfolgreich am Server anzumelden, ohne Nutzernamen oder Passwort eines Nutzers zu kennen.
	\end{teilaufgaben}
\end{aufgabe}

\end{document}
