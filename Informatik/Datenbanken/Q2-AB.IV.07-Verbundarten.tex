\documentclass[10pt, a4paper, ngerman]{arbeitsblatt}

\ladeModule{theme,tabellen}

\ladeFach[quelltexte,datenbanken]{informatik}

\aboptionen{
	name		= {J. Neugebauer},
	kuerzel 	= {Ngb},
	titel 		= {Verbundarten},
	reihe 		= {Relationale Datenbanken},
	fach 		= {Informatik},
	kurs 		= {Q2},
	nummer 		= {IV.7},
	lizenz 		= {cc-by-nc-sa-eu-4},
	version 	= {2022-05-05},
}

\begin{document}
\ReiheTitel

\begin{aufgabe}\label{aufg:1}
Ordne die vier Verbundarten \code{INNER JOIN}, \code{LEFT JOIN}, \code{RIGHT JOIN} und \code{FULL OUTER JOIN} den Diagrammen unten zu.

\begin{center}
	\begin{tikzpicture}[thick, scale=.75,
		every node/.style={scale=0.75},
		set/.style={circle, minimum size=3cm}]
		% INNER JOIN
		\node[set,fill=white] (A3) at (0,0) {A};
		\node[set,fill=white] (B3) at ($(A3)+(2,0)$) {B};
		\begin{scope}
			\clip (A3) circle(1.5cm);
			\clip (B3) circle(1.5cm);
			\fill[secondary.hg](A3) circle(1.5cm);
		\end{scope}
		\draw (A3) circle(1.5cm);
		\draw (B3) circle(1.5cm);
		\draw ($(A3)+(-1,-1.75)$) rectangle ++(4,-1);

		% LEFT JOIN
		\node[set,fill=white] (B1) at (14,0) {B};
		\node[set,fill=secondary.hg] (A1) at ($(B1)+(-2,0)$) {A};
		\draw (A1) circle(1.5cm);
		\draw (B1) circle(1.5cm);
		\draw ($(A1)+(-1,-1.75)$) rectangle ++(4,-1);

		% RIGHT JOIN
		\node[set,fill=white] (A2) at (6,0) {A};
		\node[set,fill=secondary.hg] (B2) at ($(A2)+(2,0)$) {B};
		\draw (A2) circle(1.5cm);
		\draw (B2) circle(1.5cm);
		\draw ($(A2)+(-1,-1.75)$) rectangle ++(4,-1);

		% FULL JOIN
		\node[set,fill=secondary.hg] (A4) at (18,0) {A};
		\node[set,fill=secondary.hg] (B4) at ($(A4)+(2,0)$) {B};
		\draw (A4) circle(1.5cm);
		\draw (B4) circle(1.5cm);
		\draw ($(A4)+(-1,-1.75)$) rectangle ++(4,-1);
	\end{tikzpicture}
\end{center}
\end{aufgabe}

\begin{aufgabe}
Ausgehend von den Tabellen A und B unten wurden die gezeigten Ergebnisse durch SQL-Anweisungen mit \code{JOIN}s über das Attribut \code{p1} ermittelt. Verbinde die Ergebnisstabellen mit den passenden Verbundarten aus \prettyref{aufg:1}.

\begin{enuma}
	\item \mintinline{sql}{SELECT * FROM A INNER JOIN B ON A.p1 = B.p1}
	\item \mintinline{sql}{SELECT * FROM A LEFT JOIN B ON A.p1 = B.p1}
	\item \mintinline{sql}{SELECT * FROM A RIGHT JOIN B ON A.p1 = B.p1}
	\item \mintinline{sql}{SELECT * FROM A FULL OUTER JOIN B ON A.p1 = B.p1}
\end{enuma}

\begin{tabularx}{4cm}{*{3}{X}}
	\rowcolor{ab.tabelle.kopf.hg!90!black}
	\multicolumn{3}{l}{\code{A}}       \\
	\hline\rowcolor{ab.tabelle.kopf.hg}
	\primarykey{p1} & colA & colB      \\
	\hline\hline
	1 & A & aa \\\hline
	2 & B & bb \\\hline
	3 & C & cc \\\hline
	4 & D & dd \\\hline
	5 & E & ee \\\hline
	6 & F & ff \\\hline
	7 & G & gg \\\hline
	8 & H & hh \\\hline
\end{tabularx}

\begin{tabularx}{4cm}{*{4}{X}}
	\rowcolor{ab.tabelle.kopf.hg!90!black}
	\multicolumn{4}{l}{\code{B}}       \\
	\hline\rowcolor{ab.tabelle.kopf.hg}
	\primarykey{p2} & \foreignkey{p1} & colX & colY      \\
	\hline\hline
	9001 & 1 & 55 & ABC \\\hline
	9002 & 3 & 77 & PQR \\\hline
	9003 & 3 & 95 & DEF \\\hline
	9004 & 7 & 62 & GHI \\\hline
	9005 & 5 & 50 & KLM \\\hline
	9006 & 2 & 42 & STU \\\hline
	9007 & 9 & 83 & XYZ \\\hline
\end{tabularx}

\begin{tabular}{*{7}{l}} \hline
	\rowcolor{ab.tabelle.kopf.hg}
	A.p1 & colA & colB & p2 & B.p1 & colX & colY      \\
	\hline\midrule
	1 & A & aa & 9001 & 1 & 55 & ABC \\\midrule
	2 & B & bb & 9006 & 2 & 42 & STU \\\midrule
	3 & C & cc & 9002 & 3 & 77 & PQR \\\midrule
	3 & C & cc & 9003 & 3 & 95 & DEF \\\midrule
	5 & E & ee & 9005 & 5 & 50 & KLM \\\midrule
	7 & G & gg & 9004 & 7 & 62 & GHI \\\bottomrule
\end{tabular}

\begin{tabular}{|*{7}{l|}}\hline
	\rowcolor{ab.tabelle.kopf.hg}
	A.p1 & colA & colB & p2 & B.p1 & colX & colY      \\
	\hline
	1 & A & aa & 9001 & 1 & 55 & ABC \\\hline
	2 & B & bb & 9006 & 2 & 42 & STU \\\hline
	3 & C & cc & 9002 & 3 & 77 & PQR \\\hline
	3 & C & cc & 9003 & 3 & 95 & DEF \\\hline
	4 & D & dd & NULL & NULL & NULL & NULL\\\hline
	5 & E & ee & 9005 & 5 & 50 & KLM \\\hline
	6 & F & ff & NULL & NULL & NULL & NULL \\\hline
	7 & G & gg & 9004 & 7 & 62 & GHI \\\hline
	8 & H & hh & NULL & NULL & NULL & NULL \\\hline
\end{tabular}

\begin{tabular}{|*{7}{l|}}\hline
	\rowcolor{ab.tabelle.kopf.hg}
	A.p1 & colA & colB & p2 & B.p1 & colX & colY      \\
	\hline
	1 & A & aa & 9001 & 1 & 55 & ABC \\\hline
	2 & B & bb & 9006 & 2 & 42 & STU \\\hline
	3 & C & cc & 9002 & 3 & 77 & PQR \\\hline
	3 & C & cc & 9003 & 3 & 95 & DEF \\\hline
	5 & E & ee & 9005 & 5 & 50 & KLM \\\hline
	7 & G & gg & 9004 & 7 & 62 & GHI \\\hline
	NULL & NULL & NULL & 9007 & 9 & 83 & XYZ \\\hline
\end{tabular}

\begin{tabular}{|*{7}{l|}}\hline
	\rowcolor{ab.tabelle.kopf.hg}
	A.p1 & colA & colB & p2 & B.p1 & colX & colY      \\
	\hline
	1 & A & aa & 9001 & 1 & 55 & ABC \\\hline
	2 & B & bb & 9006 & 2 & 42 & STU \\\hline
	3 & C & cc & 9002 & 3 & 77 & PQR \\\hline
	3 & C & cc & 9003 & 3 & 95 & DEF \\\hline
	4 & D & dd & NULL & NULL & NULL & NULL\\\hline
	5 & E & ee & 9005 & 5 & 50 & KLM \\\hline
	6 & F & ff & NULL & NULL & NULL & NULL \\\hline
	7 & G & gg & 9004 & 7 & 62 & GHI \\\hline
	8 & H & hh & NULL & NULL & NULL & NULL \\\hline
	NULL & NULL & NULL & 9007 & 9 & 83 & XYZ \\\hline
\end{tabular}
\end{aufgabe}

\newpage
\begin{aufgabe}
Ermittele die Ergebnisrelationen der folgenden Abfragen auf den Tabellen \code{Buch} und \code{Film}.

\begin{tabularx}{\textwidth}{XXXcc} \hline
	\rowcolor{ab.tabelle.kopf.hg!90!black}
	\multicolumn{5}{l}{\code{Buch}} \\ \hline
	\rowcolor{ab.tabelle.kopf.hg}
	\primarykey{ISBN} & Titel & Autor & Preis & VID\\ \hline\toprule
	978-3-608-93544-8 & Der Herr der Ringe & J. R. R. Tokien & 32,90 & 608 \\\midrule
	978-3-596-29433-6 & Der Zauberberg & Thomas Mann & 12,95 & 596 \\\midrule
	978-1-401-22266-6 & Watchmen & Alan Moore & 13,95 & 4012 \\\bottomrule
\end{tabularx}

\begin{tabularx}{\textwidth}{XXccX} \hline
	\rowcolor{ab.tabelle.kopf.hg!90!black}
	\multicolumn{5}{l}{\code{Film}} \\ \hline
	\rowcolor{ab.tabelle.kopf.hg}
	\primarykey{EAN} & Titel & FSK & Preis & zum\_Buch\\ \hline\toprule
	4042564141962 & Der Zauberberg & 16 & 11,99 & 978-3-596-29433-6 \\\midrule
	4011976887188 & Fack Ju Göhte & 12 & 7,99 & \\\midrule
	4010324200334 & Rubinrot & 12 & 7,99 & \\\bottomrule
\end{tabularx}

\begin{multicols}{2}
\begin{enuma}
	\item ~\\[-3em]\begin{minted}{sql}
	SELECT Titel, FSK
	FROM Film
	WHERE FSK = 12;
	\end{minted}

	\item ~\\[-3em]\begin{minted}{sql}
	SELECT Titel, Preis
	FROM Buch
	UNION
	SELECT Titel, Preis
	FROM Film;
	\end{minted}

	\columnbreak
	\item ~\\[-3em]\begin{minted}{sql}
	SELECT Buch.Titel, Buch.Preis,
		Film.Titel, Film.Preis
	FROM Buch
	  INNER JOIN Film
		ON Buch.ISBN = Film.zum_BUCH
	\end{minted}

	\item ~\\[-3em]\begin{minted}{sql}
	SELECT Buch.Titel, Buch.Preis,
		Film.Titel, Film.Preis
	FROM Buch
	  RIGHT JOIN Film
	    ON Buch.ISBN = Film.zum_BUCH
	WHERE Film.Preis < 10;
	\end{minted}
\end{enuma}
\end{multicols}
\end{aufgabe}

\begin{aufgabe}
Gegeben sind die drei Relationen, die das Ausgehverhalten einiger Jugendlicher dokumentieren.

\begin{multicols}{3}
	\begin{tabularx}{\linewidth}{XX} \hline
		\rowcolor{ab.tabelle.kopf.hg!90!black}
		\multicolumn{2}{l}{\code{Hört\_gern}} \\ \hline
		\rowcolor{ab.tabelle.kopf.hg}
		Name & Musik\-richtung \\\hline\toprule
		Tina & Pop \\\midrule
		Lisa & Latin \\\midrule
		Tina & Rock \\\midrule
		Maxi & Pop \\\bottomrule
	\end{tabularx}

	\begin{tabularx}{\linewidth}{XX} \hline
		\rowcolor{ab.tabelle.kopf.hg!90!black}
		\multicolumn{2}{l}{\code{Geht\_in}} \\ \hline
		\rowcolor{ab.tabelle.kopf.hg}
		Name & Lokal \\\hline\toprule
		Maxi & A \\\midrule
		Maxi & C \\\midrule
		Lisa & B \\\midrule
		Tina & B \\\midrule
		Flo & A \\\bottomrule
	\end{tabularx}

	\begin{tabularx}{\linewidth}{XX} \hline
		\rowcolor{ab.tabelle.kopf.hg!90!black}
		\multicolumn{2}{l}{\code{Spielt}} \\ \hline
		\rowcolor{ab.tabelle.kopf.hg}
		Lokal & Musik\-richtung \\\hline\toprule
		A & Rock \\\midrule
		A & Pop \\\midrule
		B & Latin \\\midrule
		D & Rock \\\bottomrule
	\end{tabularx}
\end{multicols}

\begin{enuma}
	\item Bilde den Vrbund der Tabellen \code{Hört\_gern} und \code{Spielt} über das gemeinsame Attribut \enquote{Musikrichtung} und beschreibe, welche Informationen die ergebnisrelation beinhaltet.

	\item Gib alle Lokale an, die Musikrichtungen spielen, die Tina mag. Entiwckele dann eine entsprechende Abfrage in SQL.

	\item Analysiere die folgenden Abfragen, indem du jeweils die Ergebnisrelation und eine umgangssprachliche Formulierung im Sachkontext für die Abfrage angibst.
	\begin{enumerate}
		\item ~\\[-3em]\begin{minted}{sql}
		SELECT Musikrichtung
		FROM Hört_gern
		  INNER JOIN Geht_in
		    ON Hört_gern.Name = Geht_in.Name
		WHERE Lokal = 'A';
		\end{minted}

		\item ~\\[-3em]\begin{minted}{sql}
		SELECT Musikrichtung
		FROM Geht_in
		  INNER JOIN Spielt
			ON Geht_in.Lokal = Spielt.Lokal
		WHERE Name = 'Maxi';
		\end{minted}

		\item ~\\[-3em]\begin{minted}{sql}
		SELECT Name
		FROM Hört_gern
		  INNER JOIN Spielt
			ON Hört_gern.Musikrichtung = Spielt.Musikrichtung
		  INNER JOIN Geht_in
		    ON Hört_gern.Name = Geht_in.Name
			  AND Spielt.Lokal = Geht_in.Lokal
		\end{minted}
	\end{enumerate}
\end{enuma}
\end{aufgabe}

\end{document}
