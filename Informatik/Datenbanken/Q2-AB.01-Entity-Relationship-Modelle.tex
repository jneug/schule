\documentclass[11pt, a4paper, ngerman]{arbeitsblatt}

\ladeModule{theme,boxen}

\ladeFach[datenbanken]{informatik}

\aboptionen{
	name		= {J. Neugebauer},
	kuerzel 	= {Ngb},
	titel 		= {Entity-Relationship-Modelle},
	reihe 		= {Relationale Datenbanken},
	fach 		= {Informatik},
	kurs 		= {Q2},
	nummer 		= {IV.1},
	lizenz 		= {cc-by-nc-sa-4},
	version 	= {2021-09-05},
}

%\renewcommand{\FussRechts}{}

\ofoot{\FussMitte}
\cfoot{\color{muted}Aufgaben zuerst veröffentlicht von J.\,Dorn unter \ccbyncsa\\\https://wi-wissen.github.io/instahub-doc-de/\#/exercices?id=modellieren-von-daten}

\begin{document}
\ReiheTitel

\begin{aufgabe}
Erstelle ein Entity-Relationship-Modell (ERM) (bestehend aus einem Entity-Relationship-Diagramm
(ERD)) zu dem folgenden Szenario.

Du kannst das Diagramm mit Hilfe des Tools unter \url{https://link.ngb.schule/erdplus} erstellen.

\begin{rahmen}\vspace*{-1em}
	\subsection*{Das Stahlwerk der Postapokalypse}

	Orientierungslos wachen Sie in einer Wüste aus Metall auf. Die Luft ist eiskalt klar.
	Der Boden besteht aus rostenden Stahl. Um Sie herum laufen orientierungslose Menschen,
	welche voller Panik versuchen kleine Stahlplatten und Stangen gegen belegte Butterbrote
	zu tauschen. Sie schleppen sich zu einer großen Halle aus der das Dröhnen von Hämmern zu
	vernehmen ist. Man erzählt Ihnen, dass dies vor dem Krieg das größte Stahlwerk der Welt
	war, aber bei einem Bombenangriff das vollständige ERM vernichtet wurde. Nun ist die
	Werksleitung außer Stande die Mitarbeiter in der Kantine mit Wurstbroten zu versorgen,
	da die Produktion am Boden liegt.

	Hilfe dem Werksleiter ein erstes ERM für die die Werkshalle \enquote{Stahlfuchs} zu entwerfen:

	\begin{quote}\itshape
		In der Halle \enquote{Stahlfuchs} haben wir mehrere Öfen. An jeden Ofen arbeitet genau ein
		Heizer. Mehrere Träger bringen Brennstäbe zum Beheizen der verschiedenen Öfen. Jeder
		Heizer und Träger hat eine Personalnummer. Jeder Ofen ist über seinen Standort genau
		bestimmt.
	\end{quote}

	\begin{enuma}[noitemsep]
		\item Erstelle ein passendes ERM.
		\item Füge dem Träger weitere sinnvolle Attribute hinzu.
		\item Erstelle anhand deines ERM folgende SQL-Abfragen:
		\begin{smallenum}
			\item Gib alle Träger aus.
			\item Gib die Träger aus, deren Personalnummer mit \code{5} beginnt.
			\item Zähle alle Heizer.
		\end{smallenum}
		\smallskip
		\item Zum Dank erhältst du ein Butterbrot und Pfefferminztee und ziehst dich damit für
		eine Pause auf das Dach der Halle zurück. Voller Stauen erblickst du, dass weitere Hallen
		bis an den Horizont reichen. Passe dein ERM so an, dass statt der Halle \enquote{Stahlfuchs}
		die scheinbar unendlich vielen Hallen mit erfasst werden können.
	\end{enuma}
\end{rahmen}
\end{aufgabe}

\newpage
\ReiheTitel

\begin{aufgabe}
Erstelle ein Entity-Relationship-Modell (ERM) (bestehend aus einem Entity-Relationship-Diagramm
(ERD)) zu dem folgenden Szenario.

Du kannst das Diagramm mit Hilfe des Tools unter \url{https://link.ngb.schule/erdplus} erstellen.

\begin{rahmen}\vspace*{-1em}
	\subsection*{Der Bibliothekar}

	Sie sind in einer dunklen Gasse falsch abgebogen und finden sich unversehens in einer
	großen Halle voller Bücher wieder. Gerade als Sie beschließen zu gehen, erhebt sich
	hinter einem Stapel veralteter Lexika ein alter Mann und blickt Sie mit traurigen,
	grauen Augen an. Schon lang ist es her, dass sich ein Besucher in seine Bibliothek
	verirrt hat. Mit reißender Stimme bittet er Sie ihn bei seinem ERM zu helfen, damit
	er bald die Bibliothek für junge Menschen erschließen kann.

	Unter gelegentlichen Schnaufen und Glucksen berichtet er seine Anforderungen an die Datenbank:

	\begin{quote}\itshape
		Ein jedes Buch hat seinen Platz in einem Regal. Dabei wird der Platz in dem Regal durch
		das Regalbrett genauer bestimmt. Alle Regale befinden sich in diesem Saal. Die Bücher
		werden verschiedenen Genres zugeordnet. Beispielsweise ist das Werk \enquote{Der Frosch
		zu Gast bei Helmut} nicht nur eine Komödie, sondern auch ein Krimi.
	\end{quote}

	\begin{enuma}[noitemsep]
		\item Erstelle ein passendes ERM.
		\item Füge den Büchern weitere sinnvolle Attribute hinzu.
		\item Erstelle anhand deines ERM folgende SQL-Abfragen:
		\begin{smallenum}
			\item Gib alle Bücher aus.
			\item Gib die Bücher mit dem Titel \enquote{Der Frosch zu Gast bei Helmut} aus.
			\item Finde alle Bücher die im Titel \enquote{Helmut} tragen.
			\item Zähle alle Bücher.
		\end{smallenum}
		\smallskip
		\item Dich beschleicht ein schrecklicher Verdacht. Hinter einem verschlissenen Vorhang
		befindet sich eine weitere Kammer mit Büchern. Bücher von epochaler Bedeutung für die
		Zukunft des europäischen Schmiedehandwerks. Passe dein ERM so an, dass statt dem einzelnen
		Saal auch die Kammer mit erfasst werden kann.
	\end{enuma}
\end{rahmen}
\end{aufgabe}


\end{document}
