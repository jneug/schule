\documentclass[9pt, a4paper]{scrartcl}

\usepackage{pgfpages}
\pgfpagesuselayout{2 on 1}[a4paper,landscape]

\usepackage{vorschule}
\usepackage[
	typ=ab,
	fach=Informatik,
	lerngruppe={Q2},
	nummer=5,
	module={Symbole,Lizenzen},
	seitenzahlen=keine,
	farbig,
	lizenz=cc-by-nc-sa-4,
]{schule}

\usepackage[
	kuerzel=Ngb,
	reihe={Relationale Datenbanken},
	version={2020-09-01},
]{ngbschule}

\author{J. Neugebauer}
\title{Normalformen}
\date{\Heute}

\setzeAufgabentemplate{ngbohne}


\begin{document}
\ReiheTitel

Ausgangspunkt ist die folgende Tabelle für die Personalverwaltung einer Firma.

\begin{center}
\begin{tabular}{|c|c|c|c|}\hline
	\rowcolor{ngb.tabelle.kopf.hg} Name & AbtNr & Abteilung & Projekte \\\hline
	Müller & 5 & EDV II & (6, MySQL, 50\%), (3, DV2010, 50\%) \\\hline
	Schulze & 3 & Rechenzentrum & (3, DV2010, 100\%) \\\hline
\end{tabular}
\end{center}

Man erkennt deutlich, dass es schwer ist, Teile der Daten zu löschen, einzufügen oder zu ändern. Daher überführt man solche Tabellen in die sogenannten Normalformen nach Codd.

\begin{rahmen}
	\subsubsection*{1. Normalform (1NF)}
	
	Eine Relation befindet sich in der ersten Normalform, wenn alle Attribute einen 
	\emph{atomaren Wertebereich} haben.
	
	\textbf{Jeder Datensatz muss durch den Primärschlüssel eindeutig identifizierbar sein.}
\end{rahmen}

Das Attribut \emph{Projekt} ist nicht atomar, mehrere Werte enthält. Außerdem fehlt ein Primärschlüssel. Wir ergänzen \emph{PersNr} und teilen das Mehrfachattribut in seine atomaren Bestandteile auf. Damit der Primärschlüssel weiterhin
eindeutig ist nehmen wir die Projektnummer in den Schlüssel auf (sonst gäbe es die \emph{PersNr} \enquote{1} zweimal).

\begin{center}
\begin{tabular}{|c|c|c|c|c|c|c|}\hline
	\rowcolor{ngb.tabelle.kopf.hg} \underline{PersNr} & Name & AbtNr & Abteilung & \underline{ProjNr} & Projekt & Zeit \\\hline
	1 & Müller & 5 & EDV II & 6 & MySQL & 50\% \\\hline
	1 & Müller & 5 & EDV II & 3 & DV2010 & 50\% \\\hline
	2 & Schulze & 3 & Rechenzentrum & 3 & DV2010 & 100\% \\\hline
\end{tabular}
\end{center}

Die Tabelle ist nun in der 1NF. Allerdings wurden bei der Umformung zusätzliche Redundanzen eingefügt.

\begin{rahmen}
	\subsubsection*{2. Normalform (2NF)}
	
	Eine Relation befindet sich in der zweiten Normalform, wenn sie in der 1NF ist und jedes Nichtschlüsselattribut funktional abhängig ist vom Gesamtschlüssel, nicht dagegen von Schlüsselteilen.
\end{rahmen}

\emph{Funktional abhängig} heißt, dass die Werte des Attributs durch ein anderes eindeutig festgelegt sind. Besteht
der Primärschlüssel nur aus einem Attribut ist dies immer der Fall. Im Beispiel zeigt sich, dass
\emph{Name}, \emph{AbtNr}, und \emph{Abteilung} nur durch \emph{PersNr} festgelegt werden, nicht aber 
von \emph{ProjNr} abhängig sind. Daher werden die drei Attribute mit dem Primärschlüssel \emph{PersNr}
in eine neue Tabelle ausgelagert. Beim Rest sieht man, dass \emph{Projekt} eindeutig durch \emph{ProjNr} festgelegt wird. Auch hier erstellen wir eine neue Tabelle.

\begin{center}
\begin{tabular}{|c|c|c|c|}\hline
	\rowcolor{ngb.tabelle.kopf.hg} \underline{PersNr} & Name & AbtNr & Abteilung \\\hline
	1 & Müller & 5 & EDV II \\\hline
	1 & Müller & 5 & EDV II \\\hline
	2 & Schulze & 3 & Rechenzentrum \\\hline
\end{tabular}
\end{center}

\begin{multicols}{2}
\begin{center}
\begin{tabular}{|c|c|c|}\hline
	\rowcolor{ngb.tabelle.kopf.hg} \underline{PersNr} & \underline{ProjNr} & Zeit \\\hline
	1 & 6 & 50\% \\\hline
	1 & 3 & 50\% \\\hline
	2 & 3 & 100\% \\\hline
\end{tabular}
\end{center}

\begin{center}
\begin{tabular}{|c|c|}\hline
	\rowcolor{ngb.tabelle.kopf.hg} \underline{ProjNr} & Projekt \\\hline
	6 & MySQL \\\hline
	3 & DV2010 \\\hline
\end{tabular}
\end{center}
\end{multicols}

\newpage 

\begin{rahmen}
	\subsubsection*{3. Normalform (3NF)}
	
	Eine Relation befindet sich in der dritten Normalform, wenn sie in der 2NF ist und es kein Nichtschlüsselattribut gibt, das \emph{transitiv abhängig} von einem Schlüsselattribut ist.
	
	Es darf also keine funnktionalen Abhängigkeiten zwischen Nichtschlüsselattributen geben.
\end{rahmen}

Das Attribut \emph{Abteilung} ist nur indirekt vom Primärschlüssel \emph{PersNr} abhängig. Vielmehr ist das Attribut \emph{Abteilung} von der \emph{AbtNr} abhängig. Also Aufspaltung in zwei Tabellen:

\begin{multicols}{2}
\begin{center}
\begin{tabular}{|c|c|c|}\hline
	\rowcolor{ngb.tabelle.kopf.hg} \underline{PersNr} & Name & AbtNr \\\hline
	1 & Müller & 5 \\\hline
	2 & Schulze & 3 \\\hline
\end{tabular}
\end{center}

\begin{center}
\begin{tabular}{|c|c|}\hline
	\rowcolor{ngb.tabelle.kopf.hg} \underline{AbtNr} & Abteilung \\\hline
	5 & EDV II \\\hline
	3 & Rechenzentrum \\\hline
\end{tabular}
\end{center}
\end{multicols}


\subsection*{Zusammenfassung}

Durch die Transformation in die dritte Normalform wurde also aus der Tabelle

\begin{center}
\begin{tabular}{|c|c|c|c|}\hline
	\rowcolor{ngb.tabelle.kopf.hg} Name & AbtNr & Abteilung & Projekte \\\hline
	Müller & 5 & EDV II & (6, MySQL, 50\%), (3, DV2010, 50\%) \\\hline
	Schulze & 3 & Rechenzentrum & (3, DV2010, 100\%) \\\hline
\end{tabular}
\end{center}

diese Aufteilung vorgenommen:

\begin{multicols}{2}
\begin{center}
\begin{tabular}{|c|c|c|}\hline
	\rowcolor{ngb.tabelle.kopf.hg} \underline{PersNr} & Name & AbtNr \\\hline
	1 & Müller & 5 \\\hline
	2 & Schulze & 3 \\\hline
\end{tabular}
\end{center}

\begin{center}
\begin{tabular}{|c|c|}\hline
	\rowcolor{ngb.tabelle.kopf.hg} \underline{AbtNr} & Abteilung \\\hline
	5 & EDV II \\\hline
	3 & Rechenzentrum \\\hline
\end{tabular}
\end{center}

\begin{center}
\begin{tabular}{|c|c|c|}\hline
	\rowcolor{ngb.tabelle.kopf.hg} \underline{PersNr} & \underline{ProjNr} & Zeit \\\hline
	1 & 6 & 50\% \\\hline
	1 & 3 & 50\% \\\hline
	2 & 3 & 100\% \\\hline
\end{tabular}
\end{center}

\begin{center}
\begin{tabular}{|c|c|}\hline
	\rowcolor{ngb.tabelle.kopf.hg} \underline{ProjNr} & Projekt \\\hline
	6 & MySQL \\\hline
	3 & DV2010 \\\hline
\end{tabular}
\end{center}
\end{multicols}


\end{document}
