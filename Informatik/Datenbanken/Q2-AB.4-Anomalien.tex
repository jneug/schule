\documentclass[10pt, a5paper,landscape]{scrartcl}

\usepackage{pgfpages}
\pgfpagesuselayout{2 on 1}[a4paper]

\usepackage{vorschule}
\usepackage[
	typ=ab,
	fach=Informatik,
	lerngruppe={Q2},
	nummer=4,
	module={Symbole,Lizenzen},
	seitenzahlen=keine,
	farbig,
	lizenz=cc-by-nc-sa-4,
]{schule}

\usepackage[
	kuerzel=Ngb,
	reihe={Relationale Datenbanken},
	version={2020-08-27},
]{ngbschule}

\author{J. Neugebauer}
\title{Anomalien}
\date{\Heute}

\setzeAufgabentemplate{ngbohne}


\begin{document}
\ReiheTitel

In der Datenbank eines Autohändlers finden sich folgende Daten:

{\footnotesize
\begin{tabular}{|*{11}{c|}}\hline
	\rowcolor{ngb.tabelle.kopf.hg} ID & Name & Vorname & Straße & Nr. & PLZ & Stadt & Marke & Baujahr & Kaufpreis & Kaufdatum \\ \hline
	512 & Heinrichs & Marko & Basaltweg & 18 & 54274 & Grünhausen & Audi A3 & 2001 & 12000 & 07.08.2003 \\\hline
	2314 & Heinrichs & Marko & Basaltweg & 18 & 54274 & Grünhausen & Audi A4 & 2006 & 16000 & 05.03.2008 \\\hline
	3290 & Heinrichs & Marko & Basaltweg & 18 & 54274 & Grünhausen & VW Touran & 2014 & 13450 & 01.06.2018 \\\hline
\end{tabular}}

Herr Heinrichs teilt dem Händler eine Adressänderung mit. Er wohnt nun an der Adresse \enquote{Am Kreuz 15, 54331 Grünhausen}.

Nachdem die Änderungen eingetragen wurden sieht die Datenbank so aus:

{\footnotesize
\begin{tabular}{|*{11}{c|}}\hline
	\rowcolor{ngb.tabelle.kopf.hg} ID & Name & Vorname & Straße & Nr. & PLZ & Stadt & Marke & Baujahr & Kaufpreis & Kaufdatum \\ \hline
	512 & Heinrichs & Marko & Basaltweg & 18 & 54274 & Grünhausen & Audi A3 & 2001 & 12000 & 07.08.2003 \\\hline
	2314 & Heinrichs & Marko & Basaltweg & 18 & 54274 & Grünhausen & Audi A4 & 2006 & 16000 & 05.03.2008 \\\hline
	3290 & Heinrichs & Marko & Am Kreuz & 15 & 54331 & Grünhausen & VW Touran & 2014 & 13450 & 01.06.2018 \\\hline
\end{tabular}}

\vspace{1em}
\begin{aufgabe}
\begin{enumerate}
	\item Untersuche die Datenbankausschnitte. Wo könnte ein Problem (eine Anomalie) auftauchen?
	\item Beschreibe die Anomalie möglichst genau. Was ist ihre Ursache?
	\item Wie könnte so eine Anomalie in Zukunft vermieden werden?
\end{enumerate}
\end{aufgabe}

\clearpage
\ReiheTitel

Frau Peters bestellt sich jedes Jahr einen Kalender in einem Onlinehandel. In der Datenbank finden sich folgende Daten:

{\footnotesize
\begin{tabular}{|*{10}{c|}}\hline
	\rowcolor{ngb.tabelle.kopf.hg} ID & Name & Vorname & Straße & Nr. & PLZ & Stadt & Geburtsdatum & Bestelldatum & Artikelnr. \\ \hline
	332 & Peters & Simone & Krähensteig & 4 & 32324 & Warendorf & 27.02.1982 & 04.04.2018 & 92002019 \\\hline
\end{tabular}}

Zur Sicherheit löscht der Händler nach einem Jahr das Bestelldatum aus der Datenbank, um keine Vorratsdaten zu sammeln.

Frau Peters ärgert es, dass sie jedes Jahr aufs neue ihre Adresse eingeben muss.

\vspace{1em}
\begin{aufgabe}
\begin{enumerate}
	\item Untersuche die Datenbankausschnitte. Warum muss Frau Peters jedes Jahr ihre Daten neu eingeben?
	\item Wie könnte der Händler die erneute Eingabe verhindern, aber dennoch die Details der Bestellung aus der Datenbank löschen?
	\item Was passiert, wenn Frau Peters öfters als einmal im Jahr etwas bestellt? 
\end{enumerate}
\end{aufgabe}

\clearpage
\ReiheTitel

In der Datenbank eines Buchhändlers finden sich folgende Daten:

{\footnotesize
\begin{tabular}{|*{7}{c|}}\hline
	\rowcolor{ngb.tabelle.kopf.hg} ID & Name & Vorname & Kaufdatum & Titel & ISBN & Autor\\ \hline
	112 & Schneider & Harry & 14.09.2018 & Harry Potter und ... & 978-3-6... & J. K. Rowling \\\hline
	154 & Brünger & Peter & 21.01.2019 & Der Herr der Ringe & 978-1-4.. & J. R. R. Tolkien \\\hline
	188 & Bleimund & Lisa & 25.04.2019 &  Harry Potter und ... & 978-3-6... & J. K. Rowling \\\hline
\end{tabular}}

Nachdem der neue Roman von Stephen King erschienen ist, muss ein Mitarbeiter das Buch einpflegen.

Nachdem die Änderungen eingetragen wurden sieht die Datenbank so aus:

{\footnotesize
\begin{tabular}{|*{7}{c|}}\hline
	\rowcolor{ngb.tabelle.kopf.hg} ID & Name & Vorname & Kaufdatum & Titel & ISBN & Autor\\ \hline
	112 & Schneider & Harry & 14.09.2018 & Harry Potter und ... & 978-3-6... & J. K. Rowling \\\hline
	154 & Brünger & Peter & 21.01.2019 & Der Herr der Ringe & 978-1-4.. & J. R. R. Tolkien \\\hline
	188 & Bleimund & Lisa & 25.04.2019 & Harry Potter und ... & 978-3-6... & J. K. Rowling \\\hline
	213 & XXXXX & YYYYYY & 00.00.0000 & Blutige Nachrichten & 978-3-4.. & Stephen King \\\hline
\end{tabular}}

\vspace{1em}
\begin{aufgabe}
\begin{enumerate}
	\item Untersuche die Datenbankausschnitte. Wo könnte ein Problem (eine Anomalie) stecken?
	\item Beschreibe die Anomalie möglichst genau. Was ist ihre Ursache?
	\item Wie könnte so eine Anomalie in Zukunft vermieden werden?
\end{enumerate}
\end{aufgabe}

\end{document}
