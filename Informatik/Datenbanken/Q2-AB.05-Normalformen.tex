\documentclass[9pt,a4paper]{arbeitsblatt}

\ladeModule{theme,tabellen,boxen}
\aboptionen{
	name 		= {J. Neugebauer},
	kuerzel 	= {Ngb},
	fach		= {Informatik},
	lerngruppe	= {Q2},
	reihe		= {Relationale Datenbanken},
	titel		= {Normalformen},
	version		= {2021-09-05},
	nummer		= {IV.5},
	lizenz		= {cc-by-nc-sa-4},
	wechselnde zeilenfarben
}
\ladeFach[datenbanken]{informatik}

\begin{document}
\ReiheTitel

Ausgangspunkt ist die folgende Tabelle für die Personalverwaltung einer Firma.

\begin{center}
	\begin{tabular}{|c|c|c|c|}\hline
		\rowcolor{ab.tabelle.kopf.hg} Name & AbtNr & Abteilung     & Projekte                            \\\hline
		Müller                             & 5     & EDV II        & (6, MySQL, 50\%), (3, DV2010, 50\%) \\\hline
		Schulze                            & 3     & Rechenzentrum & (3, DV2010, 100\%)                  \\\hline
	\end{tabular}
\end{center}

Man erkennt deutlich, dass es schwer ist, Teile der Daten zu löschen,
einzufügen oder zu ändern. Daher überführt man solche Tabellen in die
sogenannten \emph{Normalformen nach Codd}.

\begin{infobox}\vspace*{-1em}
	\subsubsection*{1. Normalform (1NF)}

	Eine Relation befindet sich in der ersten Normalform, wenn alle Attribute einen
	\emph{atomaren Wertebereich} haben.

	\textbf{Jeder Datensatz muss durch den Primärschlüssel eindeutig identifizierbar sein.}
\end{infobox}

Das Attribut \emph{Projekt} ist nicht atomar, da es mehrere Werte
enthält. Außerdem fehlt ein Primärschlüssel. Wir ergänzen
\emph{PersNr} und teilen das Mehrfachattribut in seine atomaren
Bestandteile auf. Damit der Primärschlüssel weiterhin eindeutig ist, nehmen wir
\emph{ProjNr} in den Schlüssel auf (sonst gäbe es die
\emph{PersNr} \enquote{1} zweimal).

\begin{center}
	\begin{tabular}{|c|c|c|c|c|c|c|}\hline
		\rowcolor{ab.tabelle.kopf.hg} \primarykey{PersNr} & Name    & AbtNr & Abteilung     & \primarykey{ProjNr} & Projekt & Zeit  \\\hline
		1                                                 & Müller  & 5     & EDV II        & 6                   & MySQL   & 50\%  \\\hline
		1                                                 & Müller  & 5     & EDV II        & 3                   & DV2010  & 50\%  \\\hline
		2                                                 & Schulze & 3     & Rechenzentrum & 3                   & DV2010  & 100\% \\\hline
	\end{tabular}
\end{center}

Die Tabelle ist nun in der 1NF. Allerdings wurden bei der Umformung zusätzliche
Redundanzen eingefügt.

\begin{infobox}\vspace*{-1em}
	\subsubsection*{2. Normalform (2NF)}

	Eine Relation befindet sich in der zweiten Normalform, wenn sie in der 1NF ist
	und jedes Nichtschlüsselattribut \emph{funktional abhängig} ist vom
	Gesamtschlüssel, nicht dagegen von Schlüsselteilen.
\end{infobox}

\emph{Funktional abhängig} heißt, dass die Werte des Attributs durch ein anderes
eindeutig festgelegt sind. Besteht der Primärschlüssel nur aus einem Attribut
ist dies immer der Fall. Im Beispiel zeigt sich, dass \emph{Name},
\emph{AbtNr}, und \emph{Abteilung} nur durch
\emph{PersNr} festgelegt werden, nicht aber von \emph{ProjNr}
abhängig sind. Daher werden die drei Attribute mit dem Primärschlüssel
\emph{PersNr} in eine neue Tabelle ausgelagert. Beim Rest sieht man,
dass \emph{Projekt} eindeutig durch \emph{ProjNr} festgelegt
wird. Auch hier erstellen wir eine neue Tabelle.

\begin{center}
	\begin{tabular}{|c|c|c|}\hline
		\rowcolor{ab.tabelle.kopf.hg}
		\fpkey{PersNr} & \fpkey{ProjNr} & Zeit  \\\hline
		1              & 6              & 50\%  \\\hline
		1              & 3              & 50\%  \\\hline
		2              & 3              & 100\% \\\hline
	\end{tabular}
\end{center}

\begin{multicols}{2}
	\begin{center}
		\begin{tabular}{|c|c|c|c|}\hline
			\rowcolor{ab.tabelle.kopf.hg}
			\primarykey{PersNr} & Name    & AbtNr & Abteilung     \\\hline
			1                   & Müller  & 5     & EDV II        \\\hline
			1                   & Müller  & 5     & EDV II        \\\hline
			2                   & Schulze & 3     & Rechenzentrum \\\hline
		\end{tabular}
	\end{center}

	\begin{center}
		\begin{tabular}{|c|c|}\hline
			\rowcolor{ab.tabelle.kopf.hg}
			\primarykey{ProjNr} & Projekt \\\hline
			6                   & MySQL   \\\hline
			3                   & DV2010  \\\hline
		\end{tabular}
	\end{center}
\end{multicols}

\newpage

\begin{infobox}\vspace*{-1em}
	\subsubsection*{3. Normalform (3NF)}

	Eine Relation befindet sich in der dritten Normalform, wenn sie in der 2NF ist
	und es kein Nichtschlüsselattribut gibt, das \emph{transitiv abhängig} von einem
	Schlüsselattribut ist.

	Es darf also keine \emph{funnktionalen Abhängigkeiten} zwischen Nichtschlüsselattributen
	geben.
\end{infobox}

Das Attribut \emph{Abteilung} ist nur indirekt vom Primärschlüssel
\emph{PersNr} abhängig. Vielmehr ist das Attribut
\emph{Abteilung} von der \emph{AbtNr} abhängig. Also
Aufspaltung in zwei Tabellen:

\begin{multicols}{2}
	\begin{center}
		\begin{tabular}{|c|c|c|}\hline
			\rowcolor{ab.tabelle.kopf.hg}
			\primarykey{PersNr} & Name    & \foreignkey{AbtNr} \\\hline
			1                   & Müller  & 5                  \\\hline
			2                   & Schulze & 3                  \\\hline
		\end{tabular}
	\end{center}

	\begin{center}
		\begin{tabular}{|c|c|}\hline
			\rowcolor{ab.tabelle.kopf.hg}
			\primarykey{AbtNr} & Abteilung     \\\hline
			5                  & EDV II        \\\hline
			3                  & Rechenzentrum \\\hline
		\end{tabular}
	\end{center}
\end{multicols}


\subsection*{Zusammenfassung}

Durch die Transformation in die dritte Normalform wurde also aus der Tabelle

\begin{center}
	\begin{tabular}{|c|c|c|c|}\hline
		\rowcolor{ab.tabelle.kopf.hg}
		Name    & AbtNr & Abteilung     & Projekte                            \\\hline
		Müller  & 5     & EDV II        & (6, MySQL, 50\%), (3, DV2010, 50\%) \\\hline
		Schulze & 3     & Rechenzentrum & (3, DV2010, 100\%)                  \\\hline
	\end{tabular}
\end{center}

diese Aufteilung vorgenommen:

\begin{multicols}{2}
	\begin{center}
		\begin{tabular}{|c|c|c|}\hline
			\rowcolor{ab.tabelle.kopf.hg}
			\primarykey{PersNr} & Name    & \foreignkey{AbtNr} \\\hline
			1                   & Müller  & 5                  \\\hline
			2                   & Schulze & 3                  \\\hline
		\end{tabular}
	\end{center}

	\begin{center}
		\begin{tabular}{|c|c|}\hline
			\rowcolor{ab.tabelle.kopf.hg}
			\primarykey{AbtNr} & Abteilung     \\\hline
			5                  & EDV II        \\\hline
			3                  & Rechenzentrum \\\hline
		\end{tabular}
	\end{center}

	\begin{center}
		\begin{tabular}{|c|c|c|}\hline
			\rowcolor{ab.tabelle.kopf.hg} \fpkey{PersNr} & \fpkey{ProjNr} & Zeit  \\\hline
			1                                            & 6              & 50\%  \\\hline
			1                                            & 3              & 50\%  \\\hline
			2                                            & 3              & 100\% \\\hline
		\end{tabular}
	\end{center}

	\begin{center}
		\begin{tabular}{|c|c|}\hline
			\rowcolor{ab.tabelle.kopf.hg}
			\primarykey{ProjNr} & Projekt \\\hline
			6                   & MySQL   \\\hline
			3                   & DV2010  \\\hline
		\end{tabular}
	\end{center}
\end{multicols}

Das Ergebnis enthält alle Informationen und Relationen der ersten Tabelle, aber
verringert die Gefahr des Auftretens von Anomalien.

\clearpage
\ReiheTitel
\large

\begin{aufgabe}
	Ein kleines Belegkrankenhaus stellt auf EDV um. Die Patientendaten enthalten
	die folgenden Attribute:
	\begin{description}
		\item[Parient] (Name, Gebdatum, Adresse, Datum1, Blutdruck1,
			Diagnosen1 (mehrere), Therapien1 (mehrere), Datum2, Blutdruck2, Diagnosen2
			(mehrere), Therapien2 (mehrere), ... , Krankenkasse, Station, Raum)
	\end{description}
	\hinweis{Für einen Blutdruck-Messwert benötigt man Systol-Wert und DiastolWert.}

	Überführen Sie die Tabelle in die erste, zweite und dritte Normalform nach
	Codd.
\end{aufgabe}

\begin{aufgabe}
	Hugo Unbedarft besitzt eine große Spedition. Er will seine Auftragsverwaltung
	auf EDV umstellen und macht sich dazu einen genauen Plan.

	Seine Aufträge sind immer so, dass sie nur zu einem Ziel führen, es kann
	allerdings möglich sein, dass mehrere LKWs für einen Auftrag nötig sind. Nicht
	jeder LKW-Typ ist dazu geeignet, alle Ziele zu erreichen (z.B. zu niedrige
	Brücken), und nicht jeder Fahrer kann jeden LKW-Typ fahren.

	Hugo will folgende Daten speichern: AuftragsNr. und LKW-Nr., Ziel,
	Zielentfernung, Auftragsdatum, LKW-Typ, max. Zuladung eines LKW-Typs,
	TÜV-Datum, Fahrer-Nr, Fahrer-Name.

	Erstellen Sie ein ER-Modell für die Spedition. Überführen Sie dieses Modell
	anschließend in das relationale Modell und normalisieren Sie Ihre Tabellen von
	der ersten bis zur dritten Normalform.
\end{aufgabe}

\begin{aufgabe}
	Überführen Sie das folgende ERM der KFZ-Zulassungsstelle zuerst in das
	Relationale Modell. Überführen Sie dieses Modell anschließend Schritt für
	Schritt in die erste bis dritte Normalform.

	\begin{center}
	\begin{tikzpicture}[erd]
		\node[entity](Eigentuemer) at (-5,0) {Eigentümer};

		\node[attribute](Geburtsdatum) at (-6,-1) {Geburtsdatum};
		\node[attribute](Vorname) at (-6,1) {Vorname};
		\node[attribute](Name) at (-4,1) {Name};

		\draw (Vorname) -- (Eigentuemer) -- (Name);
		\draw (Eigentuemer) -- (Geburtsdatum);


		\node[entity](KFZ) at (5,0) {KFZ};

		\node[attribute](Baujahr) at (5,1) {Baujahr};
		\node[attribute](FGNr) at (6.5,0) {FGNr};

		\draw (Baujahr) -- (KFZ) -- (FGNr);


		\node[entity](Typklasse) at (0,-4) {Typklasse};

		\node[attribute](Typcode) at (-2.5,-3.5) {Typcode};
		\node[attribute](Modell) at (-2.5,-4.5) {Modell};
		\node[attribute](Hersteller) at (0,-3) {Hersteller};
		\node[attribute](Schadstoff) at (0,-5) {Schadstoff};

		\draw (Typcode) -- (Typklasse) -- (Modell);
		\draw (Hersteller) -- (Typklasse) -- (Schadstoff);


		\node[relation](besitzt) at (0,0) {besitzt};

		\node[attribute](Kennzeichen) at (0,-1) {Kennzeichen};
		\node[attribute](Anmeldedatum) at (1.5,1.5) {Anmelde\-datum};
		\node[attribute](Abmeldedatum) at (-1,1) {Abmelde\-datum};

		\draw (Kennzeichen) -- (besitzt);
		\draw (Abmeldedatum) -- (besitzt) -- (Anmeldedatum);

		\node[relation](ist ein) at (5,-4) {ist ein};

		\draw (Eigentuemer) --node[above]{1} (besitzt) --node[above]{n} (KFZ)
			--node[right]{n} (ist ein) --node[above]{1} (Typklasse);
	\end{tikzpicture}
	\end{center}
\end{aufgabe}

\end{document}
