\documentclass[10pt, a4paper]{arbeitsblatt}

\ladeModule{theme}
\aboptionen{
	name 		= {J. Neugebauer},
	kuerzel 	= {Ngb},
	titel 		= {ERM transformieren},
	reihe 		= {Relationale Datenbanken},
	fach 		= {Informatik},
	kurs 		= {Q2},
	nummer 		= {IV.3},
	lizenz 		= {cc-by-nc-sa-4},
	version 	= {2021-09-02},
}
\ladeFach[datenbanken]{informatik}

\begin{document}
\ReiheTitel

Wenn der erste Entwurf einer Datenbank in Form eines ER-Diagramms erstellt wurde, muss dieses im nächsten Schritt
in ein \emph{Relationenschema} transformiert werden. Ein \emph{Relationenschema} beschreibt die \emph{Relationen}
(d.h. Tabellen), aus denen die relationale Datenbank am Ende bestehen soll.

Eine Relation wird in einem festen Format beschrieben:
\begin{center}\rmfamily
Name(\primarykey{Schlüsselattribut}, Attribut 1, Attribut2, ..., \foreignkey{Fremdschlüssel})
\end{center}


Die Transformation findet in vier Schritten (nach vier \emph{Regeln}) statt:
\begin{enumerate}
	\item\begin{multicols}{2}
		Jede \emph{Entität} mit ihren \emph{Attributen} wird eine Relation.

		\begin{center}
		\begin{tikzpicture}
			\node (Buch) at (0,0) [entity] {Buch};
			\node (ISBN) at (-2,0) [attribute] {\primarykey{ISBN}};
			\node (Titel) at (2,0) [attribute] {Titel};
			\draw (Buch) -- (ISBN);
			\draw (Buch) -- (Titel);
		\end{tikzpicture}
		\end{center}

		\code{Buch(\primarykey{ISBN}, Titel)}
	\end{multicols}

	\item\begin{multicols}{2}
		 Jede \code{m:n}-Beziehung wird in eine \emph{eigenständige Relation} überführt. Als
		 Attribute bekommt die neue Relation die Primärschlüssel der beiden verknüpften Entitäten.
		 Die Menge dieser Schlüsselattribute bildet den Primärschlüssel dieser neuen Relation.
		 Zusätzlich werden die Attribute der Relation aufgenommen.

		\begin{center}
		\begin{tikzpicture}[node distance=1cm]
			\node (Buch)  at (-2,0) [entity] {Buch};
			\node (ISBN)  at (-2,1) [attribute] {\primarykey{ISBN}};
			\node (Titel)  at (-2,-1) [attribute] {Titel};
			\draw (Buch) -- (ISBN);
			\draw (Buch) -- (Titel);

			\node (kauft) at (0,0) [relation] {kauft};
			\node (Anzahl) at (0,1) [attribute] {Anzahl};
			\draw (kauft) -- (Anzahl);

			\node (Kunde) at (2,0) [entity] {Kunde};
			\node (KID) at (2,1) [attribute] {\primarykey{KID}};
			\node (Name) at (2,-1) [attribute] {Name};
			\draw (Kunde) -- (KID);
			\draw (Kunde) -- (Name);

			\draw (Buch) --node[above]{m} (kauft) --node[above]{n} (Kunde);
		\end{tikzpicture}
		\end{center}

		\code{kauft(\foreignkey{\primarykey{KID}}, \foreignkey{\primarykey{ISBN}}, Anzahl)}
	\end{multicols}

	\item\begin{multicols}{2}
		 Jede \code{1:n}-Beziehung wird ohne eigene Tabelle abgebildet. Statt dessen wird
		 der Entität mit der Kardinalität \code{n} der \emph{Primärschlüssel} der anderen
		 Entität als Attribut hinzugefügt.

		\begin{center}
		\begin{tikzpicture}[node distance=1cm]
			\node (Buch)  at (-2,0) [entity] {Buch};
			\node (ISBN)  at (-2,1) [attribute] {\primarykey{ISBN}};
			\node (Titel)  at (-2,-1) [attribute] {Titel};
			\draw (Buch) -- (ISBN);
			\draw (Buch) -- (Titel);

			\node (verlegt) at (0,0) [relation] {verlegt};

			\node (Verlag) at (2,0) [entity] {Verlag};
			\node (VID) at (2,1) [attribute] {\primarykey{VID}};
			\node (Name) at (2,-1) [attribute] {Name};
			\draw (Verlag) -- (VID);
			\draw (Verlag) -- (Name);

			\draw (Buch) --node[above]{n} (verlegt) --node[above]{1} (Verlag);
		\end{tikzpicture}
		\end{center}

		\code{Buch(\primarykey{ISBN}, Titel, \foreignkey{VID})}
	\end{multicols}

	\item\begin{multicols}{2}
		Jede \code{1:1}-Beziehung wird ohne eigene Tabelle abgebildet. Dazu wird der Primärschlüssel
		\emph{einer} Entität der Relation der anderen als Attribut hinzugefügt. \columnbreak

		\begin{center}
		\begin{tikzpicture}[node distance=1cm]
			\node (Mitarbeiter)  at (-2.5,0) [entity] {Mitarbeiter};
			\node (MID)  at (-2.5,1) [attribute] {\primarykey{MID}};
			\node (Nachname)  at (-2.5,-1) [attribute] {Nachname};
			\draw (Mitarbeiter) -- (MID);
			\draw (Mitarbeiter) -- (Nachname);

			\node (ceo) at (0,0) [relation] {ist CEO};

			\node (Verlag) at (2,0) [entity] {Verlag};
			\node (VID) at (2,1) [attribute] {\primarykey{VID}};
			\node (Name) at (2,-1) [attribute] {Name};
			\draw (Verlag) -- (VID);
			\draw (Verlag) -- (Name);

			\draw (Mitarbeiter) --node[above]{1} (ceo) --node[above]{1} (Verlag);
		\end{tikzpicture}
		\end{center}

		\code{Mitarbeiter(\primarykey{MID}, Nachname, \foreignkey{VID})}

		\emph{oder}

		\code{Verlag(\primarykey{VID}, Name, \foreignkey{MID})}
		\end{multicols}
\end{enumerate}

\begin{aufgabe}
	Transformiere die drei ER-Diagramme mit den vier Regeln jeweils in ein \emph{Relationenschema}.

	\begin{teilaufgaben}
		\teilaufgabe
		\begin{tikzpicture}[baseline=(current bounding box.north)]
			\node (Stadt)  at (-2.5,0) [entity] {Stadt};
			\node (StadtID)  at (-3.5,1) [attribute] {\primarykey{Stadt\textunderscore ID}};
			\node (Name)  at (-1.5,1) [attribute] {Name};
			\draw (Stadt) -- (StadtID);
			\draw (Stadt) -- (Name);


			\node (steht) at (0,0) [relation] {steht};

			\node (Auto)  at (2.5,0) [entity] {Auto};
			\node (AutoID)  at (1.5,1) [attribute] {\primarykey{Auto\textunderscore ID}};
			\node (Farbe)  at (3.5,1) [attribute] {Farbe};
			\node (Leistung)  at (4.5,0.2) [attribute] {Leistung};
			\draw (Auto) -- (AutoID);
			\draw (Auto) -- (Farbe);
			\draw (Auto) -- (Leistung);

			\draw (Stadt) --node[above]{1} (steht) --node[above]{n} (Auto);
		\end{tikzpicture}
		\bigskip

		\teilaufgabe
		\begin{tikzpicture}[baseline=(current bounding box.north)]
			\node (Dönerladen)  at (-3.5,0) [entity] {Dönerladen};
			\node (id)  at (-4.5,1) [attribute] {\primarykey{id}};
			\node (Standort)  at (-2.5,1) [attribute] {Standort};
			\draw (Dönerladen) -- (id);
			\draw (Dönerladen) -- (Standort);

			\node (verkauft) at (0,0) [relation] {verkauft};

			\node (Produkt)  at (3.5,0) [entity] {Produkt};
			\node (id)  at (2.5,1) [attribute] {\primarykey{id}};
			\node (Name)  at (4.5,1) [attribute] {Name};
			\draw (Produkt) -- (id);
			\draw (Produkt) -- (Name);

			\draw (Dönerladen) --node[above]{n} (verkauft) --node[above]{m} (Produkt);
		\end{tikzpicture}
		\bigskip

		\teilaufgabe
		\begin{tikzpicture}[baseline=(current bounding box.north)]
			\node (Kunde) at (-3.5,0) [entity] {Kunde};
			\node (id)  at (-5,0) [attribute] {\primarykey{id}};
			\node (Nachname)  at (-4.5,1) [attribute] {Nachname};
			\node (Vorname)  at (-2,1) [attribute] {Vorname};
			\draw (Kunde) -- (id);
			\draw (Kunde) -- (Nachname);
			\draw (Kunde) -- (Vorname);

			\node (wohnt in) at (-.5,0) [relation] {wohnt in};

			\node (Ort) at (2,0) [entity] {Ort};
			\node (Postleitzahl)  at (4.5,.5) [attribute] {\primarykey{Postleitzahl}};
			\node (Name)  at (4,-.5) [attribute] {Name};
			\draw (Ort) -- (Postleitzahl);
			\draw (Ort) -- (Name);

			\draw (Kunde) --node[above]{n} (wohnt in) --node[above]{1} (Ort);

			\node (Bundesland) at (2,3) [entity] {Bundesland};
			\node (BundeslandId)  at (0,3) [attribute] {\primarykey{id}};
			\node (BundeslandName)  at (2,4) [attribute] {Name};
			\draw (Bundesland) -- (BundeslandId);
			\draw (Bundesland) -- (BundeslandName);

			\node (befindet sich in) at (2,1.5) [relation] {\small befindet sich in};

			\draw (Bundesland) --node[left]{1} (befindet sich in) --node[right]{n} (Ort);

			\node (Rabattstatus) at (-3.5,-4) [entity] {Rabattstatus};
			\node (RabattstatusId)  at (-5.5,-4) [attribute] {\primarykey{id}};
			\node (RabattstatusBezeichnung)  at (-4.5,-5.1) [attribute] {Bezeichnung};
			\node (RabattstatusProzent)  at (-2,-5.1) [attribute] {Prozent};
			\draw (Rabattstatus) -- (RabattstatusId);
			\draw (Rabattstatus) -- (RabattstatusBezeichnung);
			\draw (Rabattstatus) -- (RabattstatusProzent);

			\node (hat) at (-3.5,-2) [relation] {hat};

			\draw (Kunde) --node[left]{n} (hat) --node[right]{1} (Rabattstatus);

			\node (Spedition) at (2,-4) [entity] {Spedition};
			\node (SpeditionId)  at (3.5,-4) [attribute] {\primarykey{id}};
			\node (SpeditionStrasse)  at (1,-5.1) [attribute] {Straße};
			\node (SpeditionName)  at (3.5,-5.1) [attribute] {Name};
			\draw (Spedition) -- (SpeditionId);
			\draw (Spedition) -- (SpeditionStrasse);
			\draw (Spedition) -- (SpeditionName);

			\node (befindet sich in 2) at (2,-2) [relation] {\small befindet sich in};

			\draw (Ort) --node[left]{1} (befindet sich in 2) --node[right]{n} (Spedition);

			\node (bevorzugt) at (-1,-2) [relation] {bevorzugt};

			\draw (Kunde) --node[right]{n} (bevorzugt) --node[left]{m} (Spedition);
		\end{tikzpicture}
	\end{teilaufgaben}
\end{aufgabe}

\end{document}
