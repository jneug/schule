\documentclass[10pt, a4paper]{scrartcl}

\usepackage{vorschule}
\usepackage[
    typ=ab,
    fach=Informatik,
    lerngruppe={9Diff},
    nummer=4,
    module={Symbole,Lizenzen},
    seitenzahlen=keine,
    farbig,
    lizenz=cc-by-nc-sa-4,
]{schule}

\usepackage[
	kuerzel=Ngb,
	reihe={Der KnowHowPC},
	version={2019-10-07},
]{ngbschule}

\author{J. Neugebauer}
\title{Multiplikation}
\date{\Heute}

\setzeAufgabentemplate{ngbnormal}

%\usepackage{qrcode}



\begin{document}

\ReiheTitel

\begin{aufgabe}
	Erstelle im \emph{KnowHowPC} ein Programm, das folgendes leistet:
	
	\begin{itemize}
		\item Es berechnet R1 $:=$ R1 $\cdot$ R2.
		\item Hilfsregister werden zunächst gelöscht.
		\item Wenn das Programm endet, sind die Hilfsregister und R2 wieder leer.
	\end{itemize}
	
	\subsubsection*{Hilfestellung: Programm-Puzzle}
	Zur Hilfe kannst du diese Liste an Befehlen nutzen. Bringe sie in die richtige Reihenfolge, um die Aufgabe zu lösen.
	\begin{multicols}{4}\setlength{\parskip}{2ex}\ttfamily
		isz 3
		
		jmp 4
		
		jmp 7
		
		dec 3
		
		jmp 10
		
		jmp 7
		
		jmp 24
		
		jmp 31
		
		jmp 41
		
		dec 3
		
		jmp 13
		
		inc 2
		
		jmp 13
		
		jmp 1
		
		isz 4
		
		jmp 19
		
		inc 4
		
		isz 3
		
		inc 1
		
		jmp 34
		
		isz 2
		
		isz 1
		
		stp
		
		isz 4
		
		dec 4
		
		jmp 24
		
		jmp 16
		
		dec 2
		
		jmp 41
		
		jmp 44
		
		dec 2
		
		inc 3
		
		jmp 34
		
		jmp 16
		
		jmp 27
		
		dec 1
		
		dec 4
		
		jmp 37
		
		isz 2
	\end{multicols}
\end{aufgabe}

\end{document}
