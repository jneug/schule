\documentclass[10pt, a4paper, ngerman]{arbeitsblatt}

\ladeModule{theme,typo,icons,tabellen,aufgaben}
\aboptionen{
	name 		= {J. Neugebauer},
	kuerzel 	= {Ngb},
	titel 		= {Ein erstes Beispiel-Programm},
	reihe 		= {Der KnowHowPC},
	fach 		= {Informatik},
	kurs 		= {9Diff},
	nummer 		= {III.1},
	lizenz 		= {cc-by-nc-sa-4},
	version 	= {2021-05-26},
}

\begin{document}
\ReiheTitel

\begin{aufgabe}[icon=\iconComputer]
	\label{aufg:afg1}
	\begin{itemize}
		\item Kopiere den Ordner \ordner{KnowHowPC} aus dem Tauschordner in deinen Bereich und starte das Programm \programm{KnowHowPC.exe}.
		\item Öffne über \menu{Datei>Öffnen} die Datei \datei{Erstes Programm.khp}.
		\item Füge in das erste Register 3 Streichhölzer ein (linke Maustaste).
		\item Lass das Programm nun im \emph{Einzelschritt-Modus} (Taste $\blacktriangleright\blacktriangleright$) laufen und beobachte genau, wie das Programm abgearbeitet wird.
	\end{itemize}

	\bigskip
	\begin{itemize}
		\item Schreibe in die Kommentar-Spalte in der Tabelle unten, was die einzelnen Programmzeilen bewirken.
		\item Beschreibe, was das ganze Programm leistet und trage die Beschreibung im \programm{KnowHowPC} unterhalb des Programms in den Kommentarbereich ein.
	\end{itemize}

	\begin{center}
	\begin{tabularx}{.8\textwidth}{|c|c|c|X|}\hline
		\rowcolor{ab.tabelle.kopf.hg}
		Index & Befehl & Parameter & Kommentar \\\hline
		1: & \texttt{isz} & 2 & \Zeilenabstand\\\hline
		2: & \texttt{jmp} & 4 & \Zeilenabstand\\\hline
		3: & \texttt{stp} &   & \Zeilenabstand\\\hline
		4: & \texttt{inc} & 1 & \Zeilenabstand\\\hline
		5: & \texttt{dec} & 2 & \Zeilenabstand\\\hline
		6: & \texttt{jmp} & 1 & \Zeilenabstand\\\hline
	\end{tabularx}
	\end{center}
\end{aufgabe}

\begin{aufgabe}[icon=\iconComputer]
	Baue das Programm unten im \programm{KnowHowPC} nach und beschreibe seine Funktionsweise analog zu \prettyref{aufg:afg1}.

	\begin{center}
	\begin{tabularx}{.8\textwidth}{|c|c|c|X|}\hline
		\rowcolor{ab.tabelle.kopf.hg}
		Index & Befehl & Parameter & Kommentar \\\hline
		1: & \texttt{isz} & 1 & \Zeilenabstand\\\hline
		2: & \texttt{jmp} & 4 & \Zeilenabstand\\\hline
		3: & \texttt{stp} &   & \Zeilenabstand\\\hline
		4: & \texttt{dec} & 1 & \Zeilenabstand\\\hline
		5: & \texttt{jmp} & 1 & \Zeilenabstand\\\hline
	\end{tabularx}
	\end{center}
\end{aufgabe}


\end{document}
