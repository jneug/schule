\documentclass[11pt, a4paper, ngerman]{arbeitsblatt}

\ladeModule{theme,typo,icons,aufgaben}
\aboptionen{
	name 		= {J. Neugebauer},
	kuerzel 	= {Ngb},
	titel 		= {Multiplikation},
	reihe 		= {Der KnowHowPC},
	fach 		= {Informatik},
	kurs 		= {9Diff},
	nummer 		= {III.4},
	lizenz 		= {cc-by-nc-sa-4},
	version 	= {2021-05-26},
}

\begin{document}
\ReiheTitel

\begin{aufgabe}
	Erstelle im \programm{KnowHowPC} ein Programm, das folgendes leistet:

	\begin{itemize}
		\item Es berechnet R1 $:=$ R1 $\cdot$ R2.
		\item Hilfsregister werden zunächst gelöscht.
		\item Wenn das Programm endet, sind die Hilfsregister und R2 wieder leer.
	\end{itemize}

	\vspace{1.5cm}
	\begin{wrapfigure}{l}{0pt}
		\includegraphics[width=1.6cm]{9Diff-AB.III.4-Abb-1}
	\end{wrapfigure}
	\subsubsection*{Hilfestellung: Programm-Puzzle}
	Als Hilfe kannst du diese Liste an Befehlen nutzen. Bring sie in die richtige Reihenfolge, um die Aufgabe zu lösen.
	\begin{multicols}{4}\setlength{\parskip}{2ex}\usekomafont{code}
		isz 3

		jmp 4

		jmp 7

		dec 3

		jmp 10

		jmp 7

		jmp 24

		jmp 31

		jmp 41

		dec 3

		jmp 13

		inc 2

		jmp 13

		jmp 1

		isz 4

		jmp 19

		inc 4

		isz 3

		inc 1

		jmp 34

		isz 2

		isz 1

		stp

		isz 4

		dec 4

		jmp 24

		jmp 16

		dec 2

		jmp 41

		jmp 44

		dec 2

		inc 3

		jmp 34

		jmp 16

		jmp 27

		dec 1

		dec 4

		jmp 37

		isz 2
	\end{multicols}
\end{aufgabe}

\end{document}
