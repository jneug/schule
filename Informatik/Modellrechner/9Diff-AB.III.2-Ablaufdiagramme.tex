\documentclass[10pt, a4paper]{scrartcl}

\usepackage{qrcode}

\usepackage{vorschule}
\usepackage[
    typ=ab,
    fach=Informatik,
    lerngruppe={9Diff},
    nummer={III.2},
    module={Symbole,Lizenzen},
    seitenzahlen=keine,
%    farbig,
    lizenz=cc-by-nc-sa-4,
]{schule}

\usepackage[
	kuerzel=Ngb,
	reihe={Der KnowHowPC},
	version={2020-09-09},
]{ngbschule}

\author{J. Neugebauer}
\title{Ablaufdiagramme}
\date{\Heute}

\setzeAufgabentemplate{ngbnormal}

\begin{document}

\ReiheTitel

\begin{aufgabe}[symbol=\Large\symPartner]
	Betrachte das Diagramm unten und versuch es zu lesen. Welche Elemente gibt es und was bedeuten sie?
	
	\begin{center}
	\begin{tikzpicture}[pap]
		\node[startstop] (s1) {Start};
		\node[verzweigung, below=of s1] (v1) {\texttt{isz 2}};
		\node[aktion, below=of v1] (a1) {\texttt{dec 2}};
		\node[verzweigung, right=of v1] (v2) {\texttt{isz 1}};
		\node[aktion, below=of v2] (a2) {\texttt{dec 1}};
		\node[aktion, below=of a2] (a3) {\texttt{inc 2}};
		\node[aktion, below=of a3] (a4) {\texttt{inc 2}};
		\node[startstop, right=of v2] (e1) {\texttt{stp}};
		
		\draw[linie] (s1)--(v1);
		\draw[linie] (v1)--(a1) node[near start, right] {nein};
		\draw[linie] (a1.west) -| ($(v1.west) + (-.5,0)$) -- (v1.west);
		\draw[linie] (v1)--(v2) node[near start, above] {ja};
		\draw[linie] (v2)--(a2) node[near start, right] {nein};
		\draw[linie] (v2)--(e1) node[near start, above] {ja};
		\draw[linie] (v2)--(a2);
		\draw[linie] (a2)--(a3);
		\draw[linie] (a3)--(a4);
		\draw[linie] (a4.west) -| ($(v2.west) + (-.5,0)$);
	\end{tikzpicture}
	\end{center}
\end{aufgabe}

\begin{aufgabe}[symbol=\Large\symLaptop]
	Baue das Programm im \emph{KnowHowPC} nach und speichere es unter dem Namen \datei{Drittes Programm.kph}.
	Analysiere es mit dem \emph{Einzelschritt-Modus} und beschreibe seine Funktionsweise.
\end{aufgabe}

\begin{aufgabe}[symbol=\Large\symEinzel]
	\begin{minipage}{0.8\textwidth}
	Erstelle ein \emph{Ablaufdiagramm} zum Ersten Programm vom letzten Arbeitsblatt.
	
	Hier findest du Hilfe: \url{https://link.ngb.schule/paps}
	\end{minipage}\hfill
	\begin{minipage}{0.19\textwidth}
		\begin{center}
			\qrcode{https://link.ngb.schule/paps}
		\end{center}
	\end{minipage}
\end{aufgabe}

\end{document}
