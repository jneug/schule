\documentclass[10pt, a4paper, ngerman]{arbeitsblatt}

\ladeModule{theme,typo,icons,qrcodes,aufgaben}
\aboptionen{
	name 		= {J. Neugebauer},
	kuerzel 	= {Ngb},
	titel 		= {Ablaufdiagramme},
	reihe 		= {Der KnowHowPC},
	fach 		= {Informatik},
	kurs 		= {9Diff},
	nummer 		= {III.2},
	lizenz 		= {cc-by-nc-sa-4},
	version 	= {2021-05-27}
}

\ladeFach[algorithmen]{informatik}

\begin{document}
\ReiheTitel

\begin{aufgabe}[icon=\iconPartner\ \iconHeft]
	Betrachte das Diagramm unten und versuch es zu lesen. Welche Elemente gibt es und was bedeuten sie?

	\begin{center}
	\begin{tikzpicture}[pap,cmd/.style={font=\usekomafont{code}}]
		\node[startstop] (s1) {Start};
		\node[verzweigung, cmd, below=of s1] (v1) {isz 2};
		\node[aktion, cmd, below=of v1] (a1) {{dec 2}};
		\node[verzweigung, cmd, right=of v1] (v2) {isz 1};
		\node[aktion, cmd, below=of v2] (a2) {{dec 1}};
		\node[aktion, cmd, below=of a2] (a3) {{inc 2}};
		\node[aktion, cmd, below=of a3] (a4) {{inc 2}};
		\node[startstop, cmd, right=of v2] (e1) {{stp}};

		\draw[linie] (s1)--(v1);
		\draw[linie] (v1)--(a1) node[near start, right] {nein};
		\draw[linie] (a1.west) -| ($(v1.west) + (-.5,0)$) -- (v1.west);
		\draw[linie] (v1)--(v2) node[near start, above] {ja};
		\draw[linie] (v2)--(a2) node[near start, right] {nein};
		\draw[linie] (v2)--(e1) node[near start, above] {ja};
		\draw[linie] (v2)--(a2);
		\draw[linie] (a2)--(a3);
		\draw[linie] (a3)--(a4);
		\draw[linie] (a4.west) -| ($(v2.west) + (-.5,0)$);
	\end{tikzpicture}
	\end{center}
\end{aufgabe}

\begin{aufgabe}[icon=\iconLaptop]
	Baue das Programm im \programm{KnowHowPC} nach und speichere es unter dem Namen \datei{Zweites Programm.kph}.
	Analysiere es mit dem \emph{Einzelschritt-Modus} und beschreibe seine Funktionsweise.
\end{aufgabe}

\begin{aufgabe}[icon=\iconEinzel\ \iconHeft]
	\begin{wrapfix}
	\begin{wrapfigure}{r}{2cm}
		\qrcode[height=2cm]{https://link.ngb.schule/paps}
	\end{wrapfigure}
	Erstelle ein \emph{Ablaufdiagramm} zum \enquote{Ersten Programm} vom letzten Arbeitsblatt.

	Hier findest du Hilfe: \url{https://link.ngb.schule/paps}

	(\textit{Du kannst für das Ablaufdiagramm auch das Programm \programm{PapDesigner} verwenden und das Diagramm dann ausdrucken.})
	\end{wrapfix}
\end{aufgabe}

\end{document}
