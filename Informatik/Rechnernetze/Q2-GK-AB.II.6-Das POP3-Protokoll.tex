\documentclass[10pt, a4paper]{arbeitsblatt}

\ladeModule{theme,boxen,tabellen}
\ladeFach[]{informatik}

\aboptionen{
	name		= {J. Neugebauer},
	kuerzel		= {Ngb},
	titel		= {Das POP3-Protokoll},
	reihe		= {Rechnernetze},
	fach		= {Informatik},
	lerngruppe	= {Q2},
	nummer		= {II.06},
	lizenz		= {cc-by-nc-sa-4},
	version		= {2022-01-09},
}

\begin{document}
\ReiheTitel

\begin{infobox}
Ein \textbf{Protokoll} ist eine Spezifikation der Vorschriften und Regeln zum Informationsaustausch zwischen zwei Partnern eines Kommunikationssystems.

Festgelegt werden im Wesentlichen

\begin{tabularx}{\textwidth}{lX}
	\textbf{Vokabeln:} & Vorgegebene Befehle inklusive deren Bedeutung. \\
	\textbf{Verhalten:} & Welche Befehle sind in welcher Reihenfolge erlaubt? Welches Verhalten wird als Reaktion erwartet?  \\
	\textbf{Formate:} & Formate (Codierungen) der Befehle und der Antworten.
\end{tabularx}
\end{infobox}

\vspace*{-1em}
Das \emph{Post Office Protocol} (POP) ist ein Übertragungsprotokoll, über das ein Client E-Mails von einem E-Mail-Server abholen kann. POP3 (Version 3 des Protokolls) kommuniziert standardmäßig über den \emph{Port 110}.

Das POP3 definiert neun \emph{Vokabeln} mit jeweils vier Zeichen. Einige Vokabeln erfordern einen zusätzlichen Parameter. Gleichzeitig wird das \emph{Verhalten} durch erwartete Antworten und Reihenfolge festgelegt. POP3 hat drei Phasen. Einige Befehle dürfen nur in bestimmten Phasen ausgeführt werden, bei anderen gibt es keine Einschränkung.

\begin{center}
\begin{tabular}{|c|p{7cm}|m{4cm}|c|} \hline
	Befehl & Bedeutung & Beispiele für mögliche Antworten & Phase \\ \hline
	\code{USER \textit{u}} & \Zeilenabstand[1.5cm] &\code{+OK} \newline\newline \code{-ERR user unknown} & \textbf{2} \\ \hline
	\code{PASS \textit{p}} & \Zeilenabstand[1.5cm] & \code{+OK} \newline\newline \code{-ERR login failed} & \textbf{2} \\ \hline
	\code{STAT} & \Zeilenabstand[1.5cm] & \code{+OK 2 952} & \textbf{2} \\ \hline
	\code{NOOP} & \Zeilenabstand[1.5cm] & \code{+OK} \newline\newline \code{-ERR Unknown command.} & \textbf{1-3} \\ \hline
	\code{LIST} & \Zeilenabstand[1.5cm] & \code{+OK 2 952\newline1 436\newline2 516} & \textbf{2} \\ \hline
	\code{RETR \textit{n}} & \Zeilenabstand[1.5cm] & & \textbf{2} \\ \hline
	\code{DELE \textit{n}} & \Zeilenabstand[1.5cm] & & \textbf{2} \\ \hline
	\code{RSET} & \Zeilenabstand[1.5cm] & +OK & \textbf{2} \\ \hline
	\code{QUIT} & \Zeilenabstand[1.5cm] & \code{+OK session ended} & \textbf{3} \\ \hline
\end{tabular}
\end{center}

\clearpage
\ReiheTitel
\begin{aufgabe}[icon=\iconBlatt]
Recherchiere die Bedeutung der \emph{Vokablen} des POP3 und vervollständige die Tabelle zum Protokoll.
\end{aufgabe}

\begin{aufgabe}[icon=\iconComputer\,\iconBlatt]
Benutze den Telnet-Client, um mit Hilfe des POP3 manuell eine E-Mail abzurufen.

\begin{enumerate}
	\item Kopier den Ordner \ordner{Babypop3} aus dem Tauschordner. Führ dann einmalig durch Doppelklick die Datei \datei{babypop3\_config.reg} aus\footnote{Die Datei erstellt automatisch die nötigen Benutzerkonten für den Babypop3 E-Mail-Server.}. Start dann das Programm \programm{babypop3.exe}.
	\item Öffne eine Windows-Befehlszeile (\code{cmd}) und baue eine \texttt{telnet} Verbindung zum Server \enquote{127.0.0.1} auf dem Port \enquote{110} auf:

	\begin{center}
		\verb|C:\WINDOWS> telnet 127.0.0.1 110|
	\end{center}

	Meldet der Server eine positive Antwort (\verb|+OK Baby POP3 Server ready|), dann können die POP3-Befehle eingegeben werden.
	\item Als erstes musst du dich mit einem Nutzernamen und Passwort am Server anmelden. Es gibt fünf Nutzer mit Nutzername/Passwort \code{user1}/\code{user1} bis \code{user5}/\code{user5}.
	\item Benutze die Tabelle zum möglichen Ablauf eines E-Mail Abrufs als Hilfe und notiere jeweils die Antworten des Servers.
	\item Variiere die Befehle und versucht weitere Mails abzurufen. Probier auch fehlerhafte Befehle und Reihenfolgen aus. Wie reagiert der Server?
\end{enumerate}
\end{aufgabe}

\begin{aufgabe}[icon=\iconComputer\,\iconHeft]
	Studiere das Format der E-Mails, die vom Server gesendet werden. Für einen einfachen Zugriff kannst du im Ordner \ordner{Babypop3/maildrop} die E-Mails im Klartext als Textdatei einsehen.

	\medskip
	Welche Bestandteile der E-Mail kannst du identifizieren? Wie ist das Format aufgebaut? Versuche eine möglichst exakte Beschreibung zu erstellen.
\end{aufgabe}

\begin{aufgabe*}[icon=\iconComputer]
Erstelle eine Mail \enquote{von Hand}, indem du im Posteingang eines Nutzers (zum Beispiel \ordner{Babypop3/maildrop/user1}) eine neue Textdatei erstellst und die Mail anhand deiner Notizen in einem Texteditor schreibst.

Rufe die Mail dann via Telnet vom Server ab und prüfe, ob das Format korrekt erkannt wird.
\end{aufgabe*}

\end{document}
