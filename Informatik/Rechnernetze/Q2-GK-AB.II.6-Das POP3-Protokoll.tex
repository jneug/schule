\documentclass[10pt, a4paper]{scrartcl}

\usepackage{vorschule}
\usepackage[
	typ=ab,
	fach=Informatik,
	lerngruppe={Q2-GK},
	nummer=II.6,
	module={Symbole,Lizenzen},
	seitenzahlen=keine,
	farbig,
	lizenz=cc-by-nc-sa-4,
]{schule}

\usepackage[
	kuerzel=Ngb,
	reihe={Rechnernetze},
	version={2019-12-13},
]{ngbschule}

\author{J. Neugebauer}
\title{Das POP3-Protokoll}
\date{\Heute}

\setzeAufgabentemplate{ngbnormal}


\begin{document}
\ReiheTitel

\begin{infobox}
Ein \textbf{Protokoll} ist eine Spezifikation der Vorschriften und Regeln zum Informationsaustausch zwischen zwei Partnern eines Kommunikationssystems.

Festgelegt werden im Wesentlichen

\begin{tabularx}{\textwidth}{lX}
	\textbf{Vokabeln:} & Vorgegebene Befehle inklusive deren Bedeutung. \\
	\textbf{Verhalten:} & Welche Befehle sind in welcher Reihenfolge erlaubt? Welches Verhalten wird als Reaktion erwartet?  \\
	\textbf{Formate:} & Formate (Codierungen) der Befehle und der Antworten.
\end{tabularx}
\end{infobox}

Das \emph{Post Office Protocol} (POP) ist ein Übertragungsprotokoll, über das ein Client E-Mails von einem E-Mail-Server abholen kann. POP3 (Version 3 des Protokolls) kommuniziert standardmäßig über den \emph{Port 110}.

Das POP3 definiert neun \emph{Vokabeln} mit jeweils vier Zeichen. Einige Vokabeln erfordern einen zusätzlichen Parameter. Gleichzeitig wird das \emph{Verhalten} durch erwartete Antworten und Reihenfolge festgelegt. POP3 hat drei Phasen. Einige Befehle dürfen nur in bestimmten Phasen ausgeführt werden, bei anderen gibt es keine Einschränkung.

\begin{center}
\begin{tabular}{|c|p{7cm}|m{4cm}|c|} \hline
	Befehl & Bedeutung & Beispiele für mögliche Antworten & Phase \\ \hline
	\code{USER \textit{u}} & \Zeilenabstand[1.5cm] &\code{+OK} \newline\newline \code{-ERR user unknown} & \textbf{2} \\ \hline
	\code{PASS \textit{p}} & \Zeilenabstand[1.5cm] & \code{+OK} \newline\newline \code{-ERR login failed} & \textbf{2} \\ \hline
	\code{STAT} & \Zeilenabstand[1.5cm] & \code{+OK 2 952} & \textbf{2} \\ \hline
	\code{NOOP} & \Zeilenabstand[1.5cm] & \code{+OK} \newline\newline \code{-ERR Unknown command.} & \textbf{1-3} \\ \hline
	\code{LIST} & \Zeilenabstand[1.5cm] & \code{+OK 2 952\newline1 436\newline2 516} & \textbf{2} \\ \hline
	\code{RETR \textit{n}} & \Zeilenabstand[1.5cm] & & \textbf{2} \\ \hline
	\code{DELE \textit{n}} & \Zeilenabstand[1.5cm] & & \textbf{2} \\ \hline
	\code{RSET} & \Zeilenabstand[1.5cm] & +OK & \textbf{2} \\ \hline
	\code{QUIT} & \Zeilenabstand[1.5cm] & \code{+OK session ended} & \textbf{3} \\ \hline
\end{tabular}
\end{center}

\clearpage
\ReiheTitel
\begin{aufgabe}
	Recherchiert im Internet die Bedeutung der Vokablen des POP3 und notiert die Funde in der Tabelle.
\end{aufgabe}

\begin{aufgabe}
	Benutzt den Telnet-Client, um mit Hilfe des POP3 manuell eine E-Mail abzurufen.
	
	\begin{enumerate}
		\item Kopiert den Ordner \ordner{Babypop3} aus dem Tauschordner. Führt dann einmalig durhc Doppelklick die Datei \datei{babypop3\_config.reg} aus. Startet dann das Programm \programm{babypop3.exe}.
		\item Öffnet eine Windows-Befehlszeile und baut eine \texttt{telnet} Verbindung zum Server \enquote{127.0.0.1} auf dem Port \enquote{110} auf:

		\begin{center}
			\verb|C:\WINDOWS> telnet 127.0.0.1 110|
		\end{center}
		
		Meldet der Server eine positive Antwort (\verb|+OK Baby POP3 Server ready|), dann können die POP3-Befehle eingegeben werden.
		\item Als erstes müsst ihr euch mit einem Nutzernamen und Passwort am Server anmelden. Es gibt fünf Nutzer mit Nutzername/Passwort \texttt{user1}/\texttt{user1} bis \texttt{user5}/\texttt{user5}.
		\item Benutzt die Tabelle zum möglichen Ablauf eines E-Mail Abrufs als Hilfe und notiert jeweils die Antworten des Servers.
		\item Variiert die Befehle und versucht weitere Mails abzurufen. Probiert auch fehlerhafte Befehle und Reihenfolgen.
	\end{enumerate}
\end{aufgabe}

\begin{aufgabe}
	Studiert das Format der E-Mails, die vom Server gesendet werden. Für einen einfachen Zugriff könnt ihr im Ordner \ordner{Babypop3/maildrop} die E-Mails im Klartext als Textdatei abrufen.
	
	\medskip
	Welche Bestandteile der E-Mail könnt ihr identifizieren? Wie ist das Format aufgebaut? Versucht eine möglichst exakte Beschreibung zu erstellen.
\end{aufgabe}

\end{document}
