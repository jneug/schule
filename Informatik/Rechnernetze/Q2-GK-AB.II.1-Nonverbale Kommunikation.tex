\documentclass[10pt, a4paper]{arbeitsblatt}

\ladeModule{theme,boxen}
\aboptionen{
	name		= {J. Neugebauer},
	kuerzel		= {Ngb},
	titel		= {Nonverbale Kommunikation},
	reihe		= {Rechnernetze},
	fach		= {Informatik},
	lerngruppe	= {Q2},
	nummer		= {II.1},
	lizenz		= {cc-by-nc-sa-4},
	version		= {2021-11-21},
}

\begin{document}
\ReiheTitel

\section*{Regeln}
Markiert die für euch zutreffenden Kommunikationsregeln. In allen Fällen gilt:

\begin{infobox}\begin{center}
	\textbf{Es darf nicht gesprochen werden!}
\end{center}\end{infobox}

\begin{enuma}
	\item Kommuniziert zu dritt über einen großen Abstand (z.B. auf dem Schulhof oder im Flur).
	\item Kommuniziert zu zweit durch eine Tür, ohne weitere Hilfsmittel einzusetzen.
	\item Kommuniziert zu zweit durch eine Tür, indem ihr einen Faden unter der Tür hindurch führt. Jeder von euch hält ein Ende in der Hand.
\end{enuma}

\begin{aufgabe}
	Versucht nun eine Verabredung zu vereinbaren.

	\begin{teilaufgaben}
		\teilaufgabe
		Vereinbart gemeinsam, wie ihr zum Beispiel die Nachricht \enquote{Treffen um 10 Uhr in der Eisdiele} übermitteln könntet.
		\teilaufgabe
		Variiert dann zum Beispiel die Uhrzeit und den Ort und probiert aus, ob ihr die Botschaft übermitteln könnt.
	\end{teilaufgaben}
\end{aufgabe}

\begin{aufgabe}
	\begin{teilaufgaben}
		\teilaufgabe
		Welche Schwierigkeiten gibt es bei der Kommunikation? Wann kommt es zu welchen Missverständnissen? Welchen Grund haben diese? Haltet eure Überlegungen auf einem Notizzettel fest.
		\teilaufgabe
		Verbessert eure Kommunikation und testet sie erneut.
	\end{teilaufgaben}
\end{aufgabe}


\begin{aufgabe}
	Notiert euch die wichtigsten Merkmale eurer Kommunikationsform:
	\begin{itemize}
		\item Welche Festlegungen musstet ihr mindestens treffen?
		\item Worauf muss geachtet werden?
		\item Wo liegen die Grenzen bei eurer Art der Kommunikation?
		\item Wie könnte ein Computer helfen?
	\end{itemize}
\end{aufgabe}

\end{document}
