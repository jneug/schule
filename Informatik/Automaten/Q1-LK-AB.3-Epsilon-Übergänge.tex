\documentclass[fontsize=11pt, a4paper, ngerman]{scrartcl}

\usepackage[theme,typo,icons,aufgaben]{arbeitsblatt}
\aboptionen{
	name		= {J. Neugebauer},
	kuerzel		= {Ngb},
	titel		= {Epsilon-Übergänge},
	reihe		= {Automaten und formale Sprachen},
	fach		= {Informatik},
	lerngruppe	= {Q1-LK},
	nummer		= {IV.03},
	lizenz		= {cc-by-nc-sa-4},
	version		= {2021-04-10},
}

\ladeFach[automaten]{informatik}


\begin{document}
\ReiheTitel

Nichtdeterministische endliche Automaten (NEA) sind häufig viel übersichtlicher als
ihr deterministisches Gegenstück. In manchen Fällen hilft es zusätzlich, Übergänge
zu benutzen, die \emph{ohne Eingabe} ausgeführt werden. Solche Übergänge nennt man
\textbf{Epsilon-Übergänge} ($\varepsilon$ steht für das leere Wort).

\begin{aufgabe}
	\label{aufg:graph-1}
	Der gezeigte Automat $A$ über dem Alphabet $\Sigma = \{0,1,2,3\}$ akzeptiert
	die Sprach $L(A) = 0^{*}1^\ast2^\ast3^\ast$. Konstruiere den gezeigten NEA
	in FLACI und teste ihn.

	\begin{figure}[h]
	    \centering
	    \begin{transitiongraph}[fa]
	        \state[s]{q0}{0}{0}
	        \state{q1}{40}{0}
	        \state{q2}{80}{0}
	        \state[f]{q3}{120}{0}
	        \transition{q0}{q0}{0}
	        \transition{q0}{q1}{}
	        \transition{q1}{q1}{1}
	        \transition{q1}{q2}{}
	        \transition{q2}{q2}{2}
	        \transition{q2}{q3}{}
	        \transition{q3}{q3}{3}
	    \end{transitiongraph}
	    \caption{Übergangsgraph des Automaten $A$.}
	    \label{abb:graph_1}
	\end{figure}
\end{aufgabe}

\begin{aufgabe}
	\label{aufg:graph-2}
	Versuche zu erklären, warum der Graph aus \prettyref{abb:graph_1} durch die
	Epsilon-Übergänge übersichtlicher wird. Wie würde ein Graph für einen äquivalenter
	NEA \emph{ohne Epsilon-Übergänge} aussehen?

	\hinweis{Zu jedem NEA \emph{mit Epsilon-Übergängen} gibt es einen äquivalenten
	(der dieselbe Sprache akzeptiert) \emph{ohne Epsilon-Übergänge}.}
\end{aufgabe}

\begin{aufgabe}
	\label{aufg:graph-3}
	Um die Potenzmengenkonstruktion durchführen zu können, kannst du zuerst einen NEA
	ohne Epsilon-Übergänge konstruieren und diesen Umformen.

	Du kannst den NEA aber auch direkt umformen. Dabei werden Epsilon-Übergänge
	berücksichtigt, indem die Zustände am Ende der Übergänge immer mit in die Menge der
	Folgezustände aufgenommen werden.

	\begin{figure}[h]
		\centering
		\begin{subfigure}{.5\textwidth}
		\centering
		  \begin{transitiongraph}[fa]
		    \state[s]{q0}{0}{0}
		    \state[f]{q1}{20}{0}
		    \transition{q0}{q0}{0}
		    \transition{q0}{q1}{}
		    \transition{q1}{q1}{1}
	  	 \end{transitiongraph}
	  	 \caption{Ein NEA mit Epsilon-Übergang.}
	  	 \label{abb:grah_2}
		\end{subfigure}%
		\begin{subfigure}{.5\textwidth}
		\centering
		  \begin{transitiongraph}[fa]
		    \state[sf]{q0-q1}{0}{0}
		    \state[f]{q1}{20}{0}
		    \state{Err}{40}{0}
		    \transition{q0-q1}{q0-q1}{0}
		    \transition{q0-q1}{q1}{1}
		    \transition{q1}{q1}{1}
		    \transition{q1}{Err}{0}
		    \transition{Err}{Err}{0,1}
	  	 \end{transitiongraph}
	  	 \caption{Ergebnis als DEA.}
	  	 \label{abb:graph_3}
		\end{subfigure}
		\caption{Potenzmengenkonstruktion mit Epsilon-Übergang.}
	\end{figure}

	Konstruiere für den NEA $A$ aus \prettyref{aufg:graph-1} oben einen äquivalenten DEA.
\end{aufgabe}

\end{document}
