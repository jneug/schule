\documentclass[10pt, a4paper, ngerman]{scrartcl}

\usepackage[theme,typo,icons,aufgaben]{arbeitsblatt}
\aboptionen{
	name		= {J. Neugebauer},
	kuerzel		= {Ngb},
	titel		= {Scanner und Parser},
	reihe		= {Automaten und formale Sprachen},
	fach		= {Informatik},
	lerngruppe	= {Q1-LK},
	nummer		= {IV.11},
	lizenz		= {cc-by-nc-sa-4},
	version		= {2021-04-10},
}

\ladeFach[automaten]{informatik}

\begin{document}
% \ReiheTitel

\begin{aufgabe}[subtitle=Funktionsweise eines Compilers]
	\label{aufg:compiler}
	Klone das Projekt \ordner{rechenmaschine} und öffne es in \programm{BlueJ}.
	Das Programm ist eine (vereinfachte) Umsetzung eines Compilers (und Interpreters) für
	\emph{Plusterme}. (Auf \prettyref{ab:10} findest du den DEA zur Sprache.)

	Analysiere das Programm und wie der Übersetzungsvorgang eines Compilers
	implementiert wurde. Beachte auch die Verwendung der \code{switch}-Anweisung. (Recherchiere
	ihren Aufbau, falls sie dir neu ist.)

	Welche Aufgabe hat in diesem Beispiel die \code{scan}-Methode? Welche die \code{parse}-Methode?
	Könnte man beide zur Vereinfachung auch zusammenfassen?

	\tipp{Teste das Programm mit verschiedenen Eingaben und studiere vor allem die Fehlermeldungen
	bei ungültigen Plustermen.}
\end{aufgabe}

\begin{aufgabe}[subtitle=Minusterme]
	\label{aufg:subtraktion}

	\begin{enuma}
		\item Ändere das Programm so ab, dass statt \emph{Plustermen} nun \emph{Minusterme} erkannt und
		berechnet werden.

		\item Erweitere das Programm dann so, dass als Rechenoperation nun Addition \emph{und}
		Subtraktion erlaubt sind.
	\end{enuma}
\end{aufgabe}

\begin{aufgabe}[subtitle=Rechenterme]
	\label{aufg:rechenterme}
	Auf \prettyref{ab:5} wurde die Sprache der gültigen Rechenterme (ohne Klammerung) definiert.

	\begin{figure}[h]
		\centering
		\begin{transitiongraph}[fa]
			\state[sf]{q0}{0}{0}
			\state[f]{q1}{30}{30}
			\state[f]{q2}{30}{-30}
			\state{q3}{30}{0}
			\state[f]{q4}{50}{0}
			\state{q5}{110}{0}

			\transition[label=left]{q0}{q1}{0}
			\transition[label=left]{q0}{q2}{1..9}
			\transition[label=left]{q1}{q3}{$\cdot$}
			\transition[label=top]{q1}{q5}{+;-;:;$\cdot$}
			\transition[line=bot,label=bot]{q2}{q2}{0..9}
			\transition[label=top]{q2}{q5}{+;-;:;$\cdot$}
			\transition[label=left]{q2}{q3}{.}
			\transition[label=top]{q3}{q4}{0..9}
			\transition[line=left,label=bot]{q5}{q2}{1..9}
			\transition[line=right,label=top]{q5}{q1}{0}
			\transition{q4}{q4}{0..9}
			\transition{q4}{q5}{+;-;:;$\cdot$}
		\end{transitiongraph}
		\caption{Nichtdeterministischer Automat zu Rechenausdrücken.}
		\label{abb:graph-rechenausdruck}
	\end{figure}

	Passe das Programm aus \prettyref{aufg:subtraktion} schrittweise an (Scanner, Parser, Interpreter),
	so dass gültige Rechenterme übersetzt und korrekt ausgerechnet werden. Achte auch auf
	aussagekräftige Fehlermeldungen bei der Übersetzung.

	\hinweis{Für die Interpretation des Rechenterms ignorieren wir die Punkt-vor-Strich Regel und
	rechnen einfach von links nach rechts.}
\end{aufgabe}


\begin{aufgabe}[subtitle=Semantische Analyse,icon=\iconStern]
	\label{aufg:semantik}
	Die Sprache für gültige Rechenterme hat noch ein Problem: Tauscht im Rechenterm bei einer
	Division als Divisor die 0 (Null) auf, dann kann das Ergebnis nicht berechnet werden, auch wenn
	die Eingabe syntaktisch korrekt ist.

	Implementiere eine neue Methode \code{boolean analysiere()}, die die Tokenliste vor der
	Interpretation darauf prüft, ob nach einem Geteiltzeichen (\code{:}) eine Null auftaucht und
	mit einer entsprechenden Fehlermeldung abbricht.
\end{aufgabe}

\begin{aufgabe}[subtitle=Klammerausdrücke,icon=\iconStern]
	\label{aufg:nka}
	Erweitere die Rechenmaschine um Klammerausdrücke.
\end{aufgabe}

\end{document}
