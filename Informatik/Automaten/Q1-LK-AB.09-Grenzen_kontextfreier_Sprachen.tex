\documentclass[fontsize=10pt, a4paper, ngerman]{scrartcl}

\usepackage[theme,typo,icons,aufgaben,qrcodes]{arbeitsblatt}
\aboptionen{
	name={J. Neugebauer},
	kuerzel=Ngb,
	titel={Grenzen kontextfreier Sprachen},
	reihe={Automaten und formale Sprachen},
	fach={Informatik},
	lerngruppe={Q1-LK},
	nummer={IV.09},
	lizenz=cc-by-nc-sa-4,
	version={2021-04-10},
}

\ladeFach[automaten,graphen]{informatik}

\newcommand{\qrhinweis}[1]{%
	\begin{wrapfigure}[6]{r}{0pt}
		\qrcode[height=1cm]{#1}
	\end{wrapfigure}%
}

\begin{document}
\ReiheTitel

\emph{Kontextfreie Sprachen} spielen eine wichtige Rolle bei der
Übersetzung höherer Programmiersprachen in Maschinencode. In einem
ersten Schritt wird ein Quelltext auf \emph{syntaktische Korrektheit}
geprüft, also ob die Art und Abfolge der Zeichen in der
Programmier\-sprache gültig ist. Wir haben gesehen, dass erst
kontextfreie Sprachen die Prüfung von Klammer\-strukturen erlauben.
Daher sind reguläre Sprachen für die Syntax\-prüfung nicht ausreichend.

\begin{aufgabe}
\label{aufg:grammatik-ifthenelse}
Gegeben ist die Grammatik $G_{if} = (\{S, I, T, E\}, \{\text{if},
\text{then}, \text{else}, \text{a}, \text{b}\}, S, P_{if})$ mit den
Produktionen
\begin{align*}
P_{if} = \{ S &\rightarrow I\, T\, E \\
            I &\rightarrow \text{if b} \\
            T &\rightarrow \text{then}\,\text{a} \,|\, \text{then}\, S \\
            E &\rightarrow \text{else}\,\text{a} \,|\, \text{else}\, S \,|\, \varepsilon \}
\end{align*}

\begin{enuma}
	\item Welche Sprache wird von der Grammatik erzeugt? Wofür stehen die Terminale $b$ und $a$? Begründe, dass es sich nicht um eine reguläre Grammatik handeln kann.
	\item Reguläre Sprachen sind immer auch kontextfreie Sprachen, aber nicht alle kontextfreien Sprachen sind auch regulär. Prüfe, ob diese kontextfreie Sprache regulär ist, oder nicht.

		\hinweis{Eine Sprache ist regulär, wenn sie durch eine reguläre Grammatik erzeugt wird, oder sie von einem DEA/NEA akzeptiert wird.}
\end{enuma}
\end{aufgabe}

\begin{aufgabe}
\label{aufg:flaci-ifthenelse}
\begin{wrapfix}
\begin{wrapfigure}{r}{0pt}
	\begin{forest}
	[S,draw,circle
		[I,draw,circle
			[if,draw]
			[b,draw]
		]
		[T,draw,circle
			[then,draw]
			[a,draw]
		]
		[E,draw,circle
			[else,draw]
			[a,draw]
		]
	]
	\end{forest}
	\caption{Ableitungsbaum für das Wort \code{ifbthenaelsea}.}
	\label{abb:ableitungsbaum}
\end{wrapfigure}

\programm{FLACI} kann auch Grammatiken simulieren. Unter

\url{https://link.ngb.schule/flaci-ifthenelse}


findest Du die Grammatik $G_{if}$. Oben hinter \enquote{Ableiten}
kannst Du mit dem Zauberstab-Symbol Wörter der Sprache erzeugen lassen.
Klickst Du anschließend auf den Pfeil rechts daneben, erzeugt \programm{FLACI}
den \enquote{Ableitungsbaum} für das Wort.

\begin{enuma}
	\item Erzeuge einige Worte und deren Ableitungsbäume in \programm{FLACI}.
	\item Zeichne selber den Ableitungsbaum für das Wort
		\code{ifbthenifbthenaelsea} auf. Lass den Baum \emph{danach}
		in \programm{FLACI} erzeugen. Vergleiche die Bäume miteinander.
		Gibt es noch einen anderen Baum?
	\item Welches Problem ergibt sich für die Syntaxprüfung, wenn es
		mehrere Ableitungsbäume gibt? Notiere Stichpunkte.
\end{enuma}
\end{wrapfix}
\end{aufgabe}

\begin{aufgabe}
\label{aufg:eindeutigkeit}
Die \enquote{if-then-else} Sprache ist nicht eindeutig, aber
Eindeutig\-keit ist für eine syntaktische Über\-prüfung notwendig.
Überlege, wie die Sprache eindeutig gemacht werden kann.

\hinweis{Denk an die Klammersprache von \prettyref{ab:8}.}
\end{aufgabe}

\begin{aufgabe}
\label{aufg:grenzen-ktfs}
\qrhinweis{https://tipp.ngb.schule/4qex-2ts8-s5p3}
In der Chomsky-Hierarchie sind kontextfreie Sprachen vom Typ 2. Auch
sie haben Grenzen. Findest Du eine Sprache, die \emph{nicht kontextfrei}
ist?
\end{aufgabe}

\end{document}
