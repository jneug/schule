\documentclass[10pt, a4paper]{scrartcl}

\usepackage{vorschule}
\usepackage[
	typ=ab,
	fach=Informatik,
	lerngruppe={Q1-LK} ,
	nummer=6,
	module={Symbole,Lizenzen},
	seitenzahlen=keine,
	farbig,
	lizenz=cc-by-nc-sa-4,
]{schule}

\usepackage[
	kuerzel=Ngb,
	reihe={Automaten und formale Sprachen} ,
	version={2021-03-07} ,
]{ngbschule}

\author{J. Neugebauer}
\title{Reguläre Grammatiken}
\date{\Heute}

\setzeAufgabentemplate{ngbnormal}

\usepackage{FLaAL}

\usepackage{qrcode}

\renewcommand{\qrhinweis}[1]{%
	\begin{wrapfigure}[4]{r}{0pt}
		\qrcode[height=1cm]{#1}
	\end{wrapfigure}%
}

\begin{document}
\ReiheTitel

Eine Sprache $L(A)$, die von einem DEA $A$ akzeptiert wird, nennt man eine \emph{reguläre Sprache}. Eine \emph{Grammatik} zur Produktion der Sprache $L(A)$ nennt man eine \emph{reguläre Grammatik} $G_{L(A)}$. Man unterschiedet \emph{rechts- und linksreguläre Grammatiken}.

Die Produktionen einer \emph{rechtsregulären Grammatik} unterliegen folgenden Einschränkungen:
\begin{smallitemize}
	\item Auf der linken Seite einer Produktion stehen nur einzelne \emph{Nichtterminale}.
	\item Auf der rechten Seite einer Produktion stehen nur
	\begin{smallitemize}
		\item das leere Wort ($\epsilon$), \hspace{2cm} N $\rightarrow$ $\epsilon$
		\item ein \emph{Terminal}, oder \hspace{2cm} N $\rightarrow$ T
		\item ein \emph{Terminal} gefolgt von einem \emph{Nichtterminal}.  \hspace{2cm} N $\rightarrow$ TN
	\end{smallitemize}
\end{smallitemize}

Ein Nichtterminal darf mehrmals auf der linken Seite vorkommen (Alternative). Als Kurzschreibweise nutzen wir 
\begin{center}
\begin{tabular}{p{3cm}cp{3cm}}
N $\rightarrow$ T \newline N $\rightarrow$ TN 
& $\Rightarrow$ 
& N $\rightarrow$ T | TN
\end{tabular}
\end{center}

\begin{aufgabe}[symbol=\symEinzel\,\symHeft]
	\label{aufg:nea-zu-grammatik}
	Erstelle zum in \prettyref{abb:automat1} dargestellten Automaten $A_1$ eine rechtslineare Grammatik, die die Sprache $L(A_1)$ produziert.
	
	\begin{figure}[h]
	    \centering
	    \begin{transitiongraph}[fa]
	        \state[s]{q0}{0}{0}
	        \state{q1}{20}{0}
	        \state{q2}{40}{-15}
	        \state{q3}{40}{15}
	        \state{q4}{80}{-15}
	        \state{q5}{80}{15}
	        \state[f]{q7}{100}{0}
	        
	        \transition{q0}{q1}{0}
	        \transition{q5}{q7}{1}
	        \transition[line=straight,label=top]{q5}{q3}{1}
	        \transition[line=left]{q3}{q5}{1}
	        \transition{q1}{q3}{1}
	        \transition[label=bot]{q1}{q2}{0}
	        \transition[line=left]{q4}{q2}{0}
	    	\transition[line=straight,label=bot]{q2}{q4}{0}
	        \transition[label=bot]{q4}{q7}{1}
	        \transition[label=top,line=left]{q4}{q3}{1}
	        \transition[label=bot,line=right]{q5}{q2}{0}
	    \end{transitiongraph}
	    \caption{Übergangsgraphen eines NEA $A_1$}
	    \label{abb:automat1}
	\end{figure}
\end{aufgabe}

\begin{aufgabe}[symbol=\symPartner\,\symHeft]
	\label{aufg:grammatik-minimieren}
	\begin{wrapfig}
	\qrhinweis{Ueberlegt, welche Produktionen vereinfacht oder zusammengefasst werden koennen. Gibt es doppelte Produktionen?}
	Die Grammatik aus \prettyref{aufg:nea-zu-grammatik} ist zwar korrekt, aber (in der Regel) nicht optimal. Analysiert die Produktionsregeln und überlegt, ob ihr einige von ihnen vereinfachen könnt.
	\end{wrapfig}
\end{aufgabe}

\begin{wrapfig}
\qrhinweis{https://flaci.com/Gjf0v0ova}
\begin{aufgabe}[symbol=\symEinzel\,\symHeft]
	\label{aufg:dea-zu-grammatik}
	Erstelle eine \emph{rechtslineare Grammatik} zum Rechenterm-Akzeptor vom letzten Arbeitsblatt.
\end{aufgabe}
\end{wrapfig}

\begin{aufgabe}[symbol=\symEinzel\,\symHeft]
	\label{aufg:grammatik-zu-nea}
	Gegeben ist die Grammatik $G = (N, T, S, P)$ mit $N = \{ S, A, B, C \}$, $T = \{ a, b, c \}$ und $P = \{ S \rightarrow aB | bA | cA, A \rightarrow aB | bA | cA | \epsilon, B \rightarrow bC, C \rightarrow bA \}$.
	
	Leite einige Worte der Grammatik ab. Erstelle dann einen NEA, der die erzeugte Sprache akzeptiert.
\end{aufgabe}

\begin{aufgabe}[symbol=\symStern]
	\label{aufg:zusatz-linksreg}
	Überlege, wie eine \enquote{linksreguläre Grammatik} definiert ist und begründe die Aussage: \enquote{Zu jeder rechtsregulären Grammatik existiert eine äquivalente linksreguläre Grammatik.}
	
	Forme die Grammatik aus \prettyref{aufg:grammatik-zu-nea} in eine äquivalente linksreguläre Grammatik um. 
\end{aufgabe}

\end{document}