\documentclass[10pt, a4paper]{scrartcl}

\usepackage{vorschule}
\usepackage[
	typ=ab,
	fach=Informatik,
	lerngruppe={Q1-LK} ,
	nummer=10,
	module={Symbole,Lizenzen},
	seitenzahlen=keine,
	farbig,
	lizenz=cc-by-nc-sa-4,
]{schule}

\usepackage[
	kuerzel=Ngb,
	reihe={Automaten und formale Sprachen} ,
	version={2021-03-14} ,
]{ngbschule}

\author{J. Neugebauer}
\title{Exkurs: Implementierung von Automaten in Python}
\date{\Heute}

\setzeAufgabentemplate{ngbnormal}

\usepackage{FLaAL}

\begin{document}
\ReiheTitel

Mit \programm{Python} lässt sich leicht ein endlicher Automat in ein Programm überführen, dass eine Eingabe prüft, ob sie vom Automaten akzeptiert wird. Auch wenn Python und Java sehr unterschiedliche Sprachen sind, haben sie doch auch viele Gemeinsamkeiten. Eine Gegenüberstellung wichtiger Programmkonzepte von \programm{Python} und \programm{Java} findest Du unter \url{https://link.ngb.schule/java-python}.

\hinweis{Nutze für die Aufgaben jeweils die bereitgestellten Vorlagen.}


\begin{aufgabe}[subtitle=Implementierung eines DEA: Plusterme]
\label{aufg:py-dea}
Implementiere ein \programm{Python}-Programm, das prüft, ob eine Eingabe zur Sprache $L_\text{plus} = \{ w | w$ ist ein Plusterm der Form $a + b + \dots \}$ gehört.
\begin{center}
\begin{transitiongraph}[fa]
	\state[s]{q0}{0}{0}
	\state[f]{q1}{25}{-20}
	\state[f]{q2}{-25}{-20}
	
	\transition[line=left]{q0}{q1}{1..9}
	\transition[line=left]{q0}{q2}{0}
	\transition[line=right]{q1}{q1}{0..9}
	\transition[line=left]{q1}{q0}{+}
	\transition[line=left]{q2}{q0}{+}
\end{transitiongraph}
\end{center}
\end{aufgabe}

\begin{aufgabe}[subtitle=Implementierung eines NEA: Bond]
\label{aufg:py-nea}
Implementiere ein \programm{Python}-Programm, das den gezeigten nichtdeterministischen Automaten umsetzt und prüft, ob sich in einem Zahlenstring der Geheimagent \code{007} oder sein Kollege \code{006} versteckt.
\begin{center}
\begin{transitiongraph}[fa]
	\state[s]{q0}{0}{0}
	\state{q1}{20}{0}
	\state{q2}{40}{0}
	\state[f]{q3}{60}{0}
	
	\transition[label=left]{q0}{q0}{0..9}
	\transition{q0}{q1}{0}
	\transition[line=left]{q1}{q0}{0..9}
	\transition{q1}{q2}{0}
	\transition[line=right,label=top]{q2}{q0}{0..9}
	\transition{q2}{q3}{7,6}
	\transition[label=right]{q3}{q3}{0..9}
\end{transitiongraph}
\end{center}
\end{aufgabe}

\begin{aufgabe}[subtitle=Implementierung eines NKA: Palindrome]
\label{aufg:py-nka}
Implementiere \programm{Python}-Programm, das prüft, ob eine Eingabe zur Sprache $L_\text{palindrom} = \{ ww^R | w = w^{R} \}$ (die Sprache der Palindrome gerader Länge) gehört.
\begin{center}
\begin{transitiongraph}[pa]
	\state[s]{q0}{0}{0}
	\state{q1}{30}{0}
	\state[f]{q2}{60}{0}
	
	\transition{q0}{q0}{\#,a,A\#;\#,b,B\#;A,a,AA;A,b,BA;B,a,AB;B,b,BB}
	\transition{q0}{q1}{A,a,;B,b,}
	\transition{q1}{q1}{A,a,;B,b,}
	\transition{q1}{q2}{\#,,}
\end{transitiongraph}
\end{center}
\end{aufgabe}

\begin{aufgabe}[subtitle=Weitere Aufgaben,symbol=\symStern]
\label{aufg:py-weitere}
 zu den folgenden Sprachen:
\begin{itemize}
	\item Implementiere ein \programm{Python}-Programme zur regulären Sprache für Rechenterme (\prettyref{ab:8}).
	\item Erweitere den Palindrom NKA aus \prettyref{aufg:py-nka} für Palindrome mit ungerader Anzahl Buchstaben.
\end{itemize}
\end{aufgabe}

\end{document}