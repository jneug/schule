\documentclass[fontsize=11pt, a4paper, ngerman]{scrartcl}

\usepackage[theme,typo,icons,aufgaben]{arbeitsblatt}
\aboptionen{
	name		= {J. Neugebauer},
	kuerzel		= {Ngb},
	titel		= {Mealy-Automaten},
	reihe		= {Automaten und formale Sprachen},
	fach		= {Informatik},
	lerngruppe	= {Q1-LK},
	nummer		= {IV.04},
	lizenz		= {cc-by-nc-sa-4},
	version		= {2021-04-15},
}

\ladeFach[automaten]{informatik}

\begin{document}
\ReiheTitel

Ein \textbf{Mealy-Automat} ist ein Automat, der für jeden Übergang nicht nur eine Eingabe aus dem \emph{Eingabealphabet} festlegt, sondern auch eine Ausgabe aus einem \emph{Ausgabealphabet}.

Ein Mealy-Automat wird als 6-Tupel $A = (Q, \Sigma, \Omega, \delta, \lambda, q_0)$ definiert:
\begin{itemize}
	\item $Q$ ist die (endliche) Menge von Zuständen des Automaten.
	\item $\Sigma$ ist das Eingabealphabet.
	\item $\Omega$ ist das Ausgabealphabet.
	\item $\delta: Q\times \Sigma\rightarrow Q$ ist die Übergangsfunktion.
	\item $\lambda: \Omega\times \Sigma\rightarrow \Omega$ ist die Ausgabefunktion.
	\item $q_0\in Q$ ist der Startzustand.
\end{itemize}

(Beachte, dass es keinen Endzustand gibt. Ein Mealy-Automat akzeptiert also keine Worte, sondern erzeugt zu einem Wort eine Ausgabe.)

\begin{aufgabe}
	Konstruiere den gezeigten Mealy-Automaten in FLACI und teste ihn. Was macht der Automat?

	\begin{figure}[h]
	    \centering
	    \begin{transitiongraph}[fa]
	        \state[s]{q0}{0}{0}
			\state{q1}{30}{0}
	        \transition[label=top]{q0}{q0}{0 / 0}
	        \transition[label=top,line=left]{q0}{q1}{1 / 0}
	        \transition[label=bot,line=left]{q1}{q0}{1 / 1}
	        \transition[label=top]{q1}{q1}{0 / 1}
	    \end{transitiongraph}
		%\includegraphics[width=6cm]{Q1-LK-AB.4-Abb_Mealy_1}
		\caption{Ein Mealy-Automat}
		\label{abb:mealy-graph}
	\end{figure}
\end{aufgabe}

\begin{aufgabe}
	Konstruiere in FLACI einen Mealy-Automaten mit dem Eingabealphabet $\Sigma = \{ 0, 1, 2, 3 \}$ und dem Ausgabealphabet $\Omega = \{ 0, 1, 2, 3, 4, 5, 6 \}$, der für Eingaben aus zwei Buchstaben (z.B. $23$) die Summe der Zahlen bestimmt (also hier $5$).

	\tipp{Es reichen sechs Zustände zur Umsetzung des Automaten.}
\end{aufgabe}

\end{document}
