\documentclass[fontsize=10pt, a4paper, ngerman]{scrartcl}

\usepackage[theme,typo,icons,aufgaben]{arbeitsblatt}
\aboptionen{
	name={J. Neugebauer},
	kuerzel=Ngb,
	titel={Die Programmiersprache \enquote{list}},
	reihe={Automaten und formale Sprachen},
	fach={Informatik},
	kurs={Q1-LK},
	nummer={IV.12},
	lizenz=cc-by-nc-sa-4,
	version={2021-04-10},
}

\ladeFach[automaten]{informatik}

\begin{document}
\ReiheTitel

\begin{aufgabe}[subtitle=list 1.0]
	\label{aufg:list-v1}
	Die Programmiersprache \enquote{list} erlaubt das Programmieren von
	nummerierten Listen.

	\begin{minipage}[t]{.5\textwidth}
	Beispielhaftes Programm in \enquote{list}:
	\begin{verbatim}
list from 1 to 5
    Listeneintrag {i}
end
	\end{verbatim}
	\end{minipage}\hfill\begin{minipage}[t]{.5\textwidth}
	Ausgabe des Programms:
	\begin{verbatim}
Listeneintrag 1
Listeneintrag 2
Listeneintrag 3
Listeneintrag 4
Listeneintrag 5
	\end{verbatim}
	\end{minipage}

	Ein endlicher Automat für die Analyse der Sprache könnte (vereinfacht) so aussehen.

	\begin{center}
		\rmfamily\small
		\begin{transitiongraph}[fa]
			\state[sf]{q0}{0}{0}
			\state{q1}{20}{0}
			\state{q2}{40}{0}
			\state{q3}{60}{0}
			\state{q4}{80}{0}
			\state{q5}{100}{0}
			\state{q6}{100}{-20}
			\state{q7}{80}{-20}
			\state{q8}{50}{-20}
			\state{q9}{30}{-20}
			\state[f]{q10}{10}{-20}

			\transition[label=top]{q0}{q1}{list\textvisiblespace}
			\transition[label=top]{q1}{q2}{from\textvisiblespace}
			\transition[label=top]{q2}{q3}{1..9}
			\transition[label=top,line=top]{q3}{q3}{0..9}
			\transition[label=top]{q3}{q4}{\textvisiblespace}
			\transition[label=top]{q4}{q5}{to\textvisiblespace}
			\transition[label=right]{q5}{q6}{1..9}
			\transition[label=bot,line=bot]{q6}{q6}{0..9}
			\transition[label=bot]{q6}{q7}{\textvisiblespace}
			\transition[label=bot]{q7}{q8}{string\textvisiblespace}
			\transition[label=bot]{q8}{q9}{\{i\}\textvisiblespace}
			\transition[label=bot]{q9}{q10}{end}
		\end{transitiongraph}
	\end{center}

	Wobei zur besseren Übersicht folgende Vereinfachungen vereinbart werden:
	\begin{itemize}
		\item {\rmfamily\small 0..9} steht für die zehn Übergänge mit jeweils einer der Ziffern von 0 bis
			9 als Übergangssymbol.
		\item {\rmfamily\small string} steht für eine beliebige Zeichenfolge.
		\item Das Symbol {\rmfamily\textvisiblespace} steht für \enquote{Whitespace}, also eine
			beliebige Anzahl an Leerzeichen, Tabulatoren und Zeilenumbrüchen.
		\item Steht {\rmfamily\textvisiblespace} am Ende eines Übergangs, bedeutet dies, dass der
			Buchstabe (z.B. \texttt{list}) von mindestens einem \enquote{Whitespace} gefolgt werden
			muss. Der Übergang von \texttt{q0} nach \texttt{q1} ist also eine Kurzform von


			\begin{center}
				\scalebox{.7}{
				\begin{transitiongraph}[fa]\rmfamily\small
					\state{q0}{0}{0}
					\state{q0'}{20}{0}
					\state{q1}{40}{0}
					\transition[label=top]{q0}{q0'}{list}
					\transition[label=top]{q0'}{q1}{\textvisiblespace}
					\transition[label=top]{q1}{q1}{\textvisiblespace}
				\end{transitiongraph}}
			\end{center}
	\end{itemize}

	Implementiere einen Scanner, Parser und Interpreter für \enquote{list}.

	\hinweis{Die Projektvorlage enthält schon die Vorbereitung für einen Scanner, der den
		eingelesenen Quelltext nicht Zeichenweise durchgeht, sondern ihn jeweils an den Leerzeichen
		trennt und \enquote{Wort für Wort} durchläuft. Der Scanner muss also vor allem noch die
		Tokens aus dem Quelltext erstellen.}
\end{aufgabe}

\begin{aufgabe}[subtitle=list 2.0]
	\label{aufg:list-v2}
	Für Version 2 von \enquote{list} soll die Sprache um eine \code{by}-Nweisung ergänzt werden. Außerdem
	soll es möglich sein, als Textausgabe beliebige Kombinationen aus Strings und Zählern
	(\code{\{i\}}) anzugeben.

	\begin{minipage}[t]{.6\textwidth}
		Beispielhaftes Programm in \enquote{list2}:
		\begin{verbatim}
list from 1 to 99 by 3
    {i} bottles of beer on the wall, {i} bottles of beer.
end
		\end{verbatim}
		\end{minipage}\hfill\begin{minipage}[t]{.4\textwidth}
		Ausgabe des Programms:
		\begin{verbatim}
1 bottles of beer on the wall, 1 bottles of beer.
4 bottles of beer on the wall, 4 bottles of beer.
7 bottles of beer on the wall, 7 bottles of beer.
...
		\end{verbatim}
	\end{minipage}
\end{aufgabe}

\begin{aufgabe}[subtitle=list 3.0]
	\label{aufg:list-v3}
	\enquote{list3} soll beliebig viele \code{list}-Blöcke hintereinander erlauben.
\end{aufgabe}

\begin{aufgabe}[subtitle=list 4.0]
	\label{aufg:list-v4}
	List ist mittlerweile schon recht umfangreich. Version 4.0 von \enquote{list} soll die
	Sprache daher etwas vereinfachen und \code{from}, \code{by} und die Zählvariable
	optional machen. Die ersten beiden bekommen als Standardwert die \code{1}, die Zählvariable
	\code{i}.

	Ein minimales \enquote{list} Programm sieht dann so aus:
	\begin{verbatim}
list to 5
    Listeneintrag {i}
end
	\end{verbatim}
\end{aufgabe}

\begin{aufgabe}[subtitle=list 5.0]
	\label{aufg:list-v5}
	Version 5 wird das bisher größte Update. \enquote{list} soll nun \emph{verschachtelte}
	\code{list}-Blöcke erlauben. Dazu muss nun die Zählervariable (bisher \code{i}) immer
	mit angegeben werden.

	\begin{minipage}[t]{.6\textwidth}
		Beispielhaftes Programm in \enquote{list4}:
		\begin{verbatim}
list i from 1 to 5 by 2
    Liste {i}
    list j from 5 to 9 by 2
        - {j} Listeneintrag
    end
end
		\end{verbatim}
		\end{minipage}\hfill\begin{minipage}[t]{.4\textwidth}
		Ausgabe des Programms:
		\begin{verbatim}
Liste 1:
- 1.5 Listeneintrag
- 1.7 Listeneintrag
- 1.9 Listeneintrag
Liste 3:
- 3.5 Listeneintrag
- 3.7 Listeneintrag
...
		\end{verbatim}
	\end{minipage}

	\hinweis{Version 5 ist keine reguläre Sprache mehr. Zur Analyse muss nun also ein Kellerautomat
		verwendet werden. Konstruiere als Hilfe zunächst einen Automaten aus dem Übergangsgraphen
		in \prettyref{aufg:list-v1}. Die Kellersymbole entsprechen den Zählvariablen. Pro
		\code{list} Befehl wird die Zählvariable auf den Keller gelegt und bei \code{end} wieder entfernt.}
\end{aufgabe}

\begin{aufgabe}[subtitle=list 6.0]
	\label{aufg:list-v6}
	Das finale Update von \enquote{list} soll nochmal einige Neuerungen bringen. Die Community
	diskutiert noch, welche Features als nächstes umgesetzt werden sollten. Hier sind die Vorschläge,
	die im Moment die meisten Stimmen haben:

	\begin{itemize}
		\item Der Aufruf der Zählvariablen in der Ausgabe könnte einfache Arithmetik unterstützen.
			Zum Beispiel \code{\{i+1\}} oder \code{\{i*2\}}.
		\item Alle Zahlen (Anfangswert, Endwert, Schrittweite) dürfen auch negativ sein. (Vorsicht:
			Es könnten sich semantische Fehler einschleichen.)
		\item Die Ausgabe erlaubt derzeit nur eingeschränkte Verwendung von Leerzeichen in der Ausgabe.
			Die List \verb|1-1, 1-2, 1-3, 2-1, 2-2, 2-3| kann momentan nicht erzeugt werden. Oder
			Listeneinträge mit zwei Leerzeichen am Anfang. Das könnte geändert werden.
		\item \code{from} und \code{to} dürfen auch Buchstaben sein. \code{list from a to d \{i\} end}
			würde dann \code{a b c d} ausgeben.
	\end{itemize}
\end{aufgabe}

\end{document}
