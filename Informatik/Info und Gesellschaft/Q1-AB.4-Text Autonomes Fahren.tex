\documentclass[9pt, a4paper, twocolumn, landscape]{scrartcl}

\usepackage{vorschule}
\usepackage[
    typ=ab,
    fach=Informatik,
    lerngruppe={Q1 GK},
    nummer=4,
    module={Symbole,Lizenzen,Texte},
    seitenzahlen=immer,
    farbig,
    lizenz=cc-by-nc-sa-4,
]{schule}

\usepackage[
	kuerzel={Ngb},
	reihe={Informatik und Gesellschaft},
	version={2019-05-17},
]{ngbschule}

\author{J. Neugebauer}
\title{Autonomes Fahren - Datenschutz setzt der Freiheit Grenzen}
\date{\Heute}

\newcommand{\keineNr}[1][-1]{\addtocounter{linenumber}{#1}}
\newcommand{\seitenzahl}[1][1]{\setcounter{page}{#1}}

\begin{document}
\TITEL\vspace{-1em}

\textbf{Autonome, vernetzte Autos werden riesige Datenmengen sammeln – das weckt Ängste bei vielen Verbrauchern und alarmiert die Datenschützer.}

\begin{zeilenNr}
Ein modernes Auto sammelt schon heute pro Fahrstunde etwa 25 Gigabyte an Daten. Das entspricht dem Datenvolumen von 15 gestreamten HD-Filmen. In Zukunft werden digital vernetzte und autonom fahrende Autos Supercomputern auf Rädern gleichen, die ein Vielfaches dieser Datenmenge anhäufen – über das Fahrzeug und seinen Zustand, den Fahrer und sein Verhalten, andere Verkehrsteilnehmer, die Umwelt, die Straße, das Wetter und vieles mehr.

Mit den Daten, so hoffen Hersteller und Dienstleister, lässt sich Geld verdienen. Die Auswertung der Daten, so fürchten Datenschützer, könnte zugleich die Privatsphäre der Fahrer massiv beeinflussen. Zum Beispiel, wenn Bewegungs- und Verhaltensprofile erstellt werden oder aufgezeichnet wird, wer mit wem telefoniert oder Online-Dienste im Auto nutzt.

\enquote{Die unternehmerische Freiheit muss dort ihre Grenze finden, wo sie die informationelle Selbstbestimmung der Bürger einschränkt}, sagte die Bundesdatenschutzbeauftragte Andrea Voßhoff am Donnerstag in Berlin. Datenschutz stehe Innovationen nicht im Wege, sondern garantiere ein Grundrecht. Deshalb dürften Daten auch nie ohne ausdrückliche Einwilligung der Fahrzeugnutzer verarbeitet werden, forderte Voßhoff, es sei denn, Gesetze sprächen dagegen. Zwischen Fahrzeugen untereinander ausgetauschte Daten müssten zudem wirksam verschlüsselt und vor unbefugtem Zugriff geschützt werden. Und es müsse für den Fahrer klar erkennbar sein, welche Daten erhoben werden. \enquote{Transparenz ist hier gefordert}, sagte Voßhoff. Ähnlich wie beim Smartphone müsse der Fahrzeugnutzer personenbezogene Daten \enquote{einfach löschen} und den digitalen Status des Fahrzeugs in den Auslieferungszustand zurücksetzen können. Die Datenschutzbeauftragte veröffentlichte am Donnerstag 13 datenschutzrechtliche Empfehlungen zum automatisierten Fahren. [\dots]

Die ersten rechtlichen Rahmenbedingungen regelt ein Gesetz, das der Bundestag Ende März verabschiedet hat. Darin ist unter anderem festgelegt, dass eine \enquote{Blackbox} im Auto – ähnlich wie bei Flugzeugen – künftig Positions- und Zeitangaben speichert, wenn das Roboterauto die Steuerung übernimmt oder abgibt. Gespeichert wird für sechs Monate. Doch wer hat Zugriff auf die Daten? Und reicht die gespeicherte Datenmenge aus, um Unfälle zu rekonstruieren?

Kritisiert wird, dass das Gesetz zur Änderung des Straßenverkehrsrechts beim Datenschutz vage und unverbindlich bleibt. Es solle noch vor der Sommerpause in Kraft treten, sagte Norbert Barthle, Staatssekretär im Bundesverkehrsministerium. Aufgabe der Politik sei es, Rahmenbedingungen für die Weiterentwicklung eines \enquote{modernen, zukunftsfähigen Mobilitätsstandortes Deutschland} zu schaffen. \enquote{Ohne das Vertrauen der Menschen, dass ihre Daten sicher sind, hat das autonome Fahren keine Chance}, warnte Barthle. [\dots]

Doch die Vorbehalte der Verbraucher sind nach wie vor groß. Jürgen Bönninger, Geschäftsführer des Unternehmens FSD Fahrzeugsystemdaten, erinnerte an Umfragen, nach denen 88 Prozent der Autofahrer besorgt sind über die Offenlegung privater Daten. 86 Prozent fürchten deren kommerzielle Nutzung, 85 Prozent fragen sich, was nach einem Diebstahl persönlicher Daten passiert. Bönninger schlägt ähnlich wie Verbraucherschützer als vertrauensbildende Maßnahme ein \enquote{unabhängiges Trustcenter für Verkehrsdaten} vor. Diese neutrale Instanz soll Daten verwalten, verarbeiten und bereitstellen, die für einen reibungslosen Verkehr notwendig sind, und über den Datenschutz wachen. Außerdem dürften Hersteller oder Netzbetreiber keine \enquote{Polizeirolle} beim Datenschutz übernehmen. Vielmehr müsse es Verschlüsselungen für digitale Services geben, die der Fahrer \enquote{ohne Hintertüren} unkompliziert im Auto nutzen könne. Und: \enquote{Kunden dürfen von keinem Hersteller oder Netzbetreiber abhängig sein}, sagte Bönninger. Bei der Mobilität der Zukunft gehe es um sehr viel Geld, das Auto werde künftig Teil des Internets der Dinge. [\dots] Ein Megathema in der digitalen Welt ist für den Konzern Künstliche Intelligenz. \enquote{Maschinenlernen ist das neue Big Data}, sagte Redmer. Je größer die Datenmenge wird, die Maschinen sammeln und auswerten, desto intelligenter werden sie.
\end{zeilenNr}

\medskip
\small Quelle: Henrik Mortsiefer: \textbf{Autonomes Fahren - Datenschutz setzt der Freiheit Grenzen}, 01.06.2017,  \url{https://www.tagesspiegel.de/wirtschaft/autonomes-fahren-datenschutz-setzt-der-freiheit-grenzen/19884090.html} (Zugriff: 16.05.2019)

\begin{enumeraten}
	\item Arbeiten sie aus dem Text heraus, welche Daten Autos sammeln bzw. sammeln werden. Fallen ihnen ggf. weitere Daten ein, die von Autos gesammelt werden (könnten)? Notieren sie diese ebenfalls.
	\item Erläutern sie, welche datenschutzrechtlichen Bedenken im Text geäußert werden. Falls sie weitere datenschutzrechtliche Bedenken haben, führen sie diese ebenfalls an.
	\item Erläutern sie, wie versucht wird, gegen datenschutzrechtliche Bedenken vorzugehen. Beschreiben sie hierbei auch, welche Maßnahmen bereits durchgeführt wurden und welche ggf. noch möglich wären.
	\item Diskutieren sie ihre Ergebnisse mit ihrem Sitznachbarn und ergänzen sie ggf. gegenseitig ihre Überlegungen.
\end{enumeraten}

\end{document}