\documentclass[9pt, a4paper]{scrartcl}

\usepackage{vorschule}
\usepackage[
    typ=ab,
    fach=Informatik,
    lerngruppe={Q1 GK},
    nummer=2,
    module={Symbole,Lizenzen,Texte},
    seitenzahlen=keine,
    farbig,
    lizenz=cc-by-nc-sa-4,
]{schule}

\usepackage[
	kuerzel={Ngb},
	reihe={Informatik und Gesellschaft},
	version={2020-01-02},
]{ngbschule}

\author{J. Neugebauer}
\title{Texte zur Urheberrechtsreform 2019}
\date{\Heute}

\newcommand{\keineNr}[1][-1]{\addtocounter{linenumber}{#1}}
\newcommand{\seitenzahl}[1][1]{\setcounter{page}{#1}}
\newcommand{\neuerText}[1][1]{\clearpage\seitenzahl\resetZeilenNr}

\begin{document}

\section*{PRO: Verleger Jonathan Beck verteidigt Urheberrechtsreform}
\begin{zeilenNrZweispaltig}
Mit großer Mehrheit hat das Europäische Parlament gestern der Reform des digitalen Urheberschutzes zugestimmt. Der Verleger Jonathan Beck begrüßt die Entscheidung und findet, das EU-Parlament habe damit Stärke bewiesen, auch gegenüber YouTube und Co.

\medskip
Verleger Jonathan Beck:

\enquote{Ich bin erleichtert. Das europäische Parlament hat gestern eine Richtlinie zum Urheberrecht final verabschiedet, die, wenn sie bis 2021 in den EU-Mitgliedsstaaten in konkrete nationale Gesetze umgesetzt sein wird, es Buchverlagen in vielen Ländern und hoffentlich auch Deutschland ermöglicht, wieder gemeinsam mit ihren Autoren von sogenannten Verwertungsgesellschaften wie der VG Wort vertreten zu werden. Das wünschen sich in Deutschland nicht nur die Verlage, sondern auch die meisten Autorenverbände, denn die Interessen von Verlagen und Autoren sind zum größten Teil gleichgerichtet. Nur gemeinsam können wir diese Interessen überhaupt gegenüber den großen digitalen Plattformen erfolgreich vertreten.}

\subsection*{Content-Plattformen müssen haften}\keineNr
\enquote{Aber das ist nicht der wichtigste Teil der verabschiedeten Richtlinie. Vor allem erleichtert sie es Kulturinstitutionen wie Museen und Bibliotheken, der Allgemeinheit ihre Bestände digital zugänglich zu machen. Sie erkennt ein Leistungsschutzschutzrecht der Presseverlage auf europäischer Ebene an, und - das ist ihr umstrittenster Artikel - sie entlastet die Nutzer von großen, werbefinanzierten Content-Plattformen wie etwa YouTube von der Haftung für Urheberrechtsverletzungen. Im Gegenzug nimmt sie die Plattformen selbst in Haftung, womit diese zum Abschluss von Lizenzvereinbarungen mit den besagten Verwertungsgesellschaften verleitet werden sollen.

Wir sprechen hier von den größten, wertvollsten Unternehmen der Welt, die, wenn es um ihre neuen Geschäftsmodelle geht, gerne nach der Devise \enquote{Digital first, Bedenken second} handeln. Wenn aber für ihre, oft in Wildwestmanier betriebenen Herrschaftsbereiche vergleichbare Regeln und Regulierungsmechanismen gelten sollen wie schon lange für alle anderen Medien, dann werden die Lobbyisten und Apologeten des sogenannten freien Netzes zu großen Bedenkenträgern, man kann auch sagen, zu Panikmachern. Durch die geschickte Verbreitung von dramatischen Begriffen wie \enquote{Linksteuer}, \enquote{Uploadfilter} oder \enquote{Zensurmaschine}, durch die Bezeichnung von Ministerinnen als \enquote{Zensursula} konnten sie schon manche Regulierung des Internets abwenden. Einmal nicht geklappt hat das zuletzt beim Netzwerkdurchsetzungs-Gesetz, das im Herbst 2017 in Kraft trat und das bislang weder zu einem \enquote{Overblocking} von Inhalten noch zu anderen Einschränkungen der Meinungsfreiheit geführt hat.}

\subsection*{Sieg der Politiker über Digitalplattformen}\keineNr
\enquote{Es ist äußerst beruhigend, dass sich die meisten Abgeordneten des europäischen Parlaments nicht kirre haben machen lassen und gezeigt haben, dass politische Entscheidungen gegen die Interessen der großen Digitalplattformen noch möglich sind. Kein Gesetz ist perfekt, und jeder kann sich ab jetzt bei der Umsetzung der Richtlinie und der regelmäßigen Überarbeitung des Urheberrechts in seinem Land konstruktiv einbringen. Aber die wichtigste Nachricht ist, dass sich das europäische Parlament mit dieser Entscheidung nicht abgeschafft hat, sondern Stärke bewiesen.}
\end{zeilenNrZweispaltig}

{\small Quelle: \url{https://www.br.de/nachrichten/kultur/pro-urheberrechtsreform-verleger-beck-begruesst-verabschiedung,RLw1Y2z}}

\neuerText

\section*{Pro und Kontra: Die EU-Urheberrechtsreform}
\begin{zeilenNrZweispaltig}
Das Europäische Parlament hat am Dienstag in Straßburg die EU-Urheberrechtsreform komplett angenommen. Nach Darstellung der Befürworter soll die Reform das Urheberrecht für das Internet mit seinen großen Digital-Plattformen wie Youtube oder Google fit machen. Die bislang gültige Richtlinie stammt aus dem Jahr 2001.

Macht die EU in Sachen Urheberrechtsreform alles richtig? Die NDR Info Redakteure Michael Latz und Nils Kinkel haben unterschiedliche Ansichten. Wie ist Ihre Meinung? Schreiben Sie uns - unten auf dieser Seite.

\subsection*{Pro (Nils Kinkel)}\keineNr
Es ist gut, wenn in Europa große Plattformen wie Youtube oder Facebook endlich auch Inhalte prüfen müssen, meint Nils Kinkel.
Liebe Netzgemeinde, ich bewundere eure Kreativität und bin auch leidenschaftlicher Nutzer eurer Inhalte. Oft lache ich über GIFs und Memes - diese kleinen Schnipsel, neu aufgelegt, mit viel Humor, millionenfach geteilt. Manchmal auch illegal, weil Bilder einfach kopiert wurden.

Einige von euch sind dadurch auch reich geworden - YouTuber, Influencer. Glückwunsch! Aber nur an die Kollegen, die sich auch an das Zitat- und Bildrecht gehalten haben. Das ist vielleicht mühsam und verhindert Spontaneität. Aber nur wenn die Rechte geklärt sind, will ich eure Filme sehen. Wegen der neuen Ästhetik, der interessanten, ungewöhnlichen Erzählweise.

Keine Frage, das Internet hat die Kultur und die Kommunikation revolutioniert. Schade, dass wegen der veralteten Gesetze dabei viel zu lange Urheber abgezockt wurden. Ich kenne viele Journalisten, Fotografen und Musiker, die von ihrer kreativen Arbeit deshalb nicht mehr leben können. Obwohl große Plattformen Milliarden mit Werbung verdienen, kommt bei den Urhebern oft zu wenig an.

Wenn wir die Bevölkerung von der Digitalisierung begeistern wollen und nicht spalten, dann muss es fairer zugehen als zuletzt. Es ist deshalb gut, wenn in Europa große Plattformen wie Youtube oder Facebook endlich auch Inhalte prüfen müssen - notfalls auch mit intelligenten Suchmaschinen. Das schaffen die Programmierer schon.

Wichtiger als eine neue Technik sind aber die Menschen, die Musikern und Autoren jetzt ein faires Angebot machen müssen. Das ist gut für die Künstler - und auch gut für das Internet.

\subsection*{Kontra (Michael Latz)}\keineNr
Die EU-Urheberrechtsreform ist dazu geeignet, Kreativität und Meinungsfreiheit abzuwürgen, meint Michael Latz.
Keine Frage: Wer Melodien komponiert, Texte schreibt, Logos entwirft oder vom Fotografieren lebt, sollte auch mitentscheiden können, wo diese Werke veröffentlicht werden und Geld dafür bekommen. Das alte Netz-Motto "Für gut befunden und geklaut" fördert weder Kreativität noch Meinungsfreiheit - sondern trocknet sie aus.

Aber die EU-Urheberrechtsreform hilft da auch nicht weiter, denn sie ist dazu geeignet, Kreativität und Meinungsfreiheit abzuwürgen. Wenn Facebook, Youtube und Co. künftig Urheberrechtsverstöße verhindern sollen, dann werden sie um die heiß diskutierten Upload-Filter nicht herumkommen. Schon jetzt entscheiden zum Beispiel bei Youtube Programme, ob in den Clips geschützte Ausschnitte auftauchen. Und deshalb ist klar, wo die Grenzen liegen. Es sollen zum Beispiel schon private Urlaubsvideos blockiert worden sein, weil im Hintergrund die Melodie eines bekannten Sommerhits zu hören war.

Und nun sollen solche Filter entscheiden, ob der Schnipsel aus einem Hollywood-Film einfach nur geklaut ist oder als Satire gedacht ist? Sie sollen den Unterschied zwischen illegaler Kopie und erlaubtem Zitat erkennen? Und sie sollen auch noch die feinen Unterschiede im nationalen Urheberrecht der EU-Staaten begreifen?

Ich habe daran meine Zweifel. Und ich befürchte, dass die Online-Riesen übers Ziel hinausschießen und schon beim kleinsten Zweifel Inhalte blockieren, um jursitisch auf der sicheren Seite zu sein. Und wer in solchen Fällen glaubt, mit einem Widerspruch etwas ändern zu können, hat sich noch nie mit den namenlosen Beschwerdesystemen der Plattformen herumschlagen müssen.

Ich hätte mir einen anderen Ansatz gewünscht - einen, der das Internet offen und lebendig hält. So aber entscheiden die Plattformen, was ins Netz gelangt.

Den Kreativen hilft das kaum: Sie stehen künftig hilflos daneben, wenn ihre Verwertungsgesellschaften mit den Online-Riesen über Milliarden feilschen - und müssen es nehmen, wie es kommt. Es ist sicher nicht das Ende des freien Internets, aber wir kommen ihm ein schlechtes Stück näher. Liebe Netzgemeinde, ihr habt allen Grund, euch Sorgen zu machen!
\end{zeilenNrZweispaltig}

{\small Quelle: \url{https://www.ndr.de/nachrichten/netzwelt/Pro-und-Kontra-Die-EU-Urheberrechtsreform,urheberrechtinternet100.html}}

\neuerText\small

\section*{Das Leistungsschutzrecht – ein Zombie-Gesetz aus Deutschland wird bald in ganz Europa Realität}
\begin{zeilenNrZweispaltig}
Die EU-Kommission und die Mitgliedsstaaten arbeiten emsig an der Einführung eines europaweiten Leistungsschutzrechtes nach deutschem Vorbild. Doch kann in Europa funktionieren, was in Deutschland nie so recht klappte? Das Gesetz, das die Internetkonzerne in Deutschland zur Finanzierung von (journalistischen) Inhalten zwingen sollte, bleibt bis zum heutigen Tag ein Streitpunkt. Die Bundesregierung gibt auch fünf Jahre nach Einführung des Leistungsschutzrechts auf die Frage, ob das Gesetz auch sinnvoll ist, nur eine ausweichende Antwort. „Eine abschließende Bewertung ist nach wie vor nicht möglich. Die am 1. August 2013 in Kraft getretene Regelung ist auch heute noch umstritten", heißt es in der schriftlichen Antwort der Bundesjustizministeriums auf eine Anfrage der Grünen-Abgeordneten Tabea Rößner. Wir veröffentlichen die Antwort hier im Volltext.

Die grüne Abgeordnete Rößner sieht in der Haltung ein Zeichen der fehlenden Lernbereitschaft der Bundesregierung. „Die Antwort macht deutlich, was seit Jahren offensichtlich ist: Die Bundesregierung ist trotz anderslautenden Bekundungen die Evaluierung des Leistungsschutzrechts nicht mal angegangen und hat dies auch gar nicht vor. Und das, obwohl sie über Jahre und auf etliche unserer Anfragen hin immer wieder das Gegenteil behauptet hat", schrieb sie in eine Stellungnahme an netzpolitik.org.

\subsection*{Was am Leistungsschutzrecht faul ist}\keineNr
Worum es bei dem umstrittenen Gesetz geht: Vor fünf Jahren führte die damals noch schwarz-gelbe Bundesregierung in Deutschland das Leistungsschutzrecht für Presseverleger ein. Dieses sollte den Verlagen einen Anteil an den Werbeeinnahmen sichern, die Google und andere Plattformen mit Anzeigen rund um Links zu verlegerischen Angeboten verdient. Doch das ganze klappte nie so recht, denn die Verlage knickten rasch nach Einführung vor Google ein und erlaubten dem Konzern, Verlagsinhalte auf Google News und anderen Suchseiten kostenlos darzustellen. Für die Verlage war die Angst zu groß, von Google aus dem Index gestrichen zu werden und damit Besucher auf den eigenen Webseiten zu verlieren.

In Deutschland zerpflückt seit Jahren eine breite Gegeninitiative das Leistungsschutzrecht. Sie befürchtet, das LSR schränke die Urheberrechte von Journalistinnen und Journalisten ein und beeinträchtige überdies die Kommunikationsfreiheit aller Internet-Nutzenden, da es selbstkleine Textausschnitte und kurze Wortfolgen wie einzelne Sätze oder Überschriften zu vergütungspflichtigen Inhalten mache. Außerdem wenden die Kritikerinnen ein, dass das Gesetz ohnehin nur großen Verlagen helfe, da diese in den zu gründenden Verwertungsgesellschaften am Hebel säßen.

\subsection*{Politisch untot}\keineNr
Seit seinem Flop kurz nach der Einführung geistert das Leistungsschutzrecht als politischer Zombie durch Deutschland – als Maßnahme ist es unwirksam, doch es will nicht recht sterben. Die lange von der Bundesregierung versprochene Prüfung der Wirksamkeit des Gesetzes ist bisher ausgeblieben. Zuletzt war es kurz nach der Bundestagswahl 2017 der damalige Noch-Bundesjustizminister Heiko Maas, der die Evaluierung des Leistungsschutzrechts wiedermal aufschob.

Bei den Plänen für ein EU-Leistungsschutzrecht will sich die Bundesregierung nicht so ganz festlegen. Der Koalitionsvertrag von Unionsparteien und SPD vermied ein klares Bekenntnis dazu, die neue Digitalministerin Dorothee Bär bekräftigte in einem Interview gar ihre Ablehnung.

Das hinderte die EU-Kommission allerdings nicht, als Teil ihrer seit Jahren geplanten Urheberrechtsreform einen solchen Vorschlag zu machen. Kritiker in Brüssel warnen davor, das Leistungsschutzrecht auf europäischer Ebene bedeute eine absurde Steuerpflicht auf das Setzen von Links. Zunächst nahm die Politik die Warnung auch ernst: Ein Bericht der konservativen EU-Abgeordneten Therese Comodini an das Europaparlament aus dem Vorjahrmacht Alternativvorschläge. Auch eine wenig später vom Rechtsausschusses im EU-Parlament beauftragte Studie empfiehlt, die Idee eines EU-Leistungsschutzrechts für Presseverleger nicht weiterzuverfolgen.

\subsection*{Handeln oder tot stellen?}\keineNr
Geholfen hat das alles nichts: Der CDU-Politiker Axel Voss, der die Reform des EU-Urheberrechtes im Europaparlament seit Comodinis Abgang aus dem Europaparlament federführend verhandelt, drängt inzwischen sogar auf eine Verschärfung des Leistungsschutzrechts.

Das Gesetz könnte nun bald Realität werden. Seit vergangener Woche wird der Vorschlag von einer klaren Mehrheit der Mitgliedsstaaten formell unterstützt: Bei einem Treffen hinter verschlossenen Türen in Brüssel beschlossen die Staaten – gegen die Stimme Deutschlands – ihre gemeinsame Position für die Verhandlungen mit Parlament und Kommission. Die Große Koalition muss sich nun entscheiden, ob sie vehement Protest einlegt und zudem Druck auf den CDU-Abgeordneten Voss ausübt, den Vorschlag fallen zu lassen – oder sich tot stellt und das europäische Leistungsschutzrecht stillschweigend Gesetz werden lässt.
\end{zeilenNrZweispaltig}

{\small Quelle: \url{https://netzpolitik.org/2018/das-leistungsschutzrecht-ein-zombie-gesetz-aus-deutschland-wird-bald-in-ganz-europa-realitaet/}}
\end{document}