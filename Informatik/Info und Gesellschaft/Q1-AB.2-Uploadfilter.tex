\documentclass[10pt, a4paper]{scrartcl}

\usepackage{vorschule}
\usepackage[
    typ=ab,
    fach=Informatik,
    lerngruppe={Q1 GK},
    nummer=2,
    module={Symbole,Lizenzen},
    seitenzahlen=keine,
    farbig,
    lizenz=cc-by-nc-sa-4,
]{schule}

\usepackage[
	kuerzel={Ngb},
	reihe={Informatik und Gesellschaft},
	version={2019-05-06},
]{ngbschule}

\author{J. Neugebauer}
\title{Die Urheberrechtsreform von 2019: Uploadfilter}
\date{\Heute}

\setzeAufgabentemplate{ngbnormal}

\begin{document}
	\ReiheTitel
	
	In dem Tauschverzeichnis finden Sie einen Ordner \ordner{Uploadfilter}. Darin befinden sich zwei weitere Ordner: \ordner{vorhanden} und \ordner{hochgeladen}.
	
	Bei den Bildern in dem Ordner \ordner{vorhanden} können Sie davon ausgehen, dass eine Plattform bereits Kenntnis über diese Bilder und deren Eigentümer und Rechte besitzt.
	
	Stellen Sie sich vor, die Bilder aus dem Ordner \ordner{hochgeladen} werden nun auf diese Plattform hochgeladen und sie müssten entscheiden, ob Bilder auf der Plattform verbleiben dürfen oder nicht.
	
	\begin{aufgabe}
	Vergleichen Sie die Bilder in den einzelnen Ordnern. Fragestellungen, die dabei helfen können sind beispielsweise:
	
	\begin{itemize}
		\item Sind alle Bilder aus der selben Perspektive geschossen worden?
		\item Sind die Bilder am PC nachbearbeitet worden?
	\end{itemize}
	\end{aufgabe}
	
	\begin{aufgabe}
	Bestimmen Sie das Bild im Ordner \ordner{vorhanden}, das den Bildern aus dem Ordner \ordner{hochgeladen} zugrunde liegt.
	
	\begin{itemize}
		\item Wie lange brauchen Sie dafür?
		\item Wie lange würden Sie dafür brauchen, wenn in dem Ordner nicht nur rund 30, sondern 30.000 Bilder lägen?
		\item Könnte ein Computer Ihnen zu jeder Zeit automatisiert das korrekte Ergebnis liefern?
		\item Welche Informationen benötigt er dafür?
	\end{itemize}
	\end{aufgabe}
	
	\begin{aufgabe}
	Analysieren Sie die Schwierigkeit, die auftritt, wenn Sie versuchen das Bild zu identifizieren.
	\end{aufgabe}
	
	\begin{aufgabe}
	Nehmen Sie Stellung zu der Frage, wie ein Uploadfilter auf eines der hochgeladenen Bilder reagieren sollte. Erläutern Sie die folgenden Fragen:
	
	\begin{itemize}
		\item Sollte das Bild auf der Plattform veröffentlicht werden?
		\item Sollte das Bild gelöscht werden?
	\end{itemize}
	\end{aufgabe}
	
	\begin{aufgabe}
	Nehmen Sie Stellung zu dem Einsatz von Uploadfiltern.
	
	\begin{itemize}
		\item Sind Uploadfilter technisch umsetzbar?
		\item Wie schätzen Sie die Trefferquote von Uploadfiltern ein, d.h. werden Bilder, die urheberrechtlich geschützt sind, korrekt ausgesondert und Bilder, die nicht urheberrechtlich geschützt sind, korrekterweise erfolgreich hochgeladen?
	\end{itemize}
	\end{aufgabe}
\end{document}