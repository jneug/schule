\documentclass[10pt, a4paper]{scrartcl}

\usepackage{vorschule}
\usepackage[
	typ=ab,
	fach=Informatik,
	lerngruppe={Q1-GK},
	nummer=III.2,
	module={Symbole,Lizenzen},
	seitenzahlen=keine,
	farbig,
	lizenz=cc-by-nc-sa-4,
]{schule}

\usepackage[
	kuerzel=Ngb,
	reihe={Suchen und Sortieren},
	version={2020-02-25},
]{ngbschule}

\author{J. Neugebauer}
\title{Sortieralgorithmen}
\date{\Heute}

\setzeAufgabentemplate{ngbnormal}

\chead{}\ohead{}\ihead{}


\begin{document}
\ReiheTitel[Bubblesort]

Informiert euch im Internet über den \textbf{Bubblesort} Sortieralgorithmus. Ihr könnt mit der Adresse \url{https://link.ngb.schule/bubblesort} anfangen, oder euch eigene Informationen im Web suchen.

Versucht den Algorithmus möglichst gut verständlich, aber präzise an folgendem Beispiel zu erklären:

\begin{center}
\begin{tabular}{|*{9}{c|}}\hline
Wert & 6 & 3 & 8 & 11 & 12 & 4 & 2 & 1 \\ \hline
\rowcolor{black!20}
Index & 0 & 1 & 2 & 3 & 4 & 5 & 6 & 7 \\ \hline
\end{tabular}
\end{center}

\tipp{Im Buch ab Seite 118 wird der Algorithmus anschaulich vorgestellt.}

\rule{\textwidth}{.2pt}

\ReiheTitel[Insertionsort]

Informiert euch im Internet über den \textbf{Insertionsort} Sortieralgorithmus. Ihr könnt mit der Adresse \url{https://link.ngb.schule/insertionsort} anfangen, oder euch eigene Informationen im Web suchen.

Versucht den Algorithmus möglichst gut verständlich, aber präzise an folgendem Beispiel zu erklären:

\begin{center}
\begin{tabular}{|*{9}{c|}}\hline
Wert & 6 & 3 & 8 & 11 & 12 & 4 & 2 & 1 \\ \hline
\rowcolor{black!20}
Index & 0 & 1 & 2 & 3 & 4 & 5 & 6 & 7 \\ \hline
\end{tabular}
\end{center}

\tipp{Im Buch ab Seite 116 wird der Algorithmus anschaulich vorgestellt.}

\rule{\textwidth}{.2pt}

\ReiheTitel[Selectionsort]

Informiert euch im Internet über den \textbf{Selectionsort} Sortieralgorithmus. Ihr könnt mit der Adresse \url{https://link.ngb.schule/selectionsort} anfangen, oder euch eigene Informationen im Web suchen.

Versucht den Algorithmus möglichst gut verständlich, aber präzise an folgendem Beispiel zu erklären:

\begin{center}
\begin{tabular}{|*{9}{c|}}\hline
Wert & 6 & 3 & 8 & 11 & 12 & 4 & 2 & 1 \\ \hline
\rowcolor{black!20}
Index & 0 & 1 & 2 & 3 & 4 & 5 & 6 & 7 \\ \hline
\end{tabular}
\end{center}

\tipp{Im Buch ab Seite 114 wird der Algorithmus anschaulich vorgestellt.}

\end{document}
