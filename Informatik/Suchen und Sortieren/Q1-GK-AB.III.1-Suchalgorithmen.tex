\documentclass[10pt, a4paper]{scrartcl}

\usepackage{vorschule}
\usepackage[
	typ=ab,
	fach=Informatik,
	lerngruppe={Q1-GK},
	nummer=III.1,
	module={Symbole,Lizenzen},
	seitenzahlen=keine,
	farbig,
	lizenz=cc-by-nc-sa-4,
]{schule}

\usepackage[
	kuerzel=Ngb,
	reihe={Suchen und Sortieren},
	version={2020-01-21},
]{ngbschule}

\author{J. Neugebauer}
\title{Suchalgorithmen}
\date{\Heute}

\setzeAufgabentemplate{ngbnormal}




\begin{document}
\ReiheTitel

\begin{aufgabe}
	Kopiere das Projekt \ordner{01-Suchmaschine\_1} aus dem Tauschordner und öffne es in \programm{BlueJ}.
	
	Implementiere die Methode \code{public boolean lineareSuche(int suchzahl, int[] sucharray)}.
	
	Die Methode soll im Array \code{sucharray} nach dem Verfahren der linearen Suche nach der Zahl \code{suchzahl} suchen. Ist die Zahl vorhanden wird \code{true} zurück gegeben, sonst \code{false}.
	
	Du kannst deine Implementierung mit der Test-Klasse auf Fehler prüfen. Denk auch daran, dass du den \emph{Debugger} zur Fehlersuche nutzen kannst.
\end{aufgabe}

\begin{aufgabe}
	Implementiere die Methode \code{public boolean lineareSuche(int suchzahl, List<Integer> suchliste)}.
	
	Die Methode soll in der Liste \code{suchliste} nach dem Verfahren der linearen Suche nach der Zahl \code{suchzahl} suchen. Ist die Zahl vorhanden wird \code{true} zurück gegeben, sonst \code{false}.
\end{aufgabe}

\begin{aufgabe}
	Implementiere die Methode \code{public boolean binaereSuche(int suchzahl, int[] sucharray)}.
	
	Die Methode soll im Array \code{sucharray} nach dem Verfahren der binären Suche nach der Zahl \code{suchzahl} suchen. Ist die Zahl vorhanden wird \code{true} zurück gegeben, sonst \code{false}.
	
	Du kannst davon ausgehen, dass das Array \code{sucharray} schon aufsteigend sortiert ist.
	
	\tipp{Definiere Variablen für die untere (\code{von}) und obere (\code{bis}) Schranke des Suchbereichs. Zunächst sind sie auf \code{0} und \code{sucharray.length} gesetzt. Das mittlere Element liegt nun bei \code{(int)((von+bis)/2)}.}
	
	\tipp{Notiere dir in einer Tabelle, wie sich die Grenzen in einem kleinen Beispiel verändern und überleg dir, unter welcher Bedingung der Algorihtmus jeweils beendet wird.}
\end{aufgabe}

\end{document}
