\documentclass[11pt, a4paper, ngerman]{arbeitsblatt}

\ladeModule{theme,typo,icons,aufgaben}
\aboptionen{
	name 		= {J. Neugebauer},
	kuerzel 	= {Ngb},
	titel 		= {Suchalgorithmen},
	reihe 		= {Suchen und Sortieren},
	fach 		= {Informatik},
	kurs 		= {Q1},
	nummer 		= {IV.1},
	lizenz 		= {cc-by-nc-sa-4},
	version 	= {2021-04-20},
}


\begin{document}
\ReiheTitel

\begin{aufgabe}[subtitle=Lineare Suche I,icon=\iconComputer]
	\label{aufg:lin-array}
	Kopiere das Projekt \ordner{01-Suchmaschine} aus dem Tauschordner und öffne es in
	\programm{BlueJ}.

	Implementiere die Methode \code{public boolean lineareSuche(int suchzahl, int[] sucharray)}.

	Die Methode soll im Array \code{sucharray} nach dem Verfahren der
	\emph{linearen Suche} nach der Zahl \code{suchzahl} suchen. Ist die Zahl
	vorhanden wird \code{true} zurückgegeben, sonst
	\code{false}.

	Du kannst deine Implementierung mit der Test-Klasse auf Fehler prüfen. Denk
	auch daran, dass du den \emph{Debugger} zur Fehlersuche nutzen kannst.
\end{aufgabe}

\begin{aufgabe}[subtitle=Lineare Suche II,icon=\iconComputer]
	\label{aufg:lin-list}
	Implementiere die Methode \code{public boolean lineareSuche(int suchzahl, List<Integer> suchliste)}.

	Die Methode soll in der Liste \code{suchliste} nach dem Verfahren der
	\emph{linearen Suche} nach der Zahl \code{suchzahl} suchen. Ist die
	Zahl vorhanden wird \code{true} zurückgegeben, sonst
	\code{false}.
\end{aufgabe}

\begin{aufgabe}[subtitle=Binäre Suche,icon=\iconComputer]
	\label{aufg:bin-array}
	Implementiere die Methode \code{public boolean binaereSuche(int suchzahl, int[] sucharray)}.

	Die Methode soll im Array \code{sucharray} nach dem Verfahren der
	\emph{binären Suche} nach der Zahl \code{suchzahl} suchen. Ist die
	Zahl vorhanden wird \code{true} zurückgegeben, sonst
	\code{false}.

	Du kannst davon ausgehen, dass das Array \code{sucharray} schon
	aufsteigend sortiert ist.

	\tipp{Definiere Variablen für die untere (\code{von}) und obere
		(\code{bis}) Schranke des Suchbereichs. Zunächst sind sie auf
		\code{0} und \code{sucharray.length-1} gesetzt. Das mittlere Element
		liegt nun bei \code{(int)((von+bis)/2)}.}

	\tipp{Notiere dir in einer Tabelle, wie sich die Grenzen in einem kleinen Beispiel verändern
		und überleg dir, unter welcher Bedingung der Algorihtmus jeweils beendet wird.}
\end{aufgabe}

\begin{aufgabe}[icon=\iconHeft]
	\label{aufg:effizienz}
	\begin{enuma}
		\item Begründe, warum die binäre Suche auf linearen Listen nicht schneller als die
		      lineare Suche  sein kann.
		\item Gibt es Fälle, in denen die lineare Suche schneller als die binäre ist? Finde
		      Beispiele für Datensätze (Integer), bei denen dies der Fall ist und begründe
		      deine Entscheidung.
	\end{enuma}
\end{aufgabe}

\begin{aufgabe*}[icon=\iconLaptop]
	\label{aufg:rekursiv}
	Implementiere eine \emph{rekursive} Variante der \emph{binären Suche} (bzw. eine
	\emph{iterative}, wenn du bei \prettyref{aufg:bin-array} schon rekursiv implementiert hast).
\end{aufgabe*}

\end{document}
