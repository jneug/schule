\documentclass[10pt, a4paper, ngerman]{arbeitsblatt}

\ladeModule{theme,typo,icons,qrcodes,aufgaben}
\ladeFach[graphen]{informatik}
\aboptionen{
	name 		= {J. Neugebauer},
	kuerzel 	= {Ngb},
	titel 		= {Effizienz Quicksort},
	reihe 		= {Suchen und Sortieren},
	fach 		= {Informatik},
	kurs 		= {Q1},
	nummer 		= {IV.6},
	lizenz 		= {cc-by-nc-sa-4},
	version 	= {2021-05-20},
}

\begin{document}
\ReiheTitel

Für die Analyse des Quicksort Algorithmus hilft die Betrachtung
des \emph{Rekursionsbaums}.

Quicksort teilt den Sortierbereich in zwei
Teile auf, alle Elemente, die kleiner als das Pivot-Element sind, und alle, die
größer sind. Dann wird der Algorithmus rekursiv auf die beiden Teile angewandt.
Der Rekursionsbaum ist also ein Binärbaum. Die Aufteilung hängt dabei wesentlich
von der Wahl des Pivot-Elements ab.

Für die Eingabe \code{24,13,15,2,33,34,4,18} und die bekannte Implementierung des
Quicksort, könnte der Re\-kursions\-baum wie in \prettyref{abb:rekbaum-1} aussehen
(das Pivot-Element ist jeweils fett markiert):

\begin{figure}[ht]
	\begin{subfigure}[ht]{.5\linewidth}\centering
		\begin{forest}
		for tree={ellipse,draw}
		[{24,13,15,2,33,34,4,\textbf{18}}
			[{13,15,2,\textbf{4}}
				[{\textbf{2}}
				]
				[{13,\textbf{15}}
					[\textbf{13}]
				]
			]
			[{24,33,\textbf{34}}
				[{24,\textbf{33}}
					[\textbf{24}]
				]
			]\quad
		]
		\end{forest}
		\caption{Rekursionsbaum mit Suchbereichen}\label{abb:rekbaum-1}
	\end{subfigure}%
	\begin{subfigure}[ht]{.5\linewidth}\centering
		\begin{forest}
		for tree={circle,draw}
		[8
			[4
				[1]
				[2
					[1]
				]
			]
			[3
				[2
					[1]
				]
			]
		]
		\end{forest}
		\caption{Rekursionsbaum mit Anzahl Elementen}\label{abb:rekbaum-2}
	\end{subfigure}
	\caption{Rekursionsbaum des Quicksort}\label{abb:rekbaum}
\end{figure}

Zur Vereinfachung notieren wir nur die \emph{Anzahl der Elemente} pro rekursivem Aufruf,
wie in \prettyref{abb:rekbaum-2} zu sehen.

\begin{aufgabe}
	\label{aufg:rekursionsbaum}
	Zeichne beide Varianten des Rekursionsbaums für die Eingabe \code{3,14,5,37,28,12,21,52,19,8,32,41,44,17,1} bei Wahl
	des letzten (rechten) Elements im Sortierbereich.
\end{aufgabe}

Für die Laufzeit gibt es zwei wesentliche Faktoren: Die Teilung (Partitionierung) des Suchbereichs
abhängig vom Pivot-Element, sowie einen Verwaltungsaufwand. Für den Verwaltungsaufwand (z.B. das Tauschen
von Elementen) kann \emph{konstanter Zeitaufwand} veranschlagt werden.

\begin{aufgabe}
	\begin{wrapfix}
	\begin{wrapfigure}[4]{r}{0pt}
		\qrcode{https://qr-lernhilfen.de/mobileUrl?url=f1e0cbf3a6ba0f47}
	\end{wrapfigure}
	Bestimme den Zeitaufwand für die Partitionierung eines Sortierbereichs mit $n$ Elementen.

	\begin{itemize}
		\item Verwaltungsaufwand: $O(1)$
		\item Aufwand zur  Partitionierung: \linie[6cm]
	\end{itemize}
	\end{wrapfix}
\end{aufgabe}

\begin{aufgabe}
	\label{aufg:analyse-worst-case}
	\begin{wrapfix}
	\begin{wrapfigure}[4]{r}{0pt}
		\qrcode{https://qr-lernhilfen.de/mobileUrl?url=e70534b1f28ff954}
	\end{wrapfigure}

	Im \emph{worst case} wird immer genau das größte oder kleinste Element im Suchbereich
	als Pivot-Element gewählt.

	Zeichne den Rekursionsbaum (mit der Anzahl der Elemente) zum Beispiel oben für den worst case.

	Begründe anhand des Baumes, warum die Laufzeit im worst case in $O(n^2)$ liegt.
	\end{wrapfix}
	\hinweis{Die Summe der Zahlen von $1$ bis $n$ lässt sich zusammenfassen zu $(n+1)\cdot n\cdot 0,5$.}
\end{aufgabe}

\begin{aufgabe}
	\label{aufg:analyse-best-case}
	\begin{wrapfix}
	\begin{wrapfigure}[4]{r}{0pt}
		\qrcode{https://qr-lernhilfen.de/mobileUrl?url=82930c3bf17525c1}
	\end{wrapfigure}

	Im \emph{best case} wird immer genau das Element in der Mitte des Sortierbereichs gewählt,
	wodurch die beiden Teilbäume dieselbe Anzahl Elemente haben.

	Zeichne den Rekursionsbaum (mit der Anzahl der Elemente) zum Beispiel oben für den best case.

	Begründe anhand des Baumes, warum die Laufzeit im best case in $O(n\cdot \log{n})$ liegt.
	\end{wrapfix}
\end{aufgabe}

\begin{center}
	\qrlink{https://qr-lernhilfen.de/mobileSolution?s=39407}{Lösungen}
\end{center}

\end{document}
