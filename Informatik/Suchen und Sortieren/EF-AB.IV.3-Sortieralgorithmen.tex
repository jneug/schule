\documentclass[11pt, a5paper, landscape, ngerman]{arbeitsblatt}

\ladeModule{theme,typo,icons,aufgaben}
\aboptionen{
	name 		= {J. Neugebauer},
	kuerzel 	= {Ngb},
	titel 		= {Sortieralgorithmen},
	reihe 		= {Suchen und Sortieren},
	fach 		= {Informatik},
	kurs 		= {EF},
	nummer 		= {IV.3},
	lizenz 		= {cc-by-nc-sa-4},
	version 	= {2021-05-03},
}

\renewcommand{\Seitenzahlen}{\relax}

\begin{document}
\ReiheTitel[Bubblesort]

Informiert euch im Internet über den \textbf{Bubblesort} Sortieralgorithmus.
Ihr könnt mit der Adresse \url{https://link.ngb.schule/bubblesort} anfangen, oder euch eigene
Informationen im Web suchen.

Versucht den Algorithmus möglichst gut verständlich, aber präzise an folgendem
Beispiel zu erklären:

\begin{center}
	\begin{tabular}{|*{9}{c|}}\hline
		Wert  & 6 & 3 & 8 & 11 & 12 & 4 & 2 & 1 \\ \hline
		\rowcolor{muted.hg}
		Index & 0 & 1 & 2 & 3  & 4  & 5 & 6 & 7 \\ \hline
	\end{tabular}
\end{center}

\tipp{Im Buch ab Seite 155 wird der Algorithmus anschaulich vorgestellt.}

\rule{\textwidth}{.2pt}

\newpage
\ReiheTitel[Insertionsort]

Informiert euch im Internet über den \textbf{Insertionsort} Sortieralgorithmus.
Ihr könnt mit der Adresse \url{https://link.ngb.schule/insertionsort} anfangen, oder euch eigene
Informationen im Web suchen.

Versucht den Algorithmus möglichst gut verständlich, aber präzise an folgendem
Beispiel zu erklären:

\begin{center}
	\begin{tabular}{|*{9}{c|}}\hline
		Wert  & 6 & 3 & 8 & 11 & 12 & 4 & 2 & 1 \\ \hline
		\rowcolor{muted.hg}
		Index & 0 & 1 & 2 & 3  & 4  & 5 & 6 & 7 \\ \hline
	\end{tabular}
\end{center}

\tipp{Im Buch ab Seite 153 wird der Algorithmus anschaulich vorgestellt.}

\rule{\textwidth}{.2pt}

\newpage
\ReiheTitel[Selectionsort]

Informiert euch im Internet über den \textbf{Selectionsort} Sortieralgorithmus.
Ihr könnt mit der Adresse \url{https://link.ngb.schule/selectionsort} anfangen, oder euch eigene
Informationen im Web suchen.

Versucht den Algorithmus möglichst gut verständlich, aber präzise an folgendem
Beispiel zu erklären:

\begin{center}
	\begin{tabular}{|*{9}{c|}}\hline
		Wert  & 6 & 3 & 8 & 11 & 12 & 4 & 2 & 1 \\ \hline
		\rowcolor{muted.hg}
		Index & 0 & 1 & 2 & 3  & 4  & 5 & 6 & 7 \\ \hline
	\end{tabular}
\end{center}

\tipp{Im Buch ab Seite 151 wird der Algorithmus anschaulich vorgestellt.}

\end{document}
