\documentclass[11pt, a4paper, ngerman]{arbeitsblatt}

\ladeModule{theme,typo,icons,boxen,aufgaben}
\aboptionen{
	name 		= {J. Neugebauer},
	kuerzel 	= {Ngb},
	titel 		= {Effizienzklassen},
	reihe 		= {Suchen und Sortieren},
	fach 		= {Informatik},
	kurs 		= {Q1},
	nummer 		= {IV.4},
	lizenz 		= {cc-by-nc-sa-4},
	version 	= {2021-04-25},
}

\begin{document}
\ReiheTitel

\begin{infobox}
	\textbf{Die Landau-Notation (O-Notation)}

	In der Landau-Notation gibt es verschiedene Symbole. Für uns ist die
	\enquote{Groß-O-Notation} wichtig. Die Notation $O(n)$ bedeutet,
	dass die Laufzeit eines Algorithmus \emph{nicht schneller wächst, als die größe
	seiner Eingabe $n$}. Auf einen Suchalgorithmus bezogen
	bedeutet dies: Wenn ich in doppelt so vielen Zahlen suche, dann braucht der
	Algorithmus ungefähr doppelt so lange.

	\smallskip
	Wichtige Laufzeiten sind:\vspace{-1ex}
	\begin{multicols}{2}
		\begin{itemize}
			\item $O(1)$ - Konstante Laufzeit
			\item $O(\log(n))$ - Logarithmische Laufzeit
			\item $O(n)$ - Lineare Laufzeit
			\item $O(n\cdot\log(n))$ - Super-lineare Laufzeit
			\item $O(n^2)$ - Quadratische Laufzeit
			\item $O(n^m)$ - Polynomielle Laufzeit
			\item $O(2^n)$ - Exponentielle Laufzeit
			\item $O(n!)$ - Fakultative Laufzeit
		\end{itemize}
	\end{multicols}
	\smallskip
\end{infobox}

Um die Laufzeit der bekannten Sortieralgorithmen zu ermitteln, wollen wir die
Laufzeit für verschiedene Eingabegrößen ermitteln. Du kannst dazu deine eigenen
Implementierungen nutzen, oder die Musterlösung aus dem Tauschordner benutzen
(\ordner{03-Sortiermaschine\_ML}).

\begin{rahmen}{\Large\faStopwatch}
	Die Stoppuhr in deinem Projekt ist für unseren Zweck
	nicht genau genug. Sie misst in \emph{Millisekunden}, für kleine Arrays ist
	dies aber zu grob. Wenn du dein Projekt verwendest, musst du daher in Zeile 28
	und 34 den Aufruf \code{System.currentTimeMillis()} durch \code{System.nanoTime()} ersetzen.
	Nun misst die Uhr in \emph{Nanosekunden}!
\end{rahmen}

Wir vergleichen das Sortieren von Arrays verschiedener Größen.

\begin{aufgabe}
	Implementiert eine Methode \code{public void messen( int n )}, die ein Array mit
	$n$ Zufallszahlen erstellt und mit einer der Sortiermethoden
	sortiert. Die Zeit für das Sortieren soll mit der Stoppuhr gemessen werden
	(aber nicht das erstellen des Zufallsarrays). Die gemessene Zeit soll auf der
	Konsole ausgegeben werden. Nutze dafür den Befehl

	\code{System.out.println(n+"$\backslash$t"+zeit);}

	wobei \code{n} die Anzahl der sortierten Elemente ist und
	\code{zeit} die gemessene Zeit. \code{$\backslash$t} erzeugt einen
	Tabulator zwischen den Zahlen.

	Das zufällige Zahlenarray kann durch \code{Zahlengenerator.zufallsArray(n)} erzeugt werden.
\end{aufgabe}

\begin{aufgabe}
	Legt ein neues Tabellenblatt in \programm{LibreOffice Calc} an und notiert in der
	ersten Zeile die Überschriften \emph{n} und
	\emph{Nanosekunden}. Hier kopiert ihr die Ergebnisse der Messungen während
	des Versuchs hinein. Durch den Tabulator in der Ausgabe übernimmt
	\programm{LibreOffice} die Werte direkt in die beiden Zellen.

	Führt Messungen für verschiedene Eingabegrößen durch, zum Beispiel
	$100, 200, 300, ..., 1000, 2000, ..., 10000, ...$ und kopiert die Laufzeiten in das Tabellenblatt.
\end{aufgabe}

\begin{aufgabe}
	Erstellt ein Streudiagramm aus den Messergebnissen. Dabei soll
	$n$ auf der $x$-Achse und die
	Nanosekunden auf der $y$-Achse abgebildet werden. Was würdet
	ihr vermuten ist die Laufzeit des untersuchten Algorithmus?
\end{aufgabe}

\bigskip
Wenn noch Zeit ist wiederholt Aufgaben 2 und 3 mit den anderen
Sortieralgorithmen.
\end{document}
