\documentclass[11pt, a4paper]{arbeitsblatt}

\ladeModule{theme,tabellen}

\ladeFach[]{informatik}

\aboptionen{
	name		= {J. Neugebauer},
	kuerzel 	= {Ngb},
	titel 		= {Ticker},
	reihe 		= {Projekt: Zulda},
	fach 		= {Informatik},
	kurs 		= {Q1-LK},
	nummer 		= {2},
	lizenz 		= {cc-by-nc-sa-4},
	version 	= {2021-06-13},
}

\begin{document}
\ReiheTitel

\begin{aufgabe}[subtitle=Ticker]
	\label{aufg:ticker}
	Klassen, die das Interface \code{Ticker} implementieren, werden von
	der \programm{Engine-Alpha} in regelmäßigen Abständen aufgerufen. Sie erlauben
	so zum Beispiel die Umsetzung von Animationen und Bewegungen von
	Computergegnern.
	\begin{wrapfix}
		\begin{wrapfigure}[8]{r}{0pt}
			\begin{tabular}{|c|c|c|}\hline
				\rowcolor{ab.tabelle.kopf.hg}
				Aufruf & \code{delta} & \code{speed} \\ \hline
				       &              &              \\ \hline
				       &              &              \\ \hline
				       &              &              \\ \hline
				       &              &              \\ \hline
				       &              &              \\ \hline
				       &              &              \\ \hline
				       &              &              \\ \hline
				       &              &              \\ \hline
			\end{tabular}
		\end{wrapfigure}
		\begin{enuma}
			\item Schau dir die Methode \code{tick()} in der Klasse \code{TrankAngriff}
			      an. Wofür stehen die Variablen \code{delta} und \code{speed}?
			      Fülle in der Tabelle aus, wie sich die Werte der Variablen verändern, wenn die
			      Methode \code{tick()} mehrmals aufgerufen wird.
			\item Verändere die Initialwerte der Variablen bei ihrer Deklaration (zeilen 8 und 9)
			      und versuche zu erklären, was du siehst.
			\item Die Klasse \code{Ork} ist auch ein \code{Ticker}. Die
			      \code{tick()} Methode wird alle 250 Millisekunden aufgerufen. Dazu
			      muss sie beim \code{Manager} angemeldet werden (Zeile 70).

			      Implementiere eine Bewegung für den \code{Ork} in der
			      \code{tick()} Methode.

			      \hinweis{Nutze dazu die \code{bewege}-Methoden der Klasse
				      \code{Karte}, um den \code{Ork} über die Felder zu bewegen. \\(z.B.
				      \code{karte.bewegeRechts(this)}). Implementiere zum Beispiel eine zufällige Bewegung mit
				      der Klasse \code{Random}.}
		\end{enuma}
	\end{wrapfix}
\end{aufgabe}

\begin{aufgabe}[subtitle=Interaktionen 1]
	\label{aufg:interaktionen-1}
	Nun können sich \code{Lunk} und die Gegner in der Welt bzw. auf den
	Karten bewegen. (Wenn auch bisher nur zufällig.) Bisher gibt es aber noch keine
	Interaktion zwischen den Objekten.

	\begin{enuma}
		\item In der Klasse \code{Troll} findest Du die Variante eines Gegners, der
		      den Spieler auf der Karte verfolgt und den Spieler angreift, sobald er ihn
		      erreicht. Analysiere den Inhalt der \code{tick()} Methode und wie die
		      Objekte miteinander interagieren.
		\item Füge dem Spiel einen \code{Troll} hinzu (zum Beispiel in der
		      \code{Karte\_0} oder auf einer anderen Karte) und teste das Spiel.
	\end{enuma}
\end{aufgabe}

\begin{aufgabe}[subtitle=Interaktionen 2]
	\label{aufg:interaktionen-2}
	Interaktionen können auf verschiede Arten umgesetzt werden. Der
	\code{TrankAngriff} besitzt zum Beispiel zwar eine \code{tick()}
	Methode, dort wird aber nur die Animation des Tranks implementiert.

	\begin{enuma}
		\item Starte das Spiel und versuch den Trank einzusammeln.
		\item Viele Interaktionen finden statt, wenn sich der Spieler in der Welt bewegt.
		      Daher wird nach jeder Bewegung des Spielers das aktuelle \code{Feld}
		      auf Gegenstände geprüft. Dazu wurden in der Klasse \code{Karte}
		      einige Hilfsmethoden zur Ermittlung der Gegenstände und Gegner auf einem Feld
		      ergänzt. Studiere die Dokumentation der Klasse (in BlueJ der Reiter
		      \enquote{Dokumentation} im Editor oben rechts) und welche Methoden es gibt.
		      Analysiere dann, wie sie in den \code{bewege}-Methoden in Welt benutzt
		      werden.
		\item Implementiere einen weiteren Gegenstand (z.B. \code{TrankVerteidigung}) und teste
		      ihn im Spiel.
	\end{enuma}
\end{aufgabe}

\end{document}
