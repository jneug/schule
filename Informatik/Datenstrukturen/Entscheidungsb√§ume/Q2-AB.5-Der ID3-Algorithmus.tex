\documentclass[10pt, a4paper]{scrartcl}

\usepackage{vorschule}
\usepackage[
    typ=ab,
    fach=Informatik,
    lerngruppe={Q2},
    nummer={IV.5},
    module={Symbole,Lizenzen},
    seitenzahlen=keine,
    farbig,
    lizenz=cc-by-nc-sa-4,
]{schule}

\usepackage[
	kuerzel=Ngb,
	reihe={Nicht-lineare Datenstrukturen},
	version={2020-09-16},
]{ngbschule}

\author{J. Neugebauer}
\title{Der ID3-Algorithmus}
\date{\Heute}

\setzeAufgabentemplate{ngbnormal}

\begin{document}

\ReiheTitel

Kopiere das Projekt \ordner{06-ID3-Algorithmus} aus dem Tauschordner
und öffne es in \programm{BlueJ}.

\begin{aufgabe}
	Erstelle eine Instanz des \code{Entscheider}s und führe die Methode 
	\code{ladeDatenUndZaehle()} aus. Vergleiche die angezeigten Daten mit
	der berechneten \emph{Entropie} und dem \emph{Informationsgewinn} der Attribute.
\end{aufgabe}

\begin{aufgabe}
	Führe die Methode \code{ladeDatenUndErstelleBaum()} auf dem Objekt aus. Der
	Entschiedungsbaum wird nun rekursiv aus den Trainingsdaten aufgebaut.
	
	Skizziere den entstandenen Baum, indem du mit dem Inspektor von 
	\programm{BlueJ} die Objekte erkunden.
	
	\begin{rahmen}
		\vspace*{4cm}
	\end{rahmen}
	\medskip
\end{aufgabe}

\begin{aufgabe}
	Die Trainingsdaten werden aus der Datei \datei{titanic50.csv} im
	Projektordner geladen. Der Dateiname wird im \code{Entscheider} in der Klassenvariablen
	\code{TRAININGSDATEN} festgelegt. Statt 50 Passagierdaten sind auch Datensätze mit
	300 (\datei{titanic300.csv}) und 800 (\datei{titanic800.csv})
	vorhanden.
	
	Erstelle den Baum mit einem der anderen Trainingsdatensätze und vergleichen
	das Ergebnis mit dem ersten Baum.
	
	\symInfo\ Zum Vergleich kannst du die Methode \code{pruefeTestdaten()} verwenden, um einen Satz
	unbekannter Passagierdaten zu laden und durch den Entscheidungsbaum klassifizieren zu lassen.
\end{aufgabe}

\begin{aufgabe}
	Mit größeren Datensätzen werden die Bäume feiner ausdifferenziert und lassen sich durch
	den Inspektor nur umständlich betrachten.
	
	Implementiere die Methode \code{baumAusgeben( BinaryTree<Knoten> pBaum, int pTiefe )} wie folgt:
	
	Die Methode läuft rekursiv durch den Baum. Der Inhalt des Knotens wird ausgegeben, dann wird die Methode
	rekursiv auf dem linken, dann auf dem rechten Teilbaum aufgerufen. Die Rekursion bricht ab, wenn ein Blatt
	erreicht ist (linker und rechter Teilbaum gleich \code{null}).
	
	Gestartet wird die Ausgabe mit der Methode \code{baumAusgeben()}.
\end{aufgabe}

\begin{aufgabe}
	Die maximale Anzahl an Entscheidungen im Baum kann beim instanziieren des \code{Entscheider}s
	über den Parameter \code{pMaximaleTiefe} festgelegt werden. Teste verschiedene Werte für
	den Parameter und lass dir den entstehenden Baum mit \code{baumAusgeben()} anzeigen.
\end{aufgabe}

\begin{aufgabe}
	Das rekursive Durchlaufen des Baumes, wie in der Methode \code{baumAusgeben()}, nennt man
	\emph{Traversierung}.
	
	Lies im Buch den Abschnitt 5.2 (ab Seite 145) über Binärbäume - insbesondere über die 
	verschiedenen Arten der Traversierung. Bearbeite dann auf Seite 150 Aufgabe 3 a) und b).
	( b) in Pseudocode).
\end{aufgabe}


\end{document}
