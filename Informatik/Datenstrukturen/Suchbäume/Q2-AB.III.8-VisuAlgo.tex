\documentclass[10pt, a4paper]{scrartcl}

\usepackage{vorschule}
\usepackage[
	typ=ab,
	fach=Informatik,
	lerngruppe={Q2},
	nummer=III.8,
	module={Symbole,Lizenzen,Papiertypen},
	seitenzahlen=keine,
	farbig,
	lizenz=cc-by-nc-sa-4,
]{schule}

\usepackage[
	kuerzel=Ngb,
	reihe={Nichtlineare Datenstrukturen},
	version={2020-12-03},
]{ngbschule}

\author{J. Neugebauer}
\title{Einfügen in und löschen aus BinäreSuchbäume}
\date{\Heute}

\setzeAufgabentemplate{ngbnormal}

\usepackage{forest}

\begin{document}
\ReiheTitel

Öffne die Seite \url{https://visualgo.net/de/bst} im Browser. 

\begin{aufgabe}[subtitle=Einfügen,symbol=\symPartner\,\symLaptop]
\begin{teilaufgaben}
	\teilaufgabe Wähle unten links im Menü \menu{Erstellen > Balanced Example}. Es wird ein binärer Suchbaum mit der Wurzel \num{41} geladen.
	\teilaufgabe Wähle unten links im Menü einfügen und füge die Zahlen \num{1} und \num{77} ein. Beobachte den dargestellten Prozess des Einfügens.
	\teilaufgabe Füge folgende Zahlen in der gezeigten Reihenfolge in den Baum ein. (Du kannst mehrere Zahlen einfügen, indem Du sie mit einem Komma getrennt eingibst.)
	
		\num{25}, \num{27}, \num{26}, \num{28}, \num{23}, \num{22}, \num{24}, \num{21}
\end{teilaufgaben}
\end{aufgabe}

\begin{aufgabe}[subtitle=Löschen,symbol=\symPartner\,\symLaptop]
\begin{teilaufgaben}
	\teilaufgabe Lösche aus dem Baum aus Aufgabe 1, die Zahl \num{50}, indem Du unten links auf \menu{Entfernen} klickst.
\end{teilaufgaben}
\end{aufgabe}

\end{document}