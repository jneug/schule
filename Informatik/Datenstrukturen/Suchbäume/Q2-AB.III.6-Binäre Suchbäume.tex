\documentclass[10pt, a4paper]{scrartcl}

\usepackage{vorschule}
\usepackage[
	typ=ab,
	fach=Informatik,
	lerngruppe={Q2},
	nummer=III.6,
	module={Symbole,Lizenzen,Papiertypen},
	seitenzahlen=keine,
	farbig,
	lizenz=cc-by-nc-sa-4,
]{schule}

\usepackage[
	kuerzel=Ngb,
	reihe={Nichtlineare Datenstrukturen},
	version={2020-11-29},
]{ngbschule}

\author{J. Neugebauer}
\title{Binäre Suchbäume}
\date{\Heute}

\setzeAufgabentemplate{ngbnormal}

\usepackage{forest}


\begin{document}
\ReiheTitel

\begin{aufgabe}[symbol=\symEinzel\,\symFueller]
	\begin{center}
	\begin{forest}
		for tree={circle,draw,minimum width=8mm}
		[15
		  [8
		    [3
		      [1]
		      [4]
		    ]
		    [11
		      [,phantom]
		      [14]
		    ]
		  ]
		  [21
		    [16
		      [,phantom]
		      [18]
		    ]
		    [29
		      [25]
		      [32]
		    ]
		  ]
		]
	\end{forest}
	\end{center}
	
	\begin{teilaufgaben}
		\teilaufgabe Gib die Inorder-Ausgabe für den gezeigten Baum an.
		\teilaufgabe Erstelle zwei (möglichst ausbalancierte) Binärbäume mit derselben Inorder-Ausgabe an (die sich jeweils paarweise unterscheiden).
		\teilaufgabe Welche Eigenschaften haben die drei Binärbäume? Notiere Stichpunkte.
	\end{teilaufgaben}
\end{aufgabe}

\begin{aufgabe}[symbol=\symPartner\,\symFueller]
	Bäume dieser Art nennt man \emph{Binäre Suchbäume}.
	
	\begin{teilaufgaben}
		\teilaufgabe Vergleicht eure Notizen zur ersten Aufgabe und versucht Bäume dieser Art möglichst genau zu beschreiben.
		\teilaufgabe Formuliert gemeinsam einen präzisen Merksatz:
	\end{teilaufgaben}

	\begin{infobox}
		Ein \textbf{binärer Suchbaum} ist ein binärer Baum, bei dem 
		
		\feldLin[8mm]{4}
	\end{infobox}
\end{aufgabe}

\begin{aufgabe}[symbol=\symPartner\,\symFueller]
	Sollen neue Elemente in einen binären Suchbaum eingefügt werden, muss darauf geachtet werden, dass die Eigenschaften des Suchbaums erhalten bleiben.
	
	\begin{teilaufgaben}
		\teilaufgabe Fügt die Elemente \num{10}, \num{9}, \num{2}, \num{40} und \num{38} in den Baum oben ein.
		\teilaufgabe Notiert einen Pseudocode für einen \emph{rekursiven Algorithmus} für die einfügen Operation.
	\end{teilaufgaben}
\end{aufgabe}

\begin{aufgabe}[symbol=\symPartner\,\symLaptop]
	Ladet das Projekt \ordner{10-Muppets} in \programm{BlueJ} und erkundet den Aufbau. Beschreibt die Beziehungen der Klassen zueinander, insbesondere die Bedeutung von \code{ComparableContent<Muppet>}.
\end{aufgabe}

\end{document}
