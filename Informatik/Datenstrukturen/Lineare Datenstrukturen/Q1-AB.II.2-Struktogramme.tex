\documentclass[10pt, a4paper]{arbeitsblatt}

\ladeModule{theme,boxen}
\ladeFach[algorithmen]{informatik}

\aboptionen{
	name		= {J. Neugebauer},
	kuerzel		= {Ngb},
	titel		= {Struktogramme},
	reihe		= {Lineare dynamische Datenstrukturen},
	fach		= {Informatik},
	lerngruppe	= {Q1},
	nummer		= {II.02},
	lizenz		= {cc-by-nc-sa-4},
	version		= {2021-11-23},
}

\begin{document}
\ReiheTitel

\begin{infobox}
	Aufgabe eines \textbf{Struktogrammes} ist es, den Ablauf eines Computerprogramms
	auf dem Papier darzustellen. Dazu wurden in den 70er Jahren des vergangenen
	Jahrhunderts von \person{Isaac Nassi} und \person{Ben Shneidermann} graphische
	Grundelemente entwickelt, die es ermöglichen sollten, Programmabläufe ohne
	Sprunganweisungen darzustellen. Die Notwendigkeit ergab sich daraus, dass im
	Laufe der Zeit Computerprogramme immer komplexer und damit unübersichtlicher
	geworden waren. Mit der Einführung von Struktogrammen wurde es erforderlich,
	die Programmlogik wieder gründlich und ohne Sprünge zu planen. Man bezeichnete
	dies als \textbf{strukturierte Programmierung}.

	In der professionellen Softwareentwicklung werden Struktogramme eher selten
	eingesetzt. Dort werden vorrangig die \emph{Aktivitätsdiagramme} der UML verwendet.
\end{infobox}

\begin{aufgabe}[subtitle=Operationen der Schlange]
Unten siehst du das Struktogramm der Operation
\code{dequeue}. Studier den Aufbau und beschreib seine Elemente. Was ist
das besondere an dieser Art der Darstellung von Algorithmen? Wo könnten Vor-
und Nachteile liegen? Ergänze dann die schon bekannte
\code{enqueue}-Operation. (Hilfe findest du im Buch.)
\end{aufgabe}

\subsection*{enqueue-Operation}
\begin{rahmen}
	\vspace{5cm}
\end{rahmen}

\subsection*{dequeue-Operation}

\begin{center}
	\ttfamily\small
	\begin{struktogramm}(120,32)
	\ifthen{4}{6}{head != tail}
		\instruction[8]{head = head.getNext()}
	\change
		\ifthen{5}{5}{head != null}
			\instruction[8]{head = null}
			\instruction[8]{tail = null}
		\change
			%empty
		\ifend
	\ifend
	\end{struktogramm}
\end{center}

\end{document}
