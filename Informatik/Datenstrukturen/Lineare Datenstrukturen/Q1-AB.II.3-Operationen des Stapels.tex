\documentclass[10pt, a4paper]{arbeitsblatt}

\ladeModule{theme,boxen}
\ladeFach[algorithmen,listings,uml]{informatik}

\aboptionen{
	name		= {J. Neugebauer},
	kuerzel		= {Ngb},
	titel		= {Operationen des Stapels},
	reihe		= {Lineare dynamische Datenstrukturen},
	fach		= {Informatik},
	lerngruppe	= {Q1},
	nummer		= {II.03},
	lizenz		= {cc-by-nc-sa-4},
	version		= {2021-11-23},
}

\begin{document}
\ReiheTitel

Der generische Stapel besteht aus einer Klasse \code{Stack<ContentType>} und einer
Klasse \code{Node<ContentType>}.

\begin{center}
	\begin{tikzpicture}\small
		\begin{class}[text width=6cm]{Stack<ContentType>}{0,0}
		  \operation{+ push(content : ContentType) : void}
		  \operation{+ pop() : void}
		  \operation{+ isEmpty() : boolean}
		  \operation{+ top() : ContentType}
		\end{class}
		\begin{class}[text width=7cm]{Node<ContentType>}{8.5,0}
		  \attribute{- content : ContentType}
		  \operation{+ setNext(pNext : Node<ContentType>) : void}
		  \operation{+ getNext() : Node<ContentType>}
		  \operation{+ getContent() : ContentType}
		\end{class}
		\uses{Stack<ContentType>}{head}{0,1}{Node<ContentType>}
		\usesself{Node<ContentType>}{next}{0,1}
	  \end{tikzpicture}
\end{center}

\begin{aufgabe}
Unten sind die \emph{Struktogramme} der Operationen \code{push}
und \code{pop} abgebildet. \operator{Analysiere} ihren Aufbau und
\operator{erkläre}	das \emph{Verkettungsprinzip} der Datenstruktur
\emph{Stapel} anhand eines Beispiels.

\subsection*{push-Operation}
\begin{center}
	\ttfamily\small
	\begin{struktogramm}(100,32)
		\ifthen{5}{5}{head == null}
			\instruction[8]{node = new Node}
			\instruction[8]{head = node}
			\assign[8]{}
		\change
			\instruction[8]{node = new Node}
			\instruction[8]{node.setNext(head)}
			\instruction[8]{head = node}
		\ifend
	\end{struktogramm}
\end{center}

\subsection*{pop-Operation}
\begin{center}
	\ttfamily\small
	\begin{struktogramm}(80,16)
		\ifthen{7}{3}{head != null}
			\instruction[8]{head = head.getNext()}
		\change
			% empty
		\ifend
	\end{struktogramm}

\end{center}
\end{aufgabe}

\clearpage

\ReiheTitel

Der generische Stapel besteht aus einer Klasse \code{Stack<ContentType>} und einer
Klasse \code{Node<ContentType>}.

\begin{center}
	\begin{tikzpicture}\small
		\begin{class}[text width=6cm]{Stack<ContentType>}{0,0}
		  \operation{+ push(content : ContentType) : void}
		  \operation{+ pop() : void}
		  \operation{+ isEmpty() : boolean}
		  \operation{+ top() : ContentType}
		\end{class}
		\begin{class}[text width=7cm]{Node<ContentType>}{8.5,0}
		  \attribute{- content : ContentType}
		  \operation{+ setNext(pNext : Node<ContentType>) : void}
		  \operation{+ getNext() : Node<ContentType>}
		  \operation{+ getContent() : ContentType}
		\end{class}
		\uses{Stack<ContentType>}{head}{0,1}{Node<ContentType>}
		\usesself{Node<ContentType>}{next}{0,1}
	  \end{tikzpicture}
\end{center}

\begin{aufgabe}
Unten sind die Operationen \code{push} und \code{pop}
als \emph{Pseudocode} abgebildet. \operator{Analysiere} ihren Aufbau und
\operator{erkläre}	das \emph{Verkettungsprinzip} der Datenstruktur
\emph{Stapel} anhand eines Beispiels.

\subsection*{push-Operation}
\begin{lstlisting}
Wenn Stapel leer
	erstelle einen neuen Knoten mit dem neuen Inhalt
	setze head auf den neuen Knoten
Sonst
	erstelle einen neuen Knoten mit dem neuen Inhalt
	setze den Nachfolger des neuen Knotens auf head
	setze head auf den neuen Knoten
Ende Wenn
\end{lstlisting}

\subsection*{pop-Operation}
\begin{lstlisting}
Wenn Stapel nicht leer
	setze head auf den Nachfolger von head
Ende Wenn
\end{lstlisting}
\end{aufgabe}

\end{document}
