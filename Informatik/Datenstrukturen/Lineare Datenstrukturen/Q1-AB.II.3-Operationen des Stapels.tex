\documentclass[10pt, a4paper]{scrartcl}

\usepackage{vorschule}
\usepackage[
	typ=ab,
	fach=Informatik,
	lerngruppe={Q1},
	nummer=II.3,
	module={Symbole,Lizenzen},
	seitenzahlen=keine,
	farbig,
	lizenz=cc-by-nc-sa-4,
]{schule}

\usepackage[
	kuerzel=Ngb,
	reihe={Lineare dynamische Datenstrukturen},
	version={2020-09-09},
]{ngbschule}

\author{J. Neugebauer}
\title{Operationen des Stapels}
\date{\Heute}

\setzeAufgabentemplate{schule-keinenummer}

\begin{document}
\ReiheTitel

Der generische Stapel besteht aus einer Klasse \code{Stack<ContentType>} und einer Klasse \code{Node<ContentType>}.

\cd{stack}

\begin{aufgabe}
	Unten sind die \emph{Struktogramme} der Operationen \code{push} und \code{pop} abgebildet. \operator{Analysiere} ihren Aufbau und \operator{erkläre}  das \emph{Verkettungsprinzip} der Datenstruktur \emph{Stapel} anhand eines Beispiels.

\subsection*{push-Operation}
\nss{stack.push}

\subsection*{pop-Operation}
\nss{stack.pop}
\end{aufgabe}

\clearpage

\ReiheTitel

Der generische Stapel besteht aus einer Klasse \code{Stack<ContentType>} und einer Klasse \code{Node<ContentType>}.

\cd{stack}

\begin{aufgabe}
	Unten sind die Operationen \code{push} und \code{pop} als \emph{Pseudocode} abgebildet. \operator{Analysiere} ihren Aufbau und \operator{erkläre}  das \emph{Verkettungsprinzip} der Datenstruktur \emph{Stapel} anhand eines Beispiels.

\subsection*{push-Operation}
\begin{lstlisting}
Wenn Stapel leer 
	erstelle einen neuen Knoten mit dem neuen Inhalt
	setze head auf den neuen Knoten
Sonst
	erstelle einen neuen Knoten mit dem neuen Inhalt
	setze den Nachfolger des neuen Knotens auf head
	setze head auf den neuen Knoten
Ende Wenn
\end{lstlisting}

\subsection*{pop-Operation}
\begin{lstlisting}
Wenn Stapel nicht leer
	setze head auf den Nachfolger von head
Ende Wenn
\end{lstlisting}
\end{aufgabe}

\end{document}
