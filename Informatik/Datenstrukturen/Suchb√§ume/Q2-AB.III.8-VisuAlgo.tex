\documentclass[10pt, a4paper]{scrartcl}

\usepackage{vorschule}
\usepackage[
	typ=ab,
	fach=Informatik,
	lerngruppe={Q2},
	nummer=III.8,
	module={Symbole,Lizenzen,Papiertypen},
	seitenzahlen=keine,
	farbig,
	lizenz=cc-by-nc-sa-4,
]{schule}

\usepackage[
	kuerzel=Ngb,
	reihe={Nichtlineare Datenstrukturen},
	version={2020-12-03},
]{ngbschule}

\author{J. Neugebauer}
\title{Binäre Suchbäume}
\date{\Heute}

\setzeAufgabentemplate{ngbnormal}

\usepackage{forest}

\begin{document}
\ReiheTitel

\begin{wrapfig}
	\begin{wrapfigure}[4]{r}{0pt}
		\qrcode[height=1.5cm]{https://visualgo.net/de/bst}
	\end{wrapfigure}
	Öffnet die Seite \url{https://visualgo.net/de/bst} im Browser und schließt das Infofenster durch betätigen der \keys{Esc} Taste.
\end{wrapfig}

\begin{aufgabe}[subtitle=Einfügen,symbol=\symPartner\,\symLaptop]
\begin{teilaufgaben}
	\teilaufgabe Wählt unten links im Menü \menu{Erstellen > Balanced Example}. Es wird ein binärer Suchbaum mit der Wurzel \num{41} erstellt.
	
	\teilaufgabe Wählt unten links im Menü einfügen und fügt die Zahlen \num{1} und \num{77} ein. Beobachtet den dargestellten Prozess des Einfügens.
	
	\teilaufgabe Fügt folgende Zahlen in der gezeigten Reihenfolge in den Baum ein. (Ihr könnt mehrere Zahlen einfügen, indem Ihr sie mit einem Komma getrennt eingebt.)
	
	\begin{center}
		\num{25}, \num{27}, \num{26}, \num{28}, \num{23}, \num{22}, \num{24}, \num{21}
	\end{center}

	\hinweis{Ihr könnt die Geschwindigkeit der Simulation unten links verändern.}
\end{teilaufgaben}
\end{aufgabe}

\begin{aufgabe}[subtitle=Löschen,symbol=\symPartner\,\symLaptop]
	Beim Löschen muss darauf geachtet werden, dass die Eigenschaften eines Suchbaumes weiterhin erhalten bleiben. Dabei müssen drei unterschiedlich komplexe Situationen unterschieden werden:
	
	Der zu löschende Knoten hat \dots
	\begin{enumerate}
		\item[\dots] keine Teilbäume,
		\item[\dots] genau einen Teilbaum,
		\item[\dots] genau zwei Teilbäume.
	\end{enumerate}

	Führt folgende Aktionen aus und beobachtet genau, was passiert. Überlegt vorher, welcher Fall jeweils vorliegt.
	
	\begin{teilaufgaben}
		\teilaufgabe Löscht aus dem Baum aus Aufgabe 1, die Elemente \num{50} und \num{1}, indem Ihr unten links auf \menu{Entfernen} klickt.
		\teilaufgabe Entfernt die Elemente \num{22} und \num{65}.
		\teilaufgabe Entfernt die Elemente \num{23}, \num{41} und \num{20}. (Weitere Beispiele sind \num{21}, \num{27} und \num{29}.)
	\end{teilaufgaben}
\end{aufgabe}

\begin{aufgabe}[subtitle=Löschen II,symbol=\symPartner\,\symBuch]
	Formuliert eine Strategie für das Löschen eines Elements aus dem Suchbaum und formuliert einen Pseudocode.
	
	\tipp{Überlegt gemeinsam. Falls ihr Hilfe benötigt, lest im Buch Seite 156f nach.}
\end{aufgabe}

\begin{aufgabe}[subtitle=AVL-Bäume,symbol=\symStern\,\symPartner\,\symLaptop]
	Informiert euch im Internet über \emph{AVL-Bäume} und beschreibt den Unterschied zu \enquote{normalen} Suchbäumen. 
	
	In \emph{VisuAlgo} könnt ihr oben von \enquote{Binärer Suchbaum} auf \enquote{AVL-Baum} umschalten und den Ablauf der Operationen in dieser Art von Suchbäumen beobachten.
\end{aufgabe}

\end{document}