\documentclass[11pt, a5paper, landscape]{arbeitsblatt}

\ladeModule{theme}
\aboptionen{
	name 		= {J. Neugebauer},
	kuerzel 	= {Ngb},
	titel 		= {Eigenschaften von Algorithmen},
	reihe 		= {Einführung in die Programmierung},
	fach 		= {Informatik},
	kurs 		= {EF},
	nummer 		= {III.1},
	lizenz 		= {cc-by-nc-sa-4},
	version 	= {2021-10-05},
	%aufgaben/format = abkompakt
}

\renewcommand{\Seitenzahlen}{\relax}

\begin{document}
\ReiheTitel*

\begin{aufgabe}
Vergleicht die vorliegenden Beispiele für Algorithmen auf Gemeinsamkeiten. Was
verbindet alle drei Varianten miteinander? Versucht aus euren Entdeckungen
allgemeine Eigenschaften für Algorithmen abzuleiten.
\end{aufgabe}

\begin{aufgabe}
Wählt eines der drei Beispiele aus und erklärt daran anschaulich die beiden
folgenden Eigenschaften:
\begin{description}
	\item[Finitheit] Ein Algorithmus muss in einem \emph{endlichen Text} eindeutig beschreibbar
		sein.
	\item[Determiniertheit] Ein Algorithmus muss bei denselben Voraussetzungen (Eingaben) das gleiche
		Ergebnis (Ausgabe) liefern.
\end{description}
\end{aufgabe}

\newpage\setcounter{aufgabe}{0}
\ReiheTitel*

\begin{aufgabe}
Vergleicht die vorliegenden Beispiele für Algorithmen auf Gemeinsamkeiten. Was
verbindet alle drei Varianten miteinander? Versucht aus euren Entdeckungen
allgemeine Eigenschaften für Algorithmen abzuleiten.
\end{aufgabe}

\begin{aufgabe}
Wählt eines der drei Beispiele aus und erklärt daran anschaulich die beiden
folgenden Eigenschaften:
\begin{description}
	\item[Ausführbarkeit] Jeder Schritt eines Algorithmus muss tatsächlich ausführbar sein.
	\item[Determinismus] Die nächste anzuwendende Regel eines Algorithmus ist zu jedem Zeitpunkt
		eindeutig definiert.
\end{description}
\end{aufgabe}

\newpage\setcounter{aufgabe}{0}
\ReiheTitel*

\begin{aufgabe}
Vergleicht die vorliegenden Beispiele für Algorithmen auf Gemeinsamkeiten. Was
verbindet alle drei Varianten miteinander? Versucht aus euren Entdeckungen
allgemeine Eigenschaften für Algorithmen abzuleiten.
\end{aufgabe}

\begin{aufgabe}
Wählt eines der drei Beispiele aus und erklärt daran anschaulich die beiden
folgenden Eigenschaften:
\begin{description}
	\item[Allgemeinheit] Ein Algorithmus ist eine allgemeine Tätigkeitsbeschreibung, mit der nicht nur
		die Lösung einer einzelnen konkreten Aufgabe ermittelt wird, sondern die Lösung
		verschiedener (eventuell aller) Aufgaben einer bestimmten Klasse oder eines
		bestimmten Typs.
	\item[Determinismus] Die nächste anzuwendende Regel eines Algorithmus ist zu jedem Zeitpunkt
		eindeutig definiert.
\end{description}
\end{aufgabe}

\newpage\setcounter{aufgabe}{0}
\ReiheTitel*

\begin{aufgabe}
Vergleicht die vorliegenden Beispiele für Algorithmen auf Gemeinsamkeiten. Was
verbindet alle drei Varianten miteinander? Versucht aus euren Entdeckungen
allgemeine Eigenschaften für Algorithmen abzuleiten.
\end{aufgabe}

\begin{aufgabe}
Wählt eines der drei Beispiele aus und erklärt daran anschaulich die beiden
folgenden Eigenschaften:
\begin{description}
	\item[Terminierung] Ein Algorithmus darf nur endlich viele Schritte benötigen.
	\item[Determiniertheit] Die nächste anzuwendende Regel eines Algorithmus ist zu jedem Zeitpunkt
		eindeutig definiert.
\end{description}
\end{aufgabe}

\newpage\setcounter{aufgabe}{0}
\ReiheTitel*

\begin{aufgabe}
Vergleicht die vorliegenden Beispiele für Algorithmen auf Gemeinsamkeiten. Was
verbindet alle drei Varianten miteinander? Versucht aus euren Entdeckungen
allgemeine Eigenschaften für Algorithmen abzuleiten.
\end{aufgabe}

\begin{aufgabe}
Wählt eines der drei Beispiele aus und erklärt daran anschaulich die beiden
folgenden Eigenschaften:
\begin{description}
	\item[Terminierung] Ein Algorithmus darf nur endlich viele Schritte benötigen.
	\item[Finitheit] Ein Verfahren muss in einem endlichen Text eindeutig beschreibbar sein.
\end{description}
\end{aufgabe}

\end{document}
