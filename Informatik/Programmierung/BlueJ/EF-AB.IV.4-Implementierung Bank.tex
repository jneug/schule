\documentclass[10pt, a4paper]{scrartcl}

%\usepackage{qrcode}

\usepackage{vorschule}
\usepackage[
	typ=ab,
	fach=Informatik,
	lerngruppe={EF},
	nummer=IV.4,
	module={Symbole,Lizenzen},
	seitenzahlen=keine,
	farbig,
	lizenz=cc-by-nc-sa-4,
]{schule}

\usepackage[
	kuerzel=Ngb,
	reihe={Objektorientierte Modellierung},
	version={2020-02-20},
]{ngbschule}

\author{J. Neugebauer}
\title{Implementierung einer Banksoftware}
\date{\Heute}

\setzeAufgabentemplate{ngbnormal}

\begin{document}
\ReiheTitel

\hinweis{Auf der Rückseite seht ihr auf der rechten Seite das \emph{Implementierungsklassendiagramm} zum \emph{Entwurfsklassendiagramm} links.}

\begin{aufgabe}[symbol=\symPartner\,\symLaptop]
	Analysiert die Diagramme und vergleicht sie mit dem Entwurf aus dem Unterricht.
\end{aufgabe}

\begin{aufgabe}[symbol=\symPartner\,\symLaptop]
	Implementiert die Klasse \cls{Kunde} entsprechend des \emph{Implementierungsdiagramms}. Geht dazu so vor:
	
	\begin{itemize}
		\item Öffnet \programm{BlueJ} und erstell ein neues Projekt. Speichert das Pojekt in eurem Laufwerk (\ordner{N:\\}).
		\item Erstellt die Klasse \cls{Kunde} als leere Klasse, indem ihr auf \keys{Neue Klasse} klickt und unten \enquote{Leere Klasse} auswählt.
		\item Deklariert die Objektvariablen \attr{name}, \attr{geburtstag}, \attr{adresse} und \attr{konto} in der Klasse.
		\item Implementiert den \emph{Konstruktor} der Klasse, der die Attribute initialisiert.
		\item Implementiert die \emph{Getter} und \emph{Setter}.
	\end{itemize}
\end{aufgabe}

\begin{aufgabe}[symbol=\symPartner\,\symLaptop]
	Implementiert die Klasse \cls{Konto} entsprechend des \emph{Implementierungsdiagramms}. Geht dazu so vor:
	
	\begin{itemize}
		\item Erstellt die Klasse \cls{Konto} als leere Klasse wie oben.
		\item Deklariert die Objektvariablen der Klasse (mit Ausnahme der Transaktionen).
		\item Implementiert den \emph{Konstruktor} der Klasse, der die Attribute initialisiert. Dabei sollen \attr{kontostand}, \attr{dispo} und \attr{zinssatz} zunächst null sein und die vierstellige \attr{pin} soll zufällig generiert werden. (Siehe \cls{Random}.)
		\item Implementiert die \emph{Getter} und \emph{Setter}.
		\item Implementiert die Methode \func{einzahlen} und \func{auszahlen}. \func{auszahlen} soll \code{true} zurück gegeben werden, wenn die Auszahlung (unter Berücksichtigung des Dispos) erfolgreich war.
		\item Implementiert die Methode \func{ueberweisen}. Sie bekommt das Zielkonto als Parameter und ruft die Methode \func{einzahlen} des Zielkontos auf.
	\end{itemize}
	
	Testet die Klasse ausgiebig.
\end{aufgabe}

\begin{aufgabe}[symbol=\symPartner\,\symLaptop]
	Implementiert die Klasse \cls{Transaktion} entspechend des \emph{Implementierungsdiagramms}. Geht dazu ungefähr so vor, wie bei den anderen Klassen.
	
	Modifiziert die Klasse \cls{Konto} dann so, dass bei jeder Einzahlung, Auszahlung und Überweisung ein neues Transaktion-Objekt mit den passenden Informationen erstellt wird. Das neue Objekt soll in einer Objektvariablen \enquote{letzteTransaktion} gespeichert werden. Ergänzt auch einen entsprechenden \emph{Getter}.
\end{aufgabe}

\clearpage

\begin{tikzpicture}[scale=0.8,every node/.style={scale=0.8}]\small
%% Entwurfsdiagramm
	\begin{klasse}[text width=4cm]{Bank}{0,0}
		\attribut{name: Text}
		\attribut{blz: Zahl}
		\methode{neuerKunde}
		\methode{kontoAbrechnen}
	\end{klasse}
	
	\begin{klasse}[text width=4cm]{Kunde}{0,-3}
		\attribut{name: Text}
		\attribut{geburtstag: Text}
		\attribut{adresse: Text}
	\end{klasse}
	
	\begin{klasse}[text width=4cm]{Konto}{0,-6}
		\attribut{kontonummer: Zahl}
		\attribut{kontostand: Zahl}
		\attribut{dispo: Zahl}
		\attribut{zinssatz: Zahl}
		\attribut{pin: Zahl}
		\methode{einzahlen}
		\methode{auszahlen}
		\methode{ueberweisen}
	\end{klasse}
		
	\begin{klasse}[text width=4cm]{Transaktion}{0,-11}
		\attribut{art: Text}
		\attribut{betrag: Text}
	\end{klasse}

	\draw[umlcd style,->] ($(Bank.west) + (0,0)$) -- ($(Bank.west) + (-.5,0)$) --node[midway,left]{-kunden} ($(Kunde.west) + (-.5,.5)$) --node[below]{0..*} ($(Kunde.west) + (0,.5)$);
	
	\draw[umlcd style,->] ($(Konto.east) + (0,-1)$) -- ($(Konto.east) + (.5,-1)$) --node[midway,right]{-transaktionen} ($(Transaktion.east) + (.5,.5)$) --node[below]{0..*} ($(Transaktion.east) + (0,.5)$);
	
	\draw[umlcd style,->] ($(Bank.east) + (0,0)$) -- ($(Bank.east) + (.5,0)$) --node[midway,right]{-konten} ($(Konto.east) + (.5,.5)$) --node[below]{0..*} ($(Konto.east) + (0,.5)$);
	
	\draw[umlcd style,->] ($(Kunde.west) + (0,-.5)$) -- ($(Kunde.west) + (-.5,-.5)$) --node[midway,left]{-konto} ($(Konto.west) + (-.5,.5)$) --node[below]{1} ($(Konto.west) + (0,.5)$);
	
	\draw[umlcd style,->] ($(Transaktion.west) + (0,0)$) -- ($(Transaktion.west) + (-.5,0)$) --node[midway,left]{-zielkonto} ($(Konto.west) + (-.5,-1.2)$) --node[above]{1} ($(Konto.west) + (0,-1.2)$);

%% Implementierungsdiagramm
	\begin{klasse}[text width=6cm]{Bank}{10,0}
		\attribut{-name: String}
		\attribut{-blz: int}
		
		\methode{+Bank(pName: String, pBlz: int)}
		\methode{+neuerKunde(pName: String,\\\hspace{1em}pGeburtstag: String, pAdresse: String)}
		\methode{+kontoAbrechnen()}
	\end{klasse}
	
	\begin{klasse}[text width=6cm]{Kunde}{10,-4}
		\attribut{-name: String}
		\attribut{-geburtstag: String}
		\attribut{-adresse: String}
		
		\methode{+Kunde(pName: String, pGeburtstag: String,\\\hspace{1em}pAdresse: String, pKonto: Konto)}
		\methode{+getName(): String}
		\methode{+getGeburtstag(): String}
		\methode{+getAdresse(): String}
		\methode{+getKonto(): Konto}
	\end{klasse}
	
	\begin{klasse}[text width=6cm]{Konto}{10,-10}
		\attribut{-kontonummer: int}
		\attribut{-kontostand: double}
		\attribut{-dispo: double}
		\attribut{-zinssatz: double}
		\attribut{-pin: int}
		
		\methode{+Konto(pKontonummer: int)}
		\methode{+einzahlen(pBetrag: double)}
		\methode{+auszahlen(pBetrag: double): boolean}
		\methode{+ueberweisen(pBetrag: double,\\\hspace{1em}pZielkonto: Konto): boolean}
		\methode{+pruefePin(pEingabe: int): boolean}
		\methode{+getKontonummer(): int}
		\methode{+getKontostand(): double}
		\methode{+getDispo(): double}
		\methode{+setDispo(pDispo: double)}
		\methode{+getZinssatz(): double}
		\methode{+setZinssatz(pZinssatz: double)}
	\end{klasse}
		
	\begin{klasse}[text width=6cm]{Transaktion}{10,-19.5}
		\attribut{-art: String}
		\attribut{-betrag: double}
		
		\methode{+Transaktion(pArt: String,\\pBetrag: double)}
		\methode{+Transaktion(pBetrag: double,\\pZielkonto: Konto)}
		\methode{+getArt(): String}
		\methode{+getBetrag(): double}
		\methode{+getZielkonto(): Konto}
	\end{klasse}
	
	\draw[umlcd style,->] ($(Bank.west) + (0,0)$) -- ($(Bank.west) + (-.5,0)$) --node[midway,left]{-kunden} ($(Kunde.west) + (-.5,.5)$) --node[below]{0..*} ($(Kunde.west) + (0,.5)$);
		
	\draw[umlcd style,->] ($(Konto.east) + (0,-1)$) -- ($(Konto.east) + (.5,-1)$) --node[midway,right]{-transaktionen} ($(Transaktion.east) + (.5,.5)$) --node[below]{0..*} ($(Transaktion.east) + (0,.5)$);
	
	\draw[umlcd style,->] ($(Bank.east) + (0,0)$) -- ($(Bank.east) + (.5,0)$) --node[midway,right]{-konten} ($(Konto.east) + (.5,.5)$) --node[below]{0..*} ($(Konto.east) + (0,.5)$);
	
	\draw[umlcd style,->] ($(Kunde.west) + (0,-.5)$) -- ($(Kunde.west) + (-.5,-.5)$) --node[midway,left]{-konto} ($(Konto.west) + (-.5,.5)$) --node[below]{1} ($(Konto.west) + (0,.5)$);
	
	\draw[umlcd style,->] ($(Transaktion.west) + (0,0)$) -- ($(Transaktion.west) + (-.5,0)$) --node[midway,left]{-zielkonto} ($(Konto.west) + (-.5,-1.2)$) --node[above]{1} ($(Konto.west) + (0,-1.2)$);
\end{tikzpicture}

\end{document}