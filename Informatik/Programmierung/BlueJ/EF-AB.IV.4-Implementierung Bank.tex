\documentclass[10pt, a4paper]{scrartcl}

%\usepackage{qrcode}

\usepackage{vorschule}
\usepackage[
	typ=ab,
	fach=Informatik,
	lerngruppe={EF},
	nummer=IV.4,
	module={Symbole,Lizenzen},
	seitenzahlen=keine,
	farbig,
	lizenz=cc-by-nc-sa-4,
]{schule}

\usepackage[
	kuerzel=Ngb,
	reihe={Objektorientierte Modellierung},
	version={2020-02-20},
]{ngbschule}

\author{J. Neugebauer}
\title{Implementierung einer Banksoftware}
\date{\Heute}

\setzeAufgabentemplate{ngbnormal}


\begin{document}
\ReiheTitel

\hinweis{Auf der Rückseite seht ihr auf der rechten Seite das \emph{Implementierungsklassendiagramm} zum \emph{Entwurfsklassendiagramm} links.}

\begin{aufgabe}[symbol=\symPartner\,\symLaptop]
	Analysiert die Diagramme und vergleicht sie mit dem Entwurf aus dem Unterricht.
\end{aufgabe}

\begin{aufgabe}[symbol=\symPartner\,\symLaptop]
	Implementiert die Klasse \cls{Kunde} entsprechend des \emph{Implementierungsdiagramms}. Geht dazu so vor:
	
	\begin{itemize}
		\item Öffnet \programm{BlueJ} und erstell ein neues Projekt. Speichert das Pojekt in eurem Laufwerk (\ordner{N:\\}).
		\item Erstellt die Klasse \cls{Kunde} als leere Klasse, indem ihr auf \keys{Neue Klasse} klickt und unten \enquote{Leere Klasse} auswählt.
		\item Deklariert die Objektvariablen \attr{name}, \attr{geburtstag}, \attr{adresse} und \attr{konto} in der Klasse.
		\item Implementiert den \emph{Konstruktor} der Klasse, der die Attribute initialisiert.
		\item Implementiert die \emph{Getter} und \emph{Setter}.
	\end{itemize}
\end{aufgabe}

\begin{aufgabe}[symbol=\symPartner\,\symLaptop]
	Implementiert die Klasse \cls{Konto} entsprechend des \emph{Implementierungsdiagramms}. Geht dazu so vor:
	
	\begin{itemize}
		\item Erstellt die Klasse \cls{Konto} als leere Klasse wie oben.
		\item Deklariert die Objektvariablen der Klasse (mit Ausnahme der Transaktionen).
		\item Implementiert den \emph{Konstruktor} der Klasse, der die Attribute initialisiert. Dabei sollen \attr{kontostand}, \attr{dispo} und \attr{zinssatz} zunächst null sein und die vierstellige \attr{pin} soll zufällig generiert werden. (Siehe \cls{Random}.)
		\item Implementiert die \emph{Getter} und \emph{Setter}.
		\item Implementiert die Methode \func{einzahlen} und \func{auszahlen}. \func{auszahlen} soll \code{true} zurück gegeben werden, wenn die Auszahlung (unter Berücksichtigung des Dispos) erfolgreich war.
		\item Implementiert die Methode \func{ueberweisen}. Sie bekommt das Zielkonto als Parameter und ruft die Methode \func{einzahlen} des Zielkontos auf.
	\end{itemize}
	
	Testet die Klasse ausgiebig.
\end{aufgabe}

\begin{aufgabe}[symbol=\symPartner\,\symLaptop]
	Implementiert die Klasse \cls{Transaktion} entspechend des \emph{Implementierungsdiagramms}. Geht dazu ungefähr so vor, wie bei den anderen Klassen.
	
	Modifiziert die Klasse \cls{Konto} dann so, dass bei jeder Einzahlung, Auszahlung und Überweisung ein neues Transaktion-Objekt mit den passenden Informationen erstellt wird. Das neue Objekt soll in einer Objektvariablen \enquote{letzteTransaktion} gespeichert werden. Ergänzt auch einen entsprechenden \emph{Getter}.
\end{aufgabe}

\clearpage



\end{document}