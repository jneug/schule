\documentclass[10pt, a4paper]{scrartcl}

%\usepackage{qrcode}

\usepackage{vorschule}
\usepackage[
	typ=ab,
	fach=Informatik,
	lerngruppe={EF},
	nummer=IV.2,
	module={Symbole,Lizenzen},
	seitenzahlen=keine,
	farbig,
	lizenz=cc-by-nc-sa-4,
]{schule}

\usepackage[
	kuerzel=Ngb,
	reihe={Objektorientierte Modellierung},
	version={2020-01-28},
]{ngbschule}

\author{J. Neugebauer}
\title{Eigene Klassen erstellen}
\date{\Heute}

\setzeAufgabentemplate{ngbnormal}


\begin{document}
\ReiheTitel

\begin{aufgabe}
	Erstell ein neues, leeres \programm{BlueJ}-Projekt mit dem Namen \ordner{Bank}. Erstell dann im Projekt eine neue Klasse \code{Konto}.
\end{aufgabe}

\begin{aufgabe}
	Überleg dir, welche Eigenschaften (\emph{Objektvariablen}) und Fähigkeiten (\emph{Methoden}) eine Klasse \code{Konto} sinnvoller Weise haben sollte und \textbf{zeichne} ein \emph{Klassendiagramm}. Achte darauf, dass mindestens die Eigenschaft Kontostand (Datentyp \code{double}), sowie eine weitere Eigenschaft vorhanden sind. Außerdem muss der \emph{Konstruktor} der Klasse und zwei sinnvolle Methoden vorhanden sein. Darüber hinaus kannst du weitere Ergänzen.
	
	Implementiere \textbf{danach} dein Klassendiagramm im Projekt.
\end{aufgabe}

\begin{aufgabe}
	Auf ein Konto muss man Geld einzahlen und davon wieder abheben können. Aber vorsicht: Wenn das Konto nicht genug \emph{Deckung} aufweist (nicht genug Geld vorhanden ist), dann darf auch nichts abgehoben werden.
	
	Implementiere diese Vorgaben, falls noch nicht geschehen.
\end{aufgabe}

\begin{aufgabe}
	Viele Banken bieten ihren Kunden einen \emph{Dispositionskredit} an, damit auch Geld von einem Konto abgehoben werden kann, das nicht gedeckt ist. Jedes Konto hat einen eigenen \enquote{Dispo-Rahmen}, wie weit es \emph{überzogen} werden kann.
	
	Recherchiere den Begriff \enquote{Dispositionskredit} und was er bedeutet.

	Implementiere dann eine Dispo-Funktion im Projekt mit Hilfe einer (ggf. neuen) Objektvariablen \code{dispo} und passe die \code{abheben}-Methode entsprechend an.
\end{aufgabe}

\begin{aufgabe}
	Am Ende eines Monats wird das Konto \emph{abgerechnet}. Das bedeutet, wenn ein negativer Kontostand vorhanden ist (also der Dispo-Kredit in Anspruch genommen wurde) wird ein Zins berechnet und vom Kontostand abgezogen. Sonst bleibt der Kontostand gleich.
	
	Implementiere eine Methode \code{abrechnen}, die dies übernimmt. Der \emph{Zinssatz} ist in einer Objektvariablen (Datentyp \code{double}) gespeichert (im Nomalfall ca. \prozent{10,25} p.A.).
\end{aufgabe}

\end{document}
