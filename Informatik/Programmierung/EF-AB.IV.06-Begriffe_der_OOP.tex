\documentclass[11pt, a4paper, ngerman]{arbeitsblatt}

\ladeModule{theme,tabellen}

\ladeFach[]{informatik}

\aboptionen{
	name		= {J. Neugebauer},
	kuerzel 	= {Ngb},
	titel 		= {Begriffe der OOP},
	reihe 		= {Objektorientierte Programmierung},
	fach 		= {Informatik},
	kurs 		= {EF},
	nummer 		= {IV.6},
	lizenz 		= {cc-by-nc-sa-eu-4},
	version 	= {2022-03-11},
}

\begin{document}
\ReiheTitel

\begin{center}
\begin{longtable}{p{.2\linewidth}|p{.2\linewidth}|p{.5\linewidth}}
	\rowcolor{ab.tabelle.kopf.hg}
	Begriff & Synonyme & Bedeutung \\ \hline \hline

	Klasse & & \Zeilenabstand[2cm] \\\hline
	Objekt & & \Zeilenabstand[2cm] \\\hline
	Attribut & Objektvariable \newline Eigenschaft & \Zeilenabstand[2cm] \\\hline
	Methode & Funktion \newline Fähigkeit & \Zeilenabstand[2cm] \\\hline
	Konstruktor & & \Zeilenabstand[2cm] \\\hline
	Getter & Get-Methode & \Zeilenabstand[2cm] \\\hline
	Setter & Set-Methode & \Zeilenabstand[2cm] \\\hline
	Assoziation & Beziehung & \Zeilenabstand[2cm] \\\hline
	Zustand &  & \Zeilenabstand[2cm] \\\hline
\end{longtable}
\end{center}

\end{document}
