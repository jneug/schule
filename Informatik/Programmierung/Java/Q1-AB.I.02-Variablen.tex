\documentclass[10pt, a4paper, ngerman]{arbeitsblatt}

\ladeModule{theme,tabellen}

\ladeFach[quelltexte]{informatik}

\aboptionen{
	name		= {J. Neugebauer},
	kuerzel 	= {Ngb},
	titel 		= {Variablen},
	reihe 		= {Wiederholung der OOP},
	fach 		= {Informatik},
	kurs 		= {Q1},
	nummer 		= {I.2},
	lizenz 		= {cc-by-nc-sa-4},
	version 	= {2021-09-15},
}

\begin{document}
\ReiheTitel

\begin{aufgabe}
Studiere die Algorithmen auf der linken Seite sorgfältig. Notiere dann in der
rechten Spalte die Werte der Variablen nachdem der Algorithmus beendet ist.

\tipp{Gehe den Ablauf des Algorithmus durch und notiere die Veränderungen der
Variablen in jedem Schritt in einer Tabelle.}

\begin{enuma}
	\item
	\begin{links}[.69]
		\begin{minted}[linenos=false]{java}
		int i;
		\end{minted}
	\end{links}\begin{rechts}[.29]
		\begin{description}
			\item[\code{i}] =
		\end{description}
	\end{rechts}

	\item
	\begin{links}[.69]
		\begin{minted}[linenos=false]{java}
		double zahl = 10.5;
		\end{minted}
	\end{links}\begin{rechts}[.29]
		\begin{description}
			\item[\code{zahl}] =
		\end{description}
	\end{rechts}

	\item
	\begin{links}[.69]
		\begin{minted}[linenos=false]{java}
		double zahl1 = 2.3;
		double zahl2;
		double zahl3 = 2 * zahl1;
		zahl2 = zahl1 + zahl3;
		\end{minted}
	\end{links}\begin{rechts}[.29]
		\begin{description}
			\item[\code{zahl1}] =
			\item[\code{zahl2}] =
			\item[\code{zahl3}] =
		\end{description}
	\end{rechts}

	\item
	\begin{links}[.69]
		\begin{minted}[linenos=false]{java}
		int i = 0;
		while( i < 10 ) {
			int j = 2;
			i = i + j;
		}
		\end{minted}
	\end{links}\begin{rechts}[.29]
		\begin{description}
			\item[\code{i}] =
			\item[\code{j}] =
		\end{description}
	\end{rechts}

	\item
	\begin{links}[.69]
		\begin{minted}[linenos=false]{java}
		int zahl = 0;
		int vari = 5;
		while( vari > 0 ) {
			int j = 0;
			while( j < 5 ) {
				vari = vari - 1;
				j = j + 1;
			}
			zahl = zahl + 1;
		}
		\end{minted}
	\end{links}\begin{rechts}[.29]
		\begin{description}
			\item[\code{zahl}] =
			\item[\code{vari}] =
			\item[\code{j}] =
		\end{description}
	\end{rechts}

	\item
	\begin{links}[.69]
		\begin{minted}[linenos=false]{java}
		double intt;
		intt = 0.0;
		while( intt < 8.0 ) {
			int t = 4;
			intt = intt + 0.5;
			t = t + 1;
		}
		int f = 15;
		intt = 15.0;
		\end{minted}
	\end{links}\begin{rechts}[.29]
		\begin{description}
			\item[\code{intt}] =
			\item[\code{t}] =
			\item[\code{f}] =
		\end{description}
	\end{rechts}
\end{enuma}
\end{aufgabe}


\begin{aufgabe}[subtitle=Gültigkeitsbereiche]
Studiere die Algorithmen sorgfältig. Markiere die Gültigkeitsbereiche der
Variablen im Quelltext. Einige Variablenzugriffe sind ungültig, da sie
außerhalb des Gültigkeitsbereiches liegen. Markiere auch diese.

\begin{minted}{java}
public void test() {
	boolean wahrheit = false;
	if( wahrheit ) {
		int i = 2;
	}
	else {
		int i = 4;
	}
	if( i == 4 ) {
		wahrheit = true;
	}
}
\end{minted}

\begin{minted}{java}
public void test() {
	double intt = 0.0;
	while( intt < 8.0 ) {
		int t = 4;
		intt = intt + 0.5;
		t = t + 1;
	}
	t = t + 1;
}
\end{minted}

\begin{minted}{java}
public void test() {
	int i = 4;
	for( int j = 0; j < 5; j = j + 1 ) {
		fahre();
	}
	if( i < 10 ) {
		j = j - 1;
	}
}
\end{minted}
\end{aufgabe}

\end{document}
