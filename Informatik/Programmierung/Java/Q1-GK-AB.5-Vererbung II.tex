\documentclass[10pt, a4paper]{scrartcl}

\usepackage{vorschule}
\usepackage[
    typ=ab,
    fach=Informatik,
    lerngruppe={Q1-GK},
    nummer=5,
    module={Symbole,Lizenzen},
    seitenzahlen=keine,
    farbig,
    lizenz=cc-by-nc-sa-4,
]{schule}

\usepackage[
	kuerzel=Ngb,
	reihe={Objektorientierte Programmierung},
	version={2019-11-8},
]{ngbschule}

\author{J. Neugebauer}
\title{Vererbung II}
\date{\Heute}

\setzeAufgabentemplate{ngbnormal}

\begin{document}

\ReiheTitel

\begin{aufgabe}
	Kopiert und öffnet das Projekt \ordner{05-Vererbung\_1} aus dem Tauschordner.
	
	Die Klasse \code{Dreieck} ist nicht als \emph{Unterklasse} von \code{Form} implementiert. Das soll geändert werden.
\end{aufgabe}

\begin{aufgabe}
	Kopiert und öffnet das Projekt \ordner{05-Vererbung\_2} aus dem Tauschordner.
	
	Studiert den Aufbau. Die beiden Klassen haben einige Attribute und Methoden gemeinsam. Erstellt eine geeignete \emph{Oberklasse}, die die gemeinsamen \emph{Attribute} und \emph{Methoden} sammelt. Die bestehenden Klassen werden zu \emph{Unterklassen}.
\end{aufgabe}

\begin{aufgabe}
	Kopiert und öffnet das Projekt \ordner{05-Vererbung\_3} aus dem Tauschordner.
	
	\begin{enumerate}
		\item Ergänzt mindestens drei weitere Tiere im Projekt. Orientiert euch an den gezeigten Klassen.
		\item Überschreibt die Methode \code{sagWas()} in den \emph{Unterklassen}.
		\item Ergänzt neue \emph{Oberklassen}, die Tiere einer Art zusammenfassen (z.B: Fliegend, Schwimmend, Laufend ...).
	\end{enumerate}
\end{aufgabe}

\end{document}
