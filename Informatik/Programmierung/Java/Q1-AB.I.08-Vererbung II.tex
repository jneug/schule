\documentclass[10pt, a4paper]{arbeitsblatt}

\ladeModule{theme}
\ladeFach[]{Informatik}

\aboptionen{
	name		= {J. Neugebauer},
	kuerzel		= {Ngb},
	titel		= {Übungen zur Vererbung},
	reihe		= {Objektorientierte Programmierung},
	fach		= {Informatik},
	lerngruppe	= {Q1},
	nummer		= {I.08},
	lizenz		= {cc-by-nc-sa-4},
	version		= {2022-09-21},
}

\begin{document}
\ReiheTitel

\begin{aufgabe}[icon=\iconComputer]
	Fork und klone das Projekt \ordner{Formen} vom Git-Server.

	Die Klasse \code{Dreieck} ist nicht als \emph{Unterklasse} von \code{Form} implementiert. Das soll geändert werden.
\end{aufgabe}

\begin{aufgabe}[icon=\iconComputer]
	Fork und klone das Projekt \ordner{Autos} vom Git-Server.

	Analysiere den Aufbau. Die vorhandenen Klassen haben einige \emph{Attribute} und \emph{Methoden} gemeinsam. Erstelle eine geeignete \emph{Oberklasse} (\textbf{Generalisierung}), die die gemeinsamen \emph{Attribute} und \emph{Methoden} sammelt. Die bestehenden Klassen werden zu \emph{Unterklassen} (\textbf{Spezialisierung}).

	Erkläre die Begriffe \textbf{Generalisierung} und \textbf{Spezialisierung} mit eigenen Worten.
\end{aufgabe}

\begin{aufgabe}[icon=\iconComputer]
	Fork und klone das Projekt \ordner{Zoo} vom Git-Server.

	Analysiere den Aufbau der \emph{Gehege} Klassen und vergleiche Gemeinsamkeiten und Unterschiede.

	Erstelle Objekte der Tier-Klassen und versuche sie, den verschiedenen Gehegen zuzuweisen (auch Gehegen, in die sie eigentlich nicht gehören). Bei welchen Kombinationen klappt es, bei welchen nicht? Versuche zu erklären, woran das liegt.

	Manchmal können die Objekte einer Klasse durch Objekte einer anderen Klasse ersetzt werden (\textbf{Substitutionsprinzip}). Erkläre anhand des Zoo-Projektes, unter welchen Bedingungen dies erlaubt ist.
\end{aufgabe}

\end{document}
