\documentclass[10pt, a4paper]{arbeitsblatt}

\ladeModule{theme}
\ladeFach[]{Informatik}

\aboptionen{
	name		= {J. Neugebauer},
	kuerzel		= {Ngb},
	titel		= {Vererbung II},
	reihe		= {Objektorientierte Programmierung},
	fach		= {Informatik},
	lerngruppe	= {Q1},
	nummer		= {I.07},
	lizenz		= {cc-by-nc-sa-4},
	version		= {2022-02-08},
}

\begin{document}
\ReiheTitel

\begin{aufgabe}[icon=\iconComputer\,\iconPartner]
	Kopiert und öffnet das Projekt \ordner{Vererbung\_1} aus dem Tauschordner.

	Die Klasse \code{Dreieck} ist nicht als \emph{Unterklasse} von \code{Form} implementiert. Das soll geändert werden.
\end{aufgabe}

\begin{aufgabe}[icon=\iconComputer\,\iconPartner]
	Kopiert und öffnet das Projekt \ordner{Vererbung\_2} aus dem Tauschordner.

	Studiert den Aufbau. Die beiden Klassen haben einige Attribute und Methoden gemeinsam. Erstellt eine geeignete \emph{Oberklasse}, die die gemeinsamen \emph{Attribute} und \emph{Methoden} sammelt. Die bestehenden Klassen werden zu \emph{Unterklassen}.
\end{aufgabe}

\begin{aufgabe}[icon=\iconComputer\,\iconPartner]
	Kopiert und öffnet das Projekt \ordner{Vererbung\_3} aus dem Tauschordner.

	\begin{enumerate}
		\item Ergänzt mindestens drei weitere Tiere im Projekt. Orientiert euch an den gezeigten Klassen.
		\item Überschreibt die Methode \code{sagWas()} in den \emph{Unterklassen}.
		\item Ergänzt neue \emph{Oberklassen}, die Tiere einer Art zusammenfassen (z.B: \enquote{Fliegend}, \enquote{Schwimmend}, \enquote{Laufend}, ...).
	\end{enumerate}
\end{aufgabe}

\end{document}
