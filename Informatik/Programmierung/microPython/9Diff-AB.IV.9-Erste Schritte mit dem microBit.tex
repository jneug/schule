\documentclass[10pt, a4paper]{scrartcl}

\usepackage{vorschule}
\usepackage[
	typ=ab,
	fach=Informatik,
	lerngruppe={9Diff},
	nummer=IV.9,
	module={Symbole,Lizenzen},
	seitenzahlen=keine,
	farbig,
	lizenz=cc-by-nc-sa-4,
]{schule}

\usepackage[
	kuerzel=Ngb,
	reihe={micro:Bit Programmierung mit microPython},
	version={2020-01-20},
]{ngbschule}

\author{J. Neugebauer}
\title{Erste Schritte mit dem micro:Bit}
\date{\Heute}

\setzeAufgabentemplate{ngbnormal}

\begin{document}
\ReiheTitel

Der micro:bit ist ein winziger programmierbarer Computer, der mit verschiedenen Eingaben (Knöpfe, Beschleunigungssensor) und Ausgaben (LEDs) ausgestattet ist. Der micro:Bit kann mit verschiedenen Programmiersprachen gesteuert werden. Unter anderem mit einer visuellen Programmiersprache (Scratch) oder Python.

\begin{aufgabe}[symbol=\symLaptop]
	Starte das Programm \enquote{Mu-Editor} aus dem Ordner \ordner{Informatik}. Wähle oben rechts unter \enquote{Modus} den \enquote{micro:Bit Modus} aus. Verbinde nun den micro:Bit mit dem roten USB-Kabel mit dem Computer.
\end{aufgabe}

\begin{aufgabe}[symbol=\symLaptop]
	Übertrage das folgende Programm in den Editor.
	
	\begin{lstlisting}[language=Python]
from microbit import *

while True:
    if button_a.is_pressed():
        display.show(Image.HAPPY)
    elif button_b.is_pressed():
        break
    else:
        display.show(Image.SAD)

display.clear()
	\end{lstlisting}
	
	Klicke dann oben auf den \enquote{Übertragen}-Button. Die orangene Statusleuchte des micro:Bit beginnt während der Übertragung zu flackern. Dann startet das Programm automatisch.
	
	\hinweis{Achte beim übertragen auf die korrekten Einrückungen!}
\end{aufgabe}

\begin{aufgabe}[symbol=\symBuch]
	Analysiere das Programm und erkläre, wie es funktioniert.
	
	Notiere dir alle neuen Befehle mit einer Erklärung, wie sie funktionieren.
\end{aufgabe}

\begin{aufgabe}[symbol=\symLaptop]
	Teste die beiden gezeigten Programme und erkläre ihre Funktionsweise. Notiere dir wieder die neuen Befehle mit einer kurzen Erklärung.
	\vspace{-1em}\setlength{\columnsep}{.8cm}
	\begin{multicols}{2}
	\begin{lstlisting}[language=Python]
from microbit import *
from random import randint

while True:
	x = randint(0, 4)
	y = randint(0, 4)
	display.set_pixel(x, y, 6)
	sleep(200)
	\end{lstlisting}
	
	\columnbreak
	
	\begin{lstlisting}[language=Python]
from microbit import *
from math import floor as abrunden

i = 0
while i < 1000:
    if accelerometer.is_gesture("shake"):
        i += 1
    rows = abrunden(i/200)-1
    leds = "66666" + ":66666"*rows
    display.show(Image(leds))
    sleep(5)
	\end{lstlisting}
	\end{multicols}
\end{aufgabe}

\begin{aufgabe}[symbol=\symLaptop]
	Programmiere das Spiel \enquote{Achtung, Bombe!}. Das Spiel geht so: 
	
	\begin{rahmen}
	Beim Start wird der Timer der \enquote{Bombe} auf einen zufälligen Wert gesetzt. Beim Start beginnt der Timer abzulaufen. Der Startspieler gibt nun die \enquote{Bombe} an den nächsten Spieler weiter. Dieser an seinen Nachfolger, und so weiter. Der Spieler, bei dem die Bombe \enquote{explodiert}, hat verloren.
	\end{rahmen}
	
	Überlege dir, wie du das Spiel mit dem micro:Bit umsetzen möchtest. Das Spiel sollte starten, wenn ein Knopf betätigt wird. Die \enquote{Explosion} sollte passend dargestellt werden. Denke auch an die Beispielprogramme und welche Befehle du für dein Spiel verwenden kannst.
\end{aufgabe}

\begin{aufgabe}[symbol=\symLaptop]
	Verbessere dein \enquote{Achtung, Bombe!} Spiel mit weiteren Funktionen:
	\begin{itemize}
		\item Füge einen Effekt hinzu, der die tickende Bombe darstellt.
		\item Die Geschwindigkeit des Timers könnte durch Schütteln erhöht werden.
		\item Es könnte eine \enquote{letzte Chance} geben. Bei der Explosion kann der letzte Spieler schnell beide Knöpfe drücken und so die Explosion noch einmal aufschieben.
		\item Überlege dir weitere Verbesserungen.
	\end{itemize}
\end{aufgabe}

\begin{infobox}
Du kannst alle verfügbaren Befehle für den micro:Bit und viele Beispiele in der Dokumentation nachlesen (auf englisch): 

\url{https://link.ngb.schule/microbit}
\end{infobox}

\end{document}
