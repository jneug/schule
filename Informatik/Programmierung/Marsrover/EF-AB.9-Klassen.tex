\documentclass[10pt, a4paper, landscape, twocolumn]{scrartcl}

%\usepackage{qrcode}

\usepackage{vorschule}
\usepackage[
	typ=ab,
	fach=Informatik,
	lerngruppe={EF},
	nummer=9,
	module={Symbole,Lizenzen},
	seitenzahlen=keine,
	farbig,
	lizenz=cc-by-nc-sa-4,
]{schule}

\usepackage[
	kuerzel=Ngb,
	reihe={Objektorientierte Programmierung},
	version={2020-01-13},
]{ngbschule}

\author{J. Neugebauer}
\title{Klassen}
\date{\Heute}

\setzeAufgabentemplate{ngbnormal}


\begin{document}

\begin{lstlisting}[language=Java,frame=none]
import java.util.Random;

public class Schueler {
	
	public static int zufallszahl( int pMin, int pMax ) {
		Random r = new Random();
		return r.nextInt(pMax-pMin)+pMin;
	}	
	
	private String vorname;
	
	private String nachname;

	private int notenpunkte;
	
	private Schueler sitznachbar;
	
	
	public Schueler( String pVorname, String pNachname ) {
		vorname = pVorname;
		nachname = pNachname;
		
		notenpunkte = Schueler.zufallszahl(0,15);
	}
	
	public String getName() {
		return vorname + " " + nachname;
	}
	
	public int getNotenpunkte() {
		return note;
	}
	
	public Schueler getSitznachbar() {
		return sitznachbar;
	}
	
	public void setSitznachbar( Schueler pNeuerNachbar ) {
		sitznachbar = pNeuerNachbar;
	}
	
	public int notenschnittErmitteln() {
		if( sitznachbar != null ) {
			return Math.round((sitznachbar.getNotenpunkte() + notenpunkte)/2);
		} else {
			return notenpunkte;
		}
	}
	
	
	public static void main( String[] args ) {
		Schueler kim = new Schueler("Kim", "Meyer");
		Schueler johann = new Schueler("Johann", "Schmidt");
		kim.setSitznachbar(johann);
		System.out.println(kim.notenschnittErmitteln());
		
	}
	
}
\end{lstlisting}

\end{document}
