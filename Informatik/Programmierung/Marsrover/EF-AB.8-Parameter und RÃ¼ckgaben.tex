\documentclass[10pt, a4paper]{scrartcl}

%\usepackage{qrcode}

\usepackage{vorschule}
\usepackage[
	typ=ab,
	fach=Informatik,
	lerngruppe={EF},
	nummer=8,
	module={Symbole,Lizenzen},
	seitenzahlen=keine,
	farbig,
	lizenz=cc-by-nc-sa-4,
]{schule}

\usepackage[
	kuerzel=Ngb,
	reihe={Objektorientierte Programmierung},
	version={2019-12-16},
]{ngbschule}

\author{J. Neugebauer}
\title{Parameter und Rückgaben}
\date{\Heute}

\setzeAufgabentemplate{ngbnormal}

\chead{\LARGE\symPin}


\begin{document}
\ReiheTitel[Parameter]

Eine besondere Art von Variablen sind \emph{Parametervariablen}. Methoden können keinen, einen oder auch mehrere \emph{Parameter} besitzen. Sie sind lokale Variablen, die nur in der Methode \emph{gültig} sind. Das Besondere ist, dass sie erst beim Aufruf der Methode \emph{initialisiert} (also mit einem Wert belegt) werden.

Du kennst schon einige Methoden mit Parametern:
\begin{smallitemize}
	\item \code{void drehe( String richtung )}
	\item \code{boolean huegelVorhanden( String richtung )}
\end{smallitemize}

Die \emph{Signatur} der Methoden zeigen an, dass sie beiden eine \emph{Parametervariable} vom Typ \code{String} übergeben bekommen: \code{drehe("rechts");}

Bei jedem \emph{Aufruf} der Methode kann dem \emph{Parameter} ein anderer Wert \emph{übergegeben} werden.

\subsubsection*{Aufgabe 1}
Überlegt gemeinsam, welchen Vorteil \emph{Parametervariablen} haben. Vor allem, dass man bei jedem Aufruf andere Werte \emph{übergeben} kann.

\subsubsection*{Aufgabe 2}
Implementiert eine Methode \code{void fahren( int strecke )}, die den Rover bei jedem Aufruf so viele Felder fahren lässt, wie in der \emph{Parametervariablen} übergeben wird.

\subsubsection*{Aufgabe 3}
Implementiert eine Methode \code{void fahreRechteck( int kante )}, die die Methode \code{fahren} von oben benutzt, um den Rover ein Rechteck mit der angegebenen Kantenlänge fahren zu lassen.

\vspace{2cm}
\begin{tcolorbox}[colframe=black!60,colback=black!10,width=\textwidth,height=5cm,center,boxrule=1pt]
\begin{center}
Hier den Arbeitauftrag der zweiten Phase einkleben.
\end{center}
\end{tcolorbox}

\clearpage
\ReiheTitel[Rückgaben]

Eine besondere Art von Variablen sind \emph{Rückgabevariablen}. Methoden können keine oder eine \emph{Rückgabe} besitzen. Sie sind das \emph{Ergebnis} einer Methode, das nach dem Ende der Methode weitergenutzt werden kann. Sie sind also Werte, die aus der Methode \enquote{zurück gegeben} werden.

Du kennst \emph{Rückgaben} schon von den \emph{Anfragen} des Rovers:
\begin{smallitemize}
	\item \code{boolean gesteinVorhanden()}
	\item \code{boolean markeVorhanden()}
\end{smallitemize}

Die \emph{Signatur} der Methoden zeigen an, dass sie beide als \emph{Rückgabe} einen Wert vom Typ \code{boolean} \emph{zurück geben}: \code{boolean wahrheit = gesteinVorhanden();}

\emph{Rückgabevariablen} haben keinen eigenen \emph{Bezeichner}.

\subsubsection*{Aufgabe 1}
Überlegt gemeinsam, welchen Vorteil \emph{Rückgabevariablen} haben.

\subsubsection*{Aufgabe 2}
Implementiert eine Methode \code{int zaehleMarken()}, die den Rover die Marken auf dem Feld zählen lässt, und die Anzahl als \code{Rückgabe} hat.

\hinweis{Ihr könnt in Greenfoot mehrere Marken auf ein Feld legen.}

\subsubsection*{Aufgabe 3}
Implementiert eine Methode \code{int zaehleGesteine()}, die den Rover fünf Felder fahren, und auf dem Weg
alle Gesteine analysieren lässt. Die Anzahl Gesteine soll \code{zurück gegeben} werden.

\vspace{2cm}
\begin{tcolorbox}[colframe=black!60,colback=black!10,width=\textwidth,height=5cm,center,boxrule=1pt]
\begin{center}
Hier den Arbeitauftrag der zweiten Phase einkleben.
\end{center}
\end{tcolorbox}

\end{document}
