\documentclass[10pt, a4paper]{scrartcl}

%\usepackage{qrcode}

\usepackage{vorschule}
\usepackage[
	typ=ab,
	fach=Informatik,
	lerngruppe={EF},
	nummer=6,
	module={Symbole,Lizenzen},
	seitenzahlen=keine,
	farbig,
	lizenz=cc-by-nc-sa-4,
]{schule}

\usepackage[
	kuerzel=Ngb,
	reihe={Objektorientierte Programmierung},
	version={2019-11-27},
]{ngbschule}

\author{J. Neugebauer}
\title{Variablen}
\date{\Heute}

\setzeAufgabentemplate{ngbnormal}


\begin{document}

\ReiheTitel

\begin{aufgabe}
	Studiere die Algorithmen auf der linken Seite sorgfältig. Notiere dann in der rechten Spalte die Werte der Variablen nachdem der Algorithmus beendet ist.
	
	\tipp{Gehe den Ablauf des Algorithmus durch und notiere die Veränderungen der Variablen in jedem Schritt in einer Tabelle.}
	
	\medskip
	\begin{tabularx}{\textwidth}{|X|X|} \hline
		\rowcolor{gray!50!white} \textbf{Algorithmus} & \textbf{Variablen nach Ablauf} \\ \hline
		\code{int i;} & \code{i = } \\ \hline
		\code{int i = 5;} & \code{i = } \\ \hline
		\code{double zahl;} & \code{zahl = } \\ \hline
		\code{double zahl1 = 2.3;\newline
		double zahl2;\newline
		double zahl3 = zahl1 + zahl2;} & \code{zahl1 = }\newline\code{zahl2 = }\newline\code{zahl3 = } \\ \hline
		\code{int i = 0;\newline
while( i < 10 ) \{\newline
\hspace*{4em}int j = 2;\newline
\hspace*{4em}i = i + 1;\newline
\}} & \code{i = }\newline\code{j = } \\ \hline
		\code{String a = "x";\newline
String b = "y";\newline
int x = 0;\newline
while( x < 4 ) \{\newline
\hspace*{4em}a = a + b;\newline
\hspace*{4em}x = x + 1;\newline
\}} & \code{a = }\newline\code{b = }\newline\code{x = } \\ \hline
	\end{tabularx}
\end{aufgabe}



\end{document}
