\documentclass[10pt, a4paper]{scrartcl}

%\usepackage{qrcode}

\usepackage{vorschule}
\usepackage[
	typ=ab,
	fach=Informatik,
	lerngruppe={EF},
	nummer=6,
	module={Symbole,Lizenzen},
	seitenzahlen=keine,
	farbig,
	lizenz=cc-by-nc-sa-4,
]{schule}

\usepackage[
	kuerzel=Ngb,
	reihe={Objektorientierte Programmierung},
	version={2019-12-30},
]{ngbschule}

\author{J. Neugebauer}
\title{Variablen}
\date{\Heute}

\setzeAufgabentemplate{ngbnormal}

\lstset{
	frame=,
	backgroundcolor=
}

\begin{document}

\ReiheTitel

\begin{savelst}{task4}
\begin{lstlisting}[language=java]
double zahl1 = 2.3;
double zahl2;
double zahl3 = zahl1 + zahl2;
\end{lstlisting}
\end{savelst}

\begin{savelst}{task5}
\begin{lstlisting}[language=java]
int i = 0;
while( i < 10 ) {
	int j = 2;
	i = i + 1;
}
\end{lstlisting}
\end{savelst}

\begin{savelst}{task6}
\begin{lstlisting}[language=java]
String a = "x";
String b = "y";
int x = 0;
while( x < 4 ) {
	a = a + b;
	x = x + 1;
}
\end{lstlisting}
\end{savelst}

\begin{savelst}{task7}
\begin{lstlisting}[language=java]
int zahl = 0;
int var = 5;
while( var > 0 ) {
	int j = 0;
	while( j < 5 ) {
		var = var - 1;
		j = j + 1;
	}
	zahl = zahl + 1;
}
\end{lstlisting}
\end{savelst}

\begin{savelst}{task8}
\begin{lstlisting}[language=java]
double intt;
while( intt < 8.0 ) {
	int t = 4;
	intt = intt + 0.5;
	t = t + 1;
}
int f = 15;
intt = 15.0;
\end{lstlisting}
\end{savelst}


\begin{aufgabe}
	Studiere die Algorithmen auf der linken Seite sorgfältig. Notiere dann in der rechten Spalte die Werte der Variablen nachdem der Algorithmus beendet ist.
	
	\medskip
	\tipp{Gehe den Ablauf des Algorithmus durch und notiere die Veränderungen der Variablen in jedem Schritt in einer Tabelle.}
	
	\medskip
	\begin{tabularx}{\textwidth}{|X|p{4cm}|} \hline
		\rowcolor{gray!50!white} \textbf{Algorithmus} & \textbf{Variablen nach Ablauf} \\ \hline
		\tablelst{int i;} & \code{i = } \\ \hline
		\tablelst{int i = 5;} & \code{i = } \\ \hline
		\tablelst{double zahl;} & \code{zahl = } \\ \hline
		\loadlst{task4} & \code{zahl1 = }\newline\code{zahl2 = }\newline\code{zahl3 = } \\ \hline
		\loadlst{task5} & \code{i = }\newline\code{j = } \\ \hline
		\loadlst{task6} & \code{a = }\newline\code{b = }\newline\code{x = } \\ \hline
		\loadlst{task7} & \code{zahl = }\newline\code{var = }\newline\code{j = } \\ \hline
		\loadlst{task8} & \code{intt = }\newline\code{t = }\newline\code{f = } \\ \hline
	\end{tabularx}
\end{aufgabe}



\end{document}
