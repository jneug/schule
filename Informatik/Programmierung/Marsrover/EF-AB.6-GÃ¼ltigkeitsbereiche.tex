\documentclass[10pt, a4paper]{scrartcl}

%\usepackage{qrcode}

\usepackage{vorschule}
\usepackage[
	typ=ab,
	fach=Informatik,
	lerngruppe={EF},
	nummer=6,
	module={Symbole,Lizenzen},
	seitenzahlen=keine,
	farbig,
	lizenz=cc-by-nc-sa-4,
]{schule}

\usepackage[
	kuerzel=Ngb,
	reihe={Objektorientierte Programmierung},
	version={2019-12-30},
	ersteAufgabe=2
]{ngbschule}

\author{J. Neugebauer}
\title{Gültigkeitsbereiche}
\date{\Heute}

\setzeAufgabentemplate{ngbnormal}


\begin{document}

\ReiheTitel

\begin{aufgabe}
	Studiere die Algorithmen sorgfältig. Markiere die Gültigkeitsbereiche der Variablen im Quelltext. Einige Variablenzugriffe sind ungültig, da sie außerhalb des Gültigkeitsbereiches liegen. Markiere auch diese.
\end{aufgabe}

\begin{lstlisting}[language=java]
public void test1() {
    boolean wahrheit;
    if( wahrheit ) {
        int i = 2;
    } else {
        int i = 4;
    }
    if( i == 4 ) {
        wahrheit = true;
    }
}
\end{lstlisting}

\begin{lstlisting}[language=java]
public void test2() {
    double intt;
    while( intt < 8.0 ) {
        int t = 4;
        intt = intt + 0.5;
        t = t + 1;
    }
    t = t + 1;
}
\end{lstlisting}

\begin{lstlisting}[language=java]
public void test3() {
    int i = 4;
    for( int j = 0; j < 5; j = j + 1 ) {
        fahre();
    }
    if( i < 10 ) {
        j = j - 1;
    }
}
\end{lstlisting}

\begin{aufgabe}
	Wähle einen der drei Algorithmen oben aus und gehe seinen Ablauf zeilenweise durch. Notiere zu jedem Schritt die Werte aller zu diesem Zeitpunkt gültigen Variablen in einer Tabelle.
\end{aufgabe}


\end{document}
