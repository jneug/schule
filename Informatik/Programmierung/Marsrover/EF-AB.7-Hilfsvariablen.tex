\documentclass[10pt, a4paper]{scrartcl}

%\usepackage{qrcode}

\usepackage{vorschule}
\usepackage[
	typ=ab,
	fach=Informatik,
	lerngruppe={EF},
	nummer=7,
	module={Symbole,Lizenzen},
	seitenzahlen=keine,
	farbig,
	lizenz=cc-by-nc-sa-4,
]{schule}

\usepackage[
	kuerzel=Ngb,
	reihe={Objektorientierte Programmierung},
	version={2019-11-27},
]{ngbschule}

\author{J. Neugebauer}
\title{Hilfsvariablen}
\date{\Heute}

\setzeAufgabentemplate{ngbnormal}




\begin{document}

\ReiheTitel

\subsection*{Syntax}
\begin{center}
	\begin{tikzpicture}[syntaxdiagramm]
		\node [] {};
		\node [nonterminal] {Datentyp};
		\node [nonterminal] {Bezeichner};
		\node (opt) [point] {};
		\node [terminal] {=};
		\node [nonterminal] {Initialwert};
		\node (semi) [point] {};
		\node [terminal] {;};
		\node (end) [endpoint] {};
		{
			[start chain=opt]
			\node (optbelow) [point,below=of opt] {};
			\node (endbelow) [point,below=of semi] {};
			\draw [->] (opt) -- (optbelow) -- (endbelow) -- (semi);
		}
	\end{tikzpicture}
\end{center}

\subsection*{Datentypen in Java}
\begin{tabularx}{\textwidth}{c|c|c|X}
	\rowcolor{gray!50!white}Wertetyp & Datentyp & Initialwert & Beispiel \\\hline\hline
	Ganze Zahl & \code{int} & \code{0} & \code{int zahl;} \newline \code{int zahl = 5;} \\ \hline
	Fließkommazahl & \code{double} & \code{0.0} & \code{double kommazahl;} \newline \code{double kommazahl = 1.2;} \\ \hline
	Wahrheitswert & \code{boolean} & \code{false} & \code{boolean wahrheit;} \newline \code{boolean wahrheit = true;} \\ \hline
	Text & \code{String} & \code{null} & \code{String text;} \newline \code{String text = "Hallo, Welt!";} \\ \hline
\end{tabularx}

\subsection*{Vergleichsoperatoren in Java}
\begin{tabularx}{\textwidth}{c|c|X}
	\rowcolor{gray!50!white}Vergleich & Operator & Beispiel \\\hline\hline
	Gleichheit & \code{==} & \code{if( zahl == 4 ) \{ \dots\ \}} \\\hline
	Ungleichheit & \code{!=} & \code{int i = 10;\newline while( i != 0 ) \{ i -= 1; \}} \\\hline
	Größer / Kleiner & \code{>} / \code{<} & \code{if( a > 1 \&\& a < 8 ) \{ \dots\ \}} \\\hline
	Textvergleich & \code{equals()} & \code{String text = "Hallo, Welt!"\newline if( text.equals("Hallo, Welt!") ) \{ \dots\ \}} 
\end{tabularx}

\vspace{1cm}
\begin{aufgabe}
	Der Rover steht vor einer Reihe Steine, die mit einer Marke abgeschlossen sind. Er soll bis zur Marke vorfahren und auf dem Weg alle Gesteine einsammeln. Dann soll er umdrehen und zu seinem Startfeld zurückkehren.
	
	Versuche die Aufgabe zu lösen, ohne weitere Marken zu platzieren.
\end{aufgabe}

\begin{aufgabe}
	Verallgemeinere Aufgabe 1: Der Rover soll eine zufällige\footnote{Du kannst \code{Greenfoot.getRandomNumber(max)} verwenden.} Strecke fahren und dieselbe Strecke zurück fahren.
\end{aufgabe}

\begin{aufgabe}
	Programmiere den Rover so, dass er \emph{genau fünf Steine} aufnimmt.
\end{aufgabe}

\begin{aufgabe}
	Der Rover soll ein Quadrat mit einer Kantenlänge von 8 Feldern abfahren und an jeder Ecke eine Marke platzieren.
\end{aufgabe}

\end{document}
