\documentclass[10pt, a4paper]{scrartcl}

\usepackage{vorschule}
\usepackage[
	typ=ab,
	fach=Informatik,
	lerngruppe={9Diff},
	nummer=IV.8,
	module={Symbole,Lizenzen},
	seitenzahlen=keine,
	farbig,
	lizenz=cc-by-nc-sa-4,
]{schule}

\usepackage[
	kuerzel=Ngb,
	reihe={Imperative Programmierung mit Python},
	version={2020-01-13},
]{ngbschule}

\author{J. Neugebauer}
\title{Bedingte Schleifen}
\date{\Heute}

\setzeAufgabentemplate{ngbnormal}

\begin{document}
\ReiheTitel

\begin{aufgabe}
	\emph{Was macht die Turtle denn jetzt schon wieder?}
	
	Gib das Programm in \programm{TigerJython} ein und beschreibe seine Arbeitsweise.
	Erkläre besonders die Zeilen \num{10} und \num{15}.
	
	\begin{lstlisting}[language=python]
from gturtle import *

def dreieck():
    repeat 3:
        forward(100) 
        right(120)
        
makeTurtle()
runde = 1
while runde <= 6:   # neu!
    if runde == 1 or runde == 3 or runde == 5:
        setPenColor("red")
    else:
        setPenColor("green")
    fillToPoint(0,0)  # neu !
    dreieck()
    right(60)
    runde = runde + 1
	\end{lstlisting}
\end{aufgabe}

\begin{aufgabe}
	\begin{teilaufgaben}
		\teilaufgabe Lies im Buch auf Seite \num{96} die Box \enquote{Neue Konzepte und Befehle}.
		\teilaufgabe Bearbeite im Buch auf Seite 97 Aufgabe 35.
		\teilaufgabe Schreibe ein Programm, dass alle Quadratzahlen ausgibt, die kleiner als ein Parameter \code{zahl} sind.
		\teilaufgabe Entwickle einen Befehl, der eine Zufallszahl generiert (\code{randint(min, max)}) und dann den Benutzer nach einer Zahl fragt. Ist die Zahl größer wird \enquote{größer} angezeigt, ist die Zahl kleiner \enquote{kleiner}. Dies wird solange wiederholt, bis die Eingabe gleich der Zahl ist. Dann wird \enquote{Treffer!} angezeigt.
	\end{teilaufgaben}
\end{aufgabe}
	
	
\subsubsection*{Logische Operatoren}
\begin{tabularx}{\textwidth}{|c|X|} \hline
\code{and} & Und-Verknüpfung; beide Bedingungen müssen \emph{wahr} sein \\ \hline
\code{or} & Oder-Verknüpfung; midestens eine Bedingungen muss \emph{wahr} sein \\ \hline
\code{not} & Negation; Umkehrung des Wahrheitsgehaltes (aus \emph{wahr} wird \emph{falsch}) \\ \hline
\code{==} & prüft zwei Werte oder Variablen auf Gleichheit (auch Texte) \\ \hline
\code{!=} & prüft zwei Werte oder Variablen auf Ungleichheit (auch Texte) \\ \hline
\code{>} & prüft ob der erste Werte echt größer ist als der Zweite \\ \hline
\code{>=} & prüft ob der erste Werte größer oder gleich der Zweite ist \\ \hline
\code{<} & prüft ob der erste Werte echt kleiner ist als der Zweite \\ \hline
\code{<=} & prüft ob der erste Werte kleiner oder gleich der Zweite ist \\ \hline
\end{tabularx}


\end{document}
