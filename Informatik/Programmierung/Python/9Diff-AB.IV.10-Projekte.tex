\documentclass[9pt, a5paper]{arbeitsblatt}

\ladeModule{theme}

\aboptionen{
	name		= {J. Neugebauer},
	kuerzel		= {Ngb},
	titel		= {Projektaufgaben},
	reihe		= {Programmierung mit TigerJython},
	fach		= {Informatik},
	lerngruppe	= {9Diff},
	nummer		= {III.10},
	lizenz		= {cc-by-nc-sa-4},
	version		= {2022-01-12},
}

\begin{document}
\ReiheTitel

Auf diesem Arbeitsblatt findest du einige offene Aufgabenstellungen und Projektideen. Such dir eine aus und bearbeite sie. (Dann wiederhole gegebenenfalls.)

Du solltest die folgenden Konzepte der Programmierung mittlerweile kennen:
\begin{multicols}{3}\centering
	Wiederholungen

	Funktionen

	Bedingte Anweisungen

	Variablen

	Parameter

	Listen
\end{multicols}

\begin{itemize}
	\item Schreibe ein Programm, das eine einfache Simulation zeigt, wie die Erde um die Sonne kreist. In der Mitte befindet sich die Sonne (ein gelber Punkt), und darum herum kreist die Erde (z. B. als blauer Punkt).

	\item Programmiere eine Bahnschranke mit Ampel, die sich langsam schliesst. Während sich die Bahnschranke langsam senkt, blinken die beiden roten Licht der Ampel abwechselnd.

	\item Zeichne einen Pac-Man. Animiere den Pac-Man so, dass er sein Maul öffnet und schließt. Lass Deinen Pac-Man vorwärts gleiten und dabei eine Reihe von \enquote{Perlen} futtern.

	\item Programmiere ein Bild einer Naturszene. Animiere Bestandteile der Szene (z.B. die Wolken oder vielleicht regnet es).

	\item Zeichne die Skyline einer Stadt mit verschiedenen Gebäuden. Definiere für jedes Gebäude eine eigene Funktion und kombiniere die Gebäude zufällig zu einer Skyline.

	\item Entwirf ein eigenes Projekt. Nimm dir nicht zu viel vor, so dass du in 90 Minuten damit gut voran kommst. Klär vorher dein Vorhaben mit deinem Lehrer ab.
\end{itemize}

\end{document}
