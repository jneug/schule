\documentclass[10pt, a4paper]{scrartcl}

\usepackage{vorschule}
\usepackage[
	typ=ab,
	fach=Informatik,
	lerngruppe={9Diff},
	nummer=IV.9,
	module={Symbole,Lizenzen},
	seitenzahlen=keine,
	farbig,
	lizenz=cc-by-nc-sa-4,
]{schule}

\usepackage[
	kuerzel=Ngb,
	reihe={Imperative Programmierung mit Python},
	version={2020-11-25},
]{ngbschule}

\author{J. Neugebauer}
\title{Listen}
\date{\Heute}

\setzeAufgabentemplate{ngbnormal}

\begin{document}
\ReiheTitel

Eine \emph{Variable} kann benutzt werden, um \emph{einen} Wert zu speichern. Möchstest du \emph{mehrere} Werte speichern, kannst du eine \emph{Liste} benutzen.

Eine Liste speichert in einer Variablen beliebig viele Werte \enquote{hintereinander}.

\begin{aufgabe}
	Studiere das Programm und überlege, was es macht. Schau dir vor allem die neuen Befehle an.
	
	Übernimm dann das Programm in TigerJython und probiere es aus.
	\begin{lstlisting}[language=Python,gobble=8]
		from gturtle import *
		
		eingabe = 1
		seiten = [] 		#neu: eine leere Liste erstellen
		while eingabe > 0:
			eingabe = input("Gib eine Zahl ein. 0 um zu beenden.")
			seiten.append(eingabe) #neu: Wert von eingabe hinten an die Liste anhängen
		
		makeTurtle()
		
		for s in seiten:	#neu: s nimmt jeden Wert in seiten an
			fd(s)
			rt(90)
	\end{lstlisting}
\end{aufgabe}

\begin{aufgabe}
	Lies den Kasten \enquote{Neue Konzepte und Befehle} auf Seite 101.
	
	Bearbeite dann auf Seite 102 das Beispiel 3 und Aufg. 3 und 4.
\end{aufgabe}

\begin{aufgabe}
	Bearbeite die folgenden Aufgaben nach eigenem Ermessen:
	
	\begin{multicols}{2}
	\begin{itemize}
		\item S.103, Aufg. 5, 6, 8 
		\item S.106, Aufg. 14
		\item S.108, Aufg. 17, 19
		\item S.109, Aufg. 21; S.110, Aufg. 23
		\item S.112, Aufg. 27, 28
	\end{itemize}
	\end{multicols}
\end{aufgabe}

\begin{rahmen}\small
Der \code{repeat}-Befehl ist kein richtiger Python-Befehl. Es gibt ihn nur in \programm{TigerJython}. In Python erzeugt man statt dessen für eine n-fache Wiederholung eine Liste mit den Zahlen 0\code{0} bis \code{n-1} und durchläuft sie mit \code{for}:
\begin{lstlisting}[language=Python]
from gturtle import *

repeat 4:			# funktioniert nur in TigerJython
	fd(100)
	rt(90)

for i in range(4):	# range ergeugt eine Liste mit vier Zahlen
	fd(100)
	rt(90)
\end{lstlisting}
\end{rahmen}

\end{document}
