\documentclass[10pt, a4paper]{arbeitsblatt}

\ladeModule{theme,boxen}
\ladeFach[quelltexte,algorithmen]{informatik}

\aboptionen{
	name		= {J. Neugebauer},
	kuerzel		= {Ngb},
	titel		= {Verzweigungen},
	reihe		= {Programmierung mit TigerJython},
	fach		= {Informatik},
	lerngruppe	= {9Diff},
	nummer		= {III.07},
	lizenz		= {cc-by-nc-sa-4},
	version		= {2022-01-09},
}

\begin{document}
\ReiheTitel

\begin{aufgabe}[icon=\iconBlatt\,\iconEinzel]
	\begin{center}
		\emph{Was macht denn die Turtle?}
	\end{center}

	Die Turtle hat anscheinend etwas viel getrunken!

	\begin{minted}{python}
	from gturtle import *
	from random import randint

	makeTurtle("sprites/alien.gif")
	repeat 100:
		forward(20)
		zufall = randint(1, 100)
		if zufall < 25:
			left(45)
		elif zufall >= 25 and zufall < 50:
			# 'elif' steht kurz für 'else-if'
			right(135)
		elif zufall >= 50 and zufall < 90:
			back(50)
		else:
			playTone(220, 800)
	\end{minted}

	\begin{teilaufgaben}
		\teilaufgabe Analysiere das Programm und beschreibe seine Arbeitsweise. Notiere unbekannte Befehle und versuche ihre Funktion zu erklären.
		\teilaufgabe Wie genau werden die Zeilen 8-16 abgearbeitet, wenn die Variable \code{zufall} den Wert \code{73} enthält?
	\end{teilaufgaben}
\end{aufgabe}

\begin{aufgabe}[icon=\iconBuch\,\iconComputer\,\iconPartner]
	\begin{teilaufgaben}
		\teilaufgabe Lest im Buch auf Seite 80 die Box \enquote{Neue Konzepte und Befehle}.
		\teilaufgabe Bearbeitet im Buch auf Seite 83 das Beispiel 7 und Aufgabe 8.
		\teilaufgabe Bearbeitet im Buch auf Seite 85 das Beispiel 9 und Aufgabe 12.
	\end{teilaufgaben}
\end{aufgabe}

\begin{aufgabe*}[icon=\iconComputer\,\iconPartner]
	Ihr sollt eine Billardkugel programmieren. Kopiert dazu aus dem Tauschordner das Programm \datei{07-billard.py} und öffnet es. Analysiert das Programm und lest die Kommentare. Ergänzt dann die fehlenden Teile.

	Nutzt als Hilfe die Bilddatei \datei{07-Abb\_Arbeitsfläche.png}.
\end{aufgabe*}

\end{document}
