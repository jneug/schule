\documentclass[10pt, a4paper]{scrartcl}

\usepackage{vorschule}
\usepackage[
	typ=ab,
	fach=Informatik,
	lerngruppe={9Diff},
	nummer=IV.7,
	module={Symbole,Lizenzen},
	seitenzahlen=keine,
	farbig,
	lizenz=cc-by-nc-sa-4,
]{schule}

\usepackage[
	kuerzel=Ngb,
	reihe={Imperative Programmierung mit Python},
	version={2020-01-12},
]{ngbschule}

\author{J. Neugebauer}
\title{Verzweigungen}
\date{\Heute}

\setzeAufgabentemplate{ngbnormal}

\begin{document}
\ReiheTitel

\begin{aufgabe}
	\emph{Was macht denn die Turtle?}
	
	Die Turtle hat anscheinend etwas viel getrunken! 
	
	\begin{lstlisting}[language=python]
from gturtle import *
from random import randint

makeTurtle("sprites/alien.gif")
repeat 100:
    forward(20)
    zufall = randint(1, 100)
    if zufall < 25:
        left(45)
    elif zufall >= 25 and zufall < 50:
        # 'elif' steht kurz für 'else-if'
        right(135)
    elif zufall >= 50 and zufall < 90:
        back(50)
    else:
        playTone(220, 800)
	\end{lstlisting}
	
	\begin{teilaufgaben}
		\teilaufgabe Analysiere das Programm und beschreibe seine Arbeitsweise. Notiere unbekannte Befehle und versuche ihre Funktion zu erklären.
		\teilaufgabe Wie genau werden die Zeilen 8-16 abgearbeitet, wenn die Variable \code{zufall} den Wert \num{73} enthält?
	\end{teilaufgaben}
\end{aufgabe}
	
	\begin{aufgabe}
		\begin{teilaufgaben}
			\teilaufgabe Lies im Buch auf Seite \num{80} die Box \enquote{Neue Konzepte und Befehle}.
			\teilaufgabe Bearbeite im Buch auf Seite 83 das Beispiel 7 und Aufgabe 8.
			\teilaufgabe Bearbeite im Buch auf Seite 85 das Beispiel 9 und Aufgabe 12.
		\end{teilaufgaben}
	\end{aufgabe}

	\begin{aufgabe}
		Du sollst eine Billardkugel programmieren. Kopiere dazu aus dem Tauschordner das Programm \datei{07-billard.py} und öffne es. Analysiere das Programm und lies die Kommentare. Ergänze dann die fehlenden Teile.
		
		Nutze als Hilfe die Bilddatei \datei{07-Abb\_Arbeitsfläche.png}. 
	\end{aufgabe}
	
\end{document}
