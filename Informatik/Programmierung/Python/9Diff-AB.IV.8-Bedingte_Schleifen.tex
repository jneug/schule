\documentclass[10pt, a4paper]{arbeitsblatt}

\ladeModule{theme,boxen,tabellen}
\ladeFach[quelltexte]{informatik}

\aboptionen{
	name		= {J. Neugebauer},
	kuerzel		= {Ngb},
	titel		= {Bedingte Schleifen},
	reihe		= {Programmierung mit TigerJython},
	fach		= {Informatik},
	lerngruppe	= {9Diff},
	nummer		= {III.08},
	lizenz		= {cc-by-nc-sa-4},
	version		= {2022-01-09},
}

\begin{document}
\ReiheTitel

\begin{aufgabe}[icon=\iconComputer\,\iconPartner]
\vspace*{-1em}
\begin{center}
	\emph{Was macht die Turtle denn jetzt schon wieder?}
\end{center}

Gebt das Programm in \programm{TigerJython} ein und beschreibt seine
Arbeitsweise. Erklärt besonders die Zeilen 10 und 15.

\begin{minted}{python}
from gturtle import *

def dreieck():
	repeat 3:
		forward(100)
		right(120)

makeTurtle()
runde = 1
while runde <= 6:   # neu!
	if runde == 1 or runde == 3 or runde == 5:
		setPenColor("red")
	else:
		setPenColor("green")
	fillToPoint(0,0)  # neu !
	dreieck()
	right(60)
	runde = runde + 1
\end{minted}
\end{aufgabe}

\begin{aufgabe}[icon=\iconBuch\,\iconComputer\,\iconPartner]
\begin{teilaufgaben}
	\teilaufgabe
	Lest im Buch auf Seite 96 die Box \enquote{Neue Konzepte und Befehle}.

	\teilaufgabe
	Bearbeitet im Buch auf Seite 97 Aufgabe 35.

	\teilaufgabe
	Schreibt ein Programm, dass alle Quadratzahlen ausgibt, die kleiner als ein
	Parameter \code{zahl} sind.

	\teilaufgabe
	Entwickelt einen Befehl, der eine Zufallszahl generiert (\code{randint(min, max)})
	und dann den Benutzer nach einer Zahl fragt. Ist die Zahl größer wird
	\enquote{größer} angezeigt, ist die Zahl kleiner \enquote{kleiner}.
	Dies wird solange wiederholt, bis die Eingabe gleich der Zahl ist. Dann wird
	\enquote{Treffer!} angezeigt.
\end{teilaufgaben}
\end{aufgabe}

\vspace*{-1em}
\subsubsection*{Logische Operatoren}
\begin{tabularx}{\textwidth}{|c|X|} \hline
	\code{and} & Und-Verknüpfung; beide Bedingungen müssen \emph{wahr} sein                     \\ \hline
	\code{or}  & Oder-Verknüpfung; midestens eine Bedingungen muss \emph{wahr} sein             \\ \hline
	\code{not} & Negation; Umkehrung des Wahrheitsgehaltes (aus \emph{wahr} wird \emph{falsch}) \\ \hline
	\code{==}  & prüft zwei Werte oder Variablen auf Gleichheit (auch Texte)                    \\ \hline
	\code{!=}  & prüft zwei Werte oder Variablen auf Ungleichheit (auch Texte)                  \\ \hline
	\code{>}   & prüft ob der erste Werte echt größer ist als der Zweite                        \\ \hline
	\code{>=}  & prüft ob der erste Werte größer oder gleich der Zweite ist                     \\ \hline
	\code{<}   & prüft ob der erste Werte echt kleiner ist als der Zweite                       \\ \hline
	\code{<=}  & prüft ob der erste Werte kleiner oder gleich der Zweite ist                    \\ \hline
\end{tabularx}


\end{document}
