\documentclass[11pt, a4paper, ngerman]{arbeitsblatt}

\ladeModule{theme}

\ladeFach[quelltexte]{informatik}

\aboptionen{
	name		= {J. Neugebauer},
	kuerzel 	= {Ngb},
	titel 		= {Einführung in GameGrid},
	reihe 		= {Programmierung mit Python},
	fach 		= {Informatik},
	kurs 		= {9Diff},
	nummer 		= {V.1},
	lizenz 		= {cc-by-nc-sa-eu-4},
	version 	= {2022-03-21},
}

\begin{document}
\ReiheTitel

Das Python-Modul \code{gamegrid} erlaubt die Programmierung einfacher Spiele mit einer grafischen Oberfläche. Zum Kennenlernen des Moduls wollen wir eine Version des Spiels \enquote{Tic-Tac-Toe} programmieren.

\begin{aufgabe}[icon=\iconPartner\,\iconComputer]
Kopiert die Datei \datei{tictactoe\_vorlage.py} aus dem Tauschordner in euren eigenen Ordner. Startet \programm{TigerJython} und öffnet die Datei. Ihr seht das hier abgebildete Programm:
\begin{minted}{python}
from gamegrid import *

player = 1

def mausklick(e):
    global player
    loc = toLocation(e.getX(), e.getY())

    bg.fillCell(loc, Color.green)

    refresh()
    return True

def pruefeSiegbedingung():
	pass

makeGameGrid(3, 3, 100, Color.gray, False,
    mousePressed = mausklick)
setTitle("TicTacToe")

bg = getBg()
show()
\end{minted}

Analysiert das Programm und markiert neue Befehle und ungewohnte Programmkonstrukte. Notiert Stichpunkte zu ihrer Funktion auf dem AB.
\end{aufgabe}

\begin{aufgabe}[icon=\iconPartner\,\iconComputer]
Programmiere das Spiel Tic-Tac-Toe in der Vorlage nach. Nutze die Tipps. Du kannst dich an dieses Vorgehen halten:
\begin{enumerate}
	\item Färbe das geklickte Feld mit einer anderen Farbe ein, abhängig vom Wert der Variablen \code{player}. Ändere \code{player} nach jedem Mausklick von \code{1} auf \code{2} und zurück.
	\item Prüfe bei einem Mausklick, ob die angeklickte Zelle noch schwarz ist. Wenn ja, dann passiert nichts.
	\item Rufe nach jedem Mausklick die Funktion \code{pruefeSiegbedingung} auf.
	\item Implementiere in \code{pruefeSiegbedingung} eine Logik, die prüft, ob der Spieler mit seinem Zug das Spiel gewonnen hat.
	\item Implementiere einen Abbruch des Spiels, wenn alle Felder eingefärbt wurden.
\end{enumerate}
\end{aufgabe}

\end{document}
