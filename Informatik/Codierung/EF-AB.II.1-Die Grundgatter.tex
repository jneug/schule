\documentclass[10pt, a4paper]{scrartcl}

\usepackage{vorschule}
\usepackage[
    typ=ab,
    fach=Informatik,
    lerngruppe={EF},
    nummer={II.1},
    module={Symbole,Lizenzen},
    seitenzahlen=keine,
    farbig,
    lizenz=cc-by-nc-sa-4,
]{schule}

\usepackage[
	kuerzel=Ngb,
	reihe={Digitale Schaltungen},
	version={2020-09-4},
]{ngbschule}

\author{J. Neugebauer}
\title{Die Grundgatter}
\date{\Heute}

\setzeAufgabentemplate{ngbnormal}

%\usepackage{qrcode}

\newcommand\addvmargin[1]{%
\node[fit=(current bounding box),inner ysep=#1,inner xsep=0]{};}

\ctikzset{tripoles/european not symbol=ieee circle}

\setlength{\tabcolsep}{5mm}
\def\arraystretch{1.25}

\begin{document}

\ReiheTitel
Die Grundelemente einer digitalen Schaltung sind drei einfache
logische Schaltungen - auch \emph{Gatter} genannt. Sie stellen 
die Basiselemente dar, aus denen man alle anderen Schaltungen
konstruieren kann. Diese drei Grundgatter sind \textbf{UND}, 
\textbf{ODER} und \textbf{NICHT}.

\subsection*{Die Grundgatter}

\begin{tabularx}{\textwidth}{XXX}
	\textbf{AND} & \textbf{OR} & \textbf{NOT} \\
	\begin{circuitikz}[baseline=0pt]
	\draw (0,0) node[and port](01){}
		(01.in 1) node[anchor=east]{A}
		(01.in 2) node[anchor=east]{B}
		(01.out) node[anchor=west]{Y};
	\addvmargin{4mm}
	\end{circuitikz} &
	\begin{circuitikz}[baseline=0pt]
	\draw (0,0) node[or port](01){}
		(01.in 1) node[anchor=east]{A}
		(01.in 2) node[anchor=east]{B}
		(01.out) node[anchor=west]{Y};
	\addvmargin{4mm}
	\end{circuitikz}&
	\begin{circuitikz}[baseline=0pt]
	\draw (0,0) node[not port](01){}
		(01.in) node[anchor=east]{A}
		(01.out) node[anchor=west]{Y};
	\addvmargin{4mm}
	\end{circuitikz} \\
	\begin{tabular}{c|c|c}\texttt\small
		A & B & Y \\\hline
		0 & 0 & \\
		0 & 1 & \\
		1 & 0 & \\
		1 & 1 & \\
	\end{tabular}&
	\begin{tabular}{c|c|c}\texttt\small
		A & B & Y \\\hline
		0 & 0 & \\
		0 & 1 & \\
		1 & 0 & \\
		1 & 1 & \\
	\end{tabular}&
	\begin{tabular}{c|c}\texttt\small
		A & Y \\\hline
		0 & \\
		1 & \\
	\end{tabular}\\
\end{tabularx}

\subsection*{Weitere Gatter}

\begin{tabularx}{\textwidth}{XXX}
	\textbf{NAND} & \textbf{NOR} & \textbf{XOR} \\
	\begin{circuitikz}[baseline=0pt]
	\draw (0,0) node[nand port](01){}
		(01.in 1) node[anchor=east]{A}
		(01.in 2) node[anchor=east]{B}
		(01.out) node[anchor=west]{Y};
	\addvmargin{4mm}
	\end{circuitikz} &
	\begin{circuitikz}[baseline=0pt]
	\draw (0,0) node[nor port](01){}
		(01.in 1) node[anchor=east]{A}
		(01.in 2) node[anchor=east]{B}
		(01.out) node[anchor=west]{Y};
	\addvmargin{4mm}
	\end{circuitikz}&
	\begin{circuitikz}[baseline=0pt]
	\draw (0,0) node[xor port](01){}
		(01.in 1) node[anchor=east]{A}
		(01.in 2) node[anchor=east]{B}
		(01.out) node[anchor=west]{Y};
	\addvmargin{4mm}
	\end{circuitikz} \\
	\begin{tabular}{c|c|c}\texttt\small
		A & B & Y \\\hline
		0 & 0 & \\
		0 & 1 & \\
		1 & 0 & \\
		1 & 1 & \\
	\end{tabular}&
	\begin{tabular}{c|c|c}\texttt\small
		A & B & Y \\\hline
		0 & 0 & \\
		0 & 1 & \\
		1 & 0 & \\
		1 & 1 & \\
	\end{tabular}&
	\begin{tabular}{c|c|c}\texttt\small
		A & B & Y \\\hline
		0 & 0 & \\
		0 & 1 & \\
		1 & 0 & \\
		1 & 1 & \\
	\end{tabular}
\end{tabularx}

\vspace{1ex}
\begin{rahmen}\centering
Zeichne hier den Schaltplan des \code{XOR}-Gatters ein.
\vspace{5.5cm}
\end{rahmen}

\end{document}