\documentclass[9pt, a4paper]{arbeitsblatt}

\ladeModule{theme,boxen,tabellen}
\aboptionen{
	name 		= {J. Neugebauer},
	kuerzel 	= {Ngb},
	titel 		= {Die Grundgatter},
	reihe 		= {Digitale Schaltungen},
	fach 		= {Informatik},
	kurs 		= {EF},
	nummer 		= {II.1},
	lizenz 		= {cc-by-nc-sa-4},
	version 	= {2021-05-30},
}

\usetikzlibrary{fit}
\newcommand\addvmargin[1]{%
\node[fit=(current bounding box),inner ysep=#1,inner xsep=0]{};}

\usepackage{circuitikz}
\ctikzset{tripoles/european not symbol=ieee circle}

\setlength{\tabcolsep}{5mm}
\def\arraystretch{1.25}

\begin{document}

\ReiheTitel
Die Grundelemente einer digitalen Schaltung sind drei einfache
logische Schaltungen - auch \emph{Gatter} genannt. Sie stellen
die Basiselemente dar, aus denen man alle anderen Schaltungen
konstruieren kann. Diese drei Grundgatter sind \textbf{UND},
\textbf{ODER} und \textbf{NICHT}.

\subsection*{Die Grundgatter}

\begin{tabularx}{\textwidth}{XXX}
	\textbf{AND} & \textbf{OR} & \textbf{NOT} \\
	\begin{circuitikz}[baseline=0pt]
	\draw (0,0) node[european and port](01){}
		(01.in 1) node[anchor=east]{A}
		(01.in 2) node[anchor=east]{B}
		(01.out) node[anchor=west]{Y};
	\addvmargin{4mm}
	\end{circuitikz} &
	\begin{circuitikz}[baseline=0pt]
	\draw (0,0) node[european or port](01){}
		(01.in 1) node[anchor=east]{A}
		(01.in 2) node[anchor=east]{B}
		(01.out) node[anchor=west]{Y};
	\addvmargin{4mm}
	\end{circuitikz}&
	\begin{circuitikz}[baseline=0pt]
	\draw (0,0) node[european not port](01){}
		(01.in) node[anchor=east]{A}
		(01.out) node[anchor=west]{Y};
	\addvmargin{4mm}
	\end{circuitikz} \\
	\begin{tabular}{c|c|c}\texttt\small
		A & B & Y \\\hline
		0 & 0 & \\
		0 & 1 & \\
		1 & 0 & \\
		1 & 1 & \\
	\end{tabular}&
	\begin{tabular}{c|c|c}\texttt\small
		A & B & Y \\\hline
		0 & 0 & \\
		0 & 1 & \\
		1 & 0 & \\
		1 & 1 & \\
	\end{tabular}&
	\begin{tabular}{c|c}\texttt\small
		A & Y \\\hline
		0 & \\
		1 & \\
	\end{tabular}\\
\end{tabularx}

\subsection*{Weitere Gatter}

\begin{tabularx}{\textwidth}{XXX}
	\textbf{NAND} & \textbf{NOR} & \textbf{XOR} \\
	\begin{circuitikz}[baseline=0pt]
	\draw (0,0) node[european nand port](01){}
		(01.in 1) node[anchor=east]{A}
		(01.in 2) node[anchor=east]{B}
		(01.out) node[anchor=west]{Y};
	\addvmargin{4mm}
	\end{circuitikz} &
	\begin{circuitikz}[baseline=0pt]
	\draw (0,0) node[european nor port](01){}
		(01.in 1) node[anchor=east]{A}
		(01.in 2) node[anchor=east]{B}
		(01.out) node[anchor=west]{Y};
	\addvmargin{4mm}
	\end{circuitikz}&
	\begin{circuitikz}[baseline=0pt]
	\draw (0,0) node[european xor port](01){}
		(01.in 1) node[anchor=east]{A}
		(01.in 2) node[anchor=east]{B}
		(01.out) node[anchor=west]{Y};
	\addvmargin{4mm}
	\end{circuitikz} \\
	\begin{tabular}{c|c|c}\texttt\small
		A & B & Y \\\hline
		0 & 0 & \\
		0 & 1 & \\
		1 & 0 & \\
		1 & 1 & \\
	\end{tabular}&
	\begin{tabular}{c|c|c}\texttt\small
		A & B & Y \\\hline
		0 & 0 & \\
		0 & 1 & \\
		1 & 0 & \\
		1 & 1 & \\
	\end{tabular}&
	\begin{tabular}{c|c|c}\texttt\small
		A & B & Y \\\hline
		0 & 0 & \\
		0 & 1 & \\
		1 & 0 & \\
		1 & 1 & \\
	\end{tabular}
\end{tabularx}

\vspace{1ex}
\begin{rahmen}\centering
Zeichne hier den Schaltplan des \code{XOR}-Gatters ein.
\vspace{5.5cm}
\end{rahmen}

\end{document}
