\documentclass[9pt, a5paper, landscape, ngerman]{arbeitsblatt}

\ladeModule{theme,tabellen,boxen}

\ladeFach[]{informatik}

\aboptionen{
	name		= {J. Neugebauer},
	kuerzel 	= {Ngb},
	titel 		= {Darstellung von Zahlen},
	reihe 		= {Informationen und ihre Darstellung},
	fach 		= {Informatik},
	kurs 		= {9Diff},
	nummer 		= {I.2},
	lizenz 		= {cc-by-nc-sa-eu-4},
	version 	= {2022-08-24},
}

\begin{document}
\ReiheTitel

Ein \emph{Zahlensystem} ist ein System zur Darstellung von Zahlen. Wir
benutzen im Alltag das \emph{Dezimalsystem}, ein sogenanntes
\emph{Stellenwertsystem} zur Basis 10. Das bedeutet, es werden 10 verschiedene
Ziffern zur Darstellung benutzt. Das \emph{Binärsystem} benutzt dagegen nur zwei. Die \emph{Darstellung} von Zahlen in den Stellenwertsystemen erfolgt immer gleich. Der einzige Unterschied ist die Basis - also die Anzahl an Ziffern.

\begin{tabularx}{\textwidth}{c|c|c|X}
	        & Zahlwert  & Darstellung   & Berechnung                                                                     \\ \hline
	Binär   & \num{109} & $(1101110)_2$ & $1\cdot 2^6 + 1\cdot 2^5 + 0\cdot 2^4 + 1\cdot 2^3 + 1\cdot 2^2 + 1\cdot 2^1 +
	0\cdot 2^0$                                                                                                          \\ \hline
	Dezimal & \num{109} & $(109)_{10}$  & $1\cdot 10^2 + 0\cdot 10^1 + 9\cdot 10^0$
\end{tabularx}

\begin{aufgabe}
	Wandele die Zahlen ins Dezimalsystem um:
	\begin{tasks}(3)
		\task $(1001 0101)_2$ \task $(0010 0110)_2$ \task $(0111 0010)_2$
		\task $(E2)_{16}$ \task $(AF)_{16}$ \task $(14)_{16}$
	\end{tasks}

	\tipp{Berechne und notiere dir zunächst die Potenzen $2^0$ bis $2^7$ und $16^0$ bis $16^3$.}
\end{aufgabe}

\end{document}
