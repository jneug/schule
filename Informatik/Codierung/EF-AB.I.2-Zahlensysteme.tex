\documentclass[9pt, a4paper]{arbeitsblatt}

\ladeModule{theme,typo,icons,tabellen,boxen,aufgaben}
\aboptionen{
	name 		= {J. Neugebauer},
	kuerzel 	= {Ngb},
	titel 		= {Zahlensysteme},
	reihe 		= {Informationen, Daten und Codierung},
	fach 		= {Informatik},
	kurs 		= {EF},
	nummer 		= {I.2},
	lizenz 		= {cc-by-nc-sa-4},
	version 	= {2021-05-28},
}

\begin{document}
\ReiheTitel

Ein \emph{Zahlensystem} ist ein System zur Darstellung von Zahlen. Wir
benutzen im Alltag das \emph{Dezimalsystem}, ein sogenanntes
\emph{Stellenwertsystem} zur Basis 10. Das bedeutet, es werden 10 verschiedene
Ziffern zur Darstellung benutzt. Das \emph{Binärsystem} benutzt nur zwei,
und das \emph{Hexadezimalsystem} verwendet 16.

\subsection*{Ziffern der Zahlenwertsysteme}
\begin{tabularx}{\textwidth}{c|*{16}{|>{\centering\large\arraybackslash}X}}
	Binär   & 0 & 1 &   &   &   &   &   &   &   &   &   &   &   &   &   &   \\ \hline
	Dezimal & 0 & 1 & 2 & 3 & 4 & 5 & 6 & 7 & 8 & 9 &   &   &   &   &   &   \\\hline
	Hex     & 0 & 1 & 2 & 3 & 4 & 5 & 6 & 7 & 8 & 9 & A & B & C & D & E & F
\end{tabularx}


\subsection*{Darstellung von Zahlen}
Die Darstellung von Zahlen in den Stellenwertsystemen erfolgt immer gleich. Der
einzige Unterschied ist die Basis - also die Anzahl an Ziffern.

\begin{tabularx}{\textwidth}{c|c|c|X}
	        & Zahlwert  & Darstellung   & Berechnung                                                                     \\ \hline
	Binär   & \num{109} & $(1101110)_2$ & $1\cdot 2^6 + 1\cdot 2^5 + 0\cdot 2^4 + 1\cdot 2^3 + 1\cdot 2^2 + 1\cdot 2^1 +
	0\cdot 2^0$                                                                                                          \\ \hline
	Dezimal & \num{109} & $(109)_{10}$  & $1\cdot 10^2 + 0\cdot 10^1 + 9\cdot 10^0$                                      \\ \hline
	Hex     & \num{109} & $(6D)_{16}$   & $6\cdot 16^1 + 13\cdot 16^0$
\end{tabularx}
\bigskip

\begin{aufgabe}
	Berechne die Potenzen und trage die Ergebnisse in die Tabelle ein.
	\begin{tabularx}{\textwidth}{c|*{9}{|>{\centering\large\arraybackslash}X}}
	Binär   & $2^0 = 1$ & $2^1 = \quad$ & $2^2 = \quad$ & $2^3 = \quad$ & $2^4 = \quad$ & $2^5 = \quad$ & $2^6 = \quad$ & $2^7 = \quad$ \\ \hline
	Hex   & $16^0 = 1$ & $16^1 = \quad$ & $16^2 = \quad$ & $16^3 = \quad$ & & & & \\ \hline
\end{tabularx}
\end{aufgabe}

\begin{aufgabe}
	Wandele die Zahlen ins Dezimalsystem um:
	\begin{tasks}(3)
		\task $(1001 0101)_2$ \task $(0010 0110)_2$ \task $(0111 0010)_2$
		\task $(E2)_{16}$ \task $(AF)_{16}$ \task $(14)_{16}$
	\end{tasks}
\end{aufgabe}

\begin{aufgabe}
	Wandele die Zahlen ins Binärsystem um:
	\begin{tasks}(3)
		\task $(3E)_{16}$ \task $(91)_{16}$ \task $(7A)_{16}$
	\end{tasks}
\end{aufgabe}

\subsection*{Rechnen mit Hexadezimalzahlen}
Das Rechnen in Stellenwertsystemen unterscheidet sich auch nur durch die Basis.
Das Verfahren ist gleich: Für jede Stelle werden die Ziffern zusammengezählt.
Wird für diese Stelle die letzte Ziffer überschritten (z.B.
$F$ im Hexadezimalsystem), wird die nächste Stelle um eins
erhöht. (Vergleiche Addition im Dezimalsystem.)

\begin{aufgabe}
	Berechne im jeweiligen Zahlensystem:
	\begin{tasks}(3)
		\task $(45)_{10} +
				(13)_{10}$ \task $(23)_{10} +
				(8)_{10}$ \task $(105)_{10} +
				(77)_{10}$
		\task $(0101)_{2} +
				(1100)_{2}$ \task $(1011)_{2} +
				(1000)_{2}$ \task $(1001)_{2} +
				(1110)_{2} +
				(0100)_{2}$
		\task $(1E)_{16} +
				(A2)_{16}$ \task $(FF)_{16} +
				(06)_{16}$ \task $(FE)_{16} +
				(A4)_{16}$
	\end{tasks}
\end{aufgabe}

\end{document}
