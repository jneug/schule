\documentclass[10pt, a4paper]{scrartcl}

\usepackage{vorschule}
\usepackage[
    typ=ab,
    fach=Informatik,
    lerngruppe={EF},
    nummer={II.2},
    module={Symbole,Lizenzen},
    seitenzahlen=keine,
    farbig,
    lizenz=cc-by-nc-sa-4,
]{schule}

\usepackage[
	kuerzel=Ngb,
	reihe={Digitale Schaltungen},
	version={2020-09-4},
]{ngbschule}

\author{J. Neugebauer}
\title{Halb- und Volladdierer}
\date{\Heute}

\setzeAufgabentemplate{ngbnormal}

%\usepackage{qrcode}

\newcommand\addvmargin[1]{%
\node[fit=(current bounding box),inner ysep=#1,inner xsep=0]{};}

\ctikzset{tripoles/european not symbol=ieee circle}

\setlength{\tabcolsep}{5mm}
\def\arraystretch{1.25}

\begin{document}

\ReiheTitel

\begin{multicols}{2}
\subsection*{Der Halbaddierer}

Der \textbf{Halbaddierer} bestimmt die Summe ($s$) und den Übertrag ($c_o$) einer 1-Bit Addition zweier Binärzahlen (A und B).

\begin{center}
\begin{tabular}{c|c|c|c}\texttt\small
	A & B & $s$ & $c_o$\\\hline
	0 & 0 & & \\
	0 & 1 & & \\
	1 & 0 & & \\
	1 & 1 & & \\
\end{tabular}
\end{center}

\vspace*{1cm}

\columnbreak

\subsection*{Der Volladdierer}

Der \textbf{Volladdierer} bestimmt die Summe ($s$) und den Übertrag ($c_o$) einer 1-Bit Addition zweier Binärzahlen (A und B) und eines zusätzlichen Übertrages ($c_i$).

\begin{center}
\begin{tabular}{c|c|c|c|c}\texttt\small
	A & B & $c_o$ & $s$ & $c_o$\\\hline
	0 & 0 & 0 & & \\
	0 & 1 & 0 & & \\
	1 & 0 & 0 & & \\
	1 & 1 & 0 & & \\
	0 & 0 & 1 & & \\
	0 & 1 & 1 & & \\
	1 & 0 & 1 & & \\
	1 & 1 & 1 & & \\
\end{tabular}
\end{center}
\end{multicols}

\begin{rahmen}\centering
Schaltnetz des Halbaddierers.
\vspace{5cm}
\end{rahmen}
\begin{rahmen}\centering
Schaltnetz des Volladdierers.
\vspace{5cm}
\end{rahmen}

\end{document}