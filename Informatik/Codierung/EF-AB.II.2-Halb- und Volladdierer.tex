\documentclass[10pt, a4paper]{arbeitsblatt}

\ladeModule{theme,boxen,tabellen}
\aboptionen{
	name 		= {J. Neugebauer},
	kuerzel 	= {Ngb},
	titel 		= {Halb- und Volladdierer},
	reihe 		= {Digitale Schaltungen},
	fach 		= {Informatik},
	kurs 		= {EF},
	nummer 		= {II.2},
	lizenz 		= {cc-by-nc-sa-4},
	version 	= {2021-05-30},
}

\usetikzlibrary{fit}
\newcommand\addvmargin[1]{%
\node[fit=(current bounding box),inner ysep=#1,inner xsep=0]{};}

\usepackage{circuitikz}
\ctikzset{tripoles/european not symbol=ieee circle}

\setlength{\tabcolsep}{5mm}
\def\arraystretch{1.25}

\begin{document}

\ReiheTitel

\begin{multicols}{2}
\subsection*{Der Halbaddierer}

Der \textbf{Halbaddierer} bestimmt die Summe ($s$) und den Übertrag ($c_o$) einer 1-Bit Addition zweier Binärzahlen (A und B).

\begin{center}
\begin{tabular}{c|c|c|c}\texttt\small
	A & B & $s$ & $c_o$\\\hline
	0 & 0 & & \\
	0 & 1 & & \\
	1 & 0 & & \\
	1 & 1 & & \\
\end{tabular}
\end{center}

\vspace*{2cm}

\columnbreak

\subsection*{Der Volladdierer}

Der \textbf{Volladdierer} bestimmt die Summe ($s$) und den Übertrag ($c_o$) einer 1-Bit Addition zweier Binärzahlen (A und B) und eines zusätzlichen Übertrages ($c_i$).

\begin{center}
\begin{tabular}{c|c|c|c|c}\texttt\small
	A & B & $c_o$ & $s$ & $c_i$\\\hline
	0 & 0 & 0 & & \\
	0 & 1 & 0 & & \\
	1 & 0 & 0 & & \\
	1 & 1 & 0 & & \\
	0 & 0 & 1 & & \\
	0 & 1 & 1 & & \\
	1 & 0 & 1 & & \\
	1 & 1 & 1 & & \\
\end{tabular}
\end{center}
\end{multicols}

\begin{rahmen}\centering
Schaltnetz des Halbaddierers.
\vspace{4.5cm}
\end{rahmen}
\begin{rahmen}\centering
Schaltnetz des Volladdierers.
\vspace{4.5cm}
\end{rahmen}

\end{document}
