\documentclass[10pt, a5paper, landscape, ngerman]{arbeitsblatt}


% \usepackage{pgfpages}
% \pgfpagesuselayout{2 on 1}[a4paper]

\ladeModule{theme,qrcodes}

\ladeFach[]{informatik}

\aboptionen{
	name		= {J. Neugebauer},
	kuerzel 	= {Ngb},
	titel 		= {Arten von Codes},
	reihe 		= {Informationen und ihre Darstellung},
	fach 		= {Informatik},
	kurs 		= {9Diff},
	nummer 		= {I.1},
	lizenz 		= {cc-by-nc-sa-eu-4},
	version 	= {2022-08-16},
}

\begin{document}
\ReiheTitel[Der Morsecode]

Der Morsecode wurde nach der Erfindung des Telegrafen zur Datenübermittlung mit nur zwei Signalen entwickelt. Doch wie funktionieren er und was hat er mit Informatik zutun?

Ein Code ordnet Wörtern, Satzzeichen oder andere Informationsblöcke anderen Wörtern, Satzzeichen und Informationsblöcken zu.
\begin{enumn}
	\item Informiert euch über den Morsecode und wie er aufgebaut ist.
	\item Entschlüsselt eure Nachricht.
	\item Beschreibt, wie die Tabellen zu Codierung und Decodierung unter \url{https://link.ngb.schule/morsecode} aufgebaut sind. Warum ist die Tabelle zur Codierung für die Decodierung wenig hilfreich?
	\item Verfasst eine kurze Nachricht mit Morsezeichen, die ihr später den anderen zum Decodieren geben könnt.
	\item Bereitet eine Kurzvorstellung des Morsecodes für den Kurs vor.
\end{enumn}

% \begin{tabular}{|l|l|}\hline
% 	\textbf{Buchstabe} & \textbf{Code} \\\hline
% 	A & \textbullet\,\textemdash \\\hline
% 	B
% 	C
% 	D
% 	E
% 	F
% 	G
% 	H
% 	I
% 	J
% 	K
% 	L
% 	M
% 	N
% 	O
% 	P
% 	Q
% 	R
% 	S
% 	T
% 	U
% 	V
% 	X
% 	Y
% 	Z

% \end{tabular}

\newpage
\ReiheTitel[Die Brailleschrift]

Die Braillesschrift ist eine Blindenschrift und wird international von Blinden und stark Sehbehinderten benutz. Doch wie funktionieren sie und was hat sie mit Informatik zutun?

Ein Code ordnet Wörtern, Satzzeichen oder andere Informationsblöcke anderen Wörtern, Satzzeichen und Informationsblöcken zu.
\begin{enumn}
	\item Informiert euch über die Brailleschrift und wie sie aufgebaut ist.
	\item Entschlüsselt eure Nachricht mit Hilfe der Taelle unter \url{https://link.ngb.schule/braille}.
	\item Warum reichen Vierpunktfelder nicht aus? Hätten Fünfpunktfelder ausgereicht? Begründet.
	\item Verfasst eine kurze Nachricht mit Brailleschrift, die ihr später den anderen zum Decodieren geben könnt.
	\item Bereitet eine Kurzvorstellung der Brailleschrift für den Kurs vor.
\end{enumn}

\newpage
\ReiheTitel[Der EAN13 Barcode]

Barcodes (auch Strichcodes genannt) findest du in deinem Alltag ganz viele. Doch wie funktionieren sie und was hat sie mit Informatik zutun?

Ein Code ordnet Wörtern, Satzzeichen oder andere Informationsblöcke anderen Wörtern, Satzzeichen und Informationsblöcken zu.
\begin{enumn}
	\item Informiert euch unter \url{https://link.ngb.schule/ean13} über den EAN13 Code.
	\item Erklärt das Verfahren zur Bestimmung der Prüfziffer.
	\item Bereitet eine Kurzvorstellung des EAN13 Barcodes für den Kurs vor.
\end{enumn}

\newpage
\ReiheTitel[Der ISBN Code]

Drehst du ein Buch um, dann findest du ziemlich sicher auf der Rückseite die ISBN-Nummer. Doch wie funktionieren sie und was hat sie mit Informatik zutun?

Ein Code ordnet Wörtern, Satzzeichen oder andere Informationsblöcke anderen Wörtern, Satzzeichen und Informationsblöcken zu.
\begin{enumn}
	\item Informier dich über den ISBN Code.
	\item Arbeite die ersten drei Seiten unter \url{https://link.ngb.schule/isbn} durch.
	\item Erklärt, wie der GTIN Code und die ISBN funktionieren.
	\item Bereitet eine Kurzvorstellung des ISBN Codes und der GTIN Barcodes für den Kurs vor.
\end{enumn}

\newpage
\ReiheTitel[Der ASCII Code]

Wie werden eigentlich Buchstaben am Computer gespeichert, wenn der doch nur Nullen und Einsen verarbeiten kann?

Ein Code ordnet Wörtern, Satzzeichen oder andere Informationsblöcke anderen Wörtern, Satzzeichen und Informationsblöcken zu.
\begin{enumn}
	\item Informier dich über den ASCII Code.
	\item Arbeite die Schritte 2 und 3 unter \url{https://link.ngb.schule/ascii} durch.
	\item Verfasst eine kurze Nachricht mit ASCII Codes, die ihr später den anderen zum Decodieren geben könnt.
	\item Bereitet eine Kurzvorstellung des ASCII Codes für den Kurs vor.
\end{enumn}

\end{document}
