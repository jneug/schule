\documentclass[10pt, a5paper]{scrartcl}

\usepackage{pgfpages}
\pgfpagesuselayout{2 on 1}[a4paper,landscape]

\usepackage{vorschule}
\usepackage[
	typ=ab,
	fach=Informatik,
	lerngruppe={9Diff},
	nummer={II.2},
	module={Symbole,Lizenzen},
	seitenzahlen=keine,
	farbig,
	lizenz=cc-by-nc-sa-4,
]{schule}

\usepackage[
	kuerzel=Ngb,
	reihe={Informationen, Daten und Codierung},
	version={2020-09-11},
]{ngbschule}

\author{J. Neugebauer}
\title{Subtraktions- und Moduloverfahren}
\date{\Heute}

\setzeAufgabentemplate{schule-keinenummer}

\usepackage{scratch}

\begin{document}
\ReiheTitel

\begin{aufgabe}
	Wandle die Zahl $(42)_{10}$ in ihre binäre Darstellung um. Befolge dazu die folgenden Schritte genau:
	\begin{enumerate}
		\item Teile die Zahl \emph{mit Rest} durch 2 (die Modulo-Operation).
			\begin{align*}
				\text{z.B.}\quad7 : 2 &= 3 \quad\text{Rest}\ 1  \\
				12 : 2 &= 6 \quad\text{Rest}\ 0
			\end{align*}
		\item Wenn das Ergebnis gleich Null ist, nimm das Ergebnis als neue Zahl und gehe wieder zu Schritt 1.
			\begin{align*}
				\text{z.B.}\quad7 : 2 &= 3 \quad\text{Rest}\ 1 \\
				\text{Neue Zahl:} &\quad 3
			\end{align*}
		
			Wenn das Ergebnis gleich Null ist, dann weiter bei Schritt 3.
		\item Notiere die Reste der Rechnung von \enquote{unten nach oben}, also vom letzten berechneten Rest bis zum Rest der ersten Division.
	\end{enumerate}
\end{aufgabe}

\newpage

\ReiheTitel

\end{document}
