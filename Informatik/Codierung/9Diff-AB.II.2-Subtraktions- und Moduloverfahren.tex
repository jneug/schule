\documentclass[10pt, a5paper]{arbeitsblatt}

\ladeModule{theme,typo,icons,tabellen,aufgaben}
\aboptionen{
	name 		= {J. Neugebauer},
	kuerzel 	= {Ngb},
	titel 		= {Subtraktions- und Moduloverfahren},
	reihe 		= {Grundlagen der Informatik},
	fach 		= {Informatik},
	kurs 		= {9Diff},
	nummer 		= {II.1},
	lizenz 		= {cc-by-nc-sa-4},
	version	= {2021-09-06},
	aufgaben/format = abohne
}

\usepackage{pgfpages}
\pgfpagesuselayout{2 on 1}[a4paper,landscape]

\begin{document}
\ReiheTitel

\begin{aufgabe}
	Wandle die Zahl $(42)_{10}$ in ihre binäre Darstellung um. Befolge
	dazu die folgenden Schritte genau:
	\begin{enumerate}
		\item \label{goto-0} Teile die Zahl \emph{mit Rest} durch
		      $2$ (die Modulo-Operation).
		      \begin{align*}
			      \text{z.B.}\quad7 : 2 & = 3 \quad\text{Rest}\ 1 \\
			      12 : 2                & = 6 \quad\text{Rest}\ 0
		      \end{align*}
		\item Wenn das Ergebnis gleich Null ist, nimm das Ergebnis als neue Zahl und gehe
		      wieder zu Schritt \ref{goto-0}.
		      \begin{align*}
			      \text{z.B.}\quad7 : 2 & = 3 \quad\text{Rest}\ 1 \\
			      \text{Neue Zahl:}     & \quad 3
		      \end{align*}

		      Wenn das Ergebnis gleich Null ist, dann weiter bei Schritt
		      \ref{goto-1}.
		\item \label{goto-1} Notiere die Reste der
		      Rechnung von \enquote{unten nach oben}, also vom letzten berechneten Rest bis zum
		      Rest der ersten Division.
	\end{enumerate}
\end{aufgabe}

\newpage
\ReiheTitel

\begin{aufgabe}
	Wandle die Zahl $(42)_{10}$ in ihre binäre Darstellung um. Befolge
	dazu die folgenden Schritte genau:
	\begin{enumerate}
		\item Suche die größte Potenz von $2$, die kleiner oder gleich der
		      Zahl ist.
		      \begin{align*}
			      \text{z.B.}\quad 5 \geq 2^2 & = 4
		      \end{align*}
		\item \label{goto-2} Prüfe, ob die Zweierpotenz kleiner oder gleich der
		      Zahl ist.

		      Ist dies der Fall, notiere eine $1$.
		      \begin{align*}
			      \text{z.B.}\quad 5 \geq 2^2 & = 4 \quad\longrightarrow 1
		      \end{align*}

		      Ansonsten notiere eine $0$.
		      \begin{align*}
			      \text{z.B.}\quad 1 < 2^1 & = 2 \quad\longrightarrow 0
		      \end{align*}
		\item Subtrahiere die Zweierpotenz von der Zahl.
		      \begin{align*}
			      \text{z.B.}\quad 5-2^2 & = 1
		      \end{align*}
		\item Wenn die Differenz (im Beispiel $1$) ungleich Null ist, dann
		      wiederhole ab Schritt \ref{goto-2} und benutze die Differenz als neue
		      Zahl und der nächst kleinere Zweierpotenz (im Beispiel also
		      $1$ und $2^1$).
	\end{enumerate}
\end{aufgabe}

\end{document}
