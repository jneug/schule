\documentclass[10pt, a4paper]{scrartcl}

\usepackage{vorschule}
\usepackage[
    typ=ab,
    fach=Informatik,
    lerngruppe={EF},
    nummer={II.3},
    module={Symbole,Lizenzen},
    seitenzahlen=keine,
    farbig,
    lizenz=cc-by-nc-sa-4,
]{schule}

\usepackage[
	kuerzel=Ngb,
	reihe={Informationen, Daten und Codierung},
	version={2019-09-4},
]{ngbschule}

\author{J. Neugebauer}
\title{Zahlensysteme}
\date{\Heute}

\setzeAufgabentemplate{ngbnormal}

%\usepackage{qrcode}

\begin{document}

\ReiheTitel

Ein \emph{Zahlensystem} ist ein System zur Darstellung von Zahlen. Wir benutzen im Alltag das \emph{Dezimalsystem}, ein sogenanntes \emph{Stellenwertsystem} zur Basis 10. Das bedeutet, es werden 10 verschiedene Ziffern zur Darstellung benutzt. Das \emph{Binärsystem} benutzt nur zwei, und das \emph{Hexadezimalsystem} verwendet 16.

\subsection*{Ziffern der Zahlenwertsysteme}
\begin{tabularx}{\textwidth}{c|*{16}{|X}}
	Binär& 0& 1 &&&&&&&&&&&&&& \\ \hline
	Dezimal& 0& 1& 2& 3& 4& 5& 6& 7& 8& 9&&&&&& \\\hline
	Hex& 0& 1& 2& 3& 4& 5& 6& 7& 8& 9& A& B& C& D& E& F
\end{tabularx}

\subsection*{Darstellung von Zahlen}
Die Darstellung von Zahlen in den Stellenwertsystemen erfolgt immer gleich. Der einzige Unterschied ist die Basis - also die Anzahl an Ziffern.

\begin{tabularx}{\textwidth}{c|c|c|X}
	& Zahlwert & Darstellung & Berechnung \\ \hline
	Binär & \num{103} & $(1100111)_2$ & $1\cdot 2^6 + 1\cdot 2^5 + 0\cdot 2^4 + 0\cdot 2^3 + 1\cdot 2^2 + 1\cdot 2^1 + 1\cdot 2^0$ \\ \hline
	Dezimal & \num{103} & $(103)_{10}$ & $1\cdot 10^2 + 0\cdot 10^1 + 3\cdot 10^0$ \\ \hline
	Hex & \num{103} & $(67)_{16}$ & $6\cdot 16^1 + 7\cdot 16^0$
\end{tabularx}
\bigskip

\begin{aufgabe}
	Wandele die Zahlen ins Dezimalsystem um:
	\begin{tasks}(3)
		\task $(1001 0101)_2$ \task $(0010 0110)_2$ \task $(0111 0010)_2$
		\task $(E2)_{16}$ \task $(AF)_{16}$ \task $(14)_{16}$
	\end{tasks}
\end{aufgabe}

\begin{aufgabe}
	Wandele die Zahlen ins Binärsystem um:
	\begin{tasks}(3)
		\task $(3E)_{16}$ \task $(91)_{16}$ \task $(7A)_{16}$
	\end{tasks}
\end{aufgabe}

\subsection*{Rechnen mit Hexadezimalzahlen}
Das Rechnen in Stellenwertsystemen unterscheidet sich auch nur durch die Basis. Das Verfahren ist gleich: Für jede Stelle werden die Ziffern zusammengezählt. Wird für diese Stelle die letzte Ziffer überschritten (z.B. $F$ im Hexadezimalsystem), wird die nächste Stelle um eins erhöht. (Vergleiche Addition im Dezimalsystem.)

\begin{aufgabe}
	Berechne im jeweiligen Zahlensystem:
	\begin{tasks}(3)
		\task $(45)_{10} + (13)_{10}$ \task $(23)_{10} + (8)_{10}$ \task $(105)_{10} + (77)_{10}$
		\task $(0101)_{2} + (1100)_{2}$ \task $(1011)_{2} + (1000)_{2}$ \task $(1001)_{2} + (1110)_{2} + (0100)_{2}$
		\task $(1E)_{16} + (A2)_{16}$ \task $(FF)_{16} + (06)_{16}$ \task $(FE)_{16} + (A4)_{16}$
	\end{tasks}
\end{aufgabe}

\end{document}
