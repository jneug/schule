\documentclass[10pt, a5paper]{scrartcl}

\usepackage{pgfmorepages}
\pgfmorepagesloadextralayouts
\pgfpagesuselayout{repeated 2-up}[a4paper,landscape]

\usepackage{vorschule}
\usepackage[
	typ=ab,
	fach=Informatik,
	lerngruppe={8Diff},
	nummer={II.1},
	module={Symbole,Lizenzen},
	seitenzahlen=keine,
	farbig,
	lizenz=cc-by-nc-sa-4,
]{schule}

\usepackage[
	kuerzel=Ngb,
	reihe={Grundlagen der Informatik},
	version={2020-09-11},
]{ngbschule}

\author{J. Neugebauer}
\title{Binärzahlen I}
\date{\Heute}

\setzeAufgabentemplate{ngbnormal}

\begin{document}
\ReiheTitel

\emph{Binärzahlen} sind die Grundlage moderner Computersysteme. Im \emph{Binärsystem} gibt es nur zwei Symbole (\code{1} und \code{0}) mit denen alle Zahlen dargestellt werden können. Im Gegensatz dazu benutzen wir normalerweise zehn Symbole (\code{0} bis \code{9}) im \emph{Dezimalsystem}.

Um zu zeigen, dass eine Zahl im \emph{Binärsystem} steht, schreiben wir zum Beispiel $(1011)_2$, für das Dezimalsystem zum Beispiel $(1011)_{10}$.

\begin{aufgabe}[subtitle=Binärzahlen umrechnen]
	Wandele vom \emph{Binärsystem} ins \emph{Dezimalsystem} um.
	
	\begin{teilaufgaben}
		\teilaufgabe $(1110)_2$ =
		\teilaufgabe $(10001)_2$ = 
		\teilaufgabe $(10111)_2$ = 
	\end{teilaufgaben}
\end{aufgabe}

\begin{aufgabe}[subtitle=Vorgänger und Nachfolger]
	Notiere Vorgänger und Nachfolger im \emph{Binärsystem}.
	
	\begin{tabularx}{\textwidth}{|X|X|X|}\hline
		\rowcolor{ngb.tabelle.kopf.hg} Vorgänger & Zahl & Nachfolger \\\hline
		\Zeilenabstand & $(10001)_2$ & \\\hline
		\Zeilenabstand & & $(1111)_2$ \\\hline
		$(1000)_2$ &\Zeilenabstand & \\\hline
		\Zeilenabstand & $(101)_2$ & \\\hline
	\end{tabularx}
\end{aufgabe}

\end{document}
