\documentclass[9pt, a4paper]{arbeitsblatt}

\ladeModule{theme,typo,icons,tabellen,boxen,aufgaben}
\aboptionen{
	name 		= {J. Neugebauer},
	kuerzel 	= {Ngb},
	titel 		= {Negative Binärzahlen},
	reihe 		= {Informationen, Daten und Codierung},
	fach 		= {Informatik},
	kurs 		= {9Diff},
	nummer 		= {II.5},
	lizenz 		= {cc-by-nc-sa-4},
	version 	= {2021-05-28},
}

\newcommand{\luecke}[1]{\underline{\hspace{#1}}}

\begin{document}
\ReiheTitel

\section*{Vorzeichenbit-Darstellung}
\begin{infobox}
Bei n-Bit Zahlen in \emph{Vorzeichenbit-Darstellung} (VzD) wird das höchstwertige Bit (das Bit ganz links für die Potenz $2^{n-1}$) für das Vorzeichen reserviert ($0$=positiv, $1$=negativ). Es verbleiben also $n-1$ Bit für die Darstellung des Zahlwertes.

\begin{tabularx}{\textwidth}{XXX}
	\textbf{4-Bit VzD}: & $(0011)_2 = (3)_{10}$ & $(1011)_2 = (-3)_{10}$ \\
	\textbf{7-Bit VzD}: & $(1001000)_2 = (-8)_{10}$ & $(0001000)_1 = (8)_{10}$
\end{tabularx}
\end{infobox}

\begin{aufgabe}
	Forme die \textbf{6-Bit VzD Zahlen} ins Dezimalsystem um:

	\begin{tasks}(2)
		\task $(010101)_2 = (\luecke{2cm})_{10}$
		\task $(101010)_2 = (\luecke{2cm})_{10}$
		\task $(000000)_2 = (\luecke{2cm})_{10}$
		\task $(100000)_2 = (\luecke{2cm})_{10}$
	\end{tasks}
\end{aufgabe}

\begin{aufgabe}
	In der \textbf{4-Bit VzD} können \luecke{2cm} verschiedene Binärzahlen mit \luecke{2cm} verschiedenen Werten dargestellt werden. Die kleinste mögliche Zahl (in Dezimaldarstellung) ist \luecke{2cm}, die Größte ist \luecke{2cm}.
\end{aufgabe}

\begin{aufgabe}
	Berechne die Summe der \textbf{4-Bit VzD Zahlen} (denke an den Überlauf) und kontrolliere die Rechnung im Dezimalsystem.

	\begin{tasks}(2)
		\task $(0001)_2 + (1110)_2$
		\task $(1001)_2 + (0111)_2$
		\task $(1011)_2 + (1000)_2$
		\task $(1011)_2 + (0000)_2$
	\end{tasks}
\end{aufgabe}

\section*{Zweierkomplement-Darstellung}
\begin{infobox}
Das \emph{Einerkomplement} einer Binärzahl wird gebildet, indem die Zahl \emph{invertiert} wird (aus 0 wird 1 und aus 1 wird 0). Für das \emph{Zweierkomplement} wird zum Einerkomplement noch 1 addiert.

\begin{tabularx}{\textwidth}{XXXX}
	 & Binärzahl & Einerkomplement & Zweierkomplement \\
	\textbf{4-Bit ZkD}: & $(0011)_2$ & $(1100)_2$ & $(1101)_2$
\end{tabularx}

Bei Binärzahlen in \emph{Zweierkomplement-Darstellung} (ZkD) sind negative Zahlen das Zweierkomplement ihrer positiven Gegenzahl. Das bedeutet, dass das höchstwertige Bit auch hier anzeigt, ob die Zahl positiv (\code{0}) oder negativ (\code{1}) ist.
\end{infobox}

\begin{aufgabe}
	Bilde das Zweierkomplement der \textbf{4-Bit Zahlen} (denke an den Überlauf). Notiere darunter auch jeweils die Dezimalzahldarstellung der Zahlen.

	\begin{tasks}(3)
		\task $(0001)_2 \longrightarrow (\luecke{2cm})_2$
		\task $(1010)_2 \longrightarrow (\luecke{2cm})_2$
		\task $(0000)_2 \longrightarrow (\luecke{2cm})_2$
	\end{tasks}
\end{aufgabe}

\begin{aufgabe}
	Berechne die Summe der \textbf{4-Bit ZkD Zahlen} und kontrolliere die Rechnung im Dezimalsystem.

	\begin{tasks}(2)
		\task $(0001)_2 + (1110)_2$
		\task $(1001)_2 + (0111)_2$
		\task $(1011)_2 + (1000)_2$
		\task $(1011)_2 + (0000)_2$
	\end{tasks}
\end{aufgabe}

\begin{aufgabe}
	Vergleiche die beiden Darstellungen für negative Binärzahlen miteinander. Wo liegen Vorteile, was sind Probleme der Darstellungen?
\end{aufgabe}

\end{document}
