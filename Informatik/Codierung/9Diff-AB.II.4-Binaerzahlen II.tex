\documentclass[10pt, a4paper]{scrartcl}

\usepackage{vorschule}
\usepackage[
	typ=ab,
	fach=Informatik,
	lerngruppe={9Diff},
	nummer={II.4},
	module={Symbole,Lizenzen,Papiertypen},
	seitenzahlen=keine,
	farbig,
	lizenz=cc-by-nc-sa-4,
]{schule}

\usepackage[
	kuerzel=Ngb,
	reihe={Informationen, Daten und Codierung},
	version={2020-09-06},
]{ngbschule}

\author{J. Neugebauer}
\title{Binärzahlen II}
\date{\Heute}

\setzeAufgabentemplate{ngbnormal}

\begin{document}
\ReiheTitel

\begin{aufgabe}[symbol=\Large\symHeft]
	\begin{enumerate}
		\item[\symEinzel] Überleg dir zunächst \emph{alleine}, wie du zwei
		Binärzahlen miteinander addieren kannst. Rechne dabei die Zahlen noch
		nicht ins Dezimalsystem um, sondern versuche in Verfahren mit
		Binärzahlen zu finden.
		
		\tipp{Auch ohne Dezimalzahlen kannst du dich daran orientieren, wie
		man Dazimalzahlen addiert.}
		
		\begin{tasks}(2)
			\task \code{0101 + 1100 =}
			\task \code{1010 + 0011 =}
		\end{tasks}
		\item[\symPartner] Suche dir einen Partner/eine Partnerin und
		vergleicht eure Ideen für ein Rechenverfahren miteinander. Habt ihr
		dieselben Ergebnisse? 
		
		Rechnet weitere Beispiele und prüft die Ergebnisse, indem ihr die 
		Zahlen ins Dezimalsystem umrechnet.
		
		\begin{tasks}(2)
			\task \code{11100 + 00011 =}
			\task \code{11011 + 01001 =}
		\end{tasks}
	\end{enumerate}
\end{aufgabe}

\begin{aufgabe}[symbol=\Large\symHeft]
	Der Computer hat nicht unbegrenzt Speicherplatz. Normalerweise ist die Größe
	der Binärzahlen begrenzt, mit denen er rechnen kann. Zur Vereinfachung 
	legen wir die maximale Länge auf \emph{4-Bits} fest. Wir addieren also 
	Zahlen mit vier Bit und das Ergebnis hat auch vier Bit. (Bei weniger Bits 
	füllen wir den Rest mit Nullen auf.)
	
	Berechne folgende Ergebnisse der 4-Bit Addition:
	\begin{tasks}(2)
		\task \code{0001 + 0001 =}
		\task \code{0011 + 0001 =}
		\task \code{0111 + 0001 =}
		\task \code{1111 + 0001 =}
	\end{tasks}
	
	Welches Problem kann auftreten, wenn das Ergebnis nur maximal 4-Bit haben 
	darf?
	
	\feldLin{3}
\end{aufgabe}

\vspace{1cm}
\begin{rahmen}\centering
Klebe hier die Zusatzaufgabe ein, wenn du die anderen bearbeitest hast.
\vspace{5cm}
\end{rahmen}

\clearpage

\newsavebox{\aufgabeDrei}
\savebox{\aufgabeDrei}{%
\begin{rahmen}
	\textbf{\rmfamily\llap{\symPartner~} Aufgabe 3}\par\smallskip
	In dieser Aufgabe betrachten wir weiter 4-Bit Binärzahlen. Mit vier Bit
	können insgesamt 16 Zahlen dargestellt werden. Von $(0000)_2 = (0)_{10}$
	bis $(1111)_2 = (15)_{10}$.
	
	Was aber, wenn wir auch \emph{negative Zahlen} im Binärsystem darstellen wollen?
	
	\begin{enumerate}
		\item Überlegt euch, wie man negative Zahlen im Binärsystem darstellen 
		kann.
		\item Welche Zahlen können nach eurem Verfahren mit 4 Bit dargestellt werden?
		\item Probiert die Addition von Binärzahlen aus. Funktioniert sie noch
		wie zuvor?
	\end{enumerate}
\end{rahmen}%
}

\usebox{\aufgabeDrei}

\vfill
\usym{2701}\dotfill
\vfill

\usebox{\aufgabeDrei}

\vfill
\usym{2701}\dotfill
\vfill

\usebox{\aufgabeDrei}

\end{document}
