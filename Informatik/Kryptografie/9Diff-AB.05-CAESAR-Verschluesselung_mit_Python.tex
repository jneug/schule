\documentclass[11pt, a4paper, ngerman]{arbeitsblatt}

\ladeModule{
	theme,typo,icons,boxen,aufgaben
}

\aboptionen{
	name		= {J. Neugebauer},
	kuerzel 	= {Ngb},
	titel 		= {CAESAR-Verschlüsselung mit Python},
	reihe 		= {Kryptografie},
	fach 		= {Informatik},
	kurs 		= {9Diff},
	nummer 		= {IV.5},
	lizenz 		= {cc-by-nc-sa-4},
	version 	= {2021-04-28},
}


\ladeFach[quelltexte]{informatik}

\begin{document}
\ReiheTitel

Du weißt nun, wie die CAESAR-Verschlüsselung funktioniert und kennst dich auch
schon etwas mit \programm{Python} aus. Das ist eine gute Gelegenheit, beides zu verbinden
und ein \programm{Python}-Programm für die CAESAR-Verschlüsselung zu programmieren.

\begin{rahmen}\textbf{Zur Erinnerung einige Python-Befehle}\small\vspace{-2ex}
	\begin{multicols}{2}
	\code{print(\textquotedbl Hallo, Welt!")} Gibt den Text \enquote{Hallo, Welt!} aus.

	\code{input(\textquotedbl Gib etwas ein ein: ")} Gibt den Text \enquote{Gib etwas ein:} aus
			und wartet, bis der Nutzer die ENTER-Taste betätigt.

	\code{ord(\textquotedbl A")} Wandelt den Buchstaben \enquote{A} in seinen Unicode (hier \enquote{65}) um.

	\code{chr(65)} Wandelt den Unicode \enquote{65} in das passende Zeichen (hier \enquote{A}) um.

	\code{\textquotedbl Ein Text".upper()} Wandelt \enquote{Ein Text} in \enquote{EIN TEXT} um.

	\code{\textquotedbl Ein Text".lower()} Wandelt \enquote{Ein Text} in \enquote{ein text} um.

	\code{for x in \textquotedbl Hallo, Welt":} Setzt \code{x} auf jedes Zeichen in \enquote{Hallo, Welt}
	und führt den Code hinter \code{:} aus.
	\end{multicols}
\end{rahmen}

\begin{aufgabe}
	\label{aufg:encode}
\begin{wrapfix}
\begin{wrapfigure}[5]{r}{0pt}
	\includegraphics[width=2cm]{9Diff-AB.05-Abb_Puzzle}
\end{wrapfigure}

Bringe das folgende Programm-Puzzle in die korrekte Reihenfolge und
teste deine Lösung mit \programm{TigerJython}.

\hinweis{Du musst die Einrückungen selber noch passend setzen.}
\end{wrapfix}

\begin{itemize}[label=\faPuzzlePiece]
    \item\mintinline{python}{neuesZeichen = chr(verschoben)}
    \item\mintinline{python}{klartext = input("Klartextwort (ohne Leerzeichen): ")}
    \item\mintinline{python}{if verschoben > ord("Z"):}
    \item\mintinline{python}{geheimtext = ""}
    \item\mintinline{python}{unicode = ord(zeichen)}
    \item\mintinline{python}{schluessel = schluessel.upper()}
    \item\mintinline{python}{schluessel = ord(schluessel)-ord("A")}
    \item\mintinline{python}{print(klartext + " -> " + geheimtext)}
    \item\mintinline{python}{# Schleife über alle Zeichen im Klartext}
    \item\mintinline{python}{for zeichen in klartext:}
    \item\mintinline{python}{klartext = klartext.upper()}
    \item\mintinline{python}{verschoben = unicode + schluessel}
    \item\mintinline{python}{schluessel = input("Schlüsselbuchstabe: ")}
    \item\mintinline{python}{verschoben = verschoben - 26}
    \item\mintinline{python}{geheimtext = geheimtext + neuesZeichen}
\end{itemize}
\end{aufgabe}

\begin{aufgabe}
	\label{aufg:decode}
	\begin{enuma}
		\item Erweitere das Programm so, dass es auch mit \emph{Leerzeichen} im Klartext klarkommt.
		\item Entwickle ein Programm zum \emph{Entschlüsseln} eines Cäsar-Geheimtextes.
	\end{enuma}
\end{aufgabe}

\begin{aufgabe}
	\label{aufg:break}
	Du weißt auch schon, dass die CAESAR-Chiffre für moderne Computer recht leicht zu knacken
	ist. Entwickle ein Programm, dass einen Geheimtext \emph{ohne Kenntniss des Schlüssels}
	entschlüsseln kann.

	\hinweis{Das Programm kann nicht selber erkennen, ob der entschlüsselte Text ein sinnvoller
	Klartext ist. Daher muss diese Entscheidung vom Benutzer getroffen werden.}

	\hinweis{Nutze die Funktionen aus \prettyref{aufg:decode}.}
\end{aufgabe}

\end{document}
