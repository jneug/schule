\documentclass[10pt, a4paper]{scrartcl}

\usepackage{vorschule}
\usepackage[
	typ=ab,
	fach=Informatik,
	lerngruppe={Q2-GK},
	nummer=III.1,
	module={Symbole,Lizenzen},
	seitenzahlen=keine,
	farbig,
	lizenz=cc-by-nc-sa-4,
]{schule}

\usepackage[
	kuerzel=Ngb,
	reihe={Kryptografie},
	version={2020-01-08},
]{ngbschule}

\author{J. Neugebauer}
\title{Die CAESAR-Chiffre}
\date{\Heute}

\setzeAufgabentemplate{ngbnormal}




\begin{document}
\ReiheTitel

Bei der CAESAR-Verschlüsselung werden die Buchstaben des Alphabets (zunächst nur Großbuchstaben) um eine feste Anzahl an Stellen verschoben. Bei einer Verschiebung um 3 Stellen wird zum Beispiel aus einem \emph{B} ein \emph{E}, usw. Die Verschiebung kann auch als Buchstabe dargestellt werden. Im obigen Fall ist der Schlüsselbuchstabe \emph{D}, da bei einer Verschiebung um 3 Stellen der Buchstabe \emph{A} zu \emph{D} wird.

\begin{aufgabe}
	Chiffriere das Wort \code{KLARTEXT} durch das CAESAR-Verfahren mit dem Schlüsselbuchstaben \emph{G}.
\end{aufgabe}


\begin{aufgabe}
	Die drei Geheimtextnachrichten wurden mit dem CAESAR-Verfahren verschlüsselt. Die verwendeten Schlüssel sind auch angegeben, aber durcheinander geraten.
	
	\begin{center}
	\begin{tikzpicture}
		\node[draw,fill=black!10] at (0,0) {HMENQLZSHJ};
		\node[draw,fill=black!10] at (8,0) {CNIQTKVJOWU};
		\node[draw,fill=black!10] at (4,-1) {IFYJSXYWZPYZW};
		
		\node[draw,fill=black!25,diamond] at (1,-1) {C};
		\node[draw,fill=black!25,diamond] at (4,0) {Z};
		\node[draw,fill=black!25,diamond] at (6,-.8) {F};
	\end{tikzpicture}
	\end{center}
	
	Ordne die Schlüssel den Geheimtexten zu und entschlüssele die Botschaften.
\end{aufgabe}

\begin{aufgabe}
	Verschlüssele ein Wort oder einen Satz mit dem CAESAR-Verfahren. Tausche die verschlüsselte Botschaft dann mit einem Mitschüler/einer Mitschülerin aus, aber nicht die Schlüssel. Versucht nun jeweils die erhaltene Botschaft zu entschlüsseln.
\end{aufgabe}

\begin{aufgabe}
	Verfasst gemeinsam einen Algorithmus (Pseudocode), mit dem das CAESAR-Verfahren geknackt wernden kann (entschlüsseln, ohne das Schlüsselwort zu kennen).
\end{aufgabe}

\end{document}
