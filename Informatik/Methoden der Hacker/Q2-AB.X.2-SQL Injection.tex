\documentclass[10pt, a4paper]{scrartcl}

%\usepackage{qrcode}

\usepackage{vorschule}
\usepackage[
	typ=ab,
	fach=Informatik,
	lerngruppe={Q2},
	nummer=X.2,
	module={Symbole,Lizenzen},
	seitenzahlen=keine,
	farbig,
	lizenz=cc-by-nc-sa-4,
]{schule}

\usepackage[
	kuerzel=Ngb,
	reihe={Die Methoden der Hacker},
	version={2020-01-28},
]{ngbschule}

\author{J. Neugebauer}
\title{SQL-Injections}
\date{\Heute}

\setzeAufgabentemplate{ngbnormal}


\begin{document}
\ReiheTitel

Kopiere zur Vorbereitung die Dateien \datei{fussballem.db}, \datei{YeOldCheeseShoppe.db} und das Projekt \ordner{SuperSecureServer} aus dem Tauschordner. Du kannst eine der Datenbankdateien durch Doppelklick im Programm\programm{DB Browser for SQLite} öffnen. Hier kannst du dir den INhalt der Datenbank anzeigen lassen und SQL-Anfragen ausführen.

\begin{aufgabe}
	\begin{teilaufgaben}
		\teilaufgabe
		Öffne die Datenbank \datei{fussballem.db} und erkunde das Programm. Sende dann einige \code{SELECT} Anfragen an die Datenbank.
		
		\teilaufgabe
		Nutze den \code{INSERT} Befehl, um neue Daten in die Datenbank einzufügen. Suchen sie dazu im Internet nach Ergebnissen anderer EM-Jahrgänge.
		
		\teilaufgabe
		Nutze den \code{UPDATE} Befehl, um einige Datensätze zu verändern.
		
		\teilaufgabe
		Nutze den \code{DELETE} Befehl, um einige (oder alle) Datensätze zu löschen.
	\end{teilaufgaben}
	
	\hinweis{Unter \url{https://link.ngb.schule/sqlbefehle} finden sie eine Übersicht der SQL-Syntax und Befehle.}
	
	\hinweis{Falls du die Datenbank beim Arbeiten löscht oder \enquote{kaputt} machst, kannst du dir die Originalversion erneut aus dem Tauschordner kopieren.}
\end{aufgabe}

\begin{aufgabe}
	Öffnen die Seite \url{https://link.ngb.schule/sqlinjection} und lies den Text bis zur Überschrift \enquote{Wie können Webserver helfen?}. Erkläre dir selber, was eine \enquote{SQL-Injection} ist und welche Schwachstelle sie ausnutzt.
	
	Die Beispieldatenbank zum Text findest du in der Datei \datei{YeOldCheeseShoppe.db}. Hier kannst du die Abfragen selber nachvollziehen.
	
	\hinweis{Den ersten Teil brauchst du nur überfliegen, da die Grundlagen von SQLite beschrieben werden, die nun bekannt sein sollten. INteressant wird es ab der Überschirft \enquote{Verkettung von Zeichenfolgen: Die Wurzel allen Übels?}.}
\end{aufgabe}

\begin{aufgabe}
	Öffnen sie das Projekt \ordner{SuperSecureServer} in \programm{BlueJ}.
	
	\begin{teilaufgaben}
		\teilaufgabe
		Erstellen sie einen neuen \enquote{SuperSecureServer} und probieren sie die Methode \code{login} aus. Die Datenbank mit Nutzerkonten ist im Projektordner unter dem Namen \datei{auth.db} gespeichert.
		\teilaufgabe
		Studieren sie die Klasse \code{SuperSecureServer} und analysieren sie sie auf mögliche SQL-Injection Schwachstellen.
		\teilaufgabe
		Versuchen sie einen Weg zu finden, sich mit \code{login} anzumelden, ohne Nutzernamen oder Passwort eines Nutzers zu kennen.
	\end{teilaufgaben}
\end{aufgabe}

\end{document}
