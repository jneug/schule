\documentclass[11pt, a4paper, ngerman]{arbeitsblatt}

\ladeModule{theme,typo,icons,aufgaben}
\ladeFach[listings]{informatik}

\aboptionen{
	name 		= {J. Neugebauer},
	kuerzel 	= {Ngb},
	titel 		= {Grundstruktur einer HTML Seite},
	reihe 		= {Webseiten erstellen mit HTMLWebseiten erstellen mit HTML},
	fach 		= {Informatik},
	kurs 		= {8Diff},
	nummer 		= {II.1},
	lizenz 		= {cc-by-nc-sa-4},
	version 	= {2021-05-05},
}

\begin{document}
\ReiheTitel

\begin{quote}
HTML steht für \enquote{Hypertext Markup Language}. HTML ist eine formale Sprache zur Beschreibung der Struktur von Hypertexten (in Form verlinkter Webseiten) mit Hilfe von Auszeichnungen.
\end{quote}

\begin{aufgabe}[icon=\iconComputer]
Ruft die Adresse \url{https://link.ngb.schule/html-bailey} im Firefox Browser auf. Unten ist der Quelltext der Seite abgebildet. Studiert den Quelltext und vergleicht ihn mit der Darstellung der Seite im Browser. Beantwortet dann folgende Fragen:
\begin{enuma}\itemsep 0ex
	\item Wieso ist HTML eine \enquote{Auszeichnungssprache}?
	\item Welche Elemente erkennt ihr?
	\item Welche Grundstruktur hat eine HTML-Seite?
\end{enuma}
\begin{lstlisting}[language=HTML,basicstyle=\scriptsize\ttfamily]
<!doctype html>
<html>
<head>
	<meta charset="utf-8">
	<link href='styles/style.css' rel='stylesheet' type='text/css'>
	<title>Homepage von Bailey</title>
</head>
<body>
	<h1>Hallo, ich heiße Bailey</h1>
	<p>
	<img src="img/bailey1.jpg" alt="Foto von Bailey">
	</p>
	<p>
	Hallo, ich heiße Bailey und lebe in der Nähe von
	<a href="http://www.kaiserslautern.de">Kaiserslautern</a>.
	Ich bin ein Australian Shepherd, meine Vorfahren haben in
	Australien Schafe gehütet.
	</p>
	<p>
	Wenn ich erwachsen bin, will ich Agility-Sport treiben.
	Wisst ihr überhaupt, was das ist? Wenn nicht, dann schaut
	euch doch mal die Fotos bei
	<a href="https://de.wikipedia.org/wiki/Agility">Wikipedia</a>
	an. Ihr werdet staunen, was wir Hunde alles können!
	</p>
	<p>
	Ich kann aber auch schon ganz viel. Das könnt ihr auf
	der <a href="_bailey2.html">nächsten Seite</a> sehen.
	</p>
</body>
</html>
\end{lstlisting}
\end{aufgabe}

\begin{wrapfigure}{r}{3cm}
	\includegraphics[width=3cm]{8Diff-Ab.1-Abb_Menu.jpg}
\end{wrapfigure}
\begin{aufgabe}[icon=\iconComputer]
	Durch einen Rechtsklick im Browserfenster könnt ihr ein \menu{Element untersuchen}\\
	 und sogar \enquote{live} bearbeiten. Nutzt diesen \enquote{Inspektor}, um die Seite weiter zu \\
	 erkunden.
\end{aufgabe}

\end{document}
