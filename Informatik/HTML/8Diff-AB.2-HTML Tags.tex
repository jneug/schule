\documentclass[10pt, a4paper, ngerman]{arbeitsblatt}

\ladeModule{theme,typo,icons,tabellen,aufgaben}
\aboptionen{
	name		= {J. Neugebauer},
	kuerzel		= {Ngb},
	titel		= {Übersicht wichtiger HTML-Tags},
	reihe		= {Webseiten erstellen mit HTML},
	fach		= {Informatik},
	lerngruppe	= {8Diff},
	nummer		= {II.2},
	lizenz		= {cc-by-nc-sa-4},
	version		= {2021-05-27},
}

\usepackage{xinttools}

\begin{document}
\ReiheTitel

Ergänze die Tabelle nach und nach zu einem Spickzettel für wichtige
HTML-Tags. Beginne mit den vorhandenen Tags und recherchiere ihre
Bedeutung. Ergänze dann weitere Tags, die du kennenlernst.

%\medskip
\setlength{\zeilenhoehe}{1.1cm}
\begin{tabularx}{\textwidth}{|c|X|X|} \hline
	\rowcolor{ab.tabelle.kopf.hg}
	\textbf{Tag} & \textbf{Beispiele} & \textbf{Beschreibung} \\ \hline
	\texttt{doctype} & \texttt{<!doctype html>} & Legt fest welche Version von HTML verwendet wird. \\ \hline

	\xintFor #1 in {html,head,body}
	\do {\texttt{#1} & & \Zeilenabstand \\ \hline}

	\texttt{title} & \texttt{<title>Meine Seite</title>} & \Zeilenabstand \\ \hline
	\texttt{h1 (h2, h3, ...)} & \Zeilenabstand & Definiert Überschriften der ersten (zweiten, dritten, ...) Ebene. \\ \hline

	\xintFor #1 in {p, strong, em, u, img, a,,,,,,}
	\do {\texttt{#1} & & \Zeilenabstand \\ \hline}
\end{tabularx}

\end{document}
