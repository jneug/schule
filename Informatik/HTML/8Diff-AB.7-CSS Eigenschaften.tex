\documentclass[10pt, a4paper]{arbeitsblatt}

\ladeModule{theme,typo,icons,tabellen,boxen,aufgaben}
\aboptionen{
	name		= {J. Neugebauer},
	kuerzel		= {Ngb},
	titel		= {Übersicht wichtiger CSS-Eigenschaften},
	reihe		= {Webseiten erstellen mit HTML},
	fach		= {Informatik},
	lerngruppe	= {8Diff},
	nummer		= {II.7},
	lizenz		= {cc-by-nc-sa-4},
	version		= {2021-05-28},
}

\ladeFach[listings]{informatik}
\lstset{numbers=none}

\usepackage{xinttools}

\begin{document}
\ReiheTitel

\begin{savelst}{pseudoregel}
	\begin{lstlisting}[linewidth=6cm]
selektor {
  eigenschaft1: wert1;
  eigenschaft2: wert2;
}
\end{lstlisting}
\end{savelst}
\begin{savelst}{cssregel}
	\begin{lstlisting}[linewidth=6.2cm]
p {
  color: #ae34f2;
  text-decoration: underline;
}
\end{lstlisting}
\end{savelst}

\begin{tabularx}{\textwidth}{|X|p{6cm}|p{6.2cm}|} \hline
	\textbf{CSS-Regeln} & \loadlst{pseudoregel} & \loadlst{cssregel}\smallskip \\ \hline
\end{tabularx}\medskip

Such dir eine CSS-Eigenschaft unten aus und recherchiere seine Bedeutung auf
der Seite \url{https://www.css-wiki.com}. Erstelle ein Beispiel in deinem Steckbrief,
anhand dessen du die Eigenschaften und seine möglichen Werte demonstrieren
kannst.

\setlength{\zeilenhoehe}{.9cm}
\begin{tabularx}{\textwidth}{|c|X|X|} \hline
	\rowcolor{ab.tabelle.kopf.hg}
	\textbf{Eigenschaft} & \textbf{Beispiele} & \textbf{Beschreibung} \\ \hline
	\xintFor #1 in {color,background-color,font-family,font-size,font-style,font-weight,text-decoration,text-align,text-shadow,text-transform,border-style,border-radius,border-width,border-color,padding,margin,list-style-type,cursor}
	\do {\code{#1}     &                    & \Zeilenabstand   \\ \hline}
\end{tabularx}

\end{document}
