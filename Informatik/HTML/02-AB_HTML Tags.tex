\documentclass[a4paper,11pt,debug]{scrartcl}

\usepackage{vorschule}
\usepackage[
	typ=ab,
	fach=Informatik,
	lerngruppe={Diff 8},
	seitenzahlen=keine,
	module={},
]{schule}

\usepackage[
	kuerzel=Ngb,
	reihe={Webseiten erstellen mit HTML},
	version={0.1 (2018)}
]{ngbschule}

\author{J. Neugebauer}
\title{Übersicht wichtiger HTML-Tags}
\date{\Heute}

\usepackage{xinttools}

\zeilennummernAus
\begin{document}
\ReiheTitel

Ergänze die Tabelle nach und nach zu einem Spickzettel für wichtige HTML-Tags. Beginne mit den vorhandenen Tags und recherchiere ihre Bedeutung. Ergänze dann weitere Tags, die du kennenlernst.\medskip

\setlength{\zeilenhoehe}{1.1cm}
\begin{tabularx}{\textwidth}{|c|X|X|} \hline
	\textbf{Tag} & \textbf{Beispiele} & \textbf{Beschreibung} \\ \hline
	\texttt{doctype} & \texttt{<!doctype html>} & Legt fest welche Version von HTML verwendet wird. \\ \hline
	
	\xintFor #1 in {html,head,body}
	\do {\texttt{#1} & & \Zeilenabstand \\ \hline}
	
	\texttt{title} & \texttt{<title>Meine Seite</title>} & \Zeilenabstand \\ \hline
	\texttt{h1 (h2, h3, ...)} & \Zeilenabstand & Definiert Überschriften der ersten (zweiten, dritten, ...) Ebene. \\ \hline
	
	\xintFor #1 in {p, strong, em, u, img, a,,,,,,}
	\do {\texttt{#1} & & \Zeilenabstand \\ \hline}
\end{tabularx}

\end{document}