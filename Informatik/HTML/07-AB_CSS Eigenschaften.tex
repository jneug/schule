\documentclass[a4paper,11pt,debug]{scrartcl}

\usepackage{vorschule}
\usepackage[
	typ=ab,
	fach=Informatik,
	lerngruppe={Diff 8},
	seitenzahlen=keine,
	zitate=quotes,
	module={Lizenzen,Symbole},
	lizenz={cc-by-nc-sa-4},
]{schule}

\usepackage[
	kuerzel=Ngb,
	reihe={Webseiten erstellen mit HTML},
	version={2019-03-31},
	url={https://www.github.com/jneug/schule},
]{ngbschule}

\author{J. Neugebauer}
\title{Übersicht wichtiger CSS-Eigenschaften}
\date{\Heute}

\usepackage{xinttools}

\zeilennummernAus
\begin{document}
\ReiheTitel

\begin{savelst}{pseudoregel}
\begin{lstlisting}[linewidth=6cm]
selektor {
  eigenschaft1: wert1;
  eigenschaft2: wert2;
}
\end{lstlisting}	
\end{savelst}
\begin{savelst}{cssregel}
\begin{lstlisting}[linewidth=6.2cm]
p {
  color: #ae34f2;
  text-decoration: underline;
}
\end{lstlisting}
\end{savelst}

\begin{tabularx}{\textwidth}{|X|p{6cm}|p{6.2cm}|} \hline
	CSS-Regeln & \loadlst{pseudoregel} & \loadlst{cssregel}\smallskip \\ \hline
\end{tabularx}\medskip

Such dir eine CSS-Eigenschaft unten aus und recherchiere seine Bedeutung auf der Seite \url{https://www.css-wiki.com}.
Erstelle ein Beispiel in deinem Steckbrief, anhand dessen du die Eigenschaften und seine möglichen Werte demonstrieren kannst.

\begin{tabularx}{\textwidth}{|c|X|X|} \hline
	\textbf{Eigenschaft} & \textbf{Beispiele} & \textbf{Beschreibung} \\ \hline
	\xintFor #1 in {color,background-color,font-family,font-size,font-style,font-weight,text-decoration,text-align,text-shadow,text-transform,border-style,border-radius,border-width,border-color,padding,margin,list-style-type,cursor}
	\do {\texttt{#1} & & \Zeilenabstand[1cm] \\ \hline}
\end{tabularx}

\end{document}