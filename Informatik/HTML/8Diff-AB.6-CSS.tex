\documentclass[10pt, a4paper]{arbeitsblatt}

\ladeModule{theme,typo,icons,tabellen,boxen,aufgaben}
\aboptionen{
	name		= {J. Neugebauer},
	kuerzel		= {Ngb},
	titel		= {Gestaltung mit Cascading Style Sheets (CSS)},
	reihe		= {Webseiten erstellen mit HTML},
	fach		= {Informatik},
	lerngruppe	= {8Diff},
	nummer		= {II.6},
	lizenz		= {cc-by-nc-sa-4},
	version		= {2021-05-28},
}

\ladeFach[listings]{informatik}

\begin{document}
\ReiheTitel

Um die Elemente einer Webseite nach den eigenen Wünschen gestalten zu können
wurde die Sprache CSS entwickelt. Sie erlaubt die Trennung von Seiteninhalt
(Texte, Tabellen, Navigationselemente, …) und Gestaltung (Farben, Schriftarten,
Rahmen, …).

Indem das Aussehen der Webseite getrennt vom HTML-Quelltext in einer separaten
Datei gespeichert wird, kann ganz einfach durch den Tausch der CSS-Datei das
Aussehen der Webseite vollständig verändert werden. Auf der anderen Seite kann
dieselbe Gestaltungsdatei auf mehreren Seiten wiederverwendet werden.

\begin{aufgabe}[icon=\iconLaptop]
	Öffne deinen Steckbrief im \programm{HTML-Editor}. Links oben
	findest du in deinem Projekt einen Ordner \ordner{css}, in dem eine
	Datei mit dem Namen \datei{formate.css} gespeichert ist. In diese Datei
	kannst du deine CSS-Formatierungen schreiben.

	Damit die Datei von deiner Webseite benutzt wird, muss du sie mit der Seite
	\enquote{verlinken}. Öffne dazu deinen Steckbrief und suche das
	\lstinline[language=HTML]{<head>}-Tag ganz oben. Füge zwischen \lstinline[language=HTML]{<head>} und
	\lstinline[language=HTML]{</head>} das folgende Tag ein:
	\begin{lstlisting}[language=HTML]
<link rel="stylesheet" type="text/css" href="css/formate.css">
	\end{lstlisting}
\end{aufgabe}

\begin{aufgabe}[icon=\iconLaptop]
	Füge in die Datei \datei{formate.css} folgenden
	CSS-Quelltext ein, speichere sie ab und betrachte deinen Steckbrief nun in der
	Vorschau.
	\begin{lstlisting}[language=HTML,basicstyle=\scriptsize\ttfamily]
body {
  background-color: yellow;
}
p {
  color: magenta;
}
strong, b {
  text-decoration: underline;
  font-weight: normal;
}
em, i {
  font-style: normal;
  font-weight: bold;
}
u {
  text-decoration: none;
  font-style: italic;
}
	\end{lstlisting}
\end{aufgabe}

\begin{aufgabe}[icon=\iconLaptop]
	Schau dir den Quelltext oben und im Editor genau an und
	versuche zu erklären, was du siehst. Wie ist eine CSS-Datei aufgebaut?
\end{aufgabe}

\begin{aufgabe}[icon=\iconLaptop]
	Versuche anhand des Quelltextes die Bedeutung folgender
	Begriffe zu erklären:
	\begin{multicols}{4}
		CSS-Regel

		CSS-Selektor

		CSS-Eigenschaft

		Eigenschafts-Wert
	\end{multicols}
\end{aufgabe}

\begin{aufgabe*}
	Erkläre anhand eines (theoretischen) Beispiels, warum
	die Trennung von \emph{Gestaltung} und \emph{Inhalt} sinnvoll ist.
\end{aufgabe*}

\end{document}
